\documentclass{ximera}

% These macros are automatically generated from the "macros"
% XML element.  Make permanent edits there.
%
% History
%   2004/01/01  Initiated for FCLA, evolved from there
%   2006/09/17  Converted  _, ^  to \sb, \sp for TeX4ht
%   2014/02/01  Updated for MathBook XML projects
%               Obsolete in FCLA: \codeindent, \computerfont, \define
%               Change: MathJax wants \lt, so replaced by \lteval
%   2014/02/22  New: \orderof, \reals, \per
%   2015/08/16  Incorporated into MathBook XML version of FCLA
%
%%%%%%%%%%%%%%%%%%%%%
%
%     Conveniences
%
%%%%%%%%%%%%%%%%%%%%%
%
%  Order of (asymptotically limit of fraction is 1)
%  Usage: \orderof{some function}
%
\newcommand{\orderof}[1]{\sim #1}
%
%  Integers
%  Usage:  \Z
\newcommand{\Z}{\mathbb{Z}}
%
%  Real numbers, as set of scalars
%  Usage:  \reals
\newcommand{\reals}{\mathbb{R}}
%
%  n-space over real field
%  Usage: \complex{integer-dimension}
\newcommand{\real}[1]{\mathbb{R}^{#1}}
%
%  Complex numbers, as set of scalars
%  Usage:  \complexes
\newcommand{\complexes}{\mathbb{C}}
%
%  n-space over complex field
%  Usage: \complex{integer-dimension}
\newcommand{\complex}[1]{\mathbb{C}^{#1}}
\newcommand{\CC}{\mathbb{C}}
%
%  Complex conjugation (scalar, vector, matrix)
%  Usage: \conjugate{object}
\newcommand{\conjugate}[1]{\overline{#1}}
%
%  Complex number modulus
%  Usage: \modulus{a+bi}
%  Presumes math mode
\newcommand{\modulus}[1]{\left\lvert#1\right\rvert}
%
%  Zero vector
%  Usage: \zerovector
\newcommand{\zerovector}{\vect{0}}
%
%  Zero matrix
%  Usage: \zeromatrix, use a subscript when size is important
\newcommand{\zeromatrix}{\mathcal{O}}
%
%  Inner product (brackets, not quadratic form)
%  Usage: \innerproduct{a-vector}{a-vector}
\newcommand{\innerproduct}[2]{\left\langle#1,\,#2\right\rangle}
%
%  Norm of a vector
%  Usage: \norm{a-vector}
\newcommand{\norm}[1]{\left\lVert#1\right\rVert}
%
%  Dimension
%  Usage: \dimension{vector-space-letter}
\newcommand{\dimension}[1]{\dim\left(#1\right)}
%
%  Nullity
%  Usage: \nullity{matrix-or-lintrans-letter}
\newcommand{\nullity}[1]{n\left(#1\right)}
%
%  Rank
%  Usage: \rank{matrix-or-lintrans-letter}
\newcommand{\rank}[1]{r\left(#1\right)}
%
%  Direct sum
%  Usage: \ds between a couple of subspaces
%
\newcommand{\ds}{\oplus}
%
%  Determinant of a matrix (functional)
%  Usage: \detname{A}
\newcommand{\detname}[1]{\det\left(#1\right)}
%
%  Determinant of a matrix (vertical bars)
%  Usage: \detbars{A}
\newcommand{\detbars}[1]{\left\lvert#1\right\rvert}
%
%  Trace of a Matrix
%  Usage: \trace{matrix name}
\newcommand{\trace}[1]{t\left(#1\right)}
%
%  Square Root of a Matrix
%  Usage: \sr{a-matrix}
\newcommand{\sr}[1]{#1^{1/2}}
%
%%%%%%%%%%%%%%%%%%%%%
%
%     Subspace Constructions
%
%%%%%%%%%%%%%%%%%%%%%
%
%  Span of a set of vectors
%  \span and \sp are used by TeX for other things
%  Usage: \spn{set-of-vectors}
\newcommand{\spn}[1]{\left\langle#1\right\rangle}
%
%  Null space of a matrix
%  Usage:  \nsp{A}
\newcommand{\nsp}[1]{\mathcal{N}\!\left(#1\right)}
%
%  Column space of a matrix
%  Usage:  \csp{A}
\newcommand{\csp}[1]{\mathcal{C}\!\left(#1\right)}
%
%  Row space of a matrix
%  Usage:  \rsp{A}
\newcommand{\rsp}[1]{\mathcal{R}\!\left(#1\right)}
%
%  Left null space of a matrix
%  Usage:  \lns{A}
\newcommand{\lns}[1]{\mathcal{L}\!\left(#1\right)}
%
%  Orthogonal complement of a vector space
%  Avoiding TeX's \perp
%  Usage:  \per{A}
\newcommand{\per}[1]{#1^\perp}
%
%%%%%%%%%%%%%%%%%%%%%
%
%     Systems of Equations
%
%%%%%%%%%%%%%%%%%%%%%
%
%  In-line form of an augmented matrix for a system of equations
%  Usage: \augmented{coefficient-matrix}{constant-vector}
\newcommand{\augmented}[2]{\left\lbrack\left.#1\,\right\rvert\,#2\right\rbrack}
%
%  Notation for a linear system before introducing matrix multiplication
%  Usage: \linearsystem{coefficient-matrix}{constant-vector}
\newcommand{\linearsystem}[2]{\mathcal{LS}\!\left(#1,\,#2\right)}
%
%  Notation for a homogenous system before introducing matrix multiplication
%  Usage: \homosystem{coefficient-matrix}
\newcommand{\homosystem}[1]{\linearsystem{#1}{\zerovector}}
%
%%%%%%%%%%%%%%%%%%%%%
%
%     Row Operations, Echelon Form
%
%%%%%%%%%%%%%%%%%%%%%
%
% Row operations on matrices
%
% Three commands to shorten up descriptions of gaussian elimination
%
% Usage: \rowopswap{row-i}{row-j}
% Usage: \rowopmult{scalar}{row-i}
% Usage: \rowopadd{scalar}{row-multiplied}{row-added-to}
\newcommand{\rowopswap}[2]{R_{#1}\leftrightarrow R_{#2}}
\newcommand{\rowopmult}[2]{#1R_{#2}}
\newcommand{\rowopadd}[3]{#1R_{#2}+R_{#3}}
%
% Mark leading 1's in echelon form with fbox
% Usage: \leading{a-1-usually}
\newcommand{\leading}[1]{\fbox{#1}}
%
%  Row-reduce arrow
%  Usage:  \rref inbetween a matrix and its reduced row-echelon form
\newcommand{\rref}{\xrightarrow{\text{RREF}}}
%
%  Elementary Matrices
%  Usage: \elemswap{subscript}{subscript}
%  Usage: \elemmult{scalar}{subscript}
%  Usage: \elemadd{scalar}{subscript-mult}{subscript-target}
%
\newcommand{\elemswap}[2]{E_{#1,#2}}
\newcommand{\elemmult}[2]{E_{#2}\left(#1\right)}
\newcommand{\elemadd}[3]{E_{#2,#3}\left(#1\right)}
%
%%%%%%%%%%%%%%%%%%%%%
%
%     2-D Constructions (Lists, Vectors, Matrices)
%
%%%%%%%%%%%%%%%%%%%%%
%
%  A list of scalars of generic length
%  Usage:  \scalarlist{scalar letter}{terminal subscript}
\newcommand{\scalarlist}[2]{{#1}_{1},\,{#1}_{2},\,{#1}_{3},\,\ldots,\,{#1}_{#2}}
%
%  Vector styling, bold (or use wiggles, arrows, whatever)
%  Subscripts go outside this construction
%  Usage: \vect{a symbol to use as a vector}
%  Have to already be in math mode
%
\newcommand{\vect}[1]{\mathbf{#1}}
%
%  A column vector
%  Usage: \colvector{list-delimited-by-\\}
%
\newcommand{\colvector}[1]{\begin{bmatrix}#1\end{bmatrix}}
%
%  A generic vector with components
%  Usage: \vectorcomponents{component-letter}{final-subscript}
\newcommand{\vectorcomponents}[2]{\colvector{#1_{1}\\#1_{2}\\#1_{3}\\\vdots\\#1_{#2}}}
%
%  A list of vectors of generic length
%  Usage:  \vectorlist{vector letter}{terminal subscript}
\newcommand{\vectorlist}[2]{\vect{#1}_{1},\,\vect{#1}_{2},\,\vect{#1}_{3},\,\ldots,\,\vect{#1}_{#2}}
%
%  Vector entries, entry i of vector v
%  (vector-expession still needs \vect, etc.)
%  Usage:  \vectorentry{vector-expression}{single-subscript}
\newcommand{\vectorentry}[2]{\left\lbrack#1\right\rbrack_{#2}}
%
%  Matrix entries, entry i,j of matrix A
%  Usage:  \matrixentry{matrix-expression}{paired-subscripts}
%
\newcommand{\matrixentry}[2]{\left\lbrack#1\right\rbrack_{#2}}
%
%  A generic linear combination
%  Usage:  \lincombo{scalar letter}{vector letter}{terminal subscript}
\newcommand{\lincombo}[3]{#1_{1}\vect{#2}_{1}+#1_{2}\vect{#2}_{2}+#1_{3}\vect{#2}_{3}+\cdots +#1_{#3}\vect{#2}_{#3}}
%
%  Matrix, column by column, as vectors
%  Usage:  \matrixcolumns{matrix letter}{terminal subscript}
\newcommand{\matrixcolumns}[2]{\left\lbrack\vect{#1}_{1}|\vect{#1}_{2}|\vect{#1}_{3}|\ldots|\vect{#1}_{#2}\right\rbrack}
%
%%%%%%%%%%%%%%%%%%%%%
%
%     Special Matrices
%
%%%%%%%%%%%%%%%%%%%%%
%
%  Transpose of a matrix
%  Usage:  \transpose{A}
\newcommand{\transpose}[1]{#1^{t}}
%
%  Inverse of a matrix
%  Usage:  \inverse{A}
\newcommand{\inverse}[1]{#1^{-1}}
%
%  Submatrix (for minors, determinants)
%  Usage: \submatrix{matrix-name}{delete-row}{delete-col}
\newcommand{\submatrix}[3]{#1\left(#2|#3\right)}
%
%  Adjoint of a matrix (twice)
%  This macro is a convenience to call \transpose and \conjugate properly
%  It shouldn't need to be modified (or mathematical meanings will change)
%  Usage:  \adj{A}
\newcommand{\adj}[1]{\transpose{\left(\conjugate{#1}\right)}}
%
%  This macro controls the symbol used for the adjoint
%  It can be edited to taste
%  Usage:  \adjoint{A}
\newcommand{\adjoint}[1]{#1^\ast}
%
%%%%%%%%%%%%%%%%%%%%%
%
%     Sets
%
%%%%%%%%%%%%%%%%%%%%%
%
%  A convenience for simple sets
%  Usage:  \set{list of element}
\newcommand{\set}[1]{\left\{#1\right\}}
%
%  Sets with vertical bar, "such that", sized for objects, not condition
%  Usage:  \setparts{objects}{condition}
%
%%\newcommand{\setparts}[2]{\left\{ #1\mid#2\right\}}
%%\newcommand{\setparts}[2]{\left\{\left. #1\right\rvert#2\right\}}
\newcommand{\setparts}[2]{\left\lbrace#1\,\middle|\,#2\right\rbrace}
%
%  Set Cardinality
%  Usage:  \card{a-set-letter}
\newcommand{\card}[1]{\left\lvert#1\right\rvert}
%
%  Set Union
%  Use \cup
%
%  Set Intersection
%  Use \cap
%
%  Set Complement
%  Usage:  \setcomplement{a-set-letter}
\newcommand{\setcomplement}[1]{\overline{#1}}
%
%%%%%%%%%%%%%%%%%%%%%
%
%     Eigenvalues and Eigenspaces
%
%%%%%%%%%%%%%%%%%%%%%
%
%  Characteristic polynomial
%  Usage: \charpoly{matrix-letter}{variable-letter}
\newcommand{\charpoly}[2]{p_{#1}\left(#2\right)}
%
%  Eigenspace
%  Usage: \eigenspace{matrix-letter}{eigenvalue-letter}
\newcommand{\eigenspace}[2]{\mathcal{E}_{#1}\left(#2\right)}
%
%  2013/10/03 Including ampersands is problematic here, 
%  think about fixes later
%  2014/02/22 Limited testing, seems &amp; is fine for HTML and LaTeX
%  2016-07-20 only employed in Archetypes, MBX has gather/align override
%  Eigensystem (presumes wrapped in an mrow within md)
%  Usage: \eigensystem{matrixletter}{eigenvalue}{list of basis vectors}
\newcommand{\eigensystem}[3]{\lambda&amp;=#2&amp;\eigenspace{#1}{#2}&amp;=\spn{\set{#3}}} 
%
%  Generalized Eigenspace
%  Usage: \geneigenspace{lin-trans-letter}{eigenvalue-letter}
\newcommand{\geneigenspace}[2]{\mathcal{G}_{#1}\left(#2\right)}
%
%  Algebraic multiplicty
%  Usage: \algmult{matrix-letter}{eigenvalue-letter}
\newcommand{\algmult}[2]{\alpha_{#1}\left(#2\right)}
%
%  Geometric multiplicty
%  Usage: \geomult{matrix-letter}{eigenvalue-letter}
\newcommand{\geomult}[2]{\gamma_{#1}\left(#2\right)}
%
%  Index (of eigenvalue)
%  Usage: \indx{matrix-letter}{eigenvalue-letter}
\newcommand{\indx}[2]{\iota_{#1}\left(#2\right)}
%
%%%%%%%%%%%%%%%%%%%%%
%
%     Linear Transformations
%
%%%%%%%%%%%%%%%%%%%%%
%
%  MathJax defines \lt to ease XML confusion
%
%  Linear transformation definition
%  Usage: \ltdefn{name-letter}{domain}{range}
\newcommand{\ltdefn}[3]{#1\colon #2\rightarrow#3}
%
%  Linear transformation evaluation
%  Usage: \lteval{name-letter}{input}
%  Replaces old \lt desired by MathJax
\newcommand{\lteval}[2]{#1\left(#2\right)}
%
% Linear transformation inverse
%  Usage: \ltinverse{name-letter}
\newcommand{\ltinverse}[1]{#1^{-1}}
%
%  Linear transformation restriction
%  Usage: \restrict{name-letter}{subspace-letter}
\newcommand{\restrict}[2]{{#1}|_{#2}}
%
%  Linear transformation preimage
%  Usage: \preimage{name-letter}{codomain-element}
\newcommand{\preimage}[2]{#1^{-1}\left(#2\right)}
%
%  Range of a linear transformation
%  TeX uses \range for something else
%  Usage:  \rng{T}
\newcommand{\rng}[1]{\mathcal{R}\!\left(#1\right)}
%
%  Kernel of a linear transformation
%  TeX uses \ker to do something different
%  Usage:  \krn{T}
\newcommand{\krn}[1]{\mathcal{K}\!\left(#1\right)}
%
%  Linear transformation composition
%  Usage: \compose{function-name}{function-name}
\newcommand{\compose}[2]{{#1}\circ{#2}}
%
%  Vector space of linear transformations
%  Usage: \vslt{domains}{codomains}
%  Presumes math mode
\newcommand{\vslt}[2]{\mathcal{LT}\left(#1,\,#2\right)}
%
%%%%%%%%%%%%%%%%%%%%%
%
%     Vector and Matrix Representations
%
%%%%%%%%%%%%%%%%%%%%%
%
%  Isomorphism symbol
%  Usage: \isomorphic
\newcommand{\isomorphic}{\cong}
%
%  Similarity
%  Usage: \similar{inner-matrix}{outer-invertible-matrix}
%  Rearranging this will not "fix" all desired changes throughout
%
\newcommand{\similar}[2]{\inverse{#2}#1#2}
%
%  Vector representation function name
%  Usage: \vectrepname{basis-letter}
\newcommand{\vectrepname}[1]{\rho_{#1}}
%
%  Vector representation output
%  Usage: \vectrep{basis-letter}{input}
\newcommand{\vectrep}[2]{\lteval{\vectrepname{#1}}{#2}}
%
%  Vector representation inverse function name
%  (Added later, not used consistently in FCLA)
%  Usage: \vectrepinvname{basis-letter}
\newcommand{\vectrepinvname}[1]{\ltinverse{\vectrepname{#1}}}
%
%  Vector representation inverse output
%  Usage: \vectrepinv{basis-letter}{input}
\newcommand{\vectrepinv}[2]{\lteval{\ltinverse{\vectrepname{#1}}}{#2}}
%
%  Matrix representation
%  Usage: \matrixrep{transformation-letter}{domain-basis-letter}{codomain-basis-letter}
\newcommand{\matrixrep}[3]{M^{#1}_{#2,#3}}
%
%  Matrix representation column-by-colum
%  2016-07-20 only employed once?
%  Usage: \matrixrepcolumns{transformation-letter}{codomain-basis-letter}{codomain-basis-vector-letter}{final-index}
\newcommand{\matrixrepcolumns}[4]{\left\lbrack \left.\vectrep{#2}{\lteval{#1}{\vect{#3}_{1}}}\right|\left.\vectrep{#2}{\lteval{#1}{\vect{#3}_{2}}}\right|\left.\vectrep{#2}{\lteval{#1}{\vect{#3}_{3}}}\right|\ldots\left|\vectrep{#2}{\lteval{#1}{\vect{#3}_{#4}}}\right.\right\rbrack}
%
%  Change of basis matrix
%  Usage: \cbm{domain-basis-letter}{codomain-basis-letter}
\newcommand{\cbm}[2]{C_{#1,#2}}
%
%%%%%%%%%%%%%%%%%%%%%
%
%     Canonical Forms
%
%%%%%%%%%%%%%%%%%%%%%
%
%  Jordan blocks
%  Usage: \jordan{size}{diagonal-element}
\newcommand{\jordan}[2]{J_{#1}\left(#2\right)}
%
%%%%%%%%%%%%%%%%%%%%%
%
%     Hadamard Matrices
%     Contributed by Elizabeth Million
%
%%%%%%%%%%%%%%%%%%%%%
%
%  Hadamard Product
%  Usage: \hadamard{a-matrix}{a-matrix}
\newcommand{\hadamard}[2]{#1\circ #2}
%
%  Hadamard identity matrix
%  Usage: \hadamardidentity{paired-subscripts-size-of-matrix}
\newcommand{\hadamardidentity}[1]{J_{#1}}
%
%  Hadamard inverse matrix
%  Usage: \hadamardinverse{matrix-expression}
\newcommand{\hadamardinverse}[1]{\widehat{#1}}

\newcommand{\definedTerm}[1]{\textbf{#1}}
\newcommand{\dfn}[1]{\textbf{#1}}

\newcommand{\wt}{\widetilde}
\newcommand{\ov}{\overline}
\newcommand{\inj}{\rightarrowtail}
\newcommand{\surj}{\twoheadrightarrow}
\newcommand{\harpoon}{\overset{\rightharpoonup}}

\newenvironment{amatrix}[1]{%
  \left[\begin{array}{@{}*{#1}{c}|c@{}}
}{%
  \end{array}\right]
}


\title{Computing Determinants}

\begin{document}
\begin{abstract}
  There are a variety of ways to compute the determinant.  We will
  establish first that we can choose to mimic our definition of the
  determinant, but by using matrix entries and submatrices based on a
  row other than the first one.
\end{abstract}
\maketitle

\begin{theorem}[Determinant Expansion about Rows]
\label{theorem:DER}

Suppose that $A$ is a square matrix of size $n$.  Then for $1\leq i\leq n$
\begin{align*}
  \detname{A}&=
               (-1)^{i+1}\matrixentry{A}{i1}\detname{\submatrix{A}{i}{1}}+
               (-1)^{i+2}\matrixentry{A}{i2}\detname{\submatrix{A}{i}{2}}\\
             &\quad+(-1)^{i+3}\matrixentry{A}{i3}\detname{\submatrix{A}{i}{3}}+
               \cdots+
               (-1)^{i+n}\matrixentry{A}{in}\detname{\submatrix{A}{i}{n}}
\end{align*}
which is known as \dfn{expansion} about row $i$.

\begin{proof}
  First, the statement of the theorem coincides with
  \ref{definition:DM} when $i=1$, so throughout, we need only consider
  $i>1$.

  Given the recursive definition of the determinant, it should be no
  surprise that we will use induction for this proof
  (\ref{technique:I}).  When $n=1$, there is nothing to prove since
  there is but one row.  When $n=2$, we just examine expansion about
  the second row,
  \begin{align*}
(-1)^{2+1}\matrixentry{A}{21}&\detname{\submatrix{A}{2}{1}}+
                               (-1)^{2+2}\matrixentry{A}{22}\detname{\submatrix{A}{2}{2}}\\
                             &=-\matrixentry{A}{21}\matrixentry{A}{12}+\matrixentry{A}{22}\matrixentry{A}{11}
                             &&\ref{definition:DM}\\
&=\matrixentry{A}{11}\matrixentry{A}{22}-\matrixentry{A}{12}\matrixentry{A}{21}\\
                             &=
                               \detname{A}&&\ref{theorem:DMST}\\
  \end{align*}
  
  So the theorem is true for matrices of size $n=1$ and $n=2$.  Now
  assume the result is true for all matrices of size $n-1$ as we
  derive an expression for expansion about row $i$ for a matrix of
  size $n$.  We will abuse our notation for a submatrix slightly, so
  $\submatrix{A}{i_1,i_2}{j_1,j_2}$ will denote the matrix formed by
  removing rows $i_1$ and $i_2$, along with removing columns $j_1$ and
  $j_2$.  Also, as we take a determinant of a submatrix, we will need
  to ``jump up'' the index of summation partway through as we ``skip
  over'' a missing column.  To do this smoothly we will set
  \[
    \epsilon_{\ell j}=
    \begin{cases}
      0 & \ell<j\\
      1 & \ell>j
    \end{cases}
  \]

  Now,
  \begin{align*}
    &\detname{A}\\
&\quad=
  \sum_{j=1}^{n}(-1)^{1+j}\matrixentry{A}{1j}\detname{\submatrix{A}{1}{j}}
&&\ref{definition:DM}\\
    &\quad=
      \sum_{j=1}^{n}(-1)^{1+j}\matrixentry{A}{1j}
      \sum_{\substack{1\leq\ell\leq n\\\ell\neq j}}
    (-1)^{i-1+\ell-\epsilon_{\ell j}}\matrixentry{A}{i\ell}\detname{\submatrix{A}{1,i}{j,\ell}}
    &&\text{Induction}\\
    &\quad=
      \sum_{j=1}^{n}\sum_{\substack{1\leq\ell\leq n\\\ell\neq j}}
    (-1)^{j+i+\ell-\epsilon_{\ell j}}
\matrixentry{A}{1j}\matrixentry{A}{i\ell}\detname{\submatrix{A}{1,i}{j,\ell}}
    &&\ref{property:DCN}\\
    &\quad=
      \sum_{\ell=1}^{n}\sum_{\substack{1\leq j\leq n\\j\neq\ell}}
    (-1)^{j+i+\ell-\epsilon_{\ell j}}
    \matrixentry{A}{1j}\matrixentry{A}{i\ell}\detname{\submatrix{A}{1,i}{j,\ell}}
&&\ref{property:CACN}\\
    &\quad=
      \sum_{\ell=1}^{n}(-1)^{i+\ell}\matrixentry{A}{i\ell}
      \sum_{\substack{1\leq j\leq n\\j\neq\ell}}
    (-1)^{j-\epsilon_{\ell j}}
    \matrixentry{A}{1j}\detname{\submatrix{A}{1,i}{j,\ell}}
    &&\ref{property:DCN}\\
    &\quad=
      \sum_{\ell=1}^{n}(-1)^{i+\ell}\matrixentry{A}{i\ell}
      \sum_{\substack{1\leq j\leq n\\j\neq\ell}}
    (-1)^{\epsilon_{\ell j}+j}
    \matrixentry{A}{1j}\detname{\submatrix{A}{i,1}{\ell,j}}
    &&\text{$2\epsilon_{\ell j}$ is even}\\
    &\quad=
      \sum_{\ell=1}^{n}(-1)^{i+\ell}\matrixentry{A}{i\ell}\detname{\submatrix{A}{i}{\ell}}
&&\ref{definition:DM}
  \end{align*}

\end{proof}
\end{theorem}

We can also obtain a formula that computes a determinant by expansion
about a column, but this will be simpler if we first prove a result
about the interplay of determinants and transposes.  Notice how the
following proof makes use of the ability to compute a determinant by
expanding about \textit{any} row.

\begin{theorem}[Determinant of the Transpose]
  \label{theorem:DT}
  Suppose that $A$ is a square matrix.  Then $\detname{\transpose{A}}=\detname{A}$.

  \begin{proof}
    With our definition of the determinant (\ref{definition:DM}) and
    theorems like \ref{theorem:DER}, using induction
    (\ref{technique:I}) is a natural approach to proving properties of
    determinants.  And so it is here.  Let $n$ be the size of the
    matrix $A$, and we will use induction on $n$.

    For $n=1$, the transpose of a matrix is identical to the original
    matrix, so vacuously, the determinants are equal.

    Now assume the result is true for matrices of size $n-1$.  Then,
    \begin{align*}
      \detname{\transpose{A}}
      &=\frac{1}{n}\sum_{i=1}^{n}\detname{\transpose{A}}\\
      &=
        \frac{1}{n}\sum_{i=1}^{n}\sum_{j=1}^{n}(-1)^{i+j}
        \matrixentry{\transpose{A}}{ij}\detname{\submatrix{\transpose{A}}{i}{j}}
      &&\ref{theorem:DER}\\
      &=
        \frac{1}{n}\sum_{i=1}^{n}\sum_{j=1}^{n}(-1)^{i+j}
        \matrixentry{A}{ji}\detname{\submatrix{\transpose{A}}{i}{j}}
      &&\ref{definition:TM}\\
      &=
        \frac{1}{n}\sum_{i=1}^{n}\sum_{j=1}^{n}(-1)^{i+j}
        \matrixentry{A}{ji}\detname{\transpose{\left(\submatrix{A}{j}{i}\right)}}
      &&\ref{definition:TM}\\
      &=
        \frac{1}{n}\sum_{i=1}^{n}\sum_{j=1}^{n}(-1)^{i+j}
        \matrixentry{A}{ji}\detname{\submatrix{A}{j}{i}}
      &&\text{Induction Hypothesis}\\
      &=
        \frac{1}{n}\sum_{j=1}^{n}\sum_{i=1}^{n}(-1)^{j+i}
        \matrixentry{A}{ji}\detname{\submatrix{A}{j}{i}}
      &&\ref{property:CACN}\\
      &=
        \frac{1}{n}\sum_{j=1}^{n}\detname{A}
      &&\ref{theorem:DER}\\
      &=\detname{A}
    \end{align*}
  \end{proof}
\end{theorem}

Now we can easily get the result that a determinant can be computed by
expansion about any column as well.

\begin{theorem}[Determinant Expansion about Columns]
  \label{theorem:DEC}
  Suppose that $A$ is a square matrix of size $n$.  Then for
  $1\leq j\leq n$
  \begin{align*}
  \detname{A}&=
               (-1)^{1+j}\matrixentry{A}{1j}\detname{\submatrix{A}{1}{j}}+
               (-1)^{2+j}\matrixentry{A}{2j}\detname{\submatrix{A}{2}{j}}\\
             &\quad+(-1)^{3+j}\matrixentry{A}{3j}\detname{\submatrix{A}{3}{j}}+
               \cdots+
               (-1)^{n+j}\matrixentry{A}{nj}\detname{\submatrix{A}{n}{j}}
  \end{align*}
  which is known as \dfn{expansion} about column $j$.

  \begin{proof}

    \begin{align*}
      \detname{A}
      &=
        \detname{\transpose{A}}&&\ref{theorem:DT}\\
      &=
        \sum_{i=1}^{n}(-1)^{j+i}\matrixentry{\transpose{A}}{ji}\detname{\submatrix{\transpose{A}}{j}{i}}
                               &&\ref{theorem:DER}\\
      &=
        \sum_{i=1}^{n}(-1)^{j+i}\matrixentry{\transpose{A}}{ji}\detname{\transpose{\left(\submatrix{A}{i}{j}\right)}}
                               &&\ref{definition:TM}\\
      &=
        \sum_{i=1}^{n}(-1)^{j+i}\matrixentry{\transpose{A}}{ji}\detname{\submatrix{A}{i}{j}}
                               &&\ref{theorem:DT}\\
      &=
        \sum_{i=1}^{n}(-1)^{i+j}\matrixentry{A}{ij}\detname{\submatrix{A}{i}{j}}
                               &&\ref{definition:TM}
    \end{align*}
  \end{proof}
\end{theorem}

That the determinant of an $n\times n$ matrix can be computed in $2n$
different (albeit similar) ways is nothing short of remarkable.  For
the doubters among us, we will do an example, computing a $4\times 4$
matrix in two different ways.

\begin{example}[Two computations, same determinant]
  Let
  \[
    A=
    \begin{bmatrix}
      -2 & 3 & 0 & 1\\
      9 & -2 & 0 & 1\\
      1 & 3 & -2 & -1\\
      4 & 1 & 2 & 6
    \end{bmatrix}
  \]

  Then expanding about the fourth row (\ref{theorem:DER} with $i=4$)
  yields,
  \begin{align*}
    \detbars{A}
    &=
      (4)(-1)^{4+1}
      \begin{vmatrix}
        \answer{3} & 0 & 1\\
        \answer{-2} & 0 & 1\\
        \answer{3} & -2 & -1
      \end{vmatrix}
                 +(1)(-1)^{4+2}
                 \begin{vmatrix}
                   -2 &  0 & 1\\
                   9 &  0 & 1\\
                   1 &  -2 & -1
                 \end{vmatrix}\\
    &\quad\quad+(2)(-1)^{4+3}
      \begin{vmatrix}
        -2 & 3 &  1\\
        9 & -2 &  1\\
        1 & 3  & -1
      \end{vmatrix}
                 +(6)(-1)^{4+4}
                 \begin{vmatrix}
                   -2 & 3 & 0 \\
                   9 & -2 & 0 \\
                   1 & 3 & -2
                 \end{vmatrix}\\
    &=
      (-4)(10)+(1)(-22)+(-2)(61)+6(46)=92
  \end{align*}

  Expanding about column 3 (\ref{theorem:DEC} with $j=3$) gives
  \begin{align*}
    \detbars{A}
    &=
      (0)(-1)^{1+3}
      \begin{vmatrix}
        9 & -2 & 1\\
        1 & 3 & -1\\
        4 & 1 & 6
      \end{vmatrix}
                +
                (0)(-1)^{2+3}
                \begin{vmatrix}
                  -2 & 3 & 1\\
                  1 & 3 & -1\\
                  4 & 1 & 6
                \end{vmatrix}
                          +\\
    &\quad\quad(-2)(-1)^{3+3}
      \begin{vmatrix}
        -2 & 3 & 1\\
        9 & -2 & 1\\
        4 & 1 & 6
      \end{vmatrix}
                +
                (2)(-1)^{4+3}
                \begin{vmatrix}
                  -2 & 3 & 1\\
                  9 & -2 & 1\\
                  1 & 3 & -1
                \end{vmatrix}\\
    &=0+0+(-2)(-107)+(-2)(61)=\answer{92}
  \end{align*}

  Notice how much easier the second computation was.  By choosing to
  expand about the third column, we have two entries that are zero, so
  two $3\times 3$ determinants need not be computed at all!
\end{example}

When a matrix has all zeros above (or below) the diagonal, exploiting
the zeros by expanding about the proper row or column makes computing
a determinant insanely easy.

\begin{example}[Determinant of an upper triangular matrix]
  Suppose that
  \[
    T=
    \begin{bmatrix}
      2 & 3 & -1 & 3 & 3\\
      0 & -1 & 5 & 2 & -1\\
      0 & 0 & 3 & 9 & 2\\
      0 & 0 & 0 & -1 & 3\\
      0 & 0 & 0 & 0 & 5
    \end{bmatrix}
  \]

  We will compute the determinant of this $5\times 5$ matrix by
  consistently expanding about the first column for each submatrix
  that arises and does not have a zero entry multiplying it.
  \begin{align*}
    \detname{T}&=
                 \begin{vmatrix}
                   2 & 3 & -1 & 3 & 3\\
                   0 & -1 & 5 & 2 & -1\\
                   0 & 0 & 3 & 9 & 2\\
                   0 & 0 & 0 & -1 & 3\\
                   0 & 0 & 0 & 0 & 5
                 \end{vmatrix}\\
               &=2(-1)^{1+1}
                 \begin{vmatrix}
                   -1 & 5 & 2 & -1\\
                   0 & 3 & 9 & 2\\
                   0 & 0 & -1 & 3\\
                   0 & 0 & 0 & 5
                 \end{vmatrix}\\
               &=2(-1)(-1)^{1+1}
                 \begin{vmatrix}
                   3 & 9 & 2\\
                   0 & -1 & 3\\
                   0 & 0 & 5
                 \end{vmatrix}\\
               &=2(-1)(3)(-1)^{1+1}
                 \begin{vmatrix}
                   -1 & 3\\
                   0 & 5
                 \end{vmatrix}\\
               &=2(-1)(3)(-1)(-1)^{1+1}
                 \begin{vmatrix}
                   5
                 \end{vmatrix}\\
               &=2(-1)(3)(-1)(5)=30
  \end{align*}
\end{example}

When you consult other texts in your study of determinants, you may
run into the terms ``minor'' and ``cofactor,'' especially in a
discussion centered on expansion about rows and columns.  We have
chosen not to make these definitions formally since we have been able
to get along without them.  However, informally, a \dfn{minor} is a
determinant of a submatrix, specifically
$\detname{\submatrix{A}{i}{j}}$ and is usually referenced as the minor
of $\matrixentry{A}{ij}$.  A \dfn{cofactor} is a signed minor,
specifically the cofactor of $\matrixentry{A}{ij}$ is
$(-1)^{i+j}\detname{\submatrix{A}{i}{j}}$.

\end{document}
