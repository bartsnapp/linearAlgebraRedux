\documentclass{ximera}

% These macros are automatically generated from the "macros"
% XML element.  Make permanent edits there.
%
% History
%   2004/01/01  Initiated for FCLA, evolved from there
%   2006/09/17  Converted  _, ^  to \sb, \sp for TeX4ht
%   2014/02/01  Updated for MathBook XML projects
%               Obsolete in FCLA: \codeindent, \computerfont, \define
%               Change: MathJax wants \lt, so replaced by \lteval
%   2014/02/22  New: \orderof, \reals, \per
%   2015/08/16  Incorporated into MathBook XML version of FCLA
%
%%%%%%%%%%%%%%%%%%%%%
%
%     Conveniences
%
%%%%%%%%%%%%%%%%%%%%%
%
%  Order of (asymptotically limit of fraction is 1)
%  Usage: \orderof{some function}
%
\newcommand{\orderof}[1]{\sim #1}
%
%  Integers
%  Usage:  \Z
\newcommand{\Z}{\mathbb{Z}}
%
%  Real numbers, as set of scalars
%  Usage:  \reals
\newcommand{\reals}{\mathbb{R}}
%
%  n-space over real field
%  Usage: \complex{integer-dimension}
\newcommand{\real}[1]{\mathbb{R}^{#1}}
%
%  Complex numbers, as set of scalars
%  Usage:  \complexes
\newcommand{\complexes}{\mathbb{C}}
%
%  n-space over complex field
%  Usage: \complex{integer-dimension}
\newcommand{\complex}[1]{\mathbb{C}^{#1}}
\newcommand{\CC}{\mathbb{C}}
%
%  Complex conjugation (scalar, vector, matrix)
%  Usage: \conjugate{object}
\newcommand{\conjugate}[1]{\overline{#1}}
%
%  Complex number modulus
%  Usage: \modulus{a+bi}
%  Presumes math mode
\newcommand{\modulus}[1]{\left\lvert#1\right\rvert}
%
%  Zero vector
%  Usage: \zerovector
\newcommand{\zerovector}{\vect{0}}
%
%  Zero matrix
%  Usage: \zeromatrix, use a subscript when size is important
\newcommand{\zeromatrix}{\mathcal{O}}
%
%  Inner product (brackets, not quadratic form)
%  Usage: \innerproduct{a-vector}{a-vector}
\newcommand{\innerproduct}[2]{\left\langle#1,\,#2\right\rangle}
%
%  Norm of a vector
%  Usage: \norm{a-vector}
\newcommand{\norm}[1]{\left\lVert#1\right\rVert}
%
%  Dimension
%  Usage: \dimension{vector-space-letter}
\newcommand{\dimension}[1]{\dim\left(#1\right)}
%
%  Nullity
%  Usage: \nullity{matrix-or-lintrans-letter}
\newcommand{\nullity}[1]{n\left(#1\right)}
%
%  Rank
%  Usage: \rank{matrix-or-lintrans-letter}
\newcommand{\rank}[1]{r\left(#1\right)}
%
%  Direct sum
%  Usage: \ds between a couple of subspaces
%
\newcommand{\ds}{\oplus}
%
%  Determinant of a matrix (functional)
%  Usage: \detname{A}
\newcommand{\detname}[1]{\det\left(#1\right)}
%
%  Determinant of a matrix (vertical bars)
%  Usage: \detbars{A}
\newcommand{\detbars}[1]{\left\lvert#1\right\rvert}
%
%  Trace of a Matrix
%  Usage: \trace{matrix name}
\newcommand{\trace}[1]{t\left(#1\right)}
%
%  Square Root of a Matrix
%  Usage: \sr{a-matrix}
\newcommand{\sr}[1]{#1^{1/2}}
%
%%%%%%%%%%%%%%%%%%%%%
%
%     Subspace Constructions
%
%%%%%%%%%%%%%%%%%%%%%
%
%  Span of a set of vectors
%  \span and \sp are used by TeX for other things
%  Usage: \spn{set-of-vectors}
\newcommand{\spn}[1]{\left\langle#1\right\rangle}
%
%  Null space of a matrix
%  Usage:  \nsp{A}
\newcommand{\nsp}[1]{\mathcal{N}\!\left(#1\right)}
%
%  Column space of a matrix
%  Usage:  \csp{A}
\newcommand{\csp}[1]{\mathcal{C}\!\left(#1\right)}
%
%  Row space of a matrix
%  Usage:  \rsp{A}
\newcommand{\rsp}[1]{\mathcal{R}\!\left(#1\right)}
%
%  Left null space of a matrix
%  Usage:  \lns{A}
\newcommand{\lns}[1]{\mathcal{L}\!\left(#1\right)}
%
%  Orthogonal complement of a vector space
%  Avoiding TeX's \perp
%  Usage:  \per{A}
\newcommand{\per}[1]{#1^\perp}
%
%%%%%%%%%%%%%%%%%%%%%
%
%     Systems of Equations
%
%%%%%%%%%%%%%%%%%%%%%
%
%  In-line form of an augmented matrix for a system of equations
%  Usage: \augmented{coefficient-matrix}{constant-vector}
\newcommand{\augmented}[2]{\left\lbrack\left.#1\,\right\rvert\,#2\right\rbrack}
%
%  Notation for a linear system before introducing matrix multiplication
%  Usage: \linearsystem{coefficient-matrix}{constant-vector}
\newcommand{\linearsystem}[2]{\mathcal{LS}\!\left(#1,\,#2\right)}
%
%  Notation for a homogenous system before introducing matrix multiplication
%  Usage: \homosystem{coefficient-matrix}
\newcommand{\homosystem}[1]{\linearsystem{#1}{\zerovector}}
%
%%%%%%%%%%%%%%%%%%%%%
%
%     Row Operations, Echelon Form
%
%%%%%%%%%%%%%%%%%%%%%
%
% Row operations on matrices
%
% Three commands to shorten up descriptions of gaussian elimination
%
% Usage: \rowopswap{row-i}{row-j}
% Usage: \rowopmult{scalar}{row-i}
% Usage: \rowopadd{scalar}{row-multiplied}{row-added-to}
\newcommand{\rowopswap}[2]{R_{#1}\leftrightarrow R_{#2}}
\newcommand{\rowopmult}[2]{#1R_{#2}}
\newcommand{\rowopadd}[3]{#1R_{#2}+R_{#3}}
%
% Mark leading 1's in echelon form with fbox
% Usage: \leading{a-1-usually}
\newcommand{\leading}[1]{\fbox{#1}}
%
%  Row-reduce arrow
%  Usage:  \rref inbetween a matrix and its reduced row-echelon form
\newcommand{\rref}{\xrightarrow{\text{RREF}}}
%
%  Elementary Matrices
%  Usage: \elemswap{subscript}{subscript}
%  Usage: \elemmult{scalar}{subscript}
%  Usage: \elemadd{scalar}{subscript-mult}{subscript-target}
%
\newcommand{\elemswap}[2]{E_{#1,#2}}
\newcommand{\elemmult}[2]{E_{#2}\left(#1\right)}
\newcommand{\elemadd}[3]{E_{#2,#3}\left(#1\right)}
%
%%%%%%%%%%%%%%%%%%%%%
%
%     2-D Constructions (Lists, Vectors, Matrices)
%
%%%%%%%%%%%%%%%%%%%%%
%
%  A list of scalars of generic length
%  Usage:  \scalarlist{scalar letter}{terminal subscript}
\newcommand{\scalarlist}[2]{{#1}_{1},\,{#1}_{2},\,{#1}_{3},\,\ldots,\,{#1}_{#2}}
%
%  Vector styling, bold (or use wiggles, arrows, whatever)
%  Subscripts go outside this construction
%  Usage: \vect{a symbol to use as a vector}
%  Have to already be in math mode
%
\newcommand{\vect}[1]{\mathbf{#1}}
%
%  A column vector
%  Usage: \colvector{list-delimited-by-\\}
%
\newcommand{\colvector}[1]{\begin{bmatrix}#1\end{bmatrix}}
%
%  A generic vector with components
%  Usage: \vectorcomponents{component-letter}{final-subscript}
\newcommand{\vectorcomponents}[2]{\colvector{#1_{1}\\#1_{2}\\#1_{3}\\\vdots\\#1_{#2}}}
%
%  A list of vectors of generic length
%  Usage:  \vectorlist{vector letter}{terminal subscript}
\newcommand{\vectorlist}[2]{\vect{#1}_{1},\,\vect{#1}_{2},\,\vect{#1}_{3},\,\ldots,\,\vect{#1}_{#2}}
%
%  Vector entries, entry i of vector v
%  (vector-expession still needs \vect, etc.)
%  Usage:  \vectorentry{vector-expression}{single-subscript}
\newcommand{\vectorentry}[2]{\left\lbrack#1\right\rbrack_{#2}}
%
%  Matrix entries, entry i,j of matrix A
%  Usage:  \matrixentry{matrix-expression}{paired-subscripts}
%
\newcommand{\matrixentry}[2]{\left\lbrack#1\right\rbrack_{#2}}
%
%  A generic linear combination
%  Usage:  \lincombo{scalar letter}{vector letter}{terminal subscript}
\newcommand{\lincombo}[3]{#1_{1}\vect{#2}_{1}+#1_{2}\vect{#2}_{2}+#1_{3}\vect{#2}_{3}+\cdots +#1_{#3}\vect{#2}_{#3}}
%
%  Matrix, column by column, as vectors
%  Usage:  \matrixcolumns{matrix letter}{terminal subscript}
\newcommand{\matrixcolumns}[2]{\left\lbrack\vect{#1}_{1}|\vect{#1}_{2}|\vect{#1}_{3}|\ldots|\vect{#1}_{#2}\right\rbrack}
%
%%%%%%%%%%%%%%%%%%%%%
%
%     Special Matrices
%
%%%%%%%%%%%%%%%%%%%%%
%
%  Transpose of a matrix
%  Usage:  \transpose{A}
\newcommand{\transpose}[1]{#1^{t}}
%
%  Inverse of a matrix
%  Usage:  \inverse{A}
\newcommand{\inverse}[1]{#1^{-1}}
%
%  Submatrix (for minors, determinants)
%  Usage: \submatrix{matrix-name}{delete-row}{delete-col}
\newcommand{\submatrix}[3]{#1\left(#2|#3\right)}
%
%  Adjoint of a matrix (twice)
%  This macro is a convenience to call \transpose and \conjugate properly
%  It shouldn't need to be modified (or mathematical meanings will change)
%  Usage:  \adj{A}
\newcommand{\adj}[1]{\transpose{\left(\conjugate{#1}\right)}}
%
%  This macro controls the symbol used for the adjoint
%  It can be edited to taste
%  Usage:  \adjoint{A}
\newcommand{\adjoint}[1]{#1^\ast}
%
%%%%%%%%%%%%%%%%%%%%%
%
%     Sets
%
%%%%%%%%%%%%%%%%%%%%%
%
%  A convenience for simple sets
%  Usage:  \set{list of element}
\newcommand{\set}[1]{\left\{#1\right\}}
%
%  Sets with vertical bar, "such that", sized for objects, not condition
%  Usage:  \setparts{objects}{condition}
%
%%\newcommand{\setparts}[2]{\left\{ #1\mid#2\right\}}
%%\newcommand{\setparts}[2]{\left\{\left. #1\right\rvert#2\right\}}
\newcommand{\setparts}[2]{\left\lbrace#1\,\middle|\,#2\right\rbrace}
%
%  Set Cardinality
%  Usage:  \card{a-set-letter}
\newcommand{\card}[1]{\left\lvert#1\right\rvert}
%
%  Set Union
%  Use \cup
%
%  Set Intersection
%  Use \cap
%
%  Set Complement
%  Usage:  \setcomplement{a-set-letter}
\newcommand{\setcomplement}[1]{\overline{#1}}
%
%%%%%%%%%%%%%%%%%%%%%
%
%     Eigenvalues and Eigenspaces
%
%%%%%%%%%%%%%%%%%%%%%
%
%  Characteristic polynomial
%  Usage: \charpoly{matrix-letter}{variable-letter}
\newcommand{\charpoly}[2]{p_{#1}\left(#2\right)}
%
%  Eigenspace
%  Usage: \eigenspace{matrix-letter}{eigenvalue-letter}
\newcommand{\eigenspace}[2]{\mathcal{E}_{#1}\left(#2\right)}
%
%  2013/10/03 Including ampersands is problematic here, 
%  think about fixes later
%  2014/02/22 Limited testing, seems &amp; is fine for HTML and LaTeX
%  2016-07-20 only employed in Archetypes, MBX has gather/align override
%  Eigensystem (presumes wrapped in an mrow within md)
%  Usage: \eigensystem{matrixletter}{eigenvalue}{list of basis vectors}
\newcommand{\eigensystem}[3]{\lambda&amp;=#2&amp;\eigenspace{#1}{#2}&amp;=\spn{\set{#3}}} 
%
%  Generalized Eigenspace
%  Usage: \geneigenspace{lin-trans-letter}{eigenvalue-letter}
\newcommand{\geneigenspace}[2]{\mathcal{G}_{#1}\left(#2\right)}
%
%  Algebraic multiplicty
%  Usage: \algmult{matrix-letter}{eigenvalue-letter}
\newcommand{\algmult}[2]{\alpha_{#1}\left(#2\right)}
%
%  Geometric multiplicty
%  Usage: \geomult{matrix-letter}{eigenvalue-letter}
\newcommand{\geomult}[2]{\gamma_{#1}\left(#2\right)}
%
%  Index (of eigenvalue)
%  Usage: \indx{matrix-letter}{eigenvalue-letter}
\newcommand{\indx}[2]{\iota_{#1}\left(#2\right)}
%
%%%%%%%%%%%%%%%%%%%%%
%
%     Linear Transformations
%
%%%%%%%%%%%%%%%%%%%%%
%
%  MathJax defines \lt to ease XML confusion
%
%  Linear transformation definition
%  Usage: \ltdefn{name-letter}{domain}{range}
\newcommand{\ltdefn}[3]{#1\colon #2\rightarrow#3}
%
%  Linear transformation evaluation
%  Usage: \lteval{name-letter}{input}
%  Replaces old \lt desired by MathJax
\newcommand{\lteval}[2]{#1\left(#2\right)}
%
% Linear transformation inverse
%  Usage: \ltinverse{name-letter}
\newcommand{\ltinverse}[1]{#1^{-1}}
%
%  Linear transformation restriction
%  Usage: \restrict{name-letter}{subspace-letter}
\newcommand{\restrict}[2]{{#1}|_{#2}}
%
%  Linear transformation preimage
%  Usage: \preimage{name-letter}{codomain-element}
\newcommand{\preimage}[2]{#1^{-1}\left(#2\right)}
%
%  Range of a linear transformation
%  TeX uses \range for something else
%  Usage:  \rng{T}
\newcommand{\rng}[1]{\mathcal{R}\!\left(#1\right)}
%
%  Kernel of a linear transformation
%  TeX uses \ker to do something different
%  Usage:  \krn{T}
\newcommand{\krn}[1]{\mathcal{K}\!\left(#1\right)}
%
%  Linear transformation composition
%  Usage: \compose{function-name}{function-name}
\newcommand{\compose}[2]{{#1}\circ{#2}}
%
%  Vector space of linear transformations
%  Usage: \vslt{domains}{codomains}
%  Presumes math mode
\newcommand{\vslt}[2]{\mathcal{LT}\left(#1,\,#2\right)}
%
%%%%%%%%%%%%%%%%%%%%%
%
%     Vector and Matrix Representations
%
%%%%%%%%%%%%%%%%%%%%%
%
%  Isomorphism symbol
%  Usage: \isomorphic
\newcommand{\isomorphic}{\cong}
%
%  Similarity
%  Usage: \similar{inner-matrix}{outer-invertible-matrix}
%  Rearranging this will not "fix" all desired changes throughout
%
\newcommand{\similar}[2]{\inverse{#2}#1#2}
%
%  Vector representation function name
%  Usage: \vectrepname{basis-letter}
\newcommand{\vectrepname}[1]{\rho_{#1}}
%
%  Vector representation output
%  Usage: \vectrep{basis-letter}{input}
\newcommand{\vectrep}[2]{\lteval{\vectrepname{#1}}{#2}}
%
%  Vector representation inverse function name
%  (Added later, not used consistently in FCLA)
%  Usage: \vectrepinvname{basis-letter}
\newcommand{\vectrepinvname}[1]{\ltinverse{\vectrepname{#1}}}
%
%  Vector representation inverse output
%  Usage: \vectrepinv{basis-letter}{input}
\newcommand{\vectrepinv}[2]{\lteval{\ltinverse{\vectrepname{#1}}}{#2}}
%
%  Matrix representation
%  Usage: \matrixrep{transformation-letter}{domain-basis-letter}{codomain-basis-letter}
\newcommand{\matrixrep}[3]{M^{#1}_{#2,#3}}
%
%  Matrix representation column-by-colum
%  2016-07-20 only employed once?
%  Usage: \matrixrepcolumns{transformation-letter}{codomain-basis-letter}{codomain-basis-vector-letter}{final-index}
\newcommand{\matrixrepcolumns}[4]{\left\lbrack \left.\vectrep{#2}{\lteval{#1}{\vect{#3}_{1}}}\right|\left.\vectrep{#2}{\lteval{#1}{\vect{#3}_{2}}}\right|\left.\vectrep{#2}{\lteval{#1}{\vect{#3}_{3}}}\right|\ldots\left|\vectrep{#2}{\lteval{#1}{\vect{#3}_{#4}}}\right.\right\rbrack}
%
%  Change of basis matrix
%  Usage: \cbm{domain-basis-letter}{codomain-basis-letter}
\newcommand{\cbm}[2]{C_{#1,#2}}
%
%%%%%%%%%%%%%%%%%%%%%
%
%     Canonical Forms
%
%%%%%%%%%%%%%%%%%%%%%
%
%  Jordan blocks
%  Usage: \jordan{size}{diagonal-element}
\newcommand{\jordan}[2]{J_{#1}\left(#2\right)}
%
%%%%%%%%%%%%%%%%%%%%%
%
%     Hadamard Matrices
%     Contributed by Elizabeth Million
%
%%%%%%%%%%%%%%%%%%%%%
%
%  Hadamard Product
%  Usage: \hadamard{a-matrix}{a-matrix}
\newcommand{\hadamard}[2]{#1\circ #2}
%
%  Hadamard identity matrix
%  Usage: \hadamardidentity{paired-subscripts-size-of-matrix}
\newcommand{\hadamardidentity}[1]{J_{#1}}
%
%  Hadamard inverse matrix
%  Usage: \hadamardinverse{matrix-expression}
\newcommand{\hadamardinverse}[1]{\widehat{#1}}

\newcommand{\definedTerm}[1]{\textbf{#1}}
\newcommand{\dfn}[1]{\textbf{#1}}

\newcommand{\wt}{\widetilde}
\newcommand{\ov}{\overline}
\newcommand{\inj}{\rightarrowtail}
\newcommand{\surj}{\twoheadrightarrow}
\newcommand{\harpoon}{\overset{\rightharpoonup}}

\newenvironment{amatrix}[1]{%
  \left[\begin{array}{@{}*{#1}{c}|c@{}}
}{%
  \end{array}\right]
}


\title{Spanning Sets and Surjective Linear Transformations}

\begin{document}
\begin{abstract}
  Just as injective linear transformations are allied with linear independence, surjective linear transformations are allied with spanning sets.
\end{abstract}
\maketitle


\begin{theorem}[Spanning Set for Image of a Linear Transformation]
\label{theorem:SSRLT}

Suppose that $\ltdefn{T}{U}{V}$ is a linear transformation and
\begin{align*}
S&=\set{\vectorlist{u}{t}}
\end{align*}
spans $U$.  Then
\begin{align*}
R&=\set{\lteval{T}{\vect{u}_1},\,\lteval{T}{\vect{u}_2},\,\lteval{T}{\vect{u}_3},\,\ldots,\,\lteval{T}{\vect{u}_t}}
\end{align*}
spans $\rng{T}$.

\begin{proof}
We need to establish that $\rng{T}=\spn{R}$, a set equality.  First we establish that $\rng{T}\subseteq\spn{R}$.  To this end, choose $\vect{v}\in\rng{T}$.  Then there exists a vector $\vect{u}\in U$, such that $\lteval{T}{\vect{u}}=\vect{v}$ (\ref{definition:RLT}).  Because $S$ spans $U$ there are scalars, $\scalarlist{a}{t}$, such that
\[
\vect{u}=\lincombo{a}{u}{t}
\]




Then
\begin{align*}
\vect{v}
&=\lteval{T}{\vect{u}}&&\ref{definition:RLT}\\
&=\lteval{T}{\lincombo{a}{u}{t}}&&\ref{definition:SSVS}\\
&=a_1\lteval{T}{\vect{u}_1}+a_2\lteval{T}{\vect{u}_2}+a_3\lteval{T}{\vect{u}_3}+\ldots+a_t\lteval{T}{\vect{u}_t}&&\ref{theorem:LTLC}\\
\end{align*}
which establishes that $\vect{v}\in\spn{R}$ (\ref{definition:SS}).  So $\rng{T}\subseteq\spn{R}$.



To establish the opposite inclusion, choose an element of the span of $R$, say $\vect{v}\in\spn{R}$.  Then there are scalars $\scalarlist{b}{t}$ so that
\begin{align*}
\vect{v}
&=b_1\lteval{T}{\vect{u}_1}+b_2\lteval{T}{\vect{u}_2}+b_3\lteval{T}{\vect{u}_3}+\cdots+b_t\lteval{T}{\vect{u}_t}
&&\ref{definition:SS}\\
&=\lteval{T}{\lincombo{b}{\vect{u}}{t}}&&\ref{theorem:LTLC}
\end{align*}




This demonstrates that $\vect{v}$ is an output of the linear transformation $T$, so $\vect{v}\in\rng{T}$.  Therefore $\spn{R}\subseteq\rng{T}$, so we have the set equality $\rng{T}=\spn{R}$ (\ref{definition:SE}).  In other words, $R$ spans $\rng{T}$ (\ref{definition:SSVS}).



\end{proof}
\end{theorem}

\ref{theorem:SSRLT} provides an easy way to begin the construction of a basis for the image of a linear transformation, since the construction of a spanning set requires simply evaluating the linear transformation on a spanning set of the domain.  In practice the best choice for a spanning set of the domain would be as small as possible, in other words, a basis.  The resulting spanning set for the codomain may not be linearly independent, so to find a basis for the image might require tossing out redundant vectors from the spanning set.  Here is an example.



\begin{example}
[A basis for the image of a linear transformation]

Define the linear transformation $\ltdefn{T}{M_{22}}{P_2}$ by
\[
\lteval{T}{ \begin{bmatrix} a&b\\c&d \end{bmatrix}}
=\left(a+2b+8c+d\right)+\left(-3a+2b+5d\right)x+\left(a+b+5c\right)x^2
\]




A convenient spanning set for $M_{22}$ is the basis
\[
S=\set{
\begin{bmatrix} 1 & 0 \\ 0 & 0 \end{bmatrix},\,
\begin{bmatrix} 0 & 1 \\ 0 & 0 \end{bmatrix},\,
\begin{bmatrix} 0 & 0 \\ 1 & 0 \end{bmatrix},\,
\begin{bmatrix} 0 & 0 \\ 0 & 1 \end{bmatrix}
}
\]




So by \ref{theorem:SSRLT}, a spanning set for $\rng{T}$ is
\begin{align*}
R
&=\set{
\lteval{T}{\begin{bmatrix} 1 & 0 \\ 0 & 0 \end{bmatrix}},\,
\lteval{T}{\begin{bmatrix} 0 & 1 \\ 0 & 0 \end{bmatrix}},\,
\lteval{T}{\begin{bmatrix} 0 & 0 \\ 1 & 0 \end{bmatrix}},\,
\lteval{T}{\begin{bmatrix} 0 & 0 \\ 0 & 1 \end{bmatrix}}
}\\
&=\set{1-3x+x^2,\,2+2x+x^2,\,8+5x^2,\,1+5x}
\end{align*}

Then
\begin{multipleChoice}
\choice[correct]{The set $R$ is not linearly independent.}
\choice{The set $R$ is linearly independent.}
\end{multipleChoice}

\begin{feedback}[correct]
So if we desire a basis for $\rng{T}$, we need to eliminate some redundant vectors.  Two particular relations of linear dependence on $R$ are
\begin{align*}
(-2)(1-3x+x^2)+(-3)(2+2x+x^2)+(8+5x^2)&=0+0x+0x^2=\zerovector\\
(1-3x+x^2)+(-1)(2+2x+x^2)+(1+5x)&=0+0x+0x^2=\zerovector
\end{align*}

These, individually, allow us to remove $8+5x^2$ and $1+5x$ from $R$ without destroying the property that $R$ spans $\rng{T}$.  The two remaining vectors are linearly independent (check this!), so we can write
\[
\rng{T}=\spn{\set{1-3x+x^2,\,2+2x+x^2}}
\]
and see that $\dimension{\rng{T}}=2$.
\end{feedback}

\end{example}

Elements of the image are precisely those elements of the codomain with nonempty preimages.

\begin{theorem}[Image and Pre-Image]
\label{theorem:RPI}


Suppose that $\ltdefn{T}{U}{V}$ is a linear transformation.  Then
\[
\vect{v}\in\rng{T}\text{ if and only if }\preimage{T}{\vect{v}}\neq\emptyset
\]





\begin{proof}
($\Rightarrow$)  If $\vect{v}\in\rng{T}$, then there is a vector $\vect{u}\in U$ such that $\lteval{T}{\vect{u}}=\vect{v}$.  This qualifies $\vect{u}$ for membership in $\preimage{T}{\vect{v}}$, and thus the preimage of $\vect{v}$ is not empty.



($\Leftarrow$)  Suppose the preimage of $\vect{v}$ is not empty, so we can choose a vector $\vect{u}\in U$ such that $\lteval{T}{\vect{u}}=\vect{v}$.  Then $\vect{v}\in\rng{T}$.


\end{proof}
\end{theorem}

Now would be a good time to return to \ref{diagram:KPI} which depicted the pre-images of a non-surjective linear transformation.  The vectors $\vect{x},\,\vect{y}\in V$ were elements of the codomain whose pre-images were empty, as we expect for a non-surjective linear transformation from the characterization in \ref{theorem:RPI}.



\begin{theorem}[Surjective Linear Transformations and Bases]
\label{theorem:SLTB}


Suppose that $\ltdefn{T}{U}{V}$ is a linear transformation and
\begin{align*}
B&=\set{\vectorlist{u}{m}}
\end{align*}
is a basis of $U$.  Then $T$ is surjective if and only if
\begin{align*}
C&=\set{\lteval{T}{\vect{u}_1},\,\lteval{T}{\vect{u}_2},\,\lteval{T}{\vect{u}_3},\,\ldots,\,\lteval{T}{\vect{u}_m}}
\end{align*}
is a spanning set for $V$.




\begin{proof}
($\Rightarrow$)  Assume $T$ is surjective.  Since $B$ is a basis, we know $B$ is a spanning set of $U$ (\ref{definition:B}).  Then \ref{theorem:SSRLT} says that $C$ spans $\rng{T}$.  But the hypothesis that $T$ is surjective means $V=\rng{T}$ (\ref{theorem:RSLT}), so $C$ spans $V$.



($\Leftarrow$)  Assume that $C$ spans $V$.  To establish that $T$ is surjective, we will show that every element of $V$ is an output of $T$ for some input (\ref{definition:SLT}).  Suppose that $\vect{v}\in V$.  As an element of $V$, we can write $\vect{v}$ as a linear combination of the spanning set $C$.  So there are scalars, $\scalarlist{b}{m}$, such that
\[
\vect{v}=b_1\lteval{T}{\vect{u}_1}+b_2\lteval{T}{\vect{u}_2}+b_3\lteval{T}{\vect{u}_3}+\cdots+b_m\lteval{T}{\vect{u}_m}
\]




Now define the vector $\vect{u}\in U$ by
\[
\vect{u}=\lincombo{b}{u}{m}
\]




Then
\begin{align*}
\lteval{T}{\vect{u}}&=\lteval{T}{\lincombo{b}{u}{m}}\\
&=b_1\lteval{T}{\vect{u}_1}+b_2\lteval{T}{\vect{u}_2}+b_3\lteval{T}{\vect{u}_3}+\cdots+b_m\lteval{T}{\vect{u}_m}&&\ref{theorem:LTLC}\\
&=\vect{v}
\end{align*}




So, given any choice of a vector $\vect{v}\in V$, we can design an input $\vect{u}\in U$ to produce $\vect{v}$ as an output of $T$.  Thus, by \ref{definition:SLT}, $T$ is surjective.



\end{proof}
\end{theorem}

\end{document}

%%% Local Variables:
%%% mode: latex
%%% TeX-master: t
%%% End:
