\documentclass{ximera}

% These macros are automatically generated from the "macros"
% XML element.  Make permanent edits there.
%
% History
%   2004/01/01  Initiated for FCLA, evolved from there
%   2006/09/17  Converted  _, ^  to \sb, \sp for TeX4ht
%   2014/02/01  Updated for MathBook XML projects
%               Obsolete in FCLA: \codeindent, \computerfont, \define
%               Change: MathJax wants \lt, so replaced by \lteval
%   2014/02/22  New: \orderof, \reals, \per
%   2015/08/16  Incorporated into MathBook XML version of FCLA
%
%%%%%%%%%%%%%%%%%%%%%
%
%     Conveniences
%
%%%%%%%%%%%%%%%%%%%%%
%
%  Order of (asymptotically limit of fraction is 1)
%  Usage: \orderof{some function}
%
\newcommand{\orderof}[1]{\sim #1}
%
%  Integers
%  Usage:  \Z
\newcommand{\Z}{\mathbb{Z}}
%
%  Real numbers, as set of scalars
%  Usage:  \reals
\newcommand{\reals}{\mathbb{R}}
%
%  n-space over real field
%  Usage: \complex{integer-dimension}
\newcommand{\real}[1]{\mathbb{R}^{#1}}
%
%  Complex numbers, as set of scalars
%  Usage:  \complexes
\newcommand{\complexes}{\mathbb{C}}
%
%  n-space over complex field
%  Usage: \complex{integer-dimension}
\newcommand{\complex}[1]{\mathbb{C}^{#1}}
\newcommand{\CC}{\mathbb{C}}
%
%  Complex conjugation (scalar, vector, matrix)
%  Usage: \conjugate{object}
\newcommand{\conjugate}[1]{\overline{#1}}
%
%  Complex number modulus
%  Usage: \modulus{a+bi}
%  Presumes math mode
\newcommand{\modulus}[1]{\left\lvert#1\right\rvert}
%
%  Zero vector
%  Usage: \zerovector
\newcommand{\zerovector}{\vect{0}}
%
%  Zero matrix
%  Usage: \zeromatrix, use a subscript when size is important
\newcommand{\zeromatrix}{\mathcal{O}}
%
%  Inner product (brackets, not quadratic form)
%  Usage: \innerproduct{a-vector}{a-vector}
\newcommand{\innerproduct}[2]{\left\langle#1,\,#2\right\rangle}
%
%  Norm of a vector
%  Usage: \norm{a-vector}
\newcommand{\norm}[1]{\left\lVert#1\right\rVert}
%
%  Dimension
%  Usage: \dimension{vector-space-letter}
\newcommand{\dimension}[1]{\dim\left(#1\right)}
%
%  Nullity
%  Usage: \nullity{matrix-or-lintrans-letter}
\newcommand{\nullity}[1]{n\left(#1\right)}
%
%  Rank
%  Usage: \rank{matrix-or-lintrans-letter}
\newcommand{\rank}[1]{r\left(#1\right)}
%
%  Direct sum
%  Usage: \ds between a couple of subspaces
%
\newcommand{\ds}{\oplus}
%
%  Determinant of a matrix (functional)
%  Usage: \detname{A}
\newcommand{\detname}[1]{\det\left(#1\right)}
%
%  Determinant of a matrix (vertical bars)
%  Usage: \detbars{A}
\newcommand{\detbars}[1]{\left\lvert#1\right\rvert}
%
%  Trace of a Matrix
%  Usage: \trace{matrix name}
\newcommand{\trace}[1]{t\left(#1\right)}
%
%  Square Root of a Matrix
%  Usage: \sr{a-matrix}
\newcommand{\sr}[1]{#1^{1/2}}
%
%%%%%%%%%%%%%%%%%%%%%
%
%     Subspace Constructions
%
%%%%%%%%%%%%%%%%%%%%%
%
%  Span of a set of vectors
%  \span and \sp are used by TeX for other things
%  Usage: \spn{set-of-vectors}
\newcommand{\spn}[1]{\left\langle#1\right\rangle}
%
%  Null space of a matrix
%  Usage:  \nsp{A}
\newcommand{\nsp}[1]{\mathcal{N}\!\left(#1\right)}
%
%  Column space of a matrix
%  Usage:  \csp{A}
\newcommand{\csp}[1]{\mathcal{C}\!\left(#1\right)}
%
%  Row space of a matrix
%  Usage:  \rsp{A}
\newcommand{\rsp}[1]{\mathcal{R}\!\left(#1\right)}
%
%  Left null space of a matrix
%  Usage:  \lns{A}
\newcommand{\lns}[1]{\mathcal{L}\!\left(#1\right)}
%
%  Orthogonal complement of a vector space
%  Avoiding TeX's \perp
%  Usage:  \per{A}
\newcommand{\per}[1]{#1^\perp}
%
%%%%%%%%%%%%%%%%%%%%%
%
%     Systems of Equations
%
%%%%%%%%%%%%%%%%%%%%%
%
%  In-line form of an augmented matrix for a system of equations
%  Usage: \augmented{coefficient-matrix}{constant-vector}
\newcommand{\augmented}[2]{\left\lbrack\left.#1\,\right\rvert\,#2\right\rbrack}
%
%  Notation for a linear system before introducing matrix multiplication
%  Usage: \linearsystem{coefficient-matrix}{constant-vector}
\newcommand{\linearsystem}[2]{\mathcal{LS}\!\left(#1,\,#2\right)}
%
%  Notation for a homogenous system before introducing matrix multiplication
%  Usage: \homosystem{coefficient-matrix}
\newcommand{\homosystem}[1]{\linearsystem{#1}{\zerovector}}
%
%%%%%%%%%%%%%%%%%%%%%
%
%     Row Operations, Echelon Form
%
%%%%%%%%%%%%%%%%%%%%%
%
% Row operations on matrices
%
% Three commands to shorten up descriptions of gaussian elimination
%
% Usage: \rowopswap{row-i}{row-j}
% Usage: \rowopmult{scalar}{row-i}
% Usage: \rowopadd{scalar}{row-multiplied}{row-added-to}
\newcommand{\rowopswap}[2]{R_{#1}\leftrightarrow R_{#2}}
\newcommand{\rowopmult}[2]{#1R_{#2}}
\newcommand{\rowopadd}[3]{#1R_{#2}+R_{#3}}
%
% Mark leading 1's in echelon form with fbox
% Usage: \leading{a-1-usually}
\newcommand{\leading}[1]{\fbox{#1}}
%
%  Row-reduce arrow
%  Usage:  \rref inbetween a matrix and its reduced row-echelon form
\newcommand{\rref}{\xrightarrow{\text{RREF}}}
%
%  Elementary Matrices
%  Usage: \elemswap{subscript}{subscript}
%  Usage: \elemmult{scalar}{subscript}
%  Usage: \elemadd{scalar}{subscript-mult}{subscript-target}
%
\newcommand{\elemswap}[2]{E_{#1,#2}}
\newcommand{\elemmult}[2]{E_{#2}\left(#1\right)}
\newcommand{\elemadd}[3]{E_{#2,#3}\left(#1\right)}
%
%%%%%%%%%%%%%%%%%%%%%
%
%     2-D Constructions (Lists, Vectors, Matrices)
%
%%%%%%%%%%%%%%%%%%%%%
%
%  A list of scalars of generic length
%  Usage:  \scalarlist{scalar letter}{terminal subscript}
\newcommand{\scalarlist}[2]{{#1}_{1},\,{#1}_{2},\,{#1}_{3},\,\ldots,\,{#1}_{#2}}
%
%  Vector styling, bold (or use wiggles, arrows, whatever)
%  Subscripts go outside this construction
%  Usage: \vect{a symbol to use as a vector}
%  Have to already be in math mode
%
\newcommand{\vect}[1]{\mathbf{#1}}
%
%  A column vector
%  Usage: \colvector{list-delimited-by-\\}
%
\newcommand{\colvector}[1]{\begin{bmatrix}#1\end{bmatrix}}
%
%  A generic vector with components
%  Usage: \vectorcomponents{component-letter}{final-subscript}
\newcommand{\vectorcomponents}[2]{\colvector{#1_{1}\\#1_{2}\\#1_{3}\\\vdots\\#1_{#2}}}
%
%  A list of vectors of generic length
%  Usage:  \vectorlist{vector letter}{terminal subscript}
\newcommand{\vectorlist}[2]{\vect{#1}_{1},\,\vect{#1}_{2},\,\vect{#1}_{3},\,\ldots,\,\vect{#1}_{#2}}
%
%  Vector entries, entry i of vector v
%  (vector-expession still needs \vect, etc.)
%  Usage:  \vectorentry{vector-expression}{single-subscript}
\newcommand{\vectorentry}[2]{\left\lbrack#1\right\rbrack_{#2}}
%
%  Matrix entries, entry i,j of matrix A
%  Usage:  \matrixentry{matrix-expression}{paired-subscripts}
%
\newcommand{\matrixentry}[2]{\left\lbrack#1\right\rbrack_{#2}}
%
%  A generic linear combination
%  Usage:  \lincombo{scalar letter}{vector letter}{terminal subscript}
\newcommand{\lincombo}[3]{#1_{1}\vect{#2}_{1}+#1_{2}\vect{#2}_{2}+#1_{3}\vect{#2}_{3}+\cdots +#1_{#3}\vect{#2}_{#3}}
%
%  Matrix, column by column, as vectors
%  Usage:  \matrixcolumns{matrix letter}{terminal subscript}
\newcommand{\matrixcolumns}[2]{\left\lbrack\vect{#1}_{1}|\vect{#1}_{2}|\vect{#1}_{3}|\ldots|\vect{#1}_{#2}\right\rbrack}
%
%%%%%%%%%%%%%%%%%%%%%
%
%     Special Matrices
%
%%%%%%%%%%%%%%%%%%%%%
%
%  Transpose of a matrix
%  Usage:  \transpose{A}
\newcommand{\transpose}[1]{#1^{t}}
%
%  Inverse of a matrix
%  Usage:  \inverse{A}
\newcommand{\inverse}[1]{#1^{-1}}
%
%  Submatrix (for minors, determinants)
%  Usage: \submatrix{matrix-name}{delete-row}{delete-col}
\newcommand{\submatrix}[3]{#1\left(#2|#3\right)}
%
%  Adjoint of a matrix (twice)
%  This macro is a convenience to call \transpose and \conjugate properly
%  It shouldn't need to be modified (or mathematical meanings will change)
%  Usage:  \adj{A}
\newcommand{\adj}[1]{\transpose{\left(\conjugate{#1}\right)}}
%
%  This macro controls the symbol used for the adjoint
%  It can be edited to taste
%  Usage:  \adjoint{A}
\newcommand{\adjoint}[1]{#1^\ast}
%
%%%%%%%%%%%%%%%%%%%%%
%
%     Sets
%
%%%%%%%%%%%%%%%%%%%%%
%
%  A convenience for simple sets
%  Usage:  \set{list of element}
\newcommand{\set}[1]{\left\{#1\right\}}
%
%  Sets with vertical bar, "such that", sized for objects, not condition
%  Usage:  \setparts{objects}{condition}
%
%%\newcommand{\setparts}[2]{\left\{ #1\mid#2\right\}}
%%\newcommand{\setparts}[2]{\left\{\left. #1\right\rvert#2\right\}}
\newcommand{\setparts}[2]{\left\lbrace#1\,\middle|\,#2\right\rbrace}
%
%  Set Cardinality
%  Usage:  \card{a-set-letter}
\newcommand{\card}[1]{\left\lvert#1\right\rvert}
%
%  Set Union
%  Use \cup
%
%  Set Intersection
%  Use \cap
%
%  Set Complement
%  Usage:  \setcomplement{a-set-letter}
\newcommand{\setcomplement}[1]{\overline{#1}}
%
%%%%%%%%%%%%%%%%%%%%%
%
%     Eigenvalues and Eigenspaces
%
%%%%%%%%%%%%%%%%%%%%%
%
%  Characteristic polynomial
%  Usage: \charpoly{matrix-letter}{variable-letter}
\newcommand{\charpoly}[2]{p_{#1}\left(#2\right)}
%
%  Eigenspace
%  Usage: \eigenspace{matrix-letter}{eigenvalue-letter}
\newcommand{\eigenspace}[2]{\mathcal{E}_{#1}\left(#2\right)}
%
%  2013/10/03 Including ampersands is problematic here, 
%  think about fixes later
%  2014/02/22 Limited testing, seems &amp; is fine for HTML and LaTeX
%  2016-07-20 only employed in Archetypes, MBX has gather/align override
%  Eigensystem (presumes wrapped in an mrow within md)
%  Usage: \eigensystem{matrixletter}{eigenvalue}{list of basis vectors}
\newcommand{\eigensystem}[3]{\lambda&amp;=#2&amp;\eigenspace{#1}{#2}&amp;=\spn{\set{#3}}} 
%
%  Generalized Eigenspace
%  Usage: \geneigenspace{lin-trans-letter}{eigenvalue-letter}
\newcommand{\geneigenspace}[2]{\mathcal{G}_{#1}\left(#2\right)}
%
%  Algebraic multiplicty
%  Usage: \algmult{matrix-letter}{eigenvalue-letter}
\newcommand{\algmult}[2]{\alpha_{#1}\left(#2\right)}
%
%  Geometric multiplicty
%  Usage: \geomult{matrix-letter}{eigenvalue-letter}
\newcommand{\geomult}[2]{\gamma_{#1}\left(#2\right)}
%
%  Index (of eigenvalue)
%  Usage: \indx{matrix-letter}{eigenvalue-letter}
\newcommand{\indx}[2]{\iota_{#1}\left(#2\right)}
%
%%%%%%%%%%%%%%%%%%%%%
%
%     Linear Transformations
%
%%%%%%%%%%%%%%%%%%%%%
%
%  MathJax defines \lt to ease XML confusion
%
%  Linear transformation definition
%  Usage: \ltdefn{name-letter}{domain}{range}
\newcommand{\ltdefn}[3]{#1\colon #2\rightarrow#3}
%
%  Linear transformation evaluation
%  Usage: \lteval{name-letter}{input}
%  Replaces old \lt desired by MathJax
\newcommand{\lteval}[2]{#1\left(#2\right)}
%
% Linear transformation inverse
%  Usage: \ltinverse{name-letter}
\newcommand{\ltinverse}[1]{#1^{-1}}
%
%  Linear transformation restriction
%  Usage: \restrict{name-letter}{subspace-letter}
\newcommand{\restrict}[2]{{#1}|_{#2}}
%
%  Linear transformation preimage
%  Usage: \preimage{name-letter}{codomain-element}
\newcommand{\preimage}[2]{#1^{-1}\left(#2\right)}
%
%  Range of a linear transformation
%  TeX uses \range for something else
%  Usage:  \rng{T}
\newcommand{\rng}[1]{\mathcal{R}\!\left(#1\right)}
%
%  Kernel of a linear transformation
%  TeX uses \ker to do something different
%  Usage:  \krn{T}
\newcommand{\krn}[1]{\mathcal{K}\!\left(#1\right)}
%
%  Linear transformation composition
%  Usage: \compose{function-name}{function-name}
\newcommand{\compose}[2]{{#1}\circ{#2}}
%
%  Vector space of linear transformations
%  Usage: \vslt{domains}{codomains}
%  Presumes math mode
\newcommand{\vslt}[2]{\mathcal{LT}\left(#1,\,#2\right)}
%
%%%%%%%%%%%%%%%%%%%%%
%
%     Vector and Matrix Representations
%
%%%%%%%%%%%%%%%%%%%%%
%
%  Isomorphism symbol
%  Usage: \isomorphic
\newcommand{\isomorphic}{\cong}
%
%  Similarity
%  Usage: \similar{inner-matrix}{outer-invertible-matrix}
%  Rearranging this will not "fix" all desired changes throughout
%
\newcommand{\similar}[2]{\inverse{#2}#1#2}
%
%  Vector representation function name
%  Usage: \vectrepname{basis-letter}
\newcommand{\vectrepname}[1]{\rho_{#1}}
%
%  Vector representation output
%  Usage: \vectrep{basis-letter}{input}
\newcommand{\vectrep}[2]{\lteval{\vectrepname{#1}}{#2}}
%
%  Vector representation inverse function name
%  (Added later, not used consistently in FCLA)
%  Usage: \vectrepinvname{basis-letter}
\newcommand{\vectrepinvname}[1]{\ltinverse{\vectrepname{#1}}}
%
%  Vector representation inverse output
%  Usage: \vectrepinv{basis-letter}{input}
\newcommand{\vectrepinv}[2]{\lteval{\ltinverse{\vectrepname{#1}}}{#2}}
%
%  Matrix representation
%  Usage: \matrixrep{transformation-letter}{domain-basis-letter}{codomain-basis-letter}
\newcommand{\matrixrep}[3]{M^{#1}_{#2,#3}}
%
%  Matrix representation column-by-colum
%  2016-07-20 only employed once?
%  Usage: \matrixrepcolumns{transformation-letter}{codomain-basis-letter}{codomain-basis-vector-letter}{final-index}
\newcommand{\matrixrepcolumns}[4]{\left\lbrack \left.\vectrep{#2}{\lteval{#1}{\vect{#3}_{1}}}\right|\left.\vectrep{#2}{\lteval{#1}{\vect{#3}_{2}}}\right|\left.\vectrep{#2}{\lteval{#1}{\vect{#3}_{3}}}\right|\ldots\left|\vectrep{#2}{\lteval{#1}{\vect{#3}_{#4}}}\right.\right\rbrack}
%
%  Change of basis matrix
%  Usage: \cbm{domain-basis-letter}{codomain-basis-letter}
\newcommand{\cbm}[2]{C_{#1,#2}}
%
%%%%%%%%%%%%%%%%%%%%%
%
%     Canonical Forms
%
%%%%%%%%%%%%%%%%%%%%%
%
%  Jordan blocks
%  Usage: \jordan{size}{diagonal-element}
\newcommand{\jordan}[2]{J_{#1}\left(#2\right)}
%
%%%%%%%%%%%%%%%%%%%%%
%
%     Hadamard Matrices
%     Contributed by Elizabeth Million
%
%%%%%%%%%%%%%%%%%%%%%
%
%  Hadamard Product
%  Usage: \hadamard{a-matrix}{a-matrix}
\newcommand{\hadamard}[2]{#1\circ #2}
%
%  Hadamard identity matrix
%  Usage: \hadamardidentity{paired-subscripts-size-of-matrix}
\newcommand{\hadamardidentity}[1]{J_{#1}}
%
%  Hadamard inverse matrix
%  Usage: \hadamardinverse{matrix-expression}
\newcommand{\hadamardinverse}[1]{\widehat{#1}}

\newcommand{\definedTerm}[1]{\textbf{#1}}
\newcommand{\dfn}[1]{\textbf{#1}}

\newcommand{\wt}{\widetilde}
\newcommand{\ov}{\overline}
\newcommand{\inj}{\rightarrowtail}
\newcommand{\surj}{\twoheadrightarrow}
\newcommand{\harpoon}{\overset{\rightharpoonup}}

\newenvironment{amatrix}[1]{%
  \left[\begin{array}{@{}*{#1}{c}|c@{}}
}{%
  \end{array}\right]
}


\title{Image of a Linear Transformation}

\begin{document}
\begin{abstract}
  Informally, the image is the set of all outputs that the transformation creates when fed every possible input from the domain.
\end{abstract}
\maketitle

For a linear transformation $\ltdefn{T}{U}{V}$, the image is a subset of the codomain $V$.  Informally, it is the set of all outputs that the transformation creates when fed every possible input from the domain.  It will have some natural connections with the column space of a matrix, so we will keep the same notation, and if you think about your objects, then there should be little confusion.  Here is the careful definition.

\begin{definition}[Image of a Linear Transformation]

Suppose $\ltdefn{T}{U}{V}$ is a linear transformation.  Then the \dfn{image} of $T$ is the set
\[
\rng{T}=\setparts{\lteval{T}{\vect{u}}}{\vect{u}\in U}
\]
\end{definition}

\begin{example}
Consider
\[
\ltdefn{T}{\complex{3}}{\complex{5}},\quad
\lteval{T}{\colvector{x_1\\x_2\\x_3}}=
\colvector{-x_1 + x_2 - 3 x_3\\
-x_1 + 2 x_2 - 4 x_3\\
x_1 + x_2 + x_3\\
2 x_1 + 3 x_2 + x_3\\
x_1 + 2 x_3
}
\]


To determine the elements of $\complex{5}$ in $\rng{T}$, find those vectors $\vect{v}$ such that $\lteval{T}{\vect{u}}=\vect{v}$ for some $\vect{u}\in\complex{3}$,
\begin{align*}
\vect{v}&=\lteval{T}{\vect{u}}\\
&=\colvector{-u_1 + u_2 - 3 u_3\\
-u_1 + 2 u_2 - 4 u_3\\
u_1 + u_2 + u_3\\
2 u_1 + 3 u_2 + u_3\\
u_1 + 2 u_3
}\\
&=
\colvector{-u_1\\-u_1\\u_1\\2 u_1\\ u_1}
+
\colvector{u_2\\2u_2\\u_2\\3u_2\\ 0}
+
\colvector{-3u_3\\-4u_3\\u_3\\u_3\\ 2 u_3}\\
&=
u_1\colvector{-1\\-1\\1\\2\\1}
+
u_2\colvector{1\\2\\1\\3\\ 0}
+
u_3\colvector{-3\\-4\\1\\1\\2}
\end{align*}




This says that every output of $T$ (in other words, the vector $\vect{v}$) can be written as a linear combination of the three vectors
\begin{align*}
\colvector{-1\\-1\\1\\2\\1}
&&
\colvector{1\\2\\1\\3\\ 0}
&&
\colvector{-3\\-4\\1\\1\\2}
\end{align*}
using the scalars $u_1,\,u_2,\,u_3$.  Furthermore, since $\vect{u}$ can be any element of $\complex{3}$, every such linear combination is an output.  This means that
\[
\rng{T}=\spn{\set{
\colvector{-1\\-1\\1\\2\\1},\,
\colvector{1\\2\\1\\3\\ 0},\,
\colvector{-3\\-4\\1\\1\\2}
}}
\]




The three vectors in this spanning set for $\rng{T}$ form a linearly dependent set (check this!).  So we can find a more economical presentation by any of the various methods from \ref{section:CRS} and \ref{section:FS}.  We will place the vectors into a matrix as rows, row-reduce, toss out zero rows and appeal to \ref{theorem:BRS}, so we can describe the image of $T$ with a basis,
\[
\rng{T}=\spn{\set{
\colvector{1\\0\\-3\\-7\\-2},\,\colvector{0\\1\\2\\5\\1}
}}
\]

\end{example}

We know that the span of a set of vectors is always a subspace (\ref{theorem:SSS}), so the image computed in \ref{example:RAO} is also a subspace.  This is no accident, the image of a linear transformation is \textit{always} a subspace.



\begin{theorem}[Image of a Linear Transformation is a Subspace]
\label{theorem:RLTS}


Suppose that $\ltdefn{T}{U}{V}$ is a linear transformation.  Then the image of $T$, $\rng{T}$, is a subspace of $V$.





\begin{proof}
We can apply the three-part test of \ref{theorem:TSS}.  First, $\zerovector_U\in U$ and $\lteval{T}{\zerovector_U}=\zerovector_V$ by \ref{theorem:LTTZZ}, so $\zerovector_V\in\rng{T}$ and we know that the image is nonempty.



Suppose we assume that $\vect{x},\,\vect{y}\in\rng{T}$.  Is $\vect{x}+\vect{y}\in\rng{T}$?  If $\vect{x},\,\vect{y}\in\rng{T}$ then we know there are vectors $\vect{w},\,\vect{z}\in U$ such that $\lteval{T}{\vect{w}}=\vect{x}$ and $\lteval{T}{\vect{z}}=\vect{y}$.  Because $U$ is a vector space, additive closure (\ref{property:AC}) implies that $\vect{w}+\vect{z}\in U$.



Then
\begin{align*}
\lteval{T}{\vect{w}+\vect{z}}&=\lteval{T}{\vect{w}}+\lteval{T}{\vect{z}}&&\ref{definition:LT}\\
&=\vect{x}+\vect{y}&&\text{Definition of $\vect{w}$ and $\vect{z}$}
\end{align*}




So we have found an input, $\vect{w}+\vect{z}$, which when fed into $T$ creates $\vect{x}+\vect{y}$ as an output.  This qualifies $\vect{x}+\vect{y}$ for membership in $\rng{T}$.  So we have additive closure.



Suppose we assume that $\alpha\in\complexes$ and $\vect{x}\in\rng{T}$.  Is $\alpha\vect{x}\in\rng{T}$?  If $\vect{x}\in\rng{T}$, then there is a vector $\vect{w}\in U$ such that $\lteval{T}{\vect{w}}=\vect{x}$.  Because $U$ is a vector space, scalar closure implies that $\alpha\vect{w}\in U$.  Then
\begin{align*}
\lteval{T}{\alpha\vect{w}}&=\alpha\lteval{T}{\vect{w}}&&\ref{definition:LT}\\
&=\alpha\vect{x}&&\text{Definition of $\vect{w}$}
\end{align*}




So we have found an input ($\alpha\vect{w}$) which when fed into $T$ creates $\alpha\vect{x}$ as an output.  This qualifies $\alpha\vect{x}$ for membership in $\rng{T}$.  So we have scalar closure and \ref{theorem:TSS} tells us that $\rng{T}$ is a subspace of $V$.



\end{proof}
\end{theorem}

Let us compute another image, now that we know in advance that it will be a subspace.



\begin{example}
Consider
\[
\ltdefn{T}{\complex{5}}{\complex{3}},\quad
\lteval{T}{\colvector{x_1\\x_2\\x_3\\x_4\\x_5}}=
\colvector{2 x_1 + x_2 + 3 x_3 - 4 x_4 + 5 x_5\\
x_1 - 2 x_2 + 3 x_3 - 9 x_4 + 3 x_5\\
3 x_1 + 4 x_3 - 6 x_4 + 5 x_5}
\]

To determine the elements of $\complex{3}$ in $\rng{T}$, find those vectors $\vect{v}$ such that $\lteval{T}{\vect{u}}=\vect{v}$ for some $\vect{u}\in\complex{5}$,
\begin{align*}
\vect{v}&=\lteval{T}{\vect{u}}\\
&=
\colvector{
2 u_1 + u_2 + 3 u_3 - 4 u_4 + 5 u_5\\
u_1 - 2 u_2 + 3 u_3 - 9 u_4 + 3 u_5\\
3 u_1 + 4 u_3 - 6 u_4 + 5 u_5}\\
&=
\colvector{2u_1\\u_1\\3u_1}+
\colvector{u_2\\-2u_2\\0}+
\colvector{3u_3\\3u_3\\4u_3}+
\colvector{-4u_4\\-9u_4\\-6u_4}+
\colvector{5u_5\\3u_5\\5u_5}\\
&=
u_1\colvector{2\\1\\3}+
u_2\colvector{1\\-2\\0}+
u_3\colvector{3\\3\\4}+
u_4\colvector{-4\\-9\\-6}+
u_5\colvector{5\\3\\5}\\
\end{align*}




This says that every output of $T$ (in other words, the vector $\vect{v}$) can be written as a linear combination of the five vectors
\begin{align*}
\colvector{2\\1\\3}&&
\colvector{1\\-2\\0}&&
\colvector{3\\3\\4}&&
\colvector{-4\\-9\\-6}&&
\colvector{5\\3\\5}
\end{align*}
using the scalars $u_1,\,u_2,\,u_3,\,u_4,\,u_5$.  Furthermore, since $\vect{u}$ can be any element of $\complex{5}$, every such linear combination is an output.  This means that
\[
\rng{T}=\spn{\set{
\colvector{2\\1\\3},\,
\colvector{1\\-2\\0},\,
\colvector{3\\3\\4},\,
\colvector{-4\\-9\\-6},\,
\colvector{5\\3\\5}
}}
\]


The five vectors in this spanning set for $\rng{T}$ form a linearly dependent set (\ref{theorem:MVSLD}).  So we can find a more economical presentation by any of the various methods from \ref{section:CRS} and \ref{section:FS}.  We will place the vectors into a matrix as rows, row-reduce, toss out zero rows and appeal to \ref{theorem:BRS}, so we can describe the image of $T$ with a (nice) basis,
\[
\rng{T}=\spn{\set{
\colvector{1\\0\\0},\,\colvector{0\\1\\0},\,\colvector{0\\0\\1}
}} = \complex{3}\]

\end{example}

In contrast to injective linear transformations having small (trivial) kernels (\ref{theorem:KILT}), surjective linear transformations have large images, as indicated in the next theorem.



\begin{theorem}[Image of a Surjective Linear Transformation]
\label{theorem:RSLT}


Suppose that $\ltdefn{T}{U}{V}$ is a linear transformation.  Then $T$ is surjective if and only if the image of $T$ equals the codomain, $\rng{T}=V$.


\begin{proof}
($\Rightarrow$) By \ref{definition:RLT}, we know that $\rng{T}\subseteq V$.  To establish the reverse inclusion, assume $\vect{v}\in V$.  Then since $T$ is surjective (\ref{definition:SLT}), there exists a vector $\vect{u}\in U$ so that $\lteval{T}{\vect{u}}=\vect{v}$.  However, the existence of $\vect{u}$ gains $\vect{v}$ membership in $\rng{T}$, so $V\subseteq\rng{T}$.  Thus, $\rng{T}=V$.



($\Leftarrow$)  To establish that $T$ is surjective, choose $\vect{v}\in V$.  Since we are assuming that $\rng{T}=V$, $\vect{v}\in\rng{T}$.  This says there is a vector $\vect{u}\in U$ so that $\lteval{T}{\vect{u}}=\vect{v}$, i.e.,  $T$ is surjective.



\end{proof}
\end{theorem}

\begin{example}

Recall that
b\[\ltdefn{T}{\complex{5}}{\complex{5}},\quad
\lteval{T}{\colvector{x_1\\x_2\\x_3\\x_4\\x_5}}=
\colvector{-2 x_1 + 3 x_2 + 3 x_3 - 6 x_4 + 3 x_5\\
-16 x_1 + 9 x_2 + 12 x_3 - 28 x_4 + 28 x_5\\
-19 x_1 + 7 x_2 + 14 x_3 - 32 x_4 + 37 x_5\\
-21 x_1 + 9 x_2 + 15 x_3 - 35 x_4 + 39 x_5\\
-9 x_1 + 5 x_2 + 7 x_3 - 16 x_4 + 16 x_5}
\]
was shown to be not surjective by constructing a vector in the codomain where no element of the domain could be used to evaluate the linear transformation to create the output, thus violating \ref{definition:SLT}.  Just where did this vector come from?

The short answer is that the vector
\[
\vect{v}=\colvector{-1\\2\\3\\-1\\4}
\]
was constructed to lie outside of the image of $T$.  How was this accomplished?  First, the image of $T$ is given by
\[
\rng{T}=\spn{\set{
\colvector{1\\0\\0\\0\\1},\,\colvector{0\\1\\0\\0\\-1},\,
\colvector{0\\0\\1\\0\\-1},\,\colvector{0\\0\\0\\1\\2}
}}
\]

Suppose an element of the image $\vect{v^*}$ has its first 4 components equal to $-1$, $2$, $3$, $-1$, in that order.  Then to be an element of $\rng{T}$, we would have
\[
\vect{v^*}=(-1)\colvector{1\\0\\0\\0\\1}+(2)\colvector{0\\1\\0\\0\\-1}+(3)
\colvector{0\\0\\1\\0\\-1}+(-1)\colvector{0\\0\\0\\1\\2}
=\colvector{-1\\2\\3\\-1\\-8}
\]


So the only vector in the image with these first four components specified, must have $-8$ in the fifth component.  To set the fifth component to any other value (say, 4) will result in a vector ($\vect{v}$ in \ref{example:NSAQ})  outside of the image.  Any attempt to find an input for $T$ that will produce $\vect{v}$ as an output will be doomed to failure.



Whenever the image of a linear transformation is not the whole codomain, we can employ this device and conclude that the linear transformation is not surjective.
This is another way of viewing \ref{theorem:RSLT}.  For a surjective linear transformation, the image is all of the codomain and there is no choice for a vector $\vect{v}$ that lies in $V$, yet not in the image.  For every one of the archetypes that is not surjective, there is an example presented of exactly this form.



\end{example}

\begin{example}

The image of
\[
\ltdefn{T}{\complex{3}}{\complex{5}},\quad
\lteval{T}{\colvector{x_1\\x_2\\x_3}}=
\colvector{-x_1 + x_2 - 3 x_3\\
-x_1 + 2 x_2 - 4 x_3\\
x_1 + x_2 + x_3\\
2 x_1 + 3 x_2 + x_3\\
x_1 + 2 x_3
}
\]
 was determined to be
\[
\rng{T}=\spn{\colvector{1\\0\\-3\\-7\\-2},\,\colvector{0\\1\\2\\5\\1}}
\]
a subspace of dimension 2 in $\complex{5}$.  Since $\rng{T}\neq\complex{5}$, \ref{theorem:RSLT} says 
\begin{multipleChoice}
\choice{$T$ is surjective.}
\choice[correct]{$T$ is not surjective.}
\end{multipleChoice}
\end{example}

\begin{example}

The image of 
\[
\ltdefn{T}{\complex{5}}{\complex{3}},\quad
\lteval{T}{\colvector{x_1\\x_2\\x_3\\x_4\\x_5}}=
\colvector{2 x_1 + x_2 + 3 x_3 - 4 x_4 + 5 x_5\\
x_1 - 2 x_2 + 3 x_3 - 9 x_4 + 3 x_5\\
3 x_1 + 4 x_3 - 6 x_4 + 5 x_5}
\]
 was computed to be
\[
\rng{T}=\spn{\set{
\colvector{1\\0\\0},\,\colvector{0\\1\\0},\,\colvector{0\\0\\1}
}}\]

Since the basis for this subspace is the set of standard unit vectors for $\complex{3}$ (\ref{theorem:SUVB}), we have $\rng{T}=\complex{3}$ and by \ref{theorem:RSLT}, we may conclude
\begin{multipleChoice}
\choice[correct]{$T$ is surjective.}
\choice{$T$ is not surjective.}
\end{multipleChoice}

\end{example}

\end{document}
