\documentclass[handout]{ximera}
% These macros are automatically generated from the "macros"
% XML element.  Make permanent edits there.
%
% History
%   2004/01/01  Initiated for FCLA, evolved from there
%   2006/09/17  Converted  _, ^  to \sb, \sp for TeX4ht
%   2014/02/01  Updated for MathBook XML projects
%               Obsolete in FCLA: \codeindent, \computerfont, \define
%               Change: MathJax wants \lt, so replaced by \lteval
%   2014/02/22  New: \orderof, \reals, \per
%   2015/08/16  Incorporated into MathBook XML version of FCLA
%
%%%%%%%%%%%%%%%%%%%%%
%
%     Conveniences
%
%%%%%%%%%%%%%%%%%%%%%
%
%  Order of (asymptotically limit of fraction is 1)
%  Usage: \orderof{some function}
%
\newcommand{\orderof}[1]{\sim #1}
%
%  Integers
%  Usage:  \Z
\newcommand{\Z}{\mathbb{Z}}
%
%  Real numbers, as set of scalars
%  Usage:  \reals
\newcommand{\reals}{\mathbb{R}}
%
%  n-space over real field
%  Usage: \complex{integer-dimension}
\newcommand{\real}[1]{\mathbb{R}^{#1}}
%
%  Complex numbers, as set of scalars
%  Usage:  \complexes
\newcommand{\complexes}{\mathbb{C}}
%
%  n-space over complex field
%  Usage: \complex{integer-dimension}
\newcommand{\complex}[1]{\mathbb{C}^{#1}}
\newcommand{\CC}{\mathbb{C}}
%
%  Complex conjugation (scalar, vector, matrix)
%  Usage: \conjugate{object}
\newcommand{\conjugate}[1]{\overline{#1}}
%
%  Complex number modulus
%  Usage: \modulus{a+bi}
%  Presumes math mode
\newcommand{\modulus}[1]{\left\lvert#1\right\rvert}
%
%  Zero vector
%  Usage: \zerovector
\newcommand{\zerovector}{\vect{0}}
%
%  Zero matrix
%  Usage: \zeromatrix, use a subscript when size is important
\newcommand{\zeromatrix}{\mathcal{O}}
%
%  Inner product (brackets, not quadratic form)
%  Usage: \innerproduct{a-vector}{a-vector}
\newcommand{\innerproduct}[2]{\left\langle#1,\,#2\right\rangle}
%
%  Norm of a vector
%  Usage: \norm{a-vector}
\newcommand{\norm}[1]{\left\lVert#1\right\rVert}
%
%  Dimension
%  Usage: \dimension{vector-space-letter}
\newcommand{\dimension}[1]{\dim\left(#1\right)}
%
%  Nullity
%  Usage: \nullity{matrix-or-lintrans-letter}
\newcommand{\nullity}[1]{n\left(#1\right)}
%
%  Rank
%  Usage: \rank{matrix-or-lintrans-letter}
\newcommand{\rank}[1]{r\left(#1\right)}
%
%  Direct sum
%  Usage: \ds between a couple of subspaces
%
\newcommand{\ds}{\oplus}
%
%  Determinant of a matrix (functional)
%  Usage: \detname{A}
\newcommand{\detname}[1]{\det\left(#1\right)}
%
%  Determinant of a matrix (vertical bars)
%  Usage: \detbars{A}
\newcommand{\detbars}[1]{\left\lvert#1\right\rvert}
%
%  Trace of a Matrix
%  Usage: \trace{matrix name}
\newcommand{\trace}[1]{t\left(#1\right)}
%
%  Square Root of a Matrix
%  Usage: \sr{a-matrix}
\newcommand{\sr}[1]{#1^{1/2}}
%
%%%%%%%%%%%%%%%%%%%%%
%
%     Subspace Constructions
%
%%%%%%%%%%%%%%%%%%%%%
%
%  Span of a set of vectors
%  \span and \sp are used by TeX for other things
%  Usage: \spn{set-of-vectors}
\newcommand{\spn}[1]{\left\langle#1\right\rangle}
%
%  Null space of a matrix
%  Usage:  \nsp{A}
\newcommand{\nsp}[1]{\mathcal{N}\!\left(#1\right)}
%
%  Column space of a matrix
%  Usage:  \csp{A}
\newcommand{\csp}[1]{\mathcal{C}\!\left(#1\right)}
%
%  Row space of a matrix
%  Usage:  \rsp{A}
\newcommand{\rsp}[1]{\mathcal{R}\!\left(#1\right)}
%
%  Left null space of a matrix
%  Usage:  \lns{A}
\newcommand{\lns}[1]{\mathcal{L}\!\left(#1\right)}
%
%  Orthogonal complement of a vector space
%  Avoiding TeX's \perp
%  Usage:  \per{A}
\newcommand{\per}[1]{#1^\perp}
%
%%%%%%%%%%%%%%%%%%%%%
%
%     Systems of Equations
%
%%%%%%%%%%%%%%%%%%%%%
%
%  In-line form of an augmented matrix for a system of equations
%  Usage: \augmented{coefficient-matrix}{constant-vector}
\newcommand{\augmented}[2]{\left\lbrack\left.#1\,\right\rvert\,#2\right\rbrack}
%
%  Notation for a linear system before introducing matrix multiplication
%  Usage: \linearsystem{coefficient-matrix}{constant-vector}
\newcommand{\linearsystem}[2]{\mathcal{LS}\!\left(#1,\,#2\right)}
%
%  Notation for a homogenous system before introducing matrix multiplication
%  Usage: \homosystem{coefficient-matrix}
\newcommand{\homosystem}[1]{\linearsystem{#1}{\zerovector}}
%
%%%%%%%%%%%%%%%%%%%%%
%
%     Row Operations, Echelon Form
%
%%%%%%%%%%%%%%%%%%%%%
%
% Row operations on matrices
%
% Three commands to shorten up descriptions of gaussian elimination
%
% Usage: \rowopswap{row-i}{row-j}
% Usage: \rowopmult{scalar}{row-i}
% Usage: \rowopadd{scalar}{row-multiplied}{row-added-to}
\newcommand{\rowopswap}[2]{R_{#1}\leftrightarrow R_{#2}}
\newcommand{\rowopmult}[2]{#1R_{#2}}
\newcommand{\rowopadd}[3]{#1R_{#2}+R_{#3}}
%
% Mark leading 1's in echelon form with fbox
% Usage: \leading{a-1-usually}
\newcommand{\leading}[1]{\fbox{#1}}
%
%  Row-reduce arrow
%  Usage:  \rref inbetween a matrix and its reduced row-echelon form
\newcommand{\rref}{\xrightarrow{\text{RREF}}}
%
%  Elementary Matrices
%  Usage: \elemswap{subscript}{subscript}
%  Usage: \elemmult{scalar}{subscript}
%  Usage: \elemadd{scalar}{subscript-mult}{subscript-target}
%
\newcommand{\elemswap}[2]{E_{#1,#2}}
\newcommand{\elemmult}[2]{E_{#2}\left(#1\right)}
\newcommand{\elemadd}[3]{E_{#2,#3}\left(#1\right)}
%
%%%%%%%%%%%%%%%%%%%%%
%
%     2-D Constructions (Lists, Vectors, Matrices)
%
%%%%%%%%%%%%%%%%%%%%%
%
%  A list of scalars of generic length
%  Usage:  \scalarlist{scalar letter}{terminal subscript}
\newcommand{\scalarlist}[2]{{#1}_{1},\,{#1}_{2},\,{#1}_{3},\,\ldots,\,{#1}_{#2}}
%
%  Vector styling, bold (or use wiggles, arrows, whatever)
%  Subscripts go outside this construction
%  Usage: \vect{a symbol to use as a vector}
%  Have to already be in math mode
%
\newcommand{\vect}[1]{\mathbf{#1}}
%
%  A column vector
%  Usage: \colvector{list-delimited-by-\\}
%
\newcommand{\colvector}[1]{\begin{bmatrix}#1\end{bmatrix}}
%
%  A generic vector with components
%  Usage: \vectorcomponents{component-letter}{final-subscript}
\newcommand{\vectorcomponents}[2]{\colvector{#1_{1}\\#1_{2}\\#1_{3}\\\vdots\\#1_{#2}}}
%
%  A list of vectors of generic length
%  Usage:  \vectorlist{vector letter}{terminal subscript}
\newcommand{\vectorlist}[2]{\vect{#1}_{1},\,\vect{#1}_{2},\,\vect{#1}_{3},\,\ldots,\,\vect{#1}_{#2}}
%
%  Vector entries, entry i of vector v
%  (vector-expession still needs \vect, etc.)
%  Usage:  \vectorentry{vector-expression}{single-subscript}
\newcommand{\vectorentry}[2]{\left\lbrack#1\right\rbrack_{#2}}
%
%  Matrix entries, entry i,j of matrix A
%  Usage:  \matrixentry{matrix-expression}{paired-subscripts}
%
\newcommand{\matrixentry}[2]{\left\lbrack#1\right\rbrack_{#2}}
%
%  A generic linear combination
%  Usage:  \lincombo{scalar letter}{vector letter}{terminal subscript}
\newcommand{\lincombo}[3]{#1_{1}\vect{#2}_{1}+#1_{2}\vect{#2}_{2}+#1_{3}\vect{#2}_{3}+\cdots +#1_{#3}\vect{#2}_{#3}}
%
%  Matrix, column by column, as vectors
%  Usage:  \matrixcolumns{matrix letter}{terminal subscript}
\newcommand{\matrixcolumns}[2]{\left\lbrack\vect{#1}_{1}|\vect{#1}_{2}|\vect{#1}_{3}|\ldots|\vect{#1}_{#2}\right\rbrack}
%
%%%%%%%%%%%%%%%%%%%%%
%
%     Special Matrices
%
%%%%%%%%%%%%%%%%%%%%%
%
%  Transpose of a matrix
%  Usage:  \transpose{A}
\newcommand{\transpose}[1]{#1^{t}}
%
%  Inverse of a matrix
%  Usage:  \inverse{A}
\newcommand{\inverse}[1]{#1^{-1}}
%
%  Submatrix (for minors, determinants)
%  Usage: \submatrix{matrix-name}{delete-row}{delete-col}
\newcommand{\submatrix}[3]{#1\left(#2|#3\right)}
%
%  Adjoint of a matrix (twice)
%  This macro is a convenience to call \transpose and \conjugate properly
%  It shouldn't need to be modified (or mathematical meanings will change)
%  Usage:  \adj{A}
\newcommand{\adj}[1]{\transpose{\left(\conjugate{#1}\right)}}
%
%  This macro controls the symbol used for the adjoint
%  It can be edited to taste
%  Usage:  \adjoint{A}
\newcommand{\adjoint}[1]{#1^\ast}
%
%%%%%%%%%%%%%%%%%%%%%
%
%     Sets
%
%%%%%%%%%%%%%%%%%%%%%
%
%  A convenience for simple sets
%  Usage:  \set{list of element}
\newcommand{\set}[1]{\left\{#1\right\}}
%
%  Sets with vertical bar, "such that", sized for objects, not condition
%  Usage:  \setparts{objects}{condition}
%
%%\newcommand{\setparts}[2]{\left\{ #1\mid#2\right\}}
%%\newcommand{\setparts}[2]{\left\{\left. #1\right\rvert#2\right\}}
\newcommand{\setparts}[2]{\left\lbrace#1\,\middle|\,#2\right\rbrace}
%
%  Set Cardinality
%  Usage:  \card{a-set-letter}
\newcommand{\card}[1]{\left\lvert#1\right\rvert}
%
%  Set Union
%  Use \cup
%
%  Set Intersection
%  Use \cap
%
%  Set Complement
%  Usage:  \setcomplement{a-set-letter}
\newcommand{\setcomplement}[1]{\overline{#1}}
%
%%%%%%%%%%%%%%%%%%%%%
%
%     Eigenvalues and Eigenspaces
%
%%%%%%%%%%%%%%%%%%%%%
%
%  Characteristic polynomial
%  Usage: \charpoly{matrix-letter}{variable-letter}
\newcommand{\charpoly}[2]{p_{#1}\left(#2\right)}
%
%  Eigenspace
%  Usage: \eigenspace{matrix-letter}{eigenvalue-letter}
\newcommand{\eigenspace}[2]{\mathcal{E}_{#1}\left(#2\right)}
%
%  2013/10/03 Including ampersands is problematic here, 
%  think about fixes later
%  2014/02/22 Limited testing, seems &amp; is fine for HTML and LaTeX
%  2016-07-20 only employed in Archetypes, MBX has gather/align override
%  Eigensystem (presumes wrapped in an mrow within md)
%  Usage: \eigensystem{matrixletter}{eigenvalue}{list of basis vectors}
\newcommand{\eigensystem}[3]{\lambda&amp;=#2&amp;\eigenspace{#1}{#2}&amp;=\spn{\set{#3}}} 
%
%  Generalized Eigenspace
%  Usage: \geneigenspace{lin-trans-letter}{eigenvalue-letter}
\newcommand{\geneigenspace}[2]{\mathcal{G}_{#1}\left(#2\right)}
%
%  Algebraic multiplicty
%  Usage: \algmult{matrix-letter}{eigenvalue-letter}
\newcommand{\algmult}[2]{\alpha_{#1}\left(#2\right)}
%
%  Geometric multiplicty
%  Usage: \geomult{matrix-letter}{eigenvalue-letter}
\newcommand{\geomult}[2]{\gamma_{#1}\left(#2\right)}
%
%  Index (of eigenvalue)
%  Usage: \indx{matrix-letter}{eigenvalue-letter}
\newcommand{\indx}[2]{\iota_{#1}\left(#2\right)}
%
%%%%%%%%%%%%%%%%%%%%%
%
%     Linear Transformations
%
%%%%%%%%%%%%%%%%%%%%%
%
%  MathJax defines \lt to ease XML confusion
%
%  Linear transformation definition
%  Usage: \ltdefn{name-letter}{domain}{range}
\newcommand{\ltdefn}[3]{#1\colon #2\rightarrow#3}
%
%  Linear transformation evaluation
%  Usage: \lteval{name-letter}{input}
%  Replaces old \lt desired by MathJax
\newcommand{\lteval}[2]{#1\left(#2\right)}
%
% Linear transformation inverse
%  Usage: \ltinverse{name-letter}
\newcommand{\ltinverse}[1]{#1^{-1}}
%
%  Linear transformation restriction
%  Usage: \restrict{name-letter}{subspace-letter}
\newcommand{\restrict}[2]{{#1}|_{#2}}
%
%  Linear transformation preimage
%  Usage: \preimage{name-letter}{codomain-element}
\newcommand{\preimage}[2]{#1^{-1}\left(#2\right)}
%
%  Range of a linear transformation
%  TeX uses \range for something else
%  Usage:  \rng{T}
\newcommand{\rng}[1]{\mathcal{R}\!\left(#1\right)}
%
%  Kernel of a linear transformation
%  TeX uses \ker to do something different
%  Usage:  \krn{T}
\newcommand{\krn}[1]{\mathcal{K}\!\left(#1\right)}
%
%  Linear transformation composition
%  Usage: \compose{function-name}{function-name}
\newcommand{\compose}[2]{{#1}\circ{#2}}
%
%  Vector space of linear transformations
%  Usage: \vslt{domains}{codomains}
%  Presumes math mode
\newcommand{\vslt}[2]{\mathcal{LT}\left(#1,\,#2\right)}
%
%%%%%%%%%%%%%%%%%%%%%
%
%     Vector and Matrix Representations
%
%%%%%%%%%%%%%%%%%%%%%
%
%  Isomorphism symbol
%  Usage: \isomorphic
\newcommand{\isomorphic}{\cong}
%
%  Similarity
%  Usage: \similar{inner-matrix}{outer-invertible-matrix}
%  Rearranging this will not "fix" all desired changes throughout
%
\newcommand{\similar}[2]{\inverse{#2}#1#2}
%
%  Vector representation function name
%  Usage: \vectrepname{basis-letter}
\newcommand{\vectrepname}[1]{\rho_{#1}}
%
%  Vector representation output
%  Usage: \vectrep{basis-letter}{input}
\newcommand{\vectrep}[2]{\lteval{\vectrepname{#1}}{#2}}
%
%  Vector representation inverse function name
%  (Added later, not used consistently in FCLA)
%  Usage: \vectrepinvname{basis-letter}
\newcommand{\vectrepinvname}[1]{\ltinverse{\vectrepname{#1}}}
%
%  Vector representation inverse output
%  Usage: \vectrepinv{basis-letter}{input}
\newcommand{\vectrepinv}[2]{\lteval{\ltinverse{\vectrepname{#1}}}{#2}}
%
%  Matrix representation
%  Usage: \matrixrep{transformation-letter}{domain-basis-letter}{codomain-basis-letter}
\newcommand{\matrixrep}[3]{M^{#1}_{#2,#3}}
%
%  Matrix representation column-by-colum
%  2016-07-20 only employed once?
%  Usage: \matrixrepcolumns{transformation-letter}{codomain-basis-letter}{codomain-basis-vector-letter}{final-index}
\newcommand{\matrixrepcolumns}[4]{\left\lbrack \left.\vectrep{#2}{\lteval{#1}{\vect{#3}_{1}}}\right|\left.\vectrep{#2}{\lteval{#1}{\vect{#3}_{2}}}\right|\left.\vectrep{#2}{\lteval{#1}{\vect{#3}_{3}}}\right|\ldots\left|\vectrep{#2}{\lteval{#1}{\vect{#3}_{#4}}}\right.\right\rbrack}
%
%  Change of basis matrix
%  Usage: \cbm{domain-basis-letter}{codomain-basis-letter}
\newcommand{\cbm}[2]{C_{#1,#2}}
%
%%%%%%%%%%%%%%%%%%%%%
%
%     Canonical Forms
%
%%%%%%%%%%%%%%%%%%%%%
%
%  Jordan blocks
%  Usage: \jordan{size}{diagonal-element}
\newcommand{\jordan}[2]{J_{#1}\left(#2\right)}
%
%%%%%%%%%%%%%%%%%%%%%
%
%     Hadamard Matrices
%     Contributed by Elizabeth Million
%
%%%%%%%%%%%%%%%%%%%%%
%
%  Hadamard Product
%  Usage: \hadamard{a-matrix}{a-matrix}
\newcommand{\hadamard}[2]{#1\circ #2}
%
%  Hadamard identity matrix
%  Usage: \hadamardidentity{paired-subscripts-size-of-matrix}
\newcommand{\hadamardidentity}[1]{J_{#1}}
%
%  Hadamard inverse matrix
%  Usage: \hadamardinverse{matrix-expression}
\newcommand{\hadamardinverse}[1]{\widehat{#1}}

\newcommand{\definedTerm}[1]{\textbf{#1}}
\newcommand{\dfn}[1]{\textbf{#1}}

\newcommand{\wt}{\widetilde}
\newcommand{\ov}{\overline}
\newcommand{\inj}{\rightarrowtail}
\newcommand{\surj}{\twoheadrightarrow}
\newcommand{\harpoon}{\overset{\rightharpoonup}}

\newenvironment{amatrix}[1]{%
  \left[\begin{array}{@{}*{#1}{c}|c@{}}
}{%
  \end{array}\right]
}

\title{Linear Algebra}
\usepackage{amsmath, amsthm, amssymb,todonotes, enumerate,color,pigpen,mathtools}
\usepackage{graphicx}%authblk
\numberwithin{equation}{section}

%\theoremstyle{plain}
%\newtheorem{theorem}[equation]{Theorem}
%\newtheorem*{thm}{Theorem}
%\newtheorem{proposition}[equation]{Proposition}
%\newtheorem{lemma}[equation]{Lemma}
%\newtheorem{corollary}[equation]{Corollary}
%\newtheorem{conjecture}[equation]{Conjecture}

%\theoremstyle{definition}
%\newtheorem{definition}[equation]{Definition}
%\newtheorem{example}[equation]{Example}
%\newtheorem{exercise}[equation]{Exercise}
%\newtheorem{remark}[equation]{Remark}
%\newtheorem{observation}[equation]{Observation}
%\newtheorem{question}[equation]{Question}
\newtheorem*{ques}{Question}
%\newtheorem{construction}[equation]{Construction}
%\newtheorem{claim}[equation]{Claim}


\newcommand\wt{\widetilde}
\newcommand\ov{\overline}
\newcommand\inj{\rightarrowtail}
\newcommand\surj{\twoheadrightarrow}
\newcommand{\harpoon}{\overset{\rightharpoonup}}

\begin{document}
\begin{abstract}
Here's an abstract.
\end{abstract}
\maketitle

\part{Systems of Linear Equations}

\section{Linear Systems of equations} 

%%%%%%%%%%%%%%%%%%%%%%%%%%%%%%%%%%%%%%

\subsection{Definition} A {\it linear function in one variable} is of the form $f(x) = ax+b$ where $x$ is a variable and $a,b$ are numbers (or scalars, as they are referred to in linear algebra). It is {\it homogeneous} if $b=0$. Similarly, a {\it linear function in n variables} is one of the form
\[
f(x_1,x_2,\dots,x_n) = a_1x_1 + a_2x_2 + \dots a_nx_n + b
\]
where the $x_i$ are variables (or unknowns) and the $a_i$ are scalars. Again, the linear function is called {\it homogeneous} if the constant term $b$ is zero. Next, a {\it linear equation in n variables} is one of the form
\[
a_1x_1 + a_2x_2 + \dots a_nx_n =  b
\]
where the expression on the left is a linear homogeneous function in $n$ variables, and the term on the right is a constant. So, for example, 
\begin{equation}\label{eqn:ex1}
3x_1 - 7x_2 + 11x_3 = 14
\end{equation}

is a linear equation in three variables - $x_1, x_2$, and $x_3$. A {\it solution} to a linear equation is an assignment of values to each of the variables appearing which makes the equation hold true. Returning to the example in the equation (\ref{eqn:ex1}), we see
\[]
x_1 = 1, x_2 = 0, x_3 = 1
\]
is a solution, because when substituted into the left-hnd side it results in the value 14. Finally, a {\it system of equations} is a collection of linear equations in the same set of variables, or unknowns
\begin{align*}
a_{11}x_1 + a_{12}x_2  + {}\cdots{}  + a_{1n}x_n &=  b_1 \\
a_{21}x_1 +  a_{22}x_2 +   {}\cdots{}  + a_{2n}x_n &= b_2 \\
\vdots \\
a_{m1}x_1 + a_{m2}x_2  + {}\cdots{} + a_{mn}x_n &=  b_m 
\end{align*}

and a {\it solution} to that system is an assignment of values to the variables which make {\it each equation} hold true. Note that systems of equations, or even a single equation, need to have a solution. To illustrate, consider the equation
\[
0x_1 + 0x_2 + 0x_3 = 4
\]
Obviously, any set of values substituted into the left-hand side of this equation will produce the value zero, which is not equal to 4. But even non-zero systems might not have a solution. As an example, consider
\begin{alignat*}{3}
x_1 && - 2x_2 && =& \phantom{1}7 \\
2x_1 && - 4x_2 && =& 16 
\end{alignat*}
The left-hand side of the second equation is twice that of the first. So if we take any solution of the first equation and plug in those values on the left-hand side of the second equation, we will always get $2*7 = 14\ne 16$. Therefore, these two equations will never be simultaneously satisfied, and so the system doesn't have a solution.
\vskip,2in
A system which has at least one solution is called {\it consistent}. If it doesn't have any solutons it is called {\it inconsistent}. The main questions we need to answer are:
\vskip.1in

\begin{question} Given a system of equations, does it have one or more solutions (in other words is it consistent)?
\end{question}
\vskip.05in
\begin{question} If it is consistent, what are the solutions?
\end{question}
\vskip.3in

%%%%%%%%%%%%%%%%%%%%%%%%%%%%%%%%%%%%%%

\subsection{Finding solutions} To begin with, some terminology. A system consisting of $m$ equations in the same collection of $n$ unknowns is referred to as an $m\times n$ system, which reads as: \lq\lq m-by-n system\rq\rq; such a system is illustrated in (2.2). The number of rows $m$ and the number of columns $n$ are called the {\it dimensions} of the system. The system is
\begin{itemize}
\item {\it Underdetermined} if $m < n$ (less equations than unknowns);
\item {\it Overdetermined} if $m > n$ (more equations than unknowns);
\item {\it Balanced} if $m = n$ (same number of equatins as unknowns).
\end{itemize}
Also
\begin{itemize}
\item The {\it solution set} of a system of equations is the collection (or set) of all solutions;
\item Two $m\times n$ systems are {equivalent} if and only if they have the same solution set.
\end{itemize}

Given a system of equations, how can we find its solution set? The answer is by finding a simpler but equivalent system for which the solution set is easily read off. The method used for finding this simpler system is called {\it row reduction}, and the operations used to perform row reduction are called {\it row operations}, of which there are three types. 
\vskip.1in
\begin{description}
\item[[{\bf Type I}]] Switching two rows;
\item[[\bf{Type II}]] Multiplying a row by a non-zero scalar $\alpha$;
\item[[\bf{Type III}]]Adding a multiple of one row to another distinct row.
\end{description} 
\vskip.1in
It is quite easy to see the the first two types of row operations won't change the solution set, and it can also be shown that the third doesn't either. We'll say that two systems of the same dimensions are {\it row equivalent} if one can be gotten from the other by a sequence of row operatons. The following theorem tells us about when this happens.

\begin{theorem} Two systems of the same dimensions are row equivalent if and only if they are euqivalent.
\end{theorem}

Now to proceed in an efficient manner, it will be convenient to represent a system in terms of its essential information. This is accomplished by means of the {\it augmented coefficient matrix} or {\it ACM} corresponding to a given system. Starting with the $m\times n$ system in (2.2), the augmented coefficient matrix is given by
\[
A 
\begin{bmatrix}{4}
a_{11} &a_{12} &{}\ldots{} &a_{1n} &b_1 \\
a_{21} &a_{22} &{}\ldots{} &a_{2n} &b_2 \\
\vdotswithin{a_{m1}} &\vdotswithin{a_{m2}} &{}\ldots{} &\vdotswithin{a_{mn}} &\vdotswithin{b_m} \\
a_{m1} &a_{m2} &{}\ldots{} &a_{mn} &b_m 
\end{bmatrix}
\]
This is a matrix with $m$ rows and $(n+1)$ columns. Executing a number of row operations and then computing the ACM gives the same result as first forming the ACM and performing the row operations on that matrix. Because the ACM contains all of the information relevant for solving the system, and is less cumbersome to work with, we will always form the ACM {\it first}, and perform  row operations on that matrix. 
\vskip.2in

The simplified form in which we would like to get our matrix is referred to as {\it reduced row echelon form}. This is a term which applies to matrices in general, not just augmented coefficient matrices.

\begin{definition} A matrix $B$ of numbers is in {\it row-echelon form} if
\begin{itemize}
\item Every row of zeros lies below every non-zero row;
\item the left-most non-zero entry of a non-zero row is 1 (called a {\it leading} 1);
\item if both row $k$ and row $(k+1)$ are non-zero, the leading 1 in row $(k+1)$ appears to the right of the leading row in row $k$.
\end{itemize}
It is in {\it reduced row-echelon form} if in addition to being in row echelon form it satisfies the property
\begin{itemize}
\item In every column that contains a leading 1, that leading 1 is the only non-zero entry.
\end{itemize}
\end{definition}

\begin{example} Consider the three matrices
\[
A = \begin{bmatrix}
1 & 2 & -1 & 4\\
0 & 0 & 1 & 3\\
0 & 1 & 0 & 2
\end{bmatrix},\qquad
B = \begin{bmatrix}
1 & 2 & 3 & 4\\
0 & 1 & 5 & 7\\
0 & 0 & 0 & 1
\end{bmatrix},\qquad
C = \begin{bmatrix}
1 & 0 & 0 & 4\\
0 & 1 & 0 & 3\\
0 & 0 & 1 & 1
\end{bmatrix}
\]
\end{example}

The matrix $A$ is not in row echelon form, while $B$ is in row echelon but not reduced row echelon form, and $C$ is in reduced row echelon form (the reader should check this, and understand why for each example). The main fact we will need to know is

\begin{theorem} Every matrix of numbers is row-equivalent to one which is in reduced row echelon form.
\end{theorem}
\vskip.2in

The reduced row echelon form of a matrix is unique; for that reason we will refer to {\it the} reduced row echelon form of a matrix $A$, and write it as $rref(A)$. When $A$ is the ACM of a system of equations, $rref(A)$ tells us essentially everything we would like to know about the original system. The way it does this is summarized by the next result.

\begin{theorem} Let $A$ be the ACM of an $m\times n$ system of equations. Then the system
\begin{itemize}
\item is inconsistent precisely when $rref(A)$ contains a row of the form
\
[0\ \ 0\ \ 0\ \ 0\ \dots\ \ 0\ \ |\ \ 1];
\
\item has a unique solution precisely when each column of $rref(A)$ except the right-most column contains a leading 1;
\item has infinitely many solutions when it is i) consistent, and ii) at least one of the first n columns of $rref(A)$ does not contain a leading 1.
\end{itemize}
In the last case, the solution set is parametrized by the variables appearing in the original system which are indexed by the columns of $rref(A)$ to the left of the bar which do not contain a leading 1.
\end{theorem}

It is important to note that when $A$ is the ACM of a system, $rref(A)$ is the ACM of another system equivalent to the original one, which is the reduced row echelon form of the original system. In this reduced row echelon form, it {\it is} possible for an equation to consist of all zeros. If it does, we do not delete it from the system, because we want to maintain the original dimensions.

\begin{example} The rref of the ACM is given by
\[
rref(A) = 
\begin{bmatrix}
1 & 2 & 3 & 0 & 5\\
0 & 0 & 0 & 1 & 7 \\
0 & 0 & 0 & 0 & 1
\end{bmatrix}
\]
In this case, we would conclude that the system is {\it inconsistent}, because of the last row (to understand why a row like this makes the system inconsistent, the reader should write down the equation to which it corresponds, and see what they can say about solutions to that single equation).
\end{example}
\vskip.2in
\begin{example} The rref of the ACM is given by
\[
rref(A) = 
\begin{bmatrix}
1 & 0 & 0 & 0 & 5\\
0 & 1 & 0 & 0 & -2 \\
0 & 0 & 1 & 0 & 4\\
0 & 0 & 0 & 1 & 10
\end{bmatrix}
\]

In this case, every column to the left of the bar dividing the {\it coefficient matrix} from the augmented part contains a leading 1. Hence there is a unique solution (the reader should write down the equations corresponding to this ACM, and see how those equations actually give the exact solution).
\end{example} 
\vskip.2in

\begin{example} The rref of the ACM is given by
\[
rref(A) = 
\begin{bmatrix}
1 & 0 & 0 & 3 & 7\\
0 & 1 & 0 & 1 & 5 \\
0 & 0 & 1 & 2 & 2\\
0 & 0 & 0 & 0 & 0
\end{bmatrix}
\]
Here we have a $4\times 4$ system which has infinitely many solutions; this is seen by noting i) it is consistent, and ii) the fourth column does not contains a leading 1 (the reader should write down the four equations corresponding to this matrix and then rewrite each equation so that they express the original four variables $x_1,x_2,x_3,x_4$ as linear functions in $x_4$, which is the independent parameter in this case. Since we only have one parameter, the result is a {\it one-parameter family of solutions}).
\end{example}
\vskip.2in

In the process of putting the ACM in reduced row echelon form, it is often desirable to keep track of the precise row operations being used. For this, some notation is useful. The following table summarizes the notation we have used in class.
\vskip.2in

\begin{center}
    \begin{tabular}{|c|c|c|}\hline\hline
    \phantom{x} & \phantom{x} & \phantom{x}\\
    \Large{Type} &\Large{What it does} &\Large{Indicated by}\\ \hline
    \phantom{x} & {\large Switches} & \phantom{\Huge X}\\
    \large{Type I} &{\large $i^{th}$ and $j^{th}$} & {\large $R_i\leftrightarrow R_j$}\\ 
    \phantom{x} & {\large rows} & \phantom{x}\\ \hline
    \phantom{x} &{\large Multiplies} & \phantom{\Huge X}\\
    \large{Type II} &{\large $i^{th}$ row} & {\large $r\cdot R_i$}\\
    \phantom{x} & {\large by $r\ne 0$} & \phantom{x}\\ \hline
    \phantom{x} &{\large Adds} & \phantom{\Huge X}\\
    \large{Type III} &{\large $a$ times the $i^{th}$ row} & {\large $a\cdot R_i$ added to $R_j$}\\
    \phantom{x} & {\large to the $j^{th}$ row} & \phantom{x}\\\hline\hline
    \end{tabular}
    \end{center}
\vskip.2in

At this point, we see that the reduced row echelon form of the ACM allows us to solve the system. However, we have not discussed how the transition to that form is accomplished. The following algorithm describes that process (this appears as steps 1 - 6 on p. 20 of the text).
\vskip.2in
\begin{description}
\item[{\bf Step 1}] Determine the left-most column containing a non-zero entry (it exists if the matrix is non-zero).
\item[{\bf Step 2}] If needed, perform a type I operation so that the first non-zero column has a non-zero entry in the first row.
\item[{\bf Step 3}] If needed, perform a type II operation to make that first non-zero entry 1 (the leading 1 in the first row).
\item[{\bf Step 4}] Perform type III operations to make the entries below this leading 1 equal to 0.
\item[{\bf Step 5}] Repeat the previous four steps on the submatrix consisting of all except the first row, until reaching the end of the rows.
\item[{\bf Step 6}] For each row containing a leading 1, proceed upward using type III operations to  make zero any entry appearing above a leading 1.
\end{description}
\vskip.2in
To summarize,

\begin{theorem} Every system of equations is uniquely represented by its associated augmented coefficient matrix (ACM), and every ACM results from a unique system of equations, up to a labeling of the indeterminates. The solution set to the system can be determined by i) putting the ACM in reduced row echelon form (rref), and ii) reading off the solution(s) from the resulting matrix. Moreover, the computation of rref(ACM) can be performed in a systematic fashion, by following the algorithmic procedure listed above.
\end{theorem}

%\activity{systemsOfLinearEqutaions/solvingSystemsOfLinearEquations/notationForLinearEquations.tex}
%\activity{systemsOfLinearEqutaions/solvingSystemsOfLinearEquations/possibilitiesForSolutionSets.tex}
%\activity{systemsOfLinearEqutaions/solvingSystemsOfLinearEquations/nonlinear.tex}
%\activity{systemsOfLinearEqutaions/solvingSystemsOfLinearEquations/equivalentSystems.tex}

\end{document}
