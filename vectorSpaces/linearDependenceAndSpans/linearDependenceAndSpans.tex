\documentclass{ximera}

% These macros are automatically generated from the "macros"
% XML element.  Make permanent edits there.
%
% History
%   2004/01/01  Initiated for FCLA, evolved from there
%   2006/09/17  Converted  _, ^  to \sb, \sp for TeX4ht
%   2014/02/01  Updated for MathBook XML projects
%               Obsolete in FCLA: \codeindent, \computerfont, \define
%               Change: MathJax wants \lt, so replaced by \lteval
%   2014/02/22  New: \orderof, \reals, \per
%   2015/08/16  Incorporated into MathBook XML version of FCLA
%
%%%%%%%%%%%%%%%%%%%%%
%
%     Conveniences
%
%%%%%%%%%%%%%%%%%%%%%
%
%  Order of (asymptotically limit of fraction is 1)
%  Usage: \orderof{some function}
%
\newcommand{\orderof}[1]{\sim #1}
%
%  Integers
%  Usage:  \Z
\newcommand{\Z}{\mathbb{Z}}
%
%  Real numbers, as set of scalars
%  Usage:  \reals
\newcommand{\reals}{\mathbb{R}}
%
%  n-space over real field
%  Usage: \complex{integer-dimension}
\newcommand{\real}[1]{\mathbb{R}^{#1}}
%
%  Complex numbers, as set of scalars
%  Usage:  \complexes
\newcommand{\complexes}{\mathbb{C}}
%
%  n-space over complex field
%  Usage: \complex{integer-dimension}
\newcommand{\complex}[1]{\mathbb{C}^{#1}}
\newcommand{\CC}{\mathbb{C}}
%
%  Complex conjugation (scalar, vector, matrix)
%  Usage: \conjugate{object}
\newcommand{\conjugate}[1]{\overline{#1}}
%
%  Complex number modulus
%  Usage: \modulus{a+bi}
%  Presumes math mode
\newcommand{\modulus}[1]{\left\lvert#1\right\rvert}
%
%  Zero vector
%  Usage: \zerovector
\newcommand{\zerovector}{\vect{0}}
%
%  Zero matrix
%  Usage: \zeromatrix, use a subscript when size is important
\newcommand{\zeromatrix}{\mathcal{O}}
%
%  Inner product (brackets, not quadratic form)
%  Usage: \innerproduct{a-vector}{a-vector}
\newcommand{\innerproduct}[2]{\left\langle#1,\,#2\right\rangle}
%
%  Norm of a vector
%  Usage: \norm{a-vector}
\newcommand{\norm}[1]{\left\lVert#1\right\rVert}
%
%  Dimension
%  Usage: \dimension{vector-space-letter}
\newcommand{\dimension}[1]{\dim\left(#1\right)}
%
%  Nullity
%  Usage: \nullity{matrix-or-lintrans-letter}
\newcommand{\nullity}[1]{n\left(#1\right)}
%
%  Rank
%  Usage: \rank{matrix-or-lintrans-letter}
\newcommand{\rank}[1]{r\left(#1\right)}
%
%  Direct sum
%  Usage: \ds between a couple of subspaces
%
\newcommand{\ds}{\oplus}
%
%  Determinant of a matrix (functional)
%  Usage: \detname{A}
\newcommand{\detname}[1]{\det\left(#1\right)}
%
%  Determinant of a matrix (vertical bars)
%  Usage: \detbars{A}
\newcommand{\detbars}[1]{\left\lvert#1\right\rvert}
%
%  Trace of a Matrix
%  Usage: \trace{matrix name}
\newcommand{\trace}[1]{t\left(#1\right)}
%
%  Square Root of a Matrix
%  Usage: \sr{a-matrix}
\newcommand{\sr}[1]{#1^{1/2}}
%
%%%%%%%%%%%%%%%%%%%%%
%
%     Subspace Constructions
%
%%%%%%%%%%%%%%%%%%%%%
%
%  Span of a set of vectors
%  \span and \sp are used by TeX for other things
%  Usage: \spn{set-of-vectors}
\newcommand{\spn}[1]{\left\langle#1\right\rangle}
%
%  Null space of a matrix
%  Usage:  \nsp{A}
\newcommand{\nsp}[1]{\mathcal{N}\!\left(#1\right)}
%
%  Column space of a matrix
%  Usage:  \csp{A}
\newcommand{\csp}[1]{\mathcal{C}\!\left(#1\right)}
%
%  Row space of a matrix
%  Usage:  \rsp{A}
\newcommand{\rsp}[1]{\mathcal{R}\!\left(#1\right)}
%
%  Left null space of a matrix
%  Usage:  \lns{A}
\newcommand{\lns}[1]{\mathcal{L}\!\left(#1\right)}
%
%  Orthogonal complement of a vector space
%  Avoiding TeX's \perp
%  Usage:  \per{A}
\newcommand{\per}[1]{#1^\perp}
%
%%%%%%%%%%%%%%%%%%%%%
%
%     Systems of Equations
%
%%%%%%%%%%%%%%%%%%%%%
%
%  In-line form of an augmented matrix for a system of equations
%  Usage: \augmented{coefficient-matrix}{constant-vector}
\newcommand{\augmented}[2]{\left\lbrack\left.#1\,\right\rvert\,#2\right\rbrack}
%
%  Notation for a linear system before introducing matrix multiplication
%  Usage: \linearsystem{coefficient-matrix}{constant-vector}
\newcommand{\linearsystem}[2]{\mathcal{LS}\!\left(#1,\,#2\right)}
%
%  Notation for a homogenous system before introducing matrix multiplication
%  Usage: \homosystem{coefficient-matrix}
\newcommand{\homosystem}[1]{\linearsystem{#1}{\zerovector}}
%
%%%%%%%%%%%%%%%%%%%%%
%
%     Row Operations, Echelon Form
%
%%%%%%%%%%%%%%%%%%%%%
%
% Row operations on matrices
%
% Three commands to shorten up descriptions of gaussian elimination
%
% Usage: \rowopswap{row-i}{row-j}
% Usage: \rowopmult{scalar}{row-i}
% Usage: \rowopadd{scalar}{row-multiplied}{row-added-to}
\newcommand{\rowopswap}[2]{R_{#1}\leftrightarrow R_{#2}}
\newcommand{\rowopmult}[2]{#1R_{#2}}
\newcommand{\rowopadd}[3]{#1R_{#2}+R_{#3}}
%
% Mark leading 1's in echelon form with fbox
% Usage: \leading{a-1-usually}
\newcommand{\leading}[1]{\fbox{#1}}
%
%  Row-reduce arrow
%  Usage:  \rref inbetween a matrix and its reduced row-echelon form
\newcommand{\rref}{\xrightarrow{\text{RREF}}}
%
%  Elementary Matrices
%  Usage: \elemswap{subscript}{subscript}
%  Usage: \elemmult{scalar}{subscript}
%  Usage: \elemadd{scalar}{subscript-mult}{subscript-target}
%
\newcommand{\elemswap}[2]{E_{#1,#2}}
\newcommand{\elemmult}[2]{E_{#2}\left(#1\right)}
\newcommand{\elemadd}[3]{E_{#2,#3}\left(#1\right)}
%
%%%%%%%%%%%%%%%%%%%%%
%
%     2-D Constructions (Lists, Vectors, Matrices)
%
%%%%%%%%%%%%%%%%%%%%%
%
%  A list of scalars of generic length
%  Usage:  \scalarlist{scalar letter}{terminal subscript}
\newcommand{\scalarlist}[2]{{#1}_{1},\,{#1}_{2},\,{#1}_{3},\,\ldots,\,{#1}_{#2}}
%
%  Vector styling, bold (or use wiggles, arrows, whatever)
%  Subscripts go outside this construction
%  Usage: \vect{a symbol to use as a vector}
%  Have to already be in math mode
%
\newcommand{\vect}[1]{\mathbf{#1}}
%
%  A column vector
%  Usage: \colvector{list-delimited-by-\\}
%
\newcommand{\colvector}[1]{\begin{bmatrix}#1\end{bmatrix}}
%
%  A generic vector with components
%  Usage: \vectorcomponents{component-letter}{final-subscript}
\newcommand{\vectorcomponents}[2]{\colvector{#1_{1}\\#1_{2}\\#1_{3}\\\vdots\\#1_{#2}}}
%
%  A list of vectors of generic length
%  Usage:  \vectorlist{vector letter}{terminal subscript}
\newcommand{\vectorlist}[2]{\vect{#1}_{1},\,\vect{#1}_{2},\,\vect{#1}_{3},\,\ldots,\,\vect{#1}_{#2}}
%
%  Vector entries, entry i of vector v
%  (vector-expession still needs \vect, etc.)
%  Usage:  \vectorentry{vector-expression}{single-subscript}
\newcommand{\vectorentry}[2]{\left\lbrack#1\right\rbrack_{#2}}
%
%  Matrix entries, entry i,j of matrix A
%  Usage:  \matrixentry{matrix-expression}{paired-subscripts}
%
\newcommand{\matrixentry}[2]{\left\lbrack#1\right\rbrack_{#2}}
%
%  A generic linear combination
%  Usage:  \lincombo{scalar letter}{vector letter}{terminal subscript}
\newcommand{\lincombo}[3]{#1_{1}\vect{#2}_{1}+#1_{2}\vect{#2}_{2}+#1_{3}\vect{#2}_{3}+\cdots +#1_{#3}\vect{#2}_{#3}}
%
%  Matrix, column by column, as vectors
%  Usage:  \matrixcolumns{matrix letter}{terminal subscript}
\newcommand{\matrixcolumns}[2]{\left\lbrack\vect{#1}_{1}|\vect{#1}_{2}|\vect{#1}_{3}|\ldots|\vect{#1}_{#2}\right\rbrack}
%
%%%%%%%%%%%%%%%%%%%%%
%
%     Special Matrices
%
%%%%%%%%%%%%%%%%%%%%%
%
%  Transpose of a matrix
%  Usage:  \transpose{A}
\newcommand{\transpose}[1]{#1^{t}}
%
%  Inverse of a matrix
%  Usage:  \inverse{A}
\newcommand{\inverse}[1]{#1^{-1}}
%
%  Submatrix (for minors, determinants)
%  Usage: \submatrix{matrix-name}{delete-row}{delete-col}
\newcommand{\submatrix}[3]{#1\left(#2|#3\right)}
%
%  Adjoint of a matrix (twice)
%  This macro is a convenience to call \transpose and \conjugate properly
%  It shouldn't need to be modified (or mathematical meanings will change)
%  Usage:  \adj{A}
\newcommand{\adj}[1]{\transpose{\left(\conjugate{#1}\right)}}
%
%  This macro controls the symbol used for the adjoint
%  It can be edited to taste
%  Usage:  \adjoint{A}
\newcommand{\adjoint}[1]{#1^\ast}
%
%%%%%%%%%%%%%%%%%%%%%
%
%     Sets
%
%%%%%%%%%%%%%%%%%%%%%
%
%  A convenience for simple sets
%  Usage:  \set{list of element}
\newcommand{\set}[1]{\left\{#1\right\}}
%
%  Sets with vertical bar, "such that", sized for objects, not condition
%  Usage:  \setparts{objects}{condition}
%
%%\newcommand{\setparts}[2]{\left\{ #1\mid#2\right\}}
%%\newcommand{\setparts}[2]{\left\{\left. #1\right\rvert#2\right\}}
\newcommand{\setparts}[2]{\left\lbrace#1\,\middle|\,#2\right\rbrace}
%
%  Set Cardinality
%  Usage:  \card{a-set-letter}
\newcommand{\card}[1]{\left\lvert#1\right\rvert}
%
%  Set Union
%  Use \cup
%
%  Set Intersection
%  Use \cap
%
%  Set Complement
%  Usage:  \setcomplement{a-set-letter}
\newcommand{\setcomplement}[1]{\overline{#1}}
%
%%%%%%%%%%%%%%%%%%%%%
%
%     Eigenvalues and Eigenspaces
%
%%%%%%%%%%%%%%%%%%%%%
%
%  Characteristic polynomial
%  Usage: \charpoly{matrix-letter}{variable-letter}
\newcommand{\charpoly}[2]{p_{#1}\left(#2\right)}
%
%  Eigenspace
%  Usage: \eigenspace{matrix-letter}{eigenvalue-letter}
\newcommand{\eigenspace}[2]{\mathcal{E}_{#1}\left(#2\right)}
%
%  2013/10/03 Including ampersands is problematic here, 
%  think about fixes later
%  2014/02/22 Limited testing, seems &amp; is fine for HTML and LaTeX
%  2016-07-20 only employed in Archetypes, MBX has gather/align override
%  Eigensystem (presumes wrapped in an mrow within md)
%  Usage: \eigensystem{matrixletter}{eigenvalue}{list of basis vectors}
\newcommand{\eigensystem}[3]{\lambda&amp;=#2&amp;\eigenspace{#1}{#2}&amp;=\spn{\set{#3}}} 
%
%  Generalized Eigenspace
%  Usage: \geneigenspace{lin-trans-letter}{eigenvalue-letter}
\newcommand{\geneigenspace}[2]{\mathcal{G}_{#1}\left(#2\right)}
%
%  Algebraic multiplicty
%  Usage: \algmult{matrix-letter}{eigenvalue-letter}
\newcommand{\algmult}[2]{\alpha_{#1}\left(#2\right)}
%
%  Geometric multiplicty
%  Usage: \geomult{matrix-letter}{eigenvalue-letter}
\newcommand{\geomult}[2]{\gamma_{#1}\left(#2\right)}
%
%  Index (of eigenvalue)
%  Usage: \indx{matrix-letter}{eigenvalue-letter}
\newcommand{\indx}[2]{\iota_{#1}\left(#2\right)}
%
%%%%%%%%%%%%%%%%%%%%%
%
%     Linear Transformations
%
%%%%%%%%%%%%%%%%%%%%%
%
%  MathJax defines \lt to ease XML confusion
%
%  Linear transformation definition
%  Usage: \ltdefn{name-letter}{domain}{range}
\newcommand{\ltdefn}[3]{#1\colon #2\rightarrow#3}
%
%  Linear transformation evaluation
%  Usage: \lteval{name-letter}{input}
%  Replaces old \lt desired by MathJax
\newcommand{\lteval}[2]{#1\left(#2\right)}
%
% Linear transformation inverse
%  Usage: \ltinverse{name-letter}
\newcommand{\ltinverse}[1]{#1^{-1}}
%
%  Linear transformation restriction
%  Usage: \restrict{name-letter}{subspace-letter}
\newcommand{\restrict}[2]{{#1}|_{#2}}
%
%  Linear transformation preimage
%  Usage: \preimage{name-letter}{codomain-element}
\newcommand{\preimage}[2]{#1^{-1}\left(#2\right)}
%
%  Range of a linear transformation
%  TeX uses \range for something else
%  Usage:  \rng{T}
\newcommand{\rng}[1]{\mathcal{R}\!\left(#1\right)}
%
%  Kernel of a linear transformation
%  TeX uses \ker to do something different
%  Usage:  \krn{T}
\newcommand{\krn}[1]{\mathcal{K}\!\left(#1\right)}
%
%  Linear transformation composition
%  Usage: \compose{function-name}{function-name}
\newcommand{\compose}[2]{{#1}\circ{#2}}
%
%  Vector space of linear transformations
%  Usage: \vslt{domains}{codomains}
%  Presumes math mode
\newcommand{\vslt}[2]{\mathcal{LT}\left(#1,\,#2\right)}
%
%%%%%%%%%%%%%%%%%%%%%
%
%     Vector and Matrix Representations
%
%%%%%%%%%%%%%%%%%%%%%
%
%  Isomorphism symbol
%  Usage: \isomorphic
\newcommand{\isomorphic}{\cong}
%
%  Similarity
%  Usage: \similar{inner-matrix}{outer-invertible-matrix}
%  Rearranging this will not "fix" all desired changes throughout
%
\newcommand{\similar}[2]{\inverse{#2}#1#2}
%
%  Vector representation function name
%  Usage: \vectrepname{basis-letter}
\newcommand{\vectrepname}[1]{\rho_{#1}}
%
%  Vector representation output
%  Usage: \vectrep{basis-letter}{input}
\newcommand{\vectrep}[2]{\lteval{\vectrepname{#1}}{#2}}
%
%  Vector representation inverse function name
%  (Added later, not used consistently in FCLA)
%  Usage: \vectrepinvname{basis-letter}
\newcommand{\vectrepinvname}[1]{\ltinverse{\vectrepname{#1}}}
%
%  Vector representation inverse output
%  Usage: \vectrepinv{basis-letter}{input}
\newcommand{\vectrepinv}[2]{\lteval{\ltinverse{\vectrepname{#1}}}{#2}}
%
%  Matrix representation
%  Usage: \matrixrep{transformation-letter}{domain-basis-letter}{codomain-basis-letter}
\newcommand{\matrixrep}[3]{M^{#1}_{#2,#3}}
%
%  Matrix representation column-by-colum
%  2016-07-20 only employed once?
%  Usage: \matrixrepcolumns{transformation-letter}{codomain-basis-letter}{codomain-basis-vector-letter}{final-index}
\newcommand{\matrixrepcolumns}[4]{\left\lbrack \left.\vectrep{#2}{\lteval{#1}{\vect{#3}_{1}}}\right|\left.\vectrep{#2}{\lteval{#1}{\vect{#3}_{2}}}\right|\left.\vectrep{#2}{\lteval{#1}{\vect{#3}_{3}}}\right|\ldots\left|\vectrep{#2}{\lteval{#1}{\vect{#3}_{#4}}}\right.\right\rbrack}
%
%  Change of basis matrix
%  Usage: \cbm{domain-basis-letter}{codomain-basis-letter}
\newcommand{\cbm}[2]{C_{#1,#2}}
%
%%%%%%%%%%%%%%%%%%%%%
%
%     Canonical Forms
%
%%%%%%%%%%%%%%%%%%%%%
%
%  Jordan blocks
%  Usage: \jordan{size}{diagonal-element}
\newcommand{\jordan}[2]{J_{#1}\left(#2\right)}
%
%%%%%%%%%%%%%%%%%%%%%
%
%     Hadamard Matrices
%     Contributed by Elizabeth Million
%
%%%%%%%%%%%%%%%%%%%%%
%
%  Hadamard Product
%  Usage: \hadamard{a-matrix}{a-matrix}
\newcommand{\hadamard}[2]{#1\circ #2}
%
%  Hadamard identity matrix
%  Usage: \hadamardidentity{paired-subscripts-size-of-matrix}
\newcommand{\hadamardidentity}[1]{J_{#1}}
%
%  Hadamard inverse matrix
%  Usage: \hadamardinverse{matrix-expression}
\newcommand{\hadamardinverse}[1]{\widehat{#1}}

\newcommand{\definedTerm}[1]{\textbf{#1}}
\newcommand{\dfn}[1]{\textbf{#1}}

\newcommand{\wt}{\widetilde}
\newcommand{\ov}{\overline}
\newcommand{\inj}{\rightarrowtail}
\newcommand{\surj}{\twoheadrightarrow}
\newcommand{\harpoon}{\overset{\rightharpoonup}}

\newenvironment{amatrix}[1]{%
  \left[\begin{array}{@{}*{#1}{c}|c@{}}
}{%
  \end{array}\right]
}


\title{Linear Dependence and Spans}

\begin{document}
\begin{abstract}
  In a linearly dependent set, there is always a vector that can be written as a linear combination of the others. 
\end{abstract}
\maketitle

In any linearly dependent set there is always one vector that can be
written as a linear combination of the others.  This is the substance
of the upcoming \ref{theorem:DLDS}.  Perhaps this will explain the use
of the word ``dependent.''  In a linearly dependent set, at least one
vector ``depends'' on the others (via a linear combination).

Indeed, because \ref{theorem:DLDS} is an equivalence
(\ref{technique:E}) some authors use this condition as a definition
(\ref{technique:D}) of linear dependence.  Then linear independence is
defined as the logical opposite of linear dependence.  Of course, we
have \textit{chosen} to take \ref{definition:LICV} as our definition,
and then follow with \ref{theorem:DLDS} as a theorem.

If we use a linearly dependent set to construct a span, then we can
\textit{always} create the same infinite set with a starting set that
is one vector smaller in size.  However, this will not be possible if
we build a span from a linearly independent set.  So in a certain
sense, using a linearly independent set to formulate a span is the
best possible way---there are not any extra vectors being used to
build up all the necessary linear combinations.

\begin{theorem}[Dependency in Linearly Dependent Sets]
  \label{theorem:DLDS} Suppose that $S=\set{\vectorlist{u}{n}}$ is a
  set of vectors.  Then $S$ is a linearly dependent set if and only if
  there is an index $t$, $1\leq t\leq n$ such that $\vect{u_t}$ is a
  linear combination of the vectors
  $\vect{u}_1,\,\vect{u}_2,\,\vect{u}_3,\,\ldots,\,\vect{u}_{t-1},\,\vect{u}_{t+1},\,\ldots,\,\vect{u}_n$.

  \begin{proof}
    ($\Rightarrow$) Suppose that $S$ is linearly dependent, so there
    exists a nontrivial relation of linear dependence by
    \ref{definition:LICV}.  That is, there are scalars, $\alpha_i$,
    $1\leq i\leq n$, which are not all zero, such that
    \[
      \lincombo{\alpha}{u}{n}=\zerovector.
    \]
    Since the $\alpha_i$ cannot all be zero, choose one, say $\alpha_t$, that is nonzero.  Then,
    \begin{align*}
      \vect{u}_t
      &=\frac{-1}{\alpha_t}\left(-\alpha_t\vect{u}_t\right)&&\ref{property:MICN}\\
      &=
        \frac{-1}{\alpha_t}\left(
        \alpha_1\vect{u}_1+
        \cdots+
        \alpha_{t-1}\vect{u}_{t-1}+
        \alpha_{t+1}\vect{u}_{t+1}+
        \cdots+
        \alpha_n\vect{u}_n
        \right)&&\ref{theorem:VSPCV}\\
      &=
        \frac{-\alpha_1}{\alpha_t}\vect{u}_1+
        \cdots+
        \frac{-\alpha_{t-1}}{\alpha_t}\vect{u}_{t-1}+
        \frac{-\alpha_{t+1}}{\alpha_t}\vect{u}_{t+1}+
        \cdots+
        \frac{-\alpha_n}{\alpha_t}\vect{u}_n
                                                           &&\ref{theorem:VSPCV}
    \end{align*}
    Since the values of $\frac{\alpha_i}{\alpha_t}$ are again scalars,
    we have expressed $\vect{u}_t$ as a linear combination of the
    other elements of $S$.
    
    ($\Leftarrow$) Assume that the vector $\vect{u}_t$ is a linear
    combination of the other vectors in $S$.  Write this linear
    combination, denoting the relevant scalars as $\beta_1$, $\beta_2$,
    \ldots , $\beta_{t-1}$, $\beta_{t+1}$, \ldots , $\beta_n$, as
    \begin{align*}
      \vect{u_t}
      &=
        \beta_1\vect{u}_1+
        \beta_2\vect{u}_2+
        \cdots+
        \beta_{t-1}\vect{u}_{t-1}+
        \beta_{t+1}\vect{u}_{t+1}+
        \cdots+
        \beta_n\vect{u}_n
    \end{align*}
    
    Then we have
    \begin{align*}
      \beta_1\vect{u}_1
      &+\cdots+
        \beta_{t-1}\vect{u}_{t-1}+
        (-1)\vect{u}_t+
        \beta_{t+1}\vect{u}_{t+1}+
        \cdots+
        \beta_n\vect{u}_n\\
      &=\vect{u}_t+(-1)\vect{u}_t&&\ref{theorem:VSPCV}\\
      &=\left(1+\left(-1\right)\right)\vect{u}_t&&\ref{property:DSAC}\\
      &=0\vect{u}_t&&\ref{property:AICN}\\
      &=\zerovector&&\ref{definition:CVSM}
    \end{align*}
    
    So the scalars
    $\beta_1,\,\beta_2,\,\beta_3,\,\ldots,\,\beta_{t-1},\,\beta_t=-1,\beta_{t+1},\,\,\ldots,\,\beta_n$
    provide a
    \wordChoice{\choice[correct]{nontrivial}\choice{trivial}} linear
    combination of the vectors in $S$, thus establishing that $S$ is a
    linearly \wordChoice{\choice{independent}\choice[correct]{dependent}} set.
    
  \end{proof}
\end{theorem}

This theorem can be used, sometimes repeatedly, to whittle down the
size of a set of vectors used in a span construction.  We have seen
some of this already, but in the next example we will detail some of
the subtleties.

\begin{example}[Reducing a span in $\real{5}$]
  Consider the set of $n=4$ vectors from $\real{5}$,
  \[
    R=\set{\vect{v}_1,\,\vect{v}_2,\,\vect{v}_3,\,\vect{v}_4}
    =
    \set{
      \colvector{1\\2\\-1\\3\\2},\,
      \colvector{2\\1\\3\\1\\2},\,
      \colvector{0\\-7\\6\\-11\\-2},\,
      \colvector{4\\1\\2\\1\\6}
    }
  \]
  and define $V=\spn{R}$.
  
  To employ \ref{theorem:LIVHS}, we form a $5\times 4$ matrix, $D$, and
  row-reduce to understand solutions to the homogeneous system
  $\homosystem{D}$,
  \[
    D=
    \begin{bmatrix}
      1&2&0&4\\
      2&1&-7&1\\
      -1&3&6&2\\
      3&1&-11&1\\
      2&2&-2&6
    \end{bmatrix}
    \rref
    \begin{bmatrix}
      \leading{1}&0&0&4\\
      0&\leading{1}&0&0\\
      0&0&\leading{1}&1\\
      0&0&0&0\\
      0&0&0&0
    \end{bmatrix}
  \]

  We can find infinitely many solutions to this system, most of them
  nontrivial, and we choose any one we like to build a relation of
  linear dependence on $R$.  Let us begin with $x_4=1$, to find the
  solution
  \[
    \colvector{-4\\0\\-1\\\answer{1}}
  \]

  So we can write the relation of linear dependence,
  \[
    (-4)\vect{v}_1+0\vect{v}_2+(-1)\vect{v}_3+\answer{1}\vect{v}_4=\zerovector
  \]

  \ref{theorem:DLDS} guarantees that we can solve this relation of
  linear dependence for \textit{some} vector in $R$, but the choice of
  which one is up to us.  Notice however that $\vect{v}_2$ has a zero
  coefficient.  In this case, we cannot choose to solve for
  $\vect{v}_2$.  Maybe some other relation of linear dependence would
  produce a nonzero coefficient for $\vect{v}_2$ if we just had to
  solve for this vector.  Unfortunately, this example has been
  engineered to \textit{always} produce a zero coefficient here, as
  you can see from solving the homogeneous system.  Every solution has
  $x_2=0$!

  OK, if we are convinced that we cannot solve for $\vect{v}_2$, let
  us instead solve for $\vect{v}_3$,
  \[
    \vect{v}_3=(-4)\vect{v}_1+0\vect{v}_2+1\vect{v}_4=(-4)\vect{v}_1+1\vect{v}_4
  \]
  
  We now claim that this particular equation will allow us to write
  \[
    V=\spn{R}=
    \spn{\set{\vect{v}_1,\,\vect{v}_2,\,\vect{v}_3,\,\vect{v}_4}}=
    \spn{\set{\vect{v}_1,\,\vect{v}_2,\,\vect{v}_4}}
  \]
  in essence declaring $\vect{v}_3$ as surplus for the task of
  building $V$ as a span.  This claim is an equality of two sets, so
  we will use \ref{definition:SE} to establish it carefully.  Let
  $R^\prime=\set{\vect{v}_1,\,\vect{v}_2,\,\vect{v}_4}$ and
  $V^\prime=\spn{R^\prime}$.  We want to show that $V=V^\prime$.

  First show that $V^\prime\subseteq V$.  Since every vector of
  $R^\prime$ is in $R$, any vector we can construct in $V^\prime$ as a
  linear combination of vectors from $R^\prime$ can also be
  constructed as a vector in $V$ by the same linear combination of the
  same vectors in $R$.  That was easy, now turn it around.
  
  Next show that $V\subseteq V^\prime$.  Choose any $\vect{v}$ from $V$.  So there are scalars $\alpha_1,\,\alpha_2,\,\alpha_3,\,\alpha_4$ such that
  \begin{align*}
    \vect{v}&=
              \alpha_1\vect{v}_1+\alpha_2\vect{v}_2+\alpha_3\vect{v}_3+\alpha_4\vect{v}_4\\
            &=\alpha_1\vect{v}_1+\alpha_2\vect{v}_2+
              \alpha_3\left((-4)\vect{v}_1+1\vect{v}_4\right)+
              \alpha_4\vect{v}_4\\
            &=\alpha_1\vect{v}_1+\alpha_2\vect{v}_2+
              \left((-4\alpha_3)\vect{v}_1+\alpha_3\vect{v}_4\right)+
              \alpha_4\vect{v}_4\\
            &=\left(\alpha_1-4\alpha_3\right)\vect{v}_1+
              \alpha_2\vect{v}_2+
              \left(\alpha_3+\alpha_4\right)\vect{v}_4.
  \end{align*}
  
  This equation says that $\vect{v}$ can then be written as a linear
  combination of the vectors in $R^\prime$ and hence qualifies for
  membership in $V^\prime$.  So $V\subseteq V^\prime$ and we have
  established that $V=V^\prime$.

  If $R^\prime$ was also linearly dependent (it is not), we could
  reduce the set even further.  Notice that we could have chosen to
  eliminate any one of $\vect{v}_1$, $\vect{v}_3$ or $\vect{v}_4$, but
  somehow $\vect{v}_2$ is essential to the creation of $V$ since it
  cannot be replaced by any linear combination of $\vect{v}_1$,
  $\vect{v}_3$ or $\vect{v}_4$.

\end{example}


\end{document}