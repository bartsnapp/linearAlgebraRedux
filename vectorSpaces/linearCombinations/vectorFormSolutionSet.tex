\documentclass{ximera}

% These macros are automatically generated from the "macros"
% XML element.  Make permanent edits there.
%
% History
%   2004/01/01  Initiated for FCLA, evolved from there
%   2006/09/17  Converted  _, ^  to \sb, \sp for TeX4ht
%   2014/02/01  Updated for MathBook XML projects
%               Obsolete in FCLA: \codeindent, \computerfont, \define
%               Change: MathJax wants \lt, so replaced by \lteval
%   2014/02/22  New: \orderof, \reals, \per
%   2015/08/16  Incorporated into MathBook XML version of FCLA
%
%%%%%%%%%%%%%%%%%%%%%
%
%     Conveniences
%
%%%%%%%%%%%%%%%%%%%%%
%
%  Order of (asymptotically limit of fraction is 1)
%  Usage: \orderof{some function}
%
\newcommand{\orderof}[1]{\sim #1}
%
%  Integers
%  Usage:  \Z
\newcommand{\Z}{\mathbb{Z}}
%
%  Real numbers, as set of scalars
%  Usage:  \reals
\newcommand{\reals}{\mathbb{R}}
%
%  n-space over real field
%  Usage: \complex{integer-dimension}
\newcommand{\real}[1]{\mathbb{R}^{#1}}
%
%  Complex numbers, as set of scalars
%  Usage:  \complexes
\newcommand{\complexes}{\mathbb{C}}
%
%  n-space over complex field
%  Usage: \complex{integer-dimension}
\newcommand{\complex}[1]{\mathbb{C}^{#1}}
\newcommand{\CC}{\mathbb{C}}
%
%  Complex conjugation (scalar, vector, matrix)
%  Usage: \conjugate{object}
\newcommand{\conjugate}[1]{\overline{#1}}
%
%  Complex number modulus
%  Usage: \modulus{a+bi}
%  Presumes math mode
\newcommand{\modulus}[1]{\left\lvert#1\right\rvert}
%
%  Zero vector
%  Usage: \zerovector
\newcommand{\zerovector}{\vect{0}}
%
%  Zero matrix
%  Usage: \zeromatrix, use a subscript when size is important
\newcommand{\zeromatrix}{\mathcal{O}}
%
%  Inner product (brackets, not quadratic form)
%  Usage: \innerproduct{a-vector}{a-vector}
\newcommand{\innerproduct}[2]{\left\langle#1,\,#2\right\rangle}
%
%  Norm of a vector
%  Usage: \norm{a-vector}
\newcommand{\norm}[1]{\left\lVert#1\right\rVert}
%
%  Dimension
%  Usage: \dimension{vector-space-letter}
\newcommand{\dimension}[1]{\dim\left(#1\right)}
%
%  Nullity
%  Usage: \nullity{matrix-or-lintrans-letter}
\newcommand{\nullity}[1]{n\left(#1\right)}
%
%  Rank
%  Usage: \rank{matrix-or-lintrans-letter}
\newcommand{\rank}[1]{r\left(#1\right)}
%
%  Direct sum
%  Usage: \ds between a couple of subspaces
%
\newcommand{\ds}{\oplus}
%
%  Determinant of a matrix (functional)
%  Usage: \detname{A}
\newcommand{\detname}[1]{\det\left(#1\right)}
%
%  Determinant of a matrix (vertical bars)
%  Usage: \detbars{A}
\newcommand{\detbars}[1]{\left\lvert#1\right\rvert}
%
%  Trace of a Matrix
%  Usage: \trace{matrix name}
\newcommand{\trace}[1]{t\left(#1\right)}
%
%  Square Root of a Matrix
%  Usage: \sr{a-matrix}
\newcommand{\sr}[1]{#1^{1/2}}
%
%%%%%%%%%%%%%%%%%%%%%
%
%     Subspace Constructions
%
%%%%%%%%%%%%%%%%%%%%%
%
%  Span of a set of vectors
%  \span and \sp are used by TeX for other things
%  Usage: \spn{set-of-vectors}
\newcommand{\spn}[1]{\left\langle#1\right\rangle}
%
%  Null space of a matrix
%  Usage:  \nsp{A}
\newcommand{\nsp}[1]{\mathcal{N}\!\left(#1\right)}
%
%  Column space of a matrix
%  Usage:  \csp{A}
\newcommand{\csp}[1]{\mathcal{C}\!\left(#1\right)}
%
%  Row space of a matrix
%  Usage:  \rsp{A}
\newcommand{\rsp}[1]{\mathcal{R}\!\left(#1\right)}
%
%  Left null space of a matrix
%  Usage:  \lns{A}
\newcommand{\lns}[1]{\mathcal{L}\!\left(#1\right)}
%
%  Orthogonal complement of a vector space
%  Avoiding TeX's \perp
%  Usage:  \per{A}
\newcommand{\per}[1]{#1^\perp}
%
%%%%%%%%%%%%%%%%%%%%%
%
%     Systems of Equations
%
%%%%%%%%%%%%%%%%%%%%%
%
%  In-line form of an augmented matrix for a system of equations
%  Usage: \augmented{coefficient-matrix}{constant-vector}
\newcommand{\augmented}[2]{\left\lbrack\left.#1\,\right\rvert\,#2\right\rbrack}
%
%  Notation for a linear system before introducing matrix multiplication
%  Usage: \linearsystem{coefficient-matrix}{constant-vector}
\newcommand{\linearsystem}[2]{\mathcal{LS}\!\left(#1,\,#2\right)}
%
%  Notation for a homogenous system before introducing matrix multiplication
%  Usage: \homosystem{coefficient-matrix}
\newcommand{\homosystem}[1]{\linearsystem{#1}{\zerovector}}
%
%%%%%%%%%%%%%%%%%%%%%
%
%     Row Operations, Echelon Form
%
%%%%%%%%%%%%%%%%%%%%%
%
% Row operations on matrices
%
% Three commands to shorten up descriptions of gaussian elimination
%
% Usage: \rowopswap{row-i}{row-j}
% Usage: \rowopmult{scalar}{row-i}
% Usage: \rowopadd{scalar}{row-multiplied}{row-added-to}
\newcommand{\rowopswap}[2]{R_{#1}\leftrightarrow R_{#2}}
\newcommand{\rowopmult}[2]{#1R_{#2}}
\newcommand{\rowopadd}[3]{#1R_{#2}+R_{#3}}
%
% Mark leading 1's in echelon form with fbox
% Usage: \leading{a-1-usually}
\newcommand{\leading}[1]{\fbox{#1}}
%
%  Row-reduce arrow
%  Usage:  \rref inbetween a matrix and its reduced row-echelon form
\newcommand{\rref}{\xrightarrow{\text{RREF}}}
%
%  Elementary Matrices
%  Usage: \elemswap{subscript}{subscript}
%  Usage: \elemmult{scalar}{subscript}
%  Usage: \elemadd{scalar}{subscript-mult}{subscript-target}
%
\newcommand{\elemswap}[2]{E_{#1,#2}}
\newcommand{\elemmult}[2]{E_{#2}\left(#1\right)}
\newcommand{\elemadd}[3]{E_{#2,#3}\left(#1\right)}
%
%%%%%%%%%%%%%%%%%%%%%
%
%     2-D Constructions (Lists, Vectors, Matrices)
%
%%%%%%%%%%%%%%%%%%%%%
%
%  A list of scalars of generic length
%  Usage:  \scalarlist{scalar letter}{terminal subscript}
\newcommand{\scalarlist}[2]{{#1}_{1},\,{#1}_{2},\,{#1}_{3},\,\ldots,\,{#1}_{#2}}
%
%  Vector styling, bold (or use wiggles, arrows, whatever)
%  Subscripts go outside this construction
%  Usage: \vect{a symbol to use as a vector}
%  Have to already be in math mode
%
\newcommand{\vect}[1]{\mathbf{#1}}
%
%  A column vector
%  Usage: \colvector{list-delimited-by-\\}
%
\newcommand{\colvector}[1]{\begin{bmatrix}#1\end{bmatrix}}
%
%  A generic vector with components
%  Usage: \vectorcomponents{component-letter}{final-subscript}
\newcommand{\vectorcomponents}[2]{\colvector{#1_{1}\\#1_{2}\\#1_{3}\\\vdots\\#1_{#2}}}
%
%  A list of vectors of generic length
%  Usage:  \vectorlist{vector letter}{terminal subscript}
\newcommand{\vectorlist}[2]{\vect{#1}_{1},\,\vect{#1}_{2},\,\vect{#1}_{3},\,\ldots,\,\vect{#1}_{#2}}
%
%  Vector entries, entry i of vector v
%  (vector-expession still needs \vect, etc.)
%  Usage:  \vectorentry{vector-expression}{single-subscript}
\newcommand{\vectorentry}[2]{\left\lbrack#1\right\rbrack_{#2}}
%
%  Matrix entries, entry i,j of matrix A
%  Usage:  \matrixentry{matrix-expression}{paired-subscripts}
%
\newcommand{\matrixentry}[2]{\left\lbrack#1\right\rbrack_{#2}}
%
%  A generic linear combination
%  Usage:  \lincombo{scalar letter}{vector letter}{terminal subscript}
\newcommand{\lincombo}[3]{#1_{1}\vect{#2}_{1}+#1_{2}\vect{#2}_{2}+#1_{3}\vect{#2}_{3}+\cdots +#1_{#3}\vect{#2}_{#3}}
%
%  Matrix, column by column, as vectors
%  Usage:  \matrixcolumns{matrix letter}{terminal subscript}
\newcommand{\matrixcolumns}[2]{\left\lbrack\vect{#1}_{1}|\vect{#1}_{2}|\vect{#1}_{3}|\ldots|\vect{#1}_{#2}\right\rbrack}
%
%%%%%%%%%%%%%%%%%%%%%
%
%     Special Matrices
%
%%%%%%%%%%%%%%%%%%%%%
%
%  Transpose of a matrix
%  Usage:  \transpose{A}
\newcommand{\transpose}[1]{#1^{t}}
%
%  Inverse of a matrix
%  Usage:  \inverse{A}
\newcommand{\inverse}[1]{#1^{-1}}
%
%  Submatrix (for minors, determinants)
%  Usage: \submatrix{matrix-name}{delete-row}{delete-col}
\newcommand{\submatrix}[3]{#1\left(#2|#3\right)}
%
%  Adjoint of a matrix (twice)
%  This macro is a convenience to call \transpose and \conjugate properly
%  It shouldn't need to be modified (or mathematical meanings will change)
%  Usage:  \adj{A}
\newcommand{\adj}[1]{\transpose{\left(\conjugate{#1}\right)}}
%
%  This macro controls the symbol used for the adjoint
%  It can be edited to taste
%  Usage:  \adjoint{A}
\newcommand{\adjoint}[1]{#1^\ast}
%
%%%%%%%%%%%%%%%%%%%%%
%
%     Sets
%
%%%%%%%%%%%%%%%%%%%%%
%
%  A convenience for simple sets
%  Usage:  \set{list of element}
\newcommand{\set}[1]{\left\{#1\right\}}
%
%  Sets with vertical bar, "such that", sized for objects, not condition
%  Usage:  \setparts{objects}{condition}
%
%%\newcommand{\setparts}[2]{\left\{ #1\mid#2\right\}}
%%\newcommand{\setparts}[2]{\left\{\left. #1\right\rvert#2\right\}}
\newcommand{\setparts}[2]{\left\lbrace#1\,\middle|\,#2\right\rbrace}
%
%  Set Cardinality
%  Usage:  \card{a-set-letter}
\newcommand{\card}[1]{\left\lvert#1\right\rvert}
%
%  Set Union
%  Use \cup
%
%  Set Intersection
%  Use \cap
%
%  Set Complement
%  Usage:  \setcomplement{a-set-letter}
\newcommand{\setcomplement}[1]{\overline{#1}}
%
%%%%%%%%%%%%%%%%%%%%%
%
%     Eigenvalues and Eigenspaces
%
%%%%%%%%%%%%%%%%%%%%%
%
%  Characteristic polynomial
%  Usage: \charpoly{matrix-letter}{variable-letter}
\newcommand{\charpoly}[2]{p_{#1}\left(#2\right)}
%
%  Eigenspace
%  Usage: \eigenspace{matrix-letter}{eigenvalue-letter}
\newcommand{\eigenspace}[2]{\mathcal{E}_{#1}\left(#2\right)}
%
%  2013/10/03 Including ampersands is problematic here, 
%  think about fixes later
%  2014/02/22 Limited testing, seems &amp; is fine for HTML and LaTeX
%  2016-07-20 only employed in Archetypes, MBX has gather/align override
%  Eigensystem (presumes wrapped in an mrow within md)
%  Usage: \eigensystem{matrixletter}{eigenvalue}{list of basis vectors}
\newcommand{\eigensystem}[3]{\lambda&amp;=#2&amp;\eigenspace{#1}{#2}&amp;=\spn{\set{#3}}} 
%
%  Generalized Eigenspace
%  Usage: \geneigenspace{lin-trans-letter}{eigenvalue-letter}
\newcommand{\geneigenspace}[2]{\mathcal{G}_{#1}\left(#2\right)}
%
%  Algebraic multiplicty
%  Usage: \algmult{matrix-letter}{eigenvalue-letter}
\newcommand{\algmult}[2]{\alpha_{#1}\left(#2\right)}
%
%  Geometric multiplicty
%  Usage: \geomult{matrix-letter}{eigenvalue-letter}
\newcommand{\geomult}[2]{\gamma_{#1}\left(#2\right)}
%
%  Index (of eigenvalue)
%  Usage: \indx{matrix-letter}{eigenvalue-letter}
\newcommand{\indx}[2]{\iota_{#1}\left(#2\right)}
%
%%%%%%%%%%%%%%%%%%%%%
%
%     Linear Transformations
%
%%%%%%%%%%%%%%%%%%%%%
%
%  MathJax defines \lt to ease XML confusion
%
%  Linear transformation definition
%  Usage: \ltdefn{name-letter}{domain}{range}
\newcommand{\ltdefn}[3]{#1\colon #2\rightarrow#3}
%
%  Linear transformation evaluation
%  Usage: \lteval{name-letter}{input}
%  Replaces old \lt desired by MathJax
\newcommand{\lteval}[2]{#1\left(#2\right)}
%
% Linear transformation inverse
%  Usage: \ltinverse{name-letter}
\newcommand{\ltinverse}[1]{#1^{-1}}
%
%  Linear transformation restriction
%  Usage: \restrict{name-letter}{subspace-letter}
\newcommand{\restrict}[2]{{#1}|_{#2}}
%
%  Linear transformation preimage
%  Usage: \preimage{name-letter}{codomain-element}
\newcommand{\preimage}[2]{#1^{-1}\left(#2\right)}
%
%  Range of a linear transformation
%  TeX uses \range for something else
%  Usage:  \rng{T}
\newcommand{\rng}[1]{\mathcal{R}\!\left(#1\right)}
%
%  Kernel of a linear transformation
%  TeX uses \ker to do something different
%  Usage:  \krn{T}
\newcommand{\krn}[1]{\mathcal{K}\!\left(#1\right)}
%
%  Linear transformation composition
%  Usage: \compose{function-name}{function-name}
\newcommand{\compose}[2]{{#1}\circ{#2}}
%
%  Vector space of linear transformations
%  Usage: \vslt{domains}{codomains}
%  Presumes math mode
\newcommand{\vslt}[2]{\mathcal{LT}\left(#1,\,#2\right)}
%
%%%%%%%%%%%%%%%%%%%%%
%
%     Vector and Matrix Representations
%
%%%%%%%%%%%%%%%%%%%%%
%
%  Isomorphism symbol
%  Usage: \isomorphic
\newcommand{\isomorphic}{\cong}
%
%  Similarity
%  Usage: \similar{inner-matrix}{outer-invertible-matrix}
%  Rearranging this will not "fix" all desired changes throughout
%
\newcommand{\similar}[2]{\inverse{#2}#1#2}
%
%  Vector representation function name
%  Usage: \vectrepname{basis-letter}
\newcommand{\vectrepname}[1]{\rho_{#1}}
%
%  Vector representation output
%  Usage: \vectrep{basis-letter}{input}
\newcommand{\vectrep}[2]{\lteval{\vectrepname{#1}}{#2}}
%
%  Vector representation inverse function name
%  (Added later, not used consistently in FCLA)
%  Usage: \vectrepinvname{basis-letter}
\newcommand{\vectrepinvname}[1]{\ltinverse{\vectrepname{#1}}}
%
%  Vector representation inverse output
%  Usage: \vectrepinv{basis-letter}{input}
\newcommand{\vectrepinv}[2]{\lteval{\ltinverse{\vectrepname{#1}}}{#2}}
%
%  Matrix representation
%  Usage: \matrixrep{transformation-letter}{domain-basis-letter}{codomain-basis-letter}
\newcommand{\matrixrep}[3]{M^{#1}_{#2,#3}}
%
%  Matrix representation column-by-colum
%  2016-07-20 only employed once?
%  Usage: \matrixrepcolumns{transformation-letter}{codomain-basis-letter}{codomain-basis-vector-letter}{final-index}
\newcommand{\matrixrepcolumns}[4]{\left\lbrack \left.\vectrep{#2}{\lteval{#1}{\vect{#3}_{1}}}\right|\left.\vectrep{#2}{\lteval{#1}{\vect{#3}_{2}}}\right|\left.\vectrep{#2}{\lteval{#1}{\vect{#3}_{3}}}\right|\ldots\left|\vectrep{#2}{\lteval{#1}{\vect{#3}_{#4}}}\right.\right\rbrack}
%
%  Change of basis matrix
%  Usage: \cbm{domain-basis-letter}{codomain-basis-letter}
\newcommand{\cbm}[2]{C_{#1,#2}}
%
%%%%%%%%%%%%%%%%%%%%%
%
%     Canonical Forms
%
%%%%%%%%%%%%%%%%%%%%%
%
%  Jordan blocks
%  Usage: \jordan{size}{diagonal-element}
\newcommand{\jordan}[2]{J_{#1}\left(#2\right)}
%
%%%%%%%%%%%%%%%%%%%%%
%
%     Hadamard Matrices
%     Contributed by Elizabeth Million
%
%%%%%%%%%%%%%%%%%%%%%
%
%  Hadamard Product
%  Usage: \hadamard{a-matrix}{a-matrix}
\newcommand{\hadamard}[2]{#1\circ #2}
%
%  Hadamard identity matrix
%  Usage: \hadamardidentity{paired-subscripts-size-of-matrix}
\newcommand{\hadamardidentity}[1]{J_{#1}}
%
%  Hadamard inverse matrix
%  Usage: \hadamardinverse{matrix-expression}
\newcommand{\hadamardinverse}[1]{\widehat{#1}}

\newcommand{\definedTerm}[1]{\textbf{#1}}
\newcommand{\dfn}[1]{\textbf{#1}}

\newcommand{\wt}{\widetilde}
\newcommand{\ov}{\overline}
\newcommand{\inj}{\rightarrowtail}
\newcommand{\surj}{\twoheadrightarrow}
\newcommand{\harpoon}{\overset{\rightharpoonup}}

\newenvironment{amatrix}[1]{%
  \left[\begin{array}{@{}*{#1}{c}|c@{}}
}{%
  \end{array}\right]
}


\title{Vector Form of Solution Sets}

\begin{document}
\begin{abstract}
  We use column vectors and linear combinations to express all of the
  solutions to a linear system of equations.
\end{abstract}
\maketitle

We have written solutions to systems of equations as column vectors.
For example, we might write the solution
$x_1 = -3,\,x_2 = 5,\,x_3 = 2$ as
\[
  \vect{x}=\colvector{x_1\\x_2\\x_3}=\colvector{-3\\5\\2}
\]
Now, we will use column vectors and linear combinations to express
\textit{all} of the solutions to a linear system of equations in a
compact and understandable way.  First, here are two examples that
will motivate our next theorem.  This is a valuable technique, almost
the equal of row-reducing a matrix, so be sure you get comfortable
with it over the course of this section.

\begin{example}
  Consider the linear system of 3 equations in 4 variables:
  \begin{align*}
    2x_1  + x_2 + 7x_3 - 7x_4 &= 8, \\
    -3x_1 + 4x_2 -5x_3 - 6x_4 &=  -12, \\
    x_1 +x_2 + 4x_3 - 5x_4 &=  4.
  \end{align*}
  Row-reducing the corresponding augmented matrix yields
  \[
    \begin{bmatrix}
      \leading{1} & 0 & 3 & -2 & 4 \\
      0 & \leading{1} & 1 &  -3 & 0\\
      0 & 0 & 0 &  0 & 0
    \end{bmatrix}
  \]
  and we see $r=2$ pivot columns. Also, $D=\set{1,\,2}$ so the
  dependent variables are then $x_1$ and $x_2$.  Since
  $F=\set{3,\,4,\,5}$, the two free variables are $x_3$ and $x_4$.  We
  will express a generic solution for the system by two slightly
  different methods, though both arrive at the same conclusion.

  First, we decompose a solution vector.  Rearranging each equation
  represented in the row-reduced form of the augmented matrix by
  solving for the dependent variable in each row yields the vector
  equality,
  \begin{align*}
    \vect{x} &= \colvector{x_1\\x_2\\x_3\\x_4}=
    \colvector{4-3x_3+2x_4\\ -x_3+3x_4\\x_3\\x_4} \\
  \end{align*}
  Now we will use the definitions of column vector addition and scalar multiplication to express this vector as a linear combination,
\begin{align*}
  \vect{x}
                                  &=\colvector{4\\0\\0\\0}+
    \colvector{-3x_3\\-x_3\\x_3\\0}+
    \colvector{2x_4\\3x_4\\0\\x_4}&&\ref{definition:CVA}\\
                                  &=\colvector{4\\0\\0\\0}+
    x_3\colvector{-3\\-1\\1\\0}+
    x_4\colvector{2\\3\\0\\1}&&\ref{definition:CVSM}\\
  \end{align*}
  We will develop the same linear combination a bit quicker, using
  three steps.  While the method above is instructive, the method
  below will be our preferred approach.

  \textbf{Step 1: Building the scaffold.}  Write the vector of variables as a fixed vector,
  plus a linear combination of $n-r$ vectors, using the free variables
  as the scalars.
  \[
    \vect{x}=\colvector{x_1\\x_2\\x_3\\x_4}=
    \colvector{\ \\\ \\\ \\\ }+x_3\colvector{\ \\\ \\\ \\\ }+x_4\colvector{\ \\\ \\\ \\\ }
  \]

  \textbf{Step 2: Address the free variables.}  Use 0's and 1's to ensure equality for the entries
  of the vectors with indices in $F$ (corresponding to the free
  variables).
  \[
    \vect{x}=\colvector{x_1\\x_2\\x_3\\x_4}=
    \colvector{\ \\\ \\0\\0}+x_3\colvector{\ \\\ \\1\\0}+x_4\colvector{\ \\\ \\\answer{0}\\\answer{1}}
  \]

  \textbf{Step 3: Finish the dependent variables.}  For each dependent
  variable, use the augmented matrix to formulate an equation
  expressing the dependent variable as a constant plus multiples of
  the free variables.  Convert this equation into entries of the
  vectors that ensure equality for each dependent variable, one at a
  time.
  \begin{align*}
    x_1=4-3x_3+2x_4&&\Rightarrow&&
                                   \vect{x}=\colvector{x_1\\x_2\\x_3\\x_4}=
    \colvector{4\\\ \\0\\0}+x_3\colvector{-3\\\ \\1\\0}+x_4\colvector{\answer{2}\\\ \\0\\1}\\
    x_2=0-1x_3+3x_4&&\Rightarrow&&
                                   \vect{x}=\colvector{x_1\\x_2\\x_3\\x_4}=
    \colvector{4\\0\\0\\0}+x_3\colvector{-3\\-1\\1\\0}+x_4\colvector{2\\3\\0\\1}
  \end{align*}

  This final \textit{form} of a typical solution is especially
  pleasing and useful.  For example, we can build solutions quickly by
  choosing values for our free variables, and then compute a linear
  combination.  Such as
  \begin{align*}
    x_3=2,\,x_4=-5&&\Rightarrow&&
                                  \vect{x}=\colvector{x_1\\x_2\\x_3\\x_4}=
    \colvector{4\\0\\0\\0}+(2)\colvector{-3\\-1\\1\\0}+(-5)\colvector{2\\3\\0\\1}
    =\colvector{-12\\-17\\2\\-5}
    \end{align*}
    or,
    \begin{align*}
      x_3=1,\,x_4=3&&\Rightarrow&&
                                   \vect{x}=\colvector{x_1\\x_2\\x_3\\x_4}=
      \colvector{4\\0\\0\\0}+(1)\colvector{-3\\-1\\1\\0}+(3)\colvector{2\\3\\0\\1}
      =\colvector{7\\8\\1\\3}
  \end{align*}
  
  \begin{question}
    How many solutions are there?

    While the above form is useful for quickly creating solutions, it
    is even better because it tells us \textit{exactly} what every
    solution looks like.  We know the solution set is 
    \begin{multipleChoice}
      \choice{empty}
      \choice{finite}
      \choice[correct]{infinite}
      \end{multipleChoice}

      \begin{feedback}[correct]
        The solution set is infinite, and we can say more: we can say
        that a solution is some multiple of $\colvector{-3\\-1\\1\\0}$
        plus a multiple of $\colvector{2\\3\\0\\1}$ plus the fixed
        vector $\colvector{4\\0\\0\\0}$.  Period.  So it only takes us
        \textit{three} vectors to describe the entire infinite
        solution set, provided we also agree on how to combine the
        three vectors into a linear combination.
      \end{feedback}
    \end{question}
  \end{example}

That was an important and fundamental technique, so we will do another
example.

\begin{example}[Vector form of solutions]
  Consider a linear system of $m=5$ equations in $n=7$ variables,
  having the augmented matrix $A$.
  \[
    A=
    \begin{bmatrix}
      2 & 1 & -1 & -2 & 2 & 1 & 5 & 21 \\
      1 & 1 & -3 & 1 & 1 & 1 & 2 & -5 \\
      1 & 2 & -8 & 5 & 1 & 1 & -6 & -15 \\
      3 & 3 & -9 & 3 & 6 & 5 & 2 & -24 \\
      -2 & -1 & 1 & 2 & 1 & 1 & -9 & -30
    \end{bmatrix}
  \]
  Row-reducing we obtain the matrix
  \[
    B=
    \begin{bmatrix}
      \leading{1} & 0 & 2 & -3 & 0 & 0 & 9 &  15 \\
      0 & \leading{1} & -5 & 4 & 0 & 0 & -8 &  -10 \\
      0 & 0 & 0 & 0 & \leading{1} & 0 & -6 &  11 \\
      0 & 0 & 0 & 0 & 0 & \leading{1} & 7 &  -21 \\
      0 & 0 & 0 & 0 & 0 & 0 & 0 & 0
    \end{bmatrix}
  \]
  and we see $r=\answer{4}$ pivot columns. Also,
  $D=\set{1,\,2,\,5,\,\answer{6}}$ so the dependent variables are then
  $x_1,\,x_2,\,x_5,$ and $x_6$.  $F=\set{3,\,4,\,7,\,8}$ so the
  $n-r=3$ free variables are $x_3,\,x_4$ and $x_7$.  We will express a
  generic solution for the system by two different methods: both a
  decomposition and a construction.

  First, we will decompose a solution vector.  Rearranging each
  equation represented in the row-reduced form of the augmented matrix
  by solving for the dependent variable in each row yields the vector
  equality,
  \begin{align*}
    \vect{v}
    &= \colvector{x_1\\x_2\\x_3\\x_4\\x_5\\x_6\\x_7} \\
    &=\colvector{
      15-2x_3+3x_4-9x_7\\
    -10+5x_3-4x_4+8x_7\\
    x_3\\
    x_4\\
    11+6x_7\\
    -21-7x_7\\
    x_7
    }
  \end{align*}
  Now we will use the definitions of column vector addition and scalar multiplication to decompose this generic solution vector as a linear combination,
  \begin{align*}
    \vect{v}
                                                 &=
                                                   \colvector{15\\ -10\\ 0\\ 0\\ 11\\ -21\\ 0 }
    +
    \colvector{ -2x_3\\ 5x_3\\ x_3\\ 0\\ 0\\ 0\\ 0 }
    +
    \colvector{ 3x_4\\ -4x_4\\ 0\\ x_4\\ 0\\ 0\\ 0 }
    +
    \colvector{ -9x_7\\ 8x_7\\ 0\\ 0\\ 6x_7\\ -7x_7\\ x_7 }
                                                 &&\ref{definition:CVA}\\
                                                 &=
                                                   \colvector{15\\ -10\\ 0\\ 0\\ 11\\ -21\\ 0 }
    +
    x_3\colvector{ -2\\ 5\\ 1\\ 0\\ 0\\ 0\\ 0 }
    +
    x_4\colvector{ 3\\ -4\\ 0\\ 1\\ 0\\ 0\\ 0 }
    +
    x_7\colvector{ -9\\ 8\\ 0\\ 0\\ 6\\ -7\\ 1 }
                                                 &&\ref{definition:CVSM}
  \end{align*}
  We will now develop the same linear combination a bit quicker, using three steps.  While the method above is instructive, the method below will be our preferred approach.

  \textbf{Step 1.}  Write the vector of variables as a fixed vector,
  plus a linear combination of $n-r$ vectors, using the free variables
  as the scalars.
  \[
    \vect{x}=
    \colvector{x_1\\x_2\\x_3\\x_4\\x_5\\x_6\\x_7}=
    \colvector{\ \\\ \\\ \\\ \\\ \\\ \\\ }+x_3\colvector{\ \\\ \\\ \\\ \\\ \\\ \\\ }+x_4\colvector{\ \\\ \\\ \\\ \\\ \\\ \\\ }+x_7\colvector{\ \\\ \\\ \\\ \\\ \\\ \\\ }
  \]

  \textbf{Step 2.}  Use 0's and 1's to ensure equality for the entries
  of the vectors with indices in $F$ (corresponding to the free
  variables).
  \[
    \vect{x}=
    \colvector{x_1\\x_2\\x_3\\x_4\\x_5\\x_6\\x_7}=
    \colvector{\\ \\ 0\\ 0\\ \\ \\ 0}+x_3\colvector{\\ \\ 1\\ 0\\ \\ \\ 0}+x_4\colvector{\\ \\ 0\\ 1\\ \\ \\ 0}+x_7\colvector{\\ \\ 0\\ 0\\ \\ \\ 1}
  \]
  
  \textbf{Step 3.}  For each dependent variable, use the augmented
  matrix to formulate an equation expressing the dependent variable as
  a constant plus multiples of the free variables.  Convert this
  equation into entries of the vectors that ensure equality for each
  dependent variable, one at a time.
  \begin{align*}
    x_1&=15-2x_3+3x_4-9x_7\ \Rightarrow\\&\vect{x}=
                                           \colvector{x_1\\x_2\\x_3\\x_4\\x_5\\x_6\\x_7}=
    \colvector{15\\ \\ 0\\ 0\\ \\ \\ 0}+
    x_3\colvector{-2\\ \\ 1\\ 0\\ \\ \\ 0}+
    x_4\colvector{3\\ \\ 0\\ 1\\ \\ \\ 0}+
    x_7\colvector{-9\\ \\ 0\\ 0\\ \\ \\ 1}\\
    x_2&=-10+5x_3-4x_4+8x_7\ \Rightarrow\\&\vect{x}=
                                            \colvector{x_1\\x_2\\x_3\\x_4\\x_5\\x_6\\x_7}=
    \colvector{15\\ -10\\ 0\\ 0\\ \\ \\ 0}+
    x_3\colvector{-2\\ 5\\ 1\\ 0\\ \\ \\ 0}+
    x_4\colvector{3\\ -4\\ 0\\ 1\\ \\ \\ 0}+
    x_7\colvector{-9\\ 8\\ 0\\ 0\\ \\ \\ 1}\\
    x_5&=11+6x_7\ \Rightarrow\\&\vect{x}=
                                 \colvector{x_1\\x_2\\x_3\\x_4\\x_5\\x_6\\x_7}=
    \colvector{15\\ -10\\ 0\\ 0\\ 11\\ \\ 0}+
    x_3\colvector{-2\\ 5\\ 1\\ 0\\ 0\\ \\ 0}+
    x_4\colvector{3\\ -4\\ 0\\ 1\\ 0\\ \\ 0}+
    x_7\colvector{-9\\ 8\\ 0\\ 0\\ 6\\ \\ 1}\\
    x_6&=-21-7x_7\ \Rightarrow\\&\vect{x}=
                                  \colvector{x_1\\x_2\\x_3\\x_4\\x_5\\x_6\\x_7}=
    \colvector{15\\ -10\\ 0\\ 0\\ 11\\ -21\\ 0}+
    x_3\colvector{-2\\ 5\\ 1\\ 0\\ 0\\ 0\\ 0}+
    x_4\colvector{3\\ -4\\ 0\\ 1\\ 0\\ 0\\ 0}+
    x_7\colvector{-9\\ 8\\ 0\\ 0\\ 6\\ -7\\ 1}
  \end{align*}

  This final \textit{form} of a typical solution is especially
  pleasing and useful.  For example, we can build solutions quickly by
  choosing values for our free variables, and then compute a linear
  combination.  For example,
  \begin{align*}
    x_3&=2,\,
         x_4=-4,\,
         x_7=3
         \quad\quad\Rightarrow\\
    \vect{x}&=
              \colvector{x_1\\x_2\\x_3\\x_4\\x_5\\x_6\\x_7}=
    \colvector{15\\ -10\\ 0\\ 0\\ 11\\ -21\\ 0}+
    (\answer{2})\colvector{-2\\ 5\\ 1\\ 0\\ 0\\ 0\\ 0}+
    (-4)\colvector{3\\ -4\\ 0\\ 1\\ 0\\ 0\\ 0}+
    (3)\colvector{-9\\ 8\\ 0\\ 0\\ 6\\ -7\\ 1}
    =
    \colvector{-28\\40\\2\\-4\\29\\-42\\3}
  \end{align*}
  or perhaps,
  \begin{align*}
    x_3&=5,\,
         x_4=2,\,
         x_7=1
         \quad\quad\Rightarrow\\
    \vect{x}&=
              \colvector{x_1\\x_2\\x_3\\x_4\\x_5\\x_6\\x_7}=
    \colvector{15\\ -10\\ 0\\ 0\\ 11\\ -21\\ 0}+
    (5)\colvector{-2\\ 5\\ 1\\ 0\\ 0\\ 0\\ 0}+
    (2)\colvector{3\\ -4\\ 0\\ 1\\ 0\\ 0\\ 0}+
    (1)\colvector{-9\\ 8\\ 0\\ 0\\ 6\\ -7\\ 1}
    =
    \colvector{2\\15\\5\\2\\17\\-28\\1}
  \end{align*}
  or even,
  \begin{align*}
    x_3&=0,\,
         x_4=0,\,
         x_7=0
         \quad\quad\Rightarrow\\
    \vect{x}&=
              \colvector{x_1\\x_2\\x_3\\x_4\\x_5\\x_6\\x_7}=
    \colvector{15\\ -10\\ 0\\ 0\\ 11\\ -21\\ 0}+
    (0)\colvector{-2\\ 5\\ 1\\ 0\\ 0\\ 0\\ 0}+
    (0)\colvector{3\\ -4\\ 0\\ 1\\ 0\\ 0\\ 0}+
    (0)\colvector{-9\\ 8\\ 0\\ 0\\ 6\\ -7\\ 1}
    =
    \colvector{15\\ -10\\ 0\\ 0\\ 11\\ -21\\ 0}
  \end{align*}
  So we can compactly express \textit{all} of the solutions to this
  linear system with just $\answer{4}$ fixed vectors, provided we agree
  how to combine them in a linear combinations to create solution
  vectors.

  \begin{question}
    Is the vector 
    \[
      \vect{w}=\colvector{100\\-75\\7\\9\\-37\\35\\-8}
    \]
    a solution to this system of equations?

    \begin{multipleChoice}
      \choice{No.}
      \choice[correct]{Yes.}
    \end{multipleChoice}

    \begin{feedback}[correct]
      You could turn the problem around and write $\vect{w}$ as a
      linear combination of the four vectors $\vect{c}$, $\vect{u}_1$,
      $\vect{u}_2$, $\vect{u}_3$.
      \begin{align*}
        \vect{c}&=\colvector{15\\ -10\\ 0\\ 0\\ 11\\ -21\\ 0}
                &
                  \vect{u}_1&=\colvector{-2\\ 5\\ 1\\ 0\\ 0\\ 0\\ 0}
                &
                  \vect{u}_2&=\colvector{3\\ -4\\ 0\\ 1\\ 0\\ 0\\ 0}
                &
                  \vect{u}_3&=\colvector{-9\\ 8\\ 0\\ 0\\ 6\\ -7\\ 1}
      \end{align*}
      In that case, the coefficient of $\vect{c}$ is 1.  The
      coefficients of $\vect{u}_1$, $\vect{u}_2$, $\vect{u}_3$ lie in
      the third, fourth and seventh entries of $\vect{w}$.  Can you
      see why?  (Hint: $F=\set{3,\,4,\,7,\,8}$, so the free variables
      are $x_3,\,x_4$ and $x_7$.)
    \end{feedback}
  \end{question}
\end{example}

Did you think a few weeks ago that you could so quickly and easily
list \textit{all} the solutions to a linear system of 5 equations in 7
variables?

In the next activity, We will now formalize the last two (important)
examples as a theorem.

\end{document}
