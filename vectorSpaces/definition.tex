\documentclass{ximera}
% These macros are automatically generated from the "macros"
% XML element.  Make permanent edits there.
%
% History
%   2004/01/01  Initiated for FCLA, evolved from there
%   2006/09/17  Converted  _, ^  to \sb, \sp for TeX4ht
%   2014/02/01  Updated for MathBook XML projects
%               Obsolete in FCLA: \codeindent, \computerfont, \define
%               Change: MathJax wants \lt, so replaced by \lteval
%   2014/02/22  New: \orderof, \reals, \per
%   2015/08/16  Incorporated into MathBook XML version of FCLA
%
%%%%%%%%%%%%%%%%%%%%%
%
%     Conveniences
%
%%%%%%%%%%%%%%%%%%%%%
%
%  Order of (asymptotically limit of fraction is 1)
%  Usage: \orderof{some function}
%
\newcommand{\orderof}[1]{\sim #1}
%
%  Integers
%  Usage:  \Z
\newcommand{\Z}{\mathbb{Z}}
%
%  Real numbers, as set of scalars
%  Usage:  \reals
\newcommand{\reals}{\mathbb{R}}
%
%  n-space over real field
%  Usage: \complex{integer-dimension}
\newcommand{\real}[1]{\mathbb{R}^{#1}}
%
%  Complex numbers, as set of scalars
%  Usage:  \complexes
\newcommand{\complexes}{\mathbb{C}}
%
%  n-space over complex field
%  Usage: \complex{integer-dimension}
\newcommand{\complex}[1]{\mathbb{C}^{#1}}
\newcommand{\CC}{\mathbb{C}}
%
%  Complex conjugation (scalar, vector, matrix)
%  Usage: \conjugate{object}
\newcommand{\conjugate}[1]{\overline{#1}}
%
%  Complex number modulus
%  Usage: \modulus{a+bi}
%  Presumes math mode
\newcommand{\modulus}[1]{\left\lvert#1\right\rvert}
%
%  Zero vector
%  Usage: \zerovector
\newcommand{\zerovector}{\vect{0}}
%
%  Zero matrix
%  Usage: \zeromatrix, use a subscript when size is important
\newcommand{\zeromatrix}{\mathcal{O}}
%
%  Inner product (brackets, not quadratic form)
%  Usage: \innerproduct{a-vector}{a-vector}
\newcommand{\innerproduct}[2]{\left\langle#1,\,#2\right\rangle}
%
%  Norm of a vector
%  Usage: \norm{a-vector}
\newcommand{\norm}[1]{\left\lVert#1\right\rVert}
%
%  Dimension
%  Usage: \dimension{vector-space-letter}
\newcommand{\dimension}[1]{\dim\left(#1\right)}
%
%  Nullity
%  Usage: \nullity{matrix-or-lintrans-letter}
\newcommand{\nullity}[1]{n\left(#1\right)}
%
%  Rank
%  Usage: \rank{matrix-or-lintrans-letter}
\newcommand{\rank}[1]{r\left(#1\right)}
%
%  Direct sum
%  Usage: \ds between a couple of subspaces
%
\newcommand{\ds}{\oplus}
%
%  Determinant of a matrix (functional)
%  Usage: \detname{A}
\newcommand{\detname}[1]{\det\left(#1\right)}
%
%  Determinant of a matrix (vertical bars)
%  Usage: \detbars{A}
\newcommand{\detbars}[1]{\left\lvert#1\right\rvert}
%
%  Trace of a Matrix
%  Usage: \trace{matrix name}
\newcommand{\trace}[1]{t\left(#1\right)}
%
%  Square Root of a Matrix
%  Usage: \sr{a-matrix}
\newcommand{\sr}[1]{#1^{1/2}}
%
%%%%%%%%%%%%%%%%%%%%%
%
%     Subspace Constructions
%
%%%%%%%%%%%%%%%%%%%%%
%
%  Span of a set of vectors
%  \span and \sp are used by TeX for other things
%  Usage: \spn{set-of-vectors}
\newcommand{\spn}[1]{\left\langle#1\right\rangle}
%
%  Null space of a matrix
%  Usage:  \nsp{A}
\newcommand{\nsp}[1]{\mathcal{N}\!\left(#1\right)}
%
%  Column space of a matrix
%  Usage:  \csp{A}
\newcommand{\csp}[1]{\mathcal{C}\!\left(#1\right)}
%
%  Row space of a matrix
%  Usage:  \rsp{A}
\newcommand{\rsp}[1]{\mathcal{R}\!\left(#1\right)}
%
%  Left null space of a matrix
%  Usage:  \lns{A}
\newcommand{\lns}[1]{\mathcal{L}\!\left(#1\right)}
%
%  Orthogonal complement of a vector space
%  Avoiding TeX's \perp
%  Usage:  \per{A}
\newcommand{\per}[1]{#1^\perp}
%
%%%%%%%%%%%%%%%%%%%%%
%
%     Systems of Equations
%
%%%%%%%%%%%%%%%%%%%%%
%
%  In-line form of an augmented matrix for a system of equations
%  Usage: \augmented{coefficient-matrix}{constant-vector}
\newcommand{\augmented}[2]{\left\lbrack\left.#1\,\right\rvert\,#2\right\rbrack}
%
%  Notation for a linear system before introducing matrix multiplication
%  Usage: \linearsystem{coefficient-matrix}{constant-vector}
\newcommand{\linearsystem}[2]{\mathcal{LS}\!\left(#1,\,#2\right)}
%
%  Notation for a homogenous system before introducing matrix multiplication
%  Usage: \homosystem{coefficient-matrix}
\newcommand{\homosystem}[1]{\linearsystem{#1}{\zerovector}}
%
%%%%%%%%%%%%%%%%%%%%%
%
%     Row Operations, Echelon Form
%
%%%%%%%%%%%%%%%%%%%%%
%
% Row operations on matrices
%
% Three commands to shorten up descriptions of gaussian elimination
%
% Usage: \rowopswap{row-i}{row-j}
% Usage: \rowopmult{scalar}{row-i}
% Usage: \rowopadd{scalar}{row-multiplied}{row-added-to}
\newcommand{\rowopswap}[2]{R_{#1}\leftrightarrow R_{#2}}
\newcommand{\rowopmult}[2]{#1R_{#2}}
\newcommand{\rowopadd}[3]{#1R_{#2}+R_{#3}}
%
% Mark leading 1's in echelon form with fbox
% Usage: \leading{a-1-usually}
\newcommand{\leading}[1]{\fbox{#1}}
%
%  Row-reduce arrow
%  Usage:  \rref inbetween a matrix and its reduced row-echelon form
\newcommand{\rref}{\xrightarrow{\text{RREF}}}
%
%  Elementary Matrices
%  Usage: \elemswap{subscript}{subscript}
%  Usage: \elemmult{scalar}{subscript}
%  Usage: \elemadd{scalar}{subscript-mult}{subscript-target}
%
\newcommand{\elemswap}[2]{E_{#1,#2}}
\newcommand{\elemmult}[2]{E_{#2}\left(#1\right)}
\newcommand{\elemadd}[3]{E_{#2,#3}\left(#1\right)}
%
%%%%%%%%%%%%%%%%%%%%%
%
%     2-D Constructions (Lists, Vectors, Matrices)
%
%%%%%%%%%%%%%%%%%%%%%
%
%  A list of scalars of generic length
%  Usage:  \scalarlist{scalar letter}{terminal subscript}
\newcommand{\scalarlist}[2]{{#1}_{1},\,{#1}_{2},\,{#1}_{3},\,\ldots,\,{#1}_{#2}}
%
%  Vector styling, bold (or use wiggles, arrows, whatever)
%  Subscripts go outside this construction
%  Usage: \vect{a symbol to use as a vector}
%  Have to already be in math mode
%
\newcommand{\vect}[1]{\mathbf{#1}}
%
%  A column vector
%  Usage: \colvector{list-delimited-by-\\}
%
\newcommand{\colvector}[1]{\begin{bmatrix}#1\end{bmatrix}}
%
%  A generic vector with components
%  Usage: \vectorcomponents{component-letter}{final-subscript}
\newcommand{\vectorcomponents}[2]{\colvector{#1_{1}\\#1_{2}\\#1_{3}\\\vdots\\#1_{#2}}}
%
%  A list of vectors of generic length
%  Usage:  \vectorlist{vector letter}{terminal subscript}
\newcommand{\vectorlist}[2]{\vect{#1}_{1},\,\vect{#1}_{2},\,\vect{#1}_{3},\,\ldots,\,\vect{#1}_{#2}}
%
%  Vector entries, entry i of vector v
%  (vector-expession still needs \vect, etc.)
%  Usage:  \vectorentry{vector-expression}{single-subscript}
\newcommand{\vectorentry}[2]{\left\lbrack#1\right\rbrack_{#2}}
%
%  Matrix entries, entry i,j of matrix A
%  Usage:  \matrixentry{matrix-expression}{paired-subscripts}
%
\newcommand{\matrixentry}[2]{\left\lbrack#1\right\rbrack_{#2}}
%
%  A generic linear combination
%  Usage:  \lincombo{scalar letter}{vector letter}{terminal subscript}
\newcommand{\lincombo}[3]{#1_{1}\vect{#2}_{1}+#1_{2}\vect{#2}_{2}+#1_{3}\vect{#2}_{3}+\cdots +#1_{#3}\vect{#2}_{#3}}
%
%  Matrix, column by column, as vectors
%  Usage:  \matrixcolumns{matrix letter}{terminal subscript}
\newcommand{\matrixcolumns}[2]{\left\lbrack\vect{#1}_{1}|\vect{#1}_{2}|\vect{#1}_{3}|\ldots|\vect{#1}_{#2}\right\rbrack}
%
%%%%%%%%%%%%%%%%%%%%%
%
%     Special Matrices
%
%%%%%%%%%%%%%%%%%%%%%
%
%  Transpose of a matrix
%  Usage:  \transpose{A}
\newcommand{\transpose}[1]{#1^{t}}
%
%  Inverse of a matrix
%  Usage:  \inverse{A}
\newcommand{\inverse}[1]{#1^{-1}}
%
%  Submatrix (for minors, determinants)
%  Usage: \submatrix{matrix-name}{delete-row}{delete-col}
\newcommand{\submatrix}[3]{#1\left(#2|#3\right)}
%
%  Adjoint of a matrix (twice)
%  This macro is a convenience to call \transpose and \conjugate properly
%  It shouldn't need to be modified (or mathematical meanings will change)
%  Usage:  \adj{A}
\newcommand{\adj}[1]{\transpose{\left(\conjugate{#1}\right)}}
%
%  This macro controls the symbol used for the adjoint
%  It can be edited to taste
%  Usage:  \adjoint{A}
\newcommand{\adjoint}[1]{#1^\ast}
%
%%%%%%%%%%%%%%%%%%%%%
%
%     Sets
%
%%%%%%%%%%%%%%%%%%%%%
%
%  A convenience for simple sets
%  Usage:  \set{list of element}
\newcommand{\set}[1]{\left\{#1\right\}}
%
%  Sets with vertical bar, "such that", sized for objects, not condition
%  Usage:  \setparts{objects}{condition}
%
%%\newcommand{\setparts}[2]{\left\{ #1\mid#2\right\}}
%%\newcommand{\setparts}[2]{\left\{\left. #1\right\rvert#2\right\}}
\newcommand{\setparts}[2]{\left\lbrace#1\,\middle|\,#2\right\rbrace}
%
%  Set Cardinality
%  Usage:  \card{a-set-letter}
\newcommand{\card}[1]{\left\lvert#1\right\rvert}
%
%  Set Union
%  Use \cup
%
%  Set Intersection
%  Use \cap
%
%  Set Complement
%  Usage:  \setcomplement{a-set-letter}
\newcommand{\setcomplement}[1]{\overline{#1}}
%
%%%%%%%%%%%%%%%%%%%%%
%
%     Eigenvalues and Eigenspaces
%
%%%%%%%%%%%%%%%%%%%%%
%
%  Characteristic polynomial
%  Usage: \charpoly{matrix-letter}{variable-letter}
\newcommand{\charpoly}[2]{p_{#1}\left(#2\right)}
%
%  Eigenspace
%  Usage: \eigenspace{matrix-letter}{eigenvalue-letter}
\newcommand{\eigenspace}[2]{\mathcal{E}_{#1}\left(#2\right)}
%
%  2013/10/03 Including ampersands is problematic here, 
%  think about fixes later
%  2014/02/22 Limited testing, seems &amp; is fine for HTML and LaTeX
%  2016-07-20 only employed in Archetypes, MBX has gather/align override
%  Eigensystem (presumes wrapped in an mrow within md)
%  Usage: \eigensystem{matrixletter}{eigenvalue}{list of basis vectors}
\newcommand{\eigensystem}[3]{\lambda&amp;=#2&amp;\eigenspace{#1}{#2}&amp;=\spn{\set{#3}}} 
%
%  Generalized Eigenspace
%  Usage: \geneigenspace{lin-trans-letter}{eigenvalue-letter}
\newcommand{\geneigenspace}[2]{\mathcal{G}_{#1}\left(#2\right)}
%
%  Algebraic multiplicty
%  Usage: \algmult{matrix-letter}{eigenvalue-letter}
\newcommand{\algmult}[2]{\alpha_{#1}\left(#2\right)}
%
%  Geometric multiplicty
%  Usage: \geomult{matrix-letter}{eigenvalue-letter}
\newcommand{\geomult}[2]{\gamma_{#1}\left(#2\right)}
%
%  Index (of eigenvalue)
%  Usage: \indx{matrix-letter}{eigenvalue-letter}
\newcommand{\indx}[2]{\iota_{#1}\left(#2\right)}
%
%%%%%%%%%%%%%%%%%%%%%
%
%     Linear Transformations
%
%%%%%%%%%%%%%%%%%%%%%
%
%  MathJax defines \lt to ease XML confusion
%
%  Linear transformation definition
%  Usage: \ltdefn{name-letter}{domain}{range}
\newcommand{\ltdefn}[3]{#1\colon #2\rightarrow#3}
%
%  Linear transformation evaluation
%  Usage: \lteval{name-letter}{input}
%  Replaces old \lt desired by MathJax
\newcommand{\lteval}[2]{#1\left(#2\right)}
%
% Linear transformation inverse
%  Usage: \ltinverse{name-letter}
\newcommand{\ltinverse}[1]{#1^{-1}}
%
%  Linear transformation restriction
%  Usage: \restrict{name-letter}{subspace-letter}
\newcommand{\restrict}[2]{{#1}|_{#2}}
%
%  Linear transformation preimage
%  Usage: \preimage{name-letter}{codomain-element}
\newcommand{\preimage}[2]{#1^{-1}\left(#2\right)}
%
%  Range of a linear transformation
%  TeX uses \range for something else
%  Usage:  \rng{T}
\newcommand{\rng}[1]{\mathcal{R}\!\left(#1\right)}
%
%  Kernel of a linear transformation
%  TeX uses \ker to do something different
%  Usage:  \krn{T}
\newcommand{\krn}[1]{\mathcal{K}\!\left(#1\right)}
%
%  Linear transformation composition
%  Usage: \compose{function-name}{function-name}
\newcommand{\compose}[2]{{#1}\circ{#2}}
%
%  Vector space of linear transformations
%  Usage: \vslt{domains}{codomains}
%  Presumes math mode
\newcommand{\vslt}[2]{\mathcal{LT}\left(#1,\,#2\right)}
%
%%%%%%%%%%%%%%%%%%%%%
%
%     Vector and Matrix Representations
%
%%%%%%%%%%%%%%%%%%%%%
%
%  Isomorphism symbol
%  Usage: \isomorphic
\newcommand{\isomorphic}{\cong}
%
%  Similarity
%  Usage: \similar{inner-matrix}{outer-invertible-matrix}
%  Rearranging this will not "fix" all desired changes throughout
%
\newcommand{\similar}[2]{\inverse{#2}#1#2}
%
%  Vector representation function name
%  Usage: \vectrepname{basis-letter}
\newcommand{\vectrepname}[1]{\rho_{#1}}
%
%  Vector representation output
%  Usage: \vectrep{basis-letter}{input}
\newcommand{\vectrep}[2]{\lteval{\vectrepname{#1}}{#2}}
%
%  Vector representation inverse function name
%  (Added later, not used consistently in FCLA)
%  Usage: \vectrepinvname{basis-letter}
\newcommand{\vectrepinvname}[1]{\ltinverse{\vectrepname{#1}}}
%
%  Vector representation inverse output
%  Usage: \vectrepinv{basis-letter}{input}
\newcommand{\vectrepinv}[2]{\lteval{\ltinverse{\vectrepname{#1}}}{#2}}
%
%  Matrix representation
%  Usage: \matrixrep{transformation-letter}{domain-basis-letter}{codomain-basis-letter}
\newcommand{\matrixrep}[3]{M^{#1}_{#2,#3}}
%
%  Matrix representation column-by-colum
%  2016-07-20 only employed once?
%  Usage: \matrixrepcolumns{transformation-letter}{codomain-basis-letter}{codomain-basis-vector-letter}{final-index}
\newcommand{\matrixrepcolumns}[4]{\left\lbrack \left.\vectrep{#2}{\lteval{#1}{\vect{#3}_{1}}}\right|\left.\vectrep{#2}{\lteval{#1}{\vect{#3}_{2}}}\right|\left.\vectrep{#2}{\lteval{#1}{\vect{#3}_{3}}}\right|\ldots\left|\vectrep{#2}{\lteval{#1}{\vect{#3}_{#4}}}\right.\right\rbrack}
%
%  Change of basis matrix
%  Usage: \cbm{domain-basis-letter}{codomain-basis-letter}
\newcommand{\cbm}[2]{C_{#1,#2}}
%
%%%%%%%%%%%%%%%%%%%%%
%
%     Canonical Forms
%
%%%%%%%%%%%%%%%%%%%%%
%
%  Jordan blocks
%  Usage: \jordan{size}{diagonal-element}
\newcommand{\jordan}[2]{J_{#1}\left(#2\right)}
%
%%%%%%%%%%%%%%%%%%%%%
%
%     Hadamard Matrices
%     Contributed by Elizabeth Million
%
%%%%%%%%%%%%%%%%%%%%%
%
%  Hadamard Product
%  Usage: \hadamard{a-matrix}{a-matrix}
\newcommand{\hadamard}[2]{#1\circ #2}
%
%  Hadamard identity matrix
%  Usage: \hadamardidentity{paired-subscripts-size-of-matrix}
\newcommand{\hadamardidentity}[1]{J_{#1}}
%
%  Hadamard inverse matrix
%  Usage: \hadamardinverse{matrix-expression}
\newcommand{\hadamardinverse}[1]{\widehat{#1}}

\newcommand{\definedTerm}[1]{\textbf{#1}}
\newcommand{\dfn}[1]{\textbf{#1}}

\newcommand{\wt}{\widetilde}
\newcommand{\ov}{\overline}
\newcommand{\inj}{\rightarrowtail}
\newcommand{\surj}{\twoheadrightarrow}
\newcommand{\harpoon}{\overset{\rightharpoonup}}

\newenvironment{amatrix}[1]{%
  \left[\begin{array}{@{}*{#1}{c}|c@{}}
}{%
  \end{array}\right]
}

\title{Definition of a vector space}
\author{Crichton Ogle}

\begin{document}
\begin{abstract}
  A vector space is a set equipped with two operations, vector
  addition and scalar multiplication, satisfying certain properties.
\end{abstract}
\maketitle

%%%%%%%%%%%%%%%%%%%%%%%%%%%%%%%%%%%%%%
Generalizing the setup for $\mathbb R^n$, we have

\begin{definition} A {\it vector space} is a set $V$ equipped with two operations - vector addition \lq\lq$+$\rq\rq\ and scalar multiplication \lq\lq$\cdot$\rq\rq - which satisfy the two closure axioms C1, C2 as well as the eight vector space axioms A1 - A8:
\begin{description}
\item[C1] (Closure under vector addition) Given ${\bf v}, {\bf w}\in V$, ${\bf v} + {\bf w}\in V$.
\item[C2] (Closure under scalar multiplication) Given ${\bf v}\in V$ and a scalar $\alpha$, $\alpha{\bf v}\in V$.
\end{description}
\vskip.2in

For $\bf u$, $\bf v$, $\bf w$ arbitrary vectors in $V$, and $\alpha,\beta$ arbitrary scalars in $\mathbb R$,

\begin{description}
\item[A1] (Commutativity of addition) ${\bf v} + {\bf w} = {\bf w} + {\bf v}$.
\item[A2] (Associativity of addition) $({\bf u} + {\bf v}) + {\bf w} = {\bf u} + ({\bf v} + {\bf w})$.
\item[A3] (Existence of a zero vector) There is a vector ${\bf z}\in V$ with ${\bf z} + {\bf v} = {\bf v} + {\bf z} = {\bf v}$.
\item[A4] (Existence of additive inverses) For each $\bf v$, there is a vector $-{\bf v}\in V$ with ${\bf v} + (-{\bf v}) = (-{\bf v}) + {\bf v} = {\bf z}$.
\item[A5] (Distributivity of scalar multiplication over vector addition) $\alpha({\bf v} + {\bf w}) = \alpha{\bf v} + \alpha{\bf w}$.
\item[A6] (Distributivity of scalar addition over scalar multiplication) $(\alpha + \beta){\bf v} = \alpha{\bf v} + \beta{\bf v}$.
\item[A7] (Associativity of scalar multiplication) $(\alpha \beta){\bf v}) = (\alpha(\beta {\bf v})$.
\item[A8] (Scalar multiplication with 1 is the identity) $1{\bf v} = {\bf v}$.
\end{description}
\vskip.2in
\end{definition}

In this way, a vector space should properly be represented as a triple $(V,+,\cdot)$, to emphasize the fact that the algebraic structure depends not just on the underlying set of vectors, but on the choice of operations representing addition and scalar multiplication.
\vskip.2in

\begin{example}\label{example:Rmn} Let $V = \mathbb R^{m\times n}$, the space of $m\times n$ matrices, with addition given by matrix addition and scalar multiplication as previously defined for matrices. Then $(\mathbb R^{m\times n},+,\cdot)$ is a vector space. Again, as with $\mathbb R^n$, the closure axioms are seen to be satisfied as a direct consequence of the definitions, while the other properties follow from Theorem \ref{thm:matalg} together with direct construction of the $m\times n$ \lq\lq zero vector\rq\rq\ $0^{m\times n}$, as well as additive inverses as indicated in [A4].
\end{example}

Before proceeding to other examples, we need to discuss an important point regarding how theorems about vector spaces are typically proven. In any system of mathematics, one operates with a certain set of assumptions, called {\it axioms}, together with various results previously proven (possibly in other areas of mathematics) and which one is allowed to assume true without further verification.
\vskip.2in

In the case of {\it vector spaces over $\mathbb R$} (i.e. where the scalars are real numbers), the standing assumption is that the above list of ten properties hold for the real numbers. The fastidious reader will note that this was already assumed in the proof of Theorem \ref{thm:matalg}; in fact the proof of that theorem would have been impossible without such an assumption. To illustrate how this foundational assumption applies in a different context, we consider the space
\[
F[a,b] = \ \text{the space of real-valued functions on the closed interval }[a,b]
\]
Recall that i) a function is completely determined by the values it takes on the elements of its domain, and therefore ii) two functions $f, g$ are {\it equal} iff they have the same domain and $f(x) = g(x)$ for all elements $x$ in their common domain. So in order to show two functions $f$ and $g$ on the closed interval $[a,b]$ are equal, it suffices to verify that $f(c) = g(c)$ for all $a\le c\le b$.
\vskip.2in

Next recall that the sum of two functions is given by
\[
(f+g)(x) := f(x) + g(x)
\]
while the scalar multiple $\alpha f$ of the function $f$ is given by
\[
(\alpha f)(x) := \alpha(f(x))
\]

\begin{theorem} Equipped with addition and scalar multiplication as just defined, $(F[a,b],+,\cdot)$ is a vector space.
\end{theorem}


\begin{proof} One begins by verifying the two closure axioms. If $f,g\in F[a,b]$, they are real-valued functions with common domain $[a,b]$; hence their sum is defined by the above equation, and has the same domain, making $f+g$ a function in $F[a,b]$. Similarly, if $f\in F[a,b]$ and $\alpha\in\mathbb R$, then multiplying $f$ by $\alpha$ leaves the domain unchanged, so $\alpha f\in F[a,b]$.
\vskip.2in
The eight vector space axioms [A1] - [A8] are of two types. The third and fourth are {\it existential} - they assert the existence of the zero element and additive inverses, respectively. To verify these, one simply has to produce the candidate satisfying the requisite properties. The remaining six are {\it universal}. They involve statements which hold for all collections of vectors for which the given equality makes sense. We will verify each of the eight axioms in detail. This example, then, can be used as a template for how to proceed in other cases with verification that a proposed candidate vector space is in fact one. 
\vskip.2in

[A1]: For all $f,g\in F[a,b]$ and $x\in [a,b]$,
\begin{align*}
(f+g)(x) &= f(x) + g(x)\quad\text{by definition of addition for functions}\\
             &= g(x) + f(x)\quad\text{by commutativity of addition for real numbers}\\
             &=(g+f)(x)\quad\text{by definition of addition for real numbers}
\end{align*}
\vskip.2in

[A2]: For all $f,g,h\in F[a,b]$ and $x\in [a,b]$,
\begin{align*}
((f+g)+h)(x) &= (f+g)(x) + h(x)\quad\text{by definition of addition for functions}\\
                     &= (f(x) + g(x)) + h(x)\quad\text{by definition of addition for functions}\\
                     &=f(x) + (g(x)) + h(x))\quad\text{by associativity of addition for real numbers}\\
                     &=f(x) + (g+h)(x)\quad\text{by definition of addition for functions}\\
                     &= (f+(g+h))(x)\quad\text{by definition of addition for functions}
\end{align*}
\vskip.2in

[A3]: Define $z\in F[a,b]$ by $z(x) = 0, a\le x\le b$. Then for all $f\in F[a,b]$ and $x\in [a,b]$, 
\begin{align*}
(z+f)(x) &= z(x) + f(x)\quad\text{by definition of addition for functions}\\
             &= 0 + f(x)\quad\text{by definition of $z$}\\
             &= f(x)\quad\text{by the defining property of $0\in\mathbb R$}\\
             &= f(x) + 0\quad\text{by the defining property of $0\in\mathbb R$}\\
             &= f(x) + z(x)\quad\text{by definition of $z$}\\
             &= (f + z)(x)\quad\text{by definition of addition for functions}
\end{align*}
\vskip.2in

[A4]: For each $f\in F[a,b]$, define $(-f)(x) := -(f(x))$ (note the different placement of parentheses on the two sides of the equation). Then for all $f\in F[a,b]$ and $x\in [a,b]$, 
\begin{align*}
(f + (-f))(x) &= f(x) + (-f)(x)\quad\text{by definition of addition for functions}\\
             &= f(x) + (-f(x))\quad\text{by definition of $(-f)$}\\
             &= 0\quad\text{by the property of additive inverses in $\mathbb R$}\\
             &= z(x)\quad\text{by the definition of $z\in F[a,b]$}\\
             &= 0\quad\text{by the definition of $z\in F[a,b]$}\\
             &= -f(x) + f(x)\quad\text{by the property of additive inverses in $\mathbb R$}\\
             &= ((-f) + f)(x)\quad\text{by definition of addition for functions}
\end{align*}
\vskip.2in

[A5]: For all $f,g\in F[a,b]$ and $\alpha, x\in [a,b]$,
\begin{align*}
(\alpha (f+g))(x) &= \alpha((f+g)(x))\quad\text{by definition of scalar multiplication for functions}\\
             &= \alpha(f(x) + g(x))\quad\text{by definition of addition for functions}\\
             &= \alpha f(x) + \alpha g(x)\quad\text{by distributivity of multiplication over addition in $\mathbb R$}\\
             &= (\alpha f)(x) + (\alpha g)(x)\quad\text{by definition of scalar multiplication for functions}\\
             &= ((\alpha f) + (\alpha g))(x)\quad\text{by definition of addition for functions}
\end{align*}
\vskip.2in

[A6]: For all $f\in F[a,b]$ and $\alpha,\beta, x\in [a,b]$,
\begin{align*}
((\alpha +\beta)f)(x) &= (\alpha +\beta)f(x)\quad\text{by definition of scalar multiplication for functions}\\
             &= \alpha f(x) + \beta f(x)\quad\text{by distributivity of multiplication over addition in $\mathbb R$}\\
             &= (\alpha f)(x) + (\beta f)(x)\quad\text{by definition of scalar multiplication for functions}\\
             &= ((\alpha f) + (\beta f))(x)\quad\text{by definition of addition for functions}
\end{align*}
\vskip.2in

[A7]: For all $f\in F[a,b]$ and $\alpha,\beta, x\in [a,b]$,
\begin{align*}
((\alpha \beta)f)(x) &= (\alpha \beta)f(x)\quad\text{by definition of scalar multiplication for functions}\\
             &= \alpha(\beta f(x))\quad\text{by associativity of multiplication in $\mathbb R$}\\
             &= \alpha ((\beta f)(x))\quad\text{by definition of scalar multiplication for functions}\\
             &= (\alpha(\beta f))(x)\quad\text{by definition of scalar multiplication for functions}
\end{align*}
\vskip.2in

[A8]: For all $f\in F[a,b]$ and $x\in [a,b]$,
\begin{align*}
(1\cdot f)(x) &= 1\cdot f(x)\quad\text{by definition of scalar multiplication for functions}\\
                     &= f(x)\quad\text{by the multiplicative property of $1\in\mathbb R$}
\end{align*}
\end{proof}
\vskip.2in

\begin{exercise} Show that $(\mathbb R^{m\times n}, +,\cdot)$ is a vector space, where \lq\lq +\rq\rq\ denotes matrix addition, and \lq\lq$\cdot$\rq\rq\ denotes scalar multiplication for matrices (in other words, verify the claim in Example \ref{example:Rmn}. Hint: use the results of Theorem \ref{thm:matalg}).
\end{exercise}
\vskip.3in

\end{document}
