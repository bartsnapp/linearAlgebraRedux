\documentclass{ximera}

% These macros are automatically generated from the "macros"
% XML element.  Make permanent edits there.
%
% History
%   2004/01/01  Initiated for FCLA, evolved from there
%   2006/09/17  Converted  _, ^  to \sb, \sp for TeX4ht
%   2014/02/01  Updated for MathBook XML projects
%               Obsolete in FCLA: \codeindent, \computerfont, \define
%               Change: MathJax wants \lt, so replaced by \lteval
%   2014/02/22  New: \orderof, \reals, \per
%   2015/08/16  Incorporated into MathBook XML version of FCLA
%
%%%%%%%%%%%%%%%%%%%%%
%
%     Conveniences
%
%%%%%%%%%%%%%%%%%%%%%
%
%  Order of (asymptotically limit of fraction is 1)
%  Usage: \orderof{some function}
%
\newcommand{\orderof}[1]{\sim #1}
%
%  Integers
%  Usage:  \Z
\newcommand{\Z}{\mathbb{Z}}
%
%  Real numbers, as set of scalars
%  Usage:  \reals
\newcommand{\reals}{\mathbb{R}}
%
%  n-space over real field
%  Usage: \complex{integer-dimension}
\newcommand{\real}[1]{\mathbb{R}^{#1}}
%
%  Complex numbers, as set of scalars
%  Usage:  \complexes
\newcommand{\complexes}{\mathbb{C}}
%
%  n-space over complex field
%  Usage: \complex{integer-dimension}
\newcommand{\complex}[1]{\mathbb{C}^{#1}}
\newcommand{\CC}{\mathbb{C}}
%
%  Complex conjugation (scalar, vector, matrix)
%  Usage: \conjugate{object}
\newcommand{\conjugate}[1]{\overline{#1}}
%
%  Complex number modulus
%  Usage: \modulus{a+bi}
%  Presumes math mode
\newcommand{\modulus}[1]{\left\lvert#1\right\rvert}
%
%  Zero vector
%  Usage: \zerovector
\newcommand{\zerovector}{\vect{0}}
%
%  Zero matrix
%  Usage: \zeromatrix, use a subscript when size is important
\newcommand{\zeromatrix}{\mathcal{O}}
%
%  Inner product (brackets, not quadratic form)
%  Usage: \innerproduct{a-vector}{a-vector}
\newcommand{\innerproduct}[2]{\left\langle#1,\,#2\right\rangle}
%
%  Norm of a vector
%  Usage: \norm{a-vector}
\newcommand{\norm}[1]{\left\lVert#1\right\rVert}
%
%  Dimension
%  Usage: \dimension{vector-space-letter}
\newcommand{\dimension}[1]{\dim\left(#1\right)}
%
%  Nullity
%  Usage: \nullity{matrix-or-lintrans-letter}
\newcommand{\nullity}[1]{n\left(#1\right)}
%
%  Rank
%  Usage: \rank{matrix-or-lintrans-letter}
\newcommand{\rank}[1]{r\left(#1\right)}
%
%  Direct sum
%  Usage: \ds between a couple of subspaces
%
\newcommand{\ds}{\oplus}
%
%  Determinant of a matrix (functional)
%  Usage: \detname{A}
\newcommand{\detname}[1]{\det\left(#1\right)}
%
%  Determinant of a matrix (vertical bars)
%  Usage: \detbars{A}
\newcommand{\detbars}[1]{\left\lvert#1\right\rvert}
%
%  Trace of a Matrix
%  Usage: \trace{matrix name}
\newcommand{\trace}[1]{t\left(#1\right)}
%
%  Square Root of a Matrix
%  Usage: \sr{a-matrix}
\newcommand{\sr}[1]{#1^{1/2}}
%
%%%%%%%%%%%%%%%%%%%%%
%
%     Subspace Constructions
%
%%%%%%%%%%%%%%%%%%%%%
%
%  Span of a set of vectors
%  \span and \sp are used by TeX for other things
%  Usage: \spn{set-of-vectors}
\newcommand{\spn}[1]{\left\langle#1\right\rangle}
%
%  Null space of a matrix
%  Usage:  \nsp{A}
\newcommand{\nsp}[1]{\mathcal{N}\!\left(#1\right)}
%
%  Column space of a matrix
%  Usage:  \csp{A}
\newcommand{\csp}[1]{\mathcal{C}\!\left(#1\right)}
%
%  Row space of a matrix
%  Usage:  \rsp{A}
\newcommand{\rsp}[1]{\mathcal{R}\!\left(#1\right)}
%
%  Left null space of a matrix
%  Usage:  \lns{A}
\newcommand{\lns}[1]{\mathcal{L}\!\left(#1\right)}
%
%  Orthogonal complement of a vector space
%  Avoiding TeX's \perp
%  Usage:  \per{A}
\newcommand{\per}[1]{#1^\perp}
%
%%%%%%%%%%%%%%%%%%%%%
%
%     Systems of Equations
%
%%%%%%%%%%%%%%%%%%%%%
%
%  In-line form of an augmented matrix for a system of equations
%  Usage: \augmented{coefficient-matrix}{constant-vector}
\newcommand{\augmented}[2]{\left\lbrack\left.#1\,\right\rvert\,#2\right\rbrack}
%
%  Notation for a linear system before introducing matrix multiplication
%  Usage: \linearsystem{coefficient-matrix}{constant-vector}
\newcommand{\linearsystem}[2]{\mathcal{LS}\!\left(#1,\,#2\right)}
%
%  Notation for a homogenous system before introducing matrix multiplication
%  Usage: \homosystem{coefficient-matrix}
\newcommand{\homosystem}[1]{\linearsystem{#1}{\zerovector}}
%
%%%%%%%%%%%%%%%%%%%%%
%
%     Row Operations, Echelon Form
%
%%%%%%%%%%%%%%%%%%%%%
%
% Row operations on matrices
%
% Three commands to shorten up descriptions of gaussian elimination
%
% Usage: \rowopswap{row-i}{row-j}
% Usage: \rowopmult{scalar}{row-i}
% Usage: \rowopadd{scalar}{row-multiplied}{row-added-to}
\newcommand{\rowopswap}[2]{R_{#1}\leftrightarrow R_{#2}}
\newcommand{\rowopmult}[2]{#1R_{#2}}
\newcommand{\rowopadd}[3]{#1R_{#2}+R_{#3}}
%
% Mark leading 1's in echelon form with fbox
% Usage: \leading{a-1-usually}
\newcommand{\leading}[1]{\fbox{#1}}
%
%  Row-reduce arrow
%  Usage:  \rref inbetween a matrix and its reduced row-echelon form
\newcommand{\rref}{\xrightarrow{\text{RREF}}}
%
%  Elementary Matrices
%  Usage: \elemswap{subscript}{subscript}
%  Usage: \elemmult{scalar}{subscript}
%  Usage: \elemadd{scalar}{subscript-mult}{subscript-target}
%
\newcommand{\elemswap}[2]{E_{#1,#2}}
\newcommand{\elemmult}[2]{E_{#2}\left(#1\right)}
\newcommand{\elemadd}[3]{E_{#2,#3}\left(#1\right)}
%
%%%%%%%%%%%%%%%%%%%%%
%
%     2-D Constructions (Lists, Vectors, Matrices)
%
%%%%%%%%%%%%%%%%%%%%%
%
%  A list of scalars of generic length
%  Usage:  \scalarlist{scalar letter}{terminal subscript}
\newcommand{\scalarlist}[2]{{#1}_{1},\,{#1}_{2},\,{#1}_{3},\,\ldots,\,{#1}_{#2}}
%
%  Vector styling, bold (or use wiggles, arrows, whatever)
%  Subscripts go outside this construction
%  Usage: \vect{a symbol to use as a vector}
%  Have to already be in math mode
%
\newcommand{\vect}[1]{\mathbf{#1}}
%
%  A column vector
%  Usage: \colvector{list-delimited-by-\\}
%
\newcommand{\colvector}[1]{\begin{bmatrix}#1\end{bmatrix}}
%
%  A generic vector with components
%  Usage: \vectorcomponents{component-letter}{final-subscript}
\newcommand{\vectorcomponents}[2]{\colvector{#1_{1}\\#1_{2}\\#1_{3}\\\vdots\\#1_{#2}}}
%
%  A list of vectors of generic length
%  Usage:  \vectorlist{vector letter}{terminal subscript}
\newcommand{\vectorlist}[2]{\vect{#1}_{1},\,\vect{#1}_{2},\,\vect{#1}_{3},\,\ldots,\,\vect{#1}_{#2}}
%
%  Vector entries, entry i of vector v
%  (vector-expession still needs \vect, etc.)
%  Usage:  \vectorentry{vector-expression}{single-subscript}
\newcommand{\vectorentry}[2]{\left\lbrack#1\right\rbrack_{#2}}
%
%  Matrix entries, entry i,j of matrix A
%  Usage:  \matrixentry{matrix-expression}{paired-subscripts}
%
\newcommand{\matrixentry}[2]{\left\lbrack#1\right\rbrack_{#2}}
%
%  A generic linear combination
%  Usage:  \lincombo{scalar letter}{vector letter}{terminal subscript}
\newcommand{\lincombo}[3]{#1_{1}\vect{#2}_{1}+#1_{2}\vect{#2}_{2}+#1_{3}\vect{#2}_{3}+\cdots +#1_{#3}\vect{#2}_{#3}}
%
%  Matrix, column by column, as vectors
%  Usage:  \matrixcolumns{matrix letter}{terminal subscript}
\newcommand{\matrixcolumns}[2]{\left\lbrack\vect{#1}_{1}|\vect{#1}_{2}|\vect{#1}_{3}|\ldots|\vect{#1}_{#2}\right\rbrack}
%
%%%%%%%%%%%%%%%%%%%%%
%
%     Special Matrices
%
%%%%%%%%%%%%%%%%%%%%%
%
%  Transpose of a matrix
%  Usage:  \transpose{A}
\newcommand{\transpose}[1]{#1^{t}}
%
%  Inverse of a matrix
%  Usage:  \inverse{A}
\newcommand{\inverse}[1]{#1^{-1}}
%
%  Submatrix (for minors, determinants)
%  Usage: \submatrix{matrix-name}{delete-row}{delete-col}
\newcommand{\submatrix}[3]{#1\left(#2|#3\right)}
%
%  Adjoint of a matrix (twice)
%  This macro is a convenience to call \transpose and \conjugate properly
%  It shouldn't need to be modified (or mathematical meanings will change)
%  Usage:  \adj{A}
\newcommand{\adj}[1]{\transpose{\left(\conjugate{#1}\right)}}
%
%  This macro controls the symbol used for the adjoint
%  It can be edited to taste
%  Usage:  \adjoint{A}
\newcommand{\adjoint}[1]{#1^\ast}
%
%%%%%%%%%%%%%%%%%%%%%
%
%     Sets
%
%%%%%%%%%%%%%%%%%%%%%
%
%  A convenience for simple sets
%  Usage:  \set{list of element}
\newcommand{\set}[1]{\left\{#1\right\}}
%
%  Sets with vertical bar, "such that", sized for objects, not condition
%  Usage:  \setparts{objects}{condition}
%
%%\newcommand{\setparts}[2]{\left\{ #1\mid#2\right\}}
%%\newcommand{\setparts}[2]{\left\{\left. #1\right\rvert#2\right\}}
\newcommand{\setparts}[2]{\left\lbrace#1\,\middle|\,#2\right\rbrace}
%
%  Set Cardinality
%  Usage:  \card{a-set-letter}
\newcommand{\card}[1]{\left\lvert#1\right\rvert}
%
%  Set Union
%  Use \cup
%
%  Set Intersection
%  Use \cap
%
%  Set Complement
%  Usage:  \setcomplement{a-set-letter}
\newcommand{\setcomplement}[1]{\overline{#1}}
%
%%%%%%%%%%%%%%%%%%%%%
%
%     Eigenvalues and Eigenspaces
%
%%%%%%%%%%%%%%%%%%%%%
%
%  Characteristic polynomial
%  Usage: \charpoly{matrix-letter}{variable-letter}
\newcommand{\charpoly}[2]{p_{#1}\left(#2\right)}
%
%  Eigenspace
%  Usage: \eigenspace{matrix-letter}{eigenvalue-letter}
\newcommand{\eigenspace}[2]{\mathcal{E}_{#1}\left(#2\right)}
%
%  2013/10/03 Including ampersands is problematic here, 
%  think about fixes later
%  2014/02/22 Limited testing, seems &amp; is fine for HTML and LaTeX
%  2016-07-20 only employed in Archetypes, MBX has gather/align override
%  Eigensystem (presumes wrapped in an mrow within md)
%  Usage: \eigensystem{matrixletter}{eigenvalue}{list of basis vectors}
\newcommand{\eigensystem}[3]{\lambda&amp;=#2&amp;\eigenspace{#1}{#2}&amp;=\spn{\set{#3}}} 
%
%  Generalized Eigenspace
%  Usage: \geneigenspace{lin-trans-letter}{eigenvalue-letter}
\newcommand{\geneigenspace}[2]{\mathcal{G}_{#1}\left(#2\right)}
%
%  Algebraic multiplicty
%  Usage: \algmult{matrix-letter}{eigenvalue-letter}
\newcommand{\algmult}[2]{\alpha_{#1}\left(#2\right)}
%
%  Geometric multiplicty
%  Usage: \geomult{matrix-letter}{eigenvalue-letter}
\newcommand{\geomult}[2]{\gamma_{#1}\left(#2\right)}
%
%  Index (of eigenvalue)
%  Usage: \indx{matrix-letter}{eigenvalue-letter}
\newcommand{\indx}[2]{\iota_{#1}\left(#2\right)}
%
%%%%%%%%%%%%%%%%%%%%%
%
%     Linear Transformations
%
%%%%%%%%%%%%%%%%%%%%%
%
%  MathJax defines \lt to ease XML confusion
%
%  Linear transformation definition
%  Usage: \ltdefn{name-letter}{domain}{range}
\newcommand{\ltdefn}[3]{#1\colon #2\rightarrow#3}
%
%  Linear transformation evaluation
%  Usage: \lteval{name-letter}{input}
%  Replaces old \lt desired by MathJax
\newcommand{\lteval}[2]{#1\left(#2\right)}
%
% Linear transformation inverse
%  Usage: \ltinverse{name-letter}
\newcommand{\ltinverse}[1]{#1^{-1}}
%
%  Linear transformation restriction
%  Usage: \restrict{name-letter}{subspace-letter}
\newcommand{\restrict}[2]{{#1}|_{#2}}
%
%  Linear transformation preimage
%  Usage: \preimage{name-letter}{codomain-element}
\newcommand{\preimage}[2]{#1^{-1}\left(#2\right)}
%
%  Range of a linear transformation
%  TeX uses \range for something else
%  Usage:  \rng{T}
\newcommand{\rng}[1]{\mathcal{R}\!\left(#1\right)}
%
%  Kernel of a linear transformation
%  TeX uses \ker to do something different
%  Usage:  \krn{T}
\newcommand{\krn}[1]{\mathcal{K}\!\left(#1\right)}
%
%  Linear transformation composition
%  Usage: \compose{function-name}{function-name}
\newcommand{\compose}[2]{{#1}\circ{#2}}
%
%  Vector space of linear transformations
%  Usage: \vslt{domains}{codomains}
%  Presumes math mode
\newcommand{\vslt}[2]{\mathcal{LT}\left(#1,\,#2\right)}
%
%%%%%%%%%%%%%%%%%%%%%
%
%     Vector and Matrix Representations
%
%%%%%%%%%%%%%%%%%%%%%
%
%  Isomorphism symbol
%  Usage: \isomorphic
\newcommand{\isomorphic}{\cong}
%
%  Similarity
%  Usage: \similar{inner-matrix}{outer-invertible-matrix}
%  Rearranging this will not "fix" all desired changes throughout
%
\newcommand{\similar}[2]{\inverse{#2}#1#2}
%
%  Vector representation function name
%  Usage: \vectrepname{basis-letter}
\newcommand{\vectrepname}[1]{\rho_{#1}}
%
%  Vector representation output
%  Usage: \vectrep{basis-letter}{input}
\newcommand{\vectrep}[2]{\lteval{\vectrepname{#1}}{#2}}
%
%  Vector representation inverse function name
%  (Added later, not used consistently in FCLA)
%  Usage: \vectrepinvname{basis-letter}
\newcommand{\vectrepinvname}[1]{\ltinverse{\vectrepname{#1}}}
%
%  Vector representation inverse output
%  Usage: \vectrepinv{basis-letter}{input}
\newcommand{\vectrepinv}[2]{\lteval{\ltinverse{\vectrepname{#1}}}{#2}}
%
%  Matrix representation
%  Usage: \matrixrep{transformation-letter}{domain-basis-letter}{codomain-basis-letter}
\newcommand{\matrixrep}[3]{M^{#1}_{#2,#3}}
%
%  Matrix representation column-by-colum
%  2016-07-20 only employed once?
%  Usage: \matrixrepcolumns{transformation-letter}{codomain-basis-letter}{codomain-basis-vector-letter}{final-index}
\newcommand{\matrixrepcolumns}[4]{\left\lbrack \left.\vectrep{#2}{\lteval{#1}{\vect{#3}_{1}}}\right|\left.\vectrep{#2}{\lteval{#1}{\vect{#3}_{2}}}\right|\left.\vectrep{#2}{\lteval{#1}{\vect{#3}_{3}}}\right|\ldots\left|\vectrep{#2}{\lteval{#1}{\vect{#3}_{#4}}}\right.\right\rbrack}
%
%  Change of basis matrix
%  Usage: \cbm{domain-basis-letter}{codomain-basis-letter}
\newcommand{\cbm}[2]{C_{#1,#2}}
%
%%%%%%%%%%%%%%%%%%%%%
%
%     Canonical Forms
%
%%%%%%%%%%%%%%%%%%%%%
%
%  Jordan blocks
%  Usage: \jordan{size}{diagonal-element}
\newcommand{\jordan}[2]{J_{#1}\left(#2\right)}
%
%%%%%%%%%%%%%%%%%%%%%
%
%     Hadamard Matrices
%     Contributed by Elizabeth Million
%
%%%%%%%%%%%%%%%%%%%%%
%
%  Hadamard Product
%  Usage: \hadamard{a-matrix}{a-matrix}
\newcommand{\hadamard}[2]{#1\circ #2}
%
%  Hadamard identity matrix
%  Usage: \hadamardidentity{paired-subscripts-size-of-matrix}
\newcommand{\hadamardidentity}[1]{J_{#1}}
%
%  Hadamard inverse matrix
%  Usage: \hadamardinverse{matrix-expression}
\newcommand{\hadamardinverse}[1]{\widehat{#1}}

\newcommand{\definedTerm}[1]{\textbf{#1}}
\newcommand{\dfn}[1]{\textbf{#1}}

\newcommand{\wt}{\widetilde}
\newcommand{\ov}{\overline}
\newcommand{\inj}{\rightarrowtail}
\newcommand{\surj}{\twoheadrightarrow}
\newcommand{\harpoon}{\overset{\rightharpoonup}}

\newenvironment{amatrix}[1]{%
  \left[\begin{array}{@{}*{#1}{c}|c@{}}
}{%
  \end{array}\right]
}


\title{Dimension}

\begin{document}
\begin{abstract}
  Dimension is a measure of the size of a vector space.
\end{abstract}
\maketitle

Almost every vector space we have encountered has been infinite in
size (an exception is \ref{example:VSS}).  But some are ``bigger''
than others.  Dimension, once suitably defined, will be a measure of
the size of a vector space, and a useful tool for studying its
properties.  You probably already have a rough notion of what a
mathematical definition of dimension might be---try to forget these
imprecise ideas and go with the new ones given here.

\begin{definition}[Dimension]\label{definition:D}
  Suppose that $V$ is a vector space and $\set{\vectorlist{v}{t}}$ is
  a basis of $V$.  Then the \dfn{dimension} of $V$ is defined by
  $\dimension{V}=t$.  If $V$ has no finite bases, we say $V$ has
  infinite dimension.
\end{definition}

This is a very simple definition, which belies its power.  Grab a
basis, any basis, and count up the number of vectors it contains.
That is the dimension.  However, this simplicity causes a problem.
Given a vector space, you and I could each construct different bases
<mdash /> remember that a vector space might have many bases.  And
what if your basis and my basis had different sizes?  Applying
\ref{definition:D} we would arrive at different numbers!  With our
current knowledge about vector spaces, we would have to say that
dimension is not ``well-defined.''  Fortunately, there is a theorem
that will correct this problem.

In a strictly logical progression, the next two theorems would
\textit{precede} the definition of dimension.  Many subsequent
theorems will trace their lineage back to the following fundamental
result.

\begin{theorem}[Spanning Sets and Linear Dependence]
  \label{theorem:SSLD}

  Suppose that $S=\set{\vectorlist{v}{t}}$ is a finite set of vectors
  which spans the vector space $V$.  Then any set of $t+1$ or more
  vectors from $V$ is linearly dependent.

\begin{proof}
  We want to prove that any set of $t+1$ or more vectors from $V$ is
  linearly dependent.  So we will begin with a totally arbitrary set
  of vectors from $V$, $R=\set{\vectorlist{u}{m}}$, where
  $\answer{m}>t$.  We will now construct a nontrivial relation of
  linear dependence on $R$.

  Each vector $\vectorlist{u}{m}$ can be written as a linear
  combination of the vectors $\vectorlist{v}{t}$ since $S$ is a
  spanning set of $V$.  This means there exist scalars $a_{ij}$,
  $1\leq i\leq t$, $1\leq j\leq m$, so that
  \begin{align*}
    \vect{u}_1&=a_{11}\vect{v}_1+a_{21}\vect{v}_2+a_{31}\vect{v}_3+\cdots+a_{t1}\vect{v}_t\\
    \vect{u}_2&=a_{12}\vect{v}_1+a_{22}\vect{v}_2+a_{32}\vect{v}_3+\cdots+a_{t2}\vect{v}_t\\
    \vect{u}_3&=a_{13}\vect{v}_1+a_{23}\vect{v}_2+a_{33}\vect{v}_3+\cdots+a_{t3}\vect{v}_t\\
              &\quad\quad\vdots\\
    \vect{u}_m&=a_{1m}\vect{v}_1+a_{2m}\vect{v}_2+a_{3m}\vect{v}_3+\cdots+a_{tm}\vect{v}_t
  \end{align*}

  Now we form, unmotivated, the homogeneous system of $t$ equations in
  the $m$ variables, $x_1,\,x_2,\,x_3,\,\ldots,\,x_m$, where the
  coefficients are the just-discovered scalars $a_{ij}$,
  \begin{align*}
    a_{11}x_1+a_{12}x_2+a_{13}x_3+\cdots+a_{1m}x_m&=0\\
    a_{21}x_1+a_{22}x_2+a_{23}x_3+\cdots+a_{2m}x_m&=0\\
    a_{31}x_1+a_{32}x_2+a_{33}x_3+\cdots+a_{3m}x_m&=0\\
    \vdots\quad\quad&\\
    a_{t1}x_1+a_{t2}x_2+a_{t3}x_3+\cdots+a_{tm}x_m&=0\\
  \end{align*}

  This is a homogeneous system with more variables than equations (our
  hypothesis is expressed as $m>t$), so by \ref{theorem:HMVEI} there
  are infinitely many solutions.  Choose a nontrivial solution and
  denote it by $x_1=c_1,\,x_2=c_2,\,x_3=c_3,\,\ldots,\,x_m=c_m$.  As a
  solution to the homogeneous system, we then have
  \begin{align*}
    a_{11}c_1+a_{12}c_2+a_{13}c_3+\cdots+a_{1m}c_m&=0\\
    a_{21}c_1+a_{22}c_2+a_{23}c_3+\cdots+a_{2m}c_m&=0\\
    a_{31}c_1+a_{32}c_2+a_{33}c_3+\cdots+a_{3m}c_m&=0\\
    \vdots\quad\quad&\\
    a_{t1}c_1+a_{t2}c_2+a_{t3}c_3+\cdots+a_{tm}c_m&=0\\
  \end{align*}

  As a collection of nontrivial scalars,
  $c_1,\,c_2,\,c_3,\,\dots,\,c_m$ will provide the nontrivial relation
  of linear dependence we desire,
  \begin{align*}
    &\lincombo{c}{u}{m}\\
    &=c_{1}\left(a_{11}\vect{v}_1+a_{21}\vect{v}_2+a_{31}\vect{v}_3+\cdots+a_{t1}\vect{v}_t\right)
    &&\ref{definition:SSVS}\\
    &\quad\quad+c_{2}\left(a_{12}\vect{v}_1+a_{22}\vect{v}_2+a_{32}\vect{v}_3+\cdots+a_{t2}\vect{v}_t\right)\\
    &\quad\quad+c_{3}\left(a_{13}\vect{v}_1+a_{23}\vect{v}_2+a_{33}\vect{v}_3+\cdots+a_{t3}\vect{v}_t\right)\\
    &\quad\quad\quad\quad\vdots\\
    &\quad\quad+c_{m}\left(a_{1m}\vect{v}_1+a_{2m}\vect{v}_2+a_{3m}\vect{v}_3+\cdots+a_{tm}\vect{v}_t\right)\\
    &=c_{1}a_{11}\vect{v}_1+c_{1}a_{21}\vect{v}_2+c_{1}a_{31}\vect{v}_3+\cdots+c_{1}a_{t1}\vect{v}_t
    &&\ref{property:DVA}\\
    &\quad\quad+c_{2}a_{12}\vect{v}_1+c_{2}a_{22}\vect{v}_2+c_{2}a_{32}\vect{v}_3+\cdots+c_{2}a_{t2}\vect{v}_t\\
    &\quad\quad+c_{3}a_{13}\vect{v}_1+c_{3}a_{23}\vect{v}_2+c_{3}a_{33}\vect{v}_3+\cdots+c_{3}a_{t3}\vect{v}_t\\
    &\quad\quad\quad\quad\vdots\\
    &\quad\quad+c_{m}a_{1m}\vect{v}_1+c_{m}a_{2m}\vect{v}_2+c_{m}a_{3m}\vect{v}_3+\cdots+c_{m}a_{tm}\vect{v}_t\\
    &=\left(c_{1}a_{11}+c_{2}a_{12}+c_{3}a_{13}+\cdots+c_{m}a_{1m}\right)\vect{v}_1
    &&\ref{property:DSA}\\
    &\quad\quad+\left(c_{1}a_{21}+c_{2}a_{22}+c_{3}a_{23}+\cdots+c_{m}a_{2m}\right)\vect{v}_2\\
    &\quad\quad+\left(c_{1}a_{31}+c_{2}a_{32}+c_{3}a_{33}+\cdots+c_{m}a_{3m}\right)\vect{v}_3\\
    &\quad\quad\quad\quad\vdots\\
    &\quad\quad+\left(c_{1}a_{t1}+c_{2}a_{t2}+c_{3}a_{t3}+\cdots+c_{m}a_{tm}\right)\vect{v}_t\\
    &=\left(a_{11}c_{1}+a_{12}c_{2}+a_{13}c_{3}+\cdots+a_{1m}c_{m}\right)\vect{v}_1
    &&\ref{property:CMCN}\\
    &\quad\quad+\left(a_{21}c_{1}+a_{22}c_{2}+a_{23}c_{3}+\cdots+a_{2m}c_{m}\right)\vect{v}_2\\
    &\quad\quad+\left(a_{31}c_{1}+a_{32}c_{2}+a_{33}c_{3}+\cdots+a_{3m}c_{m}\right)\vect{v}_3\\
    &\quad\quad\quad\quad\vdots\\
    &\quad\quad+\left(a_{t1}c_{1}+a_{t2}c_{2}+a_{t3}c_{3}+\cdots+a_{tm}c_{m}\right)\vect{v}_t\\
    &=0\vect{v}_1+0\vect{v}_2+0\vect{v}_3+\cdots+0\vect{v}_t
    &&\text{$c_j$ as solution}\\
    &=\zerovector+\zerovector+\zerovector+\cdots+\zerovector
    &&\ref{theorem:ZSSM}\\
    &=\zerovector
    &&\ref{property:Z}
  \end{align*}

  That does it.  $R$ has been undeniably shown to be a linearly
  \wordChoice{\choice{independent}\choice[correct]{dependent}} set.

  The proof just given has some monstrous expressions in it, mostly
  owing to the double subscripts present.  Now is a great opportunity
  to show the value of a more compact notation.  We will rewrite the
  key steps of the previous proof using summation notation, resulting
  in a more economical presentation, and even greater insight into the
  key aspects of the proof.  So here is an alternate proof---study it
  carefully.

  \textbf{Alternate Proof:} We want to prove that any set of $t+1$ or
  more vectors from $V$ is linearly dependent.  So we will begin with
  a totally arbitrary set of vectors from $V$,
  $R=\setparts{\vect{u}_j}{1\leq j\leq m}$, where $m>t$.  We will now
  construct a nontrivial relation of linear dependence on $R$.

  Each vector $\vect{u_j}$, $1\leq j\leq m$ can be written as a linear
  combination of $\vect{v}_i$, $1\leq i\leq t$ since $S$ is a spanning
  set of $V$.  This means there are scalars $a_{ij}$, $1\leq i\leq t$,
  $1\leq j\leq m$, so that
  \begin{align*}
    \vect{u}_j&=\sum_{i=1}^{t}a_{ij}\vect{v}_i&&1\leq j\leq m
  \end{align*}

  Now we form, unmotivated, the homogeneous system of $t$ equations in
  the $m$ variables, $x_j$, $1\leq j\leq m$, where the coefficients
  are the just-discovered scalars $a_{ij}$,
  \begin{align*}
    \sum_{j=1}^{m}a_{ij}x_j=0&&1\leq i\leq t
  \end{align*}
  
  This is a homogeneous system with more variables than equations (our
  hypothesis is expressed as $m>t$), so by \ref{theorem:HMVEI} there
  are infinitely many solutions.  Choose one of these solutions that
  is not trivial and denote it by $x_j=c_j$, $1\leq j\leq m$.  As a
  solution to the homogeneous system, we then have
  $\sum_{j=1}^{m}a_{ij}c_{j}=0$ for $1\leq i\leq t$.  As a collection
  of nontrivial scalars, $c_j$, $1\leq j\leq m$, will provide the
  nontrivial relation of linear dependence we desire,
  \begin{align*}
    \sum_{j=1}^{m}c_{j}\vect{u}_j
    &=\sum_{j=1}^{m}c_{j}\left(\sum_{i=1}^{t}a_{ij}\vect{v}_i\right)
    &&\ref{definition:SSVS}\\
    &=\sum_{j=1}^{m}\sum_{i=1}^{t}c_{j}a_{ij}\vect{v}_i
    &&\ref{property:DVA}\\
    &=\sum_{i=1}^{t}\sum_{j=1}^{m}c_{j}a_{ij}\vect{v}_i
    &&\ref{property:C}\\
    &=\sum_{i=1}^{t}\sum_{j=1}^{m}a_{ij}c_{j}\vect{v}_i
    &&\ref{property:CMCN}\\
    &=\sum_{i=1}^{t}\left(\sum_{j=1}^{m}a_{ij}c_{j}\right)\vect{v}_i
    &&\ref{property:DSA}\\
    &=\sum_{i=1}^{t}0\vect{v}_i
    &&\text{$c_j$ as solution}\\
    &=\sum_{i=1}^{t}\zerovector
    &&\ref{theorem:ZSSM}\\
    &=\zerovector
    &&\ref{property:Z}
  \end{align*}
  
  That does it.  $R$ is a linearly dependent set.

\end{proof}
\end{theorem}

Notice how the swap of the two summations is so much easier in the
third step above, as opposed to all the rearranging and regrouping
that takes place in the previous proof.  And using only about half the
space.  And there are no ellipses (\ldots)..

\ref{theorem:SSLD} can be viewed as a generalization of
\ref{theorem:MVSLD}.  We know that $\complex{m}$ has a basis with $m$
vectors in it (\ref{theorem:SUVB}), so it is a set of $m$ vectors that
spans $\complex{m}$.  By \ref{theorem:SSLD}, any set of more than $m$
vectors from $\complex{m}$ will be linearly dependent.  But this is
exactly the conclusion we have in \ref{theorem:MVSLD}.  Maybe this is
not a total shock, as the proofs of both theorems rely heavily on
\ref{theorem:HMVEI}.  The beauty of \ref{theorem:SSLD} is that it
applies in any vector space.  We illustrate the generality of this
theorem, and hint at its power, in the next example.

\begin{example}[Linearly dependent set in $P_4$]
  We showed that
  \[
    S=\set{x-2,\,x^2-4x+4,\,x^3-6x^2+12x-8,\,x^4-8x^3+24x^2-32x+16}
  \]
  is a spanning set for $W=\setparts{p(x)}{p\in P_4,\ p(2)=0}$.  So we
  can apply \ref{theorem:SSLD} to $W$ with $t=4$.  Here is a set of
  five vectors from $W$, as you may check by verifying that each is a
  polynomial of degree 4 or less and has $x=2$ as a root,
  \begin{align*}
    T&=\set{p_1,\,p_2,\,p_3,\,p_4,\,p_5}\subseteq W\\
     &\ \\
    p_1&=x^4-2x^3+2x^2-8x+8\\
    p_2&=-x^3+6x^2-5x-6\\
    p_3&=2x^4-5x^3+5x^2-7x+2\\
    p_4&=-x^4+4x^3-7x^2+6x\\
    p_5&=4x^3-9x^2+5x-6
  \end{align*}
  
  By \ref{theorem:SSLD} we conclude that $T$ is linearly dependent,
  with no further computations.
\end{example}

\ref{theorem:SSLD} is indeed powerful, but our main purpose in proving
it right now was to make sure that our definition of dimension
(\ref{definition:D}) is well-defined.  Here is the theorem.

\begin{theorem}[Bases have Identical Sizes]
  \label{theorem:BIS}

  Suppose that $V$ is a vector space with a finite basis $B$ and a
  second basis $C$.  Then $B$ and $C$ have the same size.

  \begin{proof}
    Suppose that $C$ has more vectors than $B$.  (Allowing for the
    possibility that $C$ is infinite, we can replace $C$ by a subset
    that has more vectors than $B$.)  As a basis, $B$ is a spanning
    set for $V$ (\ref{definition:B}), so \ref{theorem:SSLD} says that
    $C$ is linearly dependent.  However, this contradicts the fact
    that as a basis $C$ is linearly independent (\ref{definition:B}).
    So $C$ must also be a finite set, with size less than, or equal
    to, that of $B$.

    Suppose that $B$ has more vectors than $C$.  As a basis, $C$ is a
    spanning set for $V$ (\ref{definition:B}), so \ref{theorem:SSLD}
    says that $B$ is linearly dependent.  However, this contradicts
    the fact that as a basis $B$ is linearly independent
    (\ref{definition:B}).  So $C$ cannot be strictly smaller than $B$.

    The only possibility left for the sizes of $B$ and $C$ is for them
    to be equal.
  \end{proof}
\end{theorem}

\ref{theorem:BIS} tells us that if we find one finite basis in a
vector space, then they all have the same size.  This (finally) makes
\ref{definition:D} unambiguous.

\end{document}
