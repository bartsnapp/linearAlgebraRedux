\documentclass{ximera}
\author{Rob Beezer}
% These macros are automatically generated from the "macros"
% XML element.  Make permanent edits there.
%
% History
%   2004/01/01  Initiated for FCLA, evolved from there
%   2006/09/17  Converted  _, ^  to \sb, \sp for TeX4ht
%   2014/02/01  Updated for MathBook XML projects
%               Obsolete in FCLA: \codeindent, \computerfont, \define
%               Change: MathJax wants \lt, so replaced by \lteval
%   2014/02/22  New: \orderof, \reals, \per
%   2015/08/16  Incorporated into MathBook XML version of FCLA
%
%%%%%%%%%%%%%%%%%%%%%
%
%     Conveniences
%
%%%%%%%%%%%%%%%%%%%%%
%
%  Order of (asymptotically limit of fraction is 1)
%  Usage: \orderof{some function}
%
\newcommand{\orderof}[1]{\sim #1}
%
%  Integers
%  Usage:  \Z
\newcommand{\Z}{\mathbb{Z}}
%
%  Real numbers, as set of scalars
%  Usage:  \reals
\newcommand{\reals}{\mathbb{R}}
%
%  n-space over real field
%  Usage: \complex{integer-dimension}
\newcommand{\real}[1]{\mathbb{R}^{#1}}
%
%  Complex numbers, as set of scalars
%  Usage:  \complexes
\newcommand{\complexes}{\mathbb{C}}
%
%  n-space over complex field
%  Usage: \complex{integer-dimension}
\newcommand{\complex}[1]{\mathbb{C}^{#1}}
\newcommand{\CC}{\mathbb{C}}
%
%  Complex conjugation (scalar, vector, matrix)
%  Usage: \conjugate{object}
\newcommand{\conjugate}[1]{\overline{#1}}
%
%  Complex number modulus
%  Usage: \modulus{a+bi}
%  Presumes math mode
\newcommand{\modulus}[1]{\left\lvert#1\right\rvert}
%
%  Zero vector
%  Usage: \zerovector
\newcommand{\zerovector}{\vect{0}}
%
%  Zero matrix
%  Usage: \zeromatrix, use a subscript when size is important
\newcommand{\zeromatrix}{\mathcal{O}}
%
%  Inner product (brackets, not quadratic form)
%  Usage: \innerproduct{a-vector}{a-vector}
\newcommand{\innerproduct}[2]{\left\langle#1,\,#2\right\rangle}
%
%  Norm of a vector
%  Usage: \norm{a-vector}
\newcommand{\norm}[1]{\left\lVert#1\right\rVert}
%
%  Dimension
%  Usage: \dimension{vector-space-letter}
\newcommand{\dimension}[1]{\dim\left(#1\right)}
%
%  Nullity
%  Usage: \nullity{matrix-or-lintrans-letter}
\newcommand{\nullity}[1]{n\left(#1\right)}
%
%  Rank
%  Usage: \rank{matrix-or-lintrans-letter}
\newcommand{\rank}[1]{r\left(#1\right)}
%
%  Direct sum
%  Usage: \ds between a couple of subspaces
%
\newcommand{\ds}{\oplus}
%
%  Determinant of a matrix (functional)
%  Usage: \detname{A}
\newcommand{\detname}[1]{\det\left(#1\right)}
%
%  Determinant of a matrix (vertical bars)
%  Usage: \detbars{A}
\newcommand{\detbars}[1]{\left\lvert#1\right\rvert}
%
%  Trace of a Matrix
%  Usage: \trace{matrix name}
\newcommand{\trace}[1]{t\left(#1\right)}
%
%  Square Root of a Matrix
%  Usage: \sr{a-matrix}
\newcommand{\sr}[1]{#1^{1/2}}
%
%%%%%%%%%%%%%%%%%%%%%
%
%     Subspace Constructions
%
%%%%%%%%%%%%%%%%%%%%%
%
%  Span of a set of vectors
%  \span and \sp are used by TeX for other things
%  Usage: \spn{set-of-vectors}
\newcommand{\spn}[1]{\left\langle#1\right\rangle}
%
%  Null space of a matrix
%  Usage:  \nsp{A}
\newcommand{\nsp}[1]{\mathcal{N}\!\left(#1\right)}
%
%  Column space of a matrix
%  Usage:  \csp{A}
\newcommand{\csp}[1]{\mathcal{C}\!\left(#1\right)}
%
%  Row space of a matrix
%  Usage:  \rsp{A}
\newcommand{\rsp}[1]{\mathcal{R}\!\left(#1\right)}
%
%  Left null space of a matrix
%  Usage:  \lns{A}
\newcommand{\lns}[1]{\mathcal{L}\!\left(#1\right)}
%
%  Orthogonal complement of a vector space
%  Avoiding TeX's \perp
%  Usage:  \per{A}
\newcommand{\per}[1]{#1^\perp}
%
%%%%%%%%%%%%%%%%%%%%%
%
%     Systems of Equations
%
%%%%%%%%%%%%%%%%%%%%%
%
%  In-line form of an augmented matrix for a system of equations
%  Usage: \augmented{coefficient-matrix}{constant-vector}
\newcommand{\augmented}[2]{\left\lbrack\left.#1\,\right\rvert\,#2\right\rbrack}
%
%  Notation for a linear system before introducing matrix multiplication
%  Usage: \linearsystem{coefficient-matrix}{constant-vector}
\newcommand{\linearsystem}[2]{\mathcal{LS}\!\left(#1,\,#2\right)}
%
%  Notation for a homogenous system before introducing matrix multiplication
%  Usage: \homosystem{coefficient-matrix}
\newcommand{\homosystem}[1]{\linearsystem{#1}{\zerovector}}
%
%%%%%%%%%%%%%%%%%%%%%
%
%     Row Operations, Echelon Form
%
%%%%%%%%%%%%%%%%%%%%%
%
% Row operations on matrices
%
% Three commands to shorten up descriptions of gaussian elimination
%
% Usage: \rowopswap{row-i}{row-j}
% Usage: \rowopmult{scalar}{row-i}
% Usage: \rowopadd{scalar}{row-multiplied}{row-added-to}
\newcommand{\rowopswap}[2]{R_{#1}\leftrightarrow R_{#2}}
\newcommand{\rowopmult}[2]{#1R_{#2}}
\newcommand{\rowopadd}[3]{#1R_{#2}+R_{#3}}
%
% Mark leading 1's in echelon form with fbox
% Usage: \leading{a-1-usually}
\newcommand{\leading}[1]{\fbox{#1}}
%
%  Row-reduce arrow
%  Usage:  \rref inbetween a matrix and its reduced row-echelon form
\newcommand{\rref}{\xrightarrow{\text{RREF}}}
%
%  Elementary Matrices
%  Usage: \elemswap{subscript}{subscript}
%  Usage: \elemmult{scalar}{subscript}
%  Usage: \elemadd{scalar}{subscript-mult}{subscript-target}
%
\newcommand{\elemswap}[2]{E_{#1,#2}}
\newcommand{\elemmult}[2]{E_{#2}\left(#1\right)}
\newcommand{\elemadd}[3]{E_{#2,#3}\left(#1\right)}
%
%%%%%%%%%%%%%%%%%%%%%
%
%     2-D Constructions (Lists, Vectors, Matrices)
%
%%%%%%%%%%%%%%%%%%%%%
%
%  A list of scalars of generic length
%  Usage:  \scalarlist{scalar letter}{terminal subscript}
\newcommand{\scalarlist}[2]{{#1}_{1},\,{#1}_{2},\,{#1}_{3},\,\ldots,\,{#1}_{#2}}
%
%  Vector styling, bold (or use wiggles, arrows, whatever)
%  Subscripts go outside this construction
%  Usage: \vect{a symbol to use as a vector}
%  Have to already be in math mode
%
\newcommand{\vect}[1]{\mathbf{#1}}
%
%  A column vector
%  Usage: \colvector{list-delimited-by-\\}
%
\newcommand{\colvector}[1]{\begin{bmatrix}#1\end{bmatrix}}
%
%  A generic vector with components
%  Usage: \vectorcomponents{component-letter}{final-subscript}
\newcommand{\vectorcomponents}[2]{\colvector{#1_{1}\\#1_{2}\\#1_{3}\\\vdots\\#1_{#2}}}
%
%  A list of vectors of generic length
%  Usage:  \vectorlist{vector letter}{terminal subscript}
\newcommand{\vectorlist}[2]{\vect{#1}_{1},\,\vect{#1}_{2},\,\vect{#1}_{3},\,\ldots,\,\vect{#1}_{#2}}
%
%  Vector entries, entry i of vector v
%  (vector-expession still needs \vect, etc.)
%  Usage:  \vectorentry{vector-expression}{single-subscript}
\newcommand{\vectorentry}[2]{\left\lbrack#1\right\rbrack_{#2}}
%
%  Matrix entries, entry i,j of matrix A
%  Usage:  \matrixentry{matrix-expression}{paired-subscripts}
%
\newcommand{\matrixentry}[2]{\left\lbrack#1\right\rbrack_{#2}}
%
%  A generic linear combination
%  Usage:  \lincombo{scalar letter}{vector letter}{terminal subscript}
\newcommand{\lincombo}[3]{#1_{1}\vect{#2}_{1}+#1_{2}\vect{#2}_{2}+#1_{3}\vect{#2}_{3}+\cdots +#1_{#3}\vect{#2}_{#3}}
%
%  Matrix, column by column, as vectors
%  Usage:  \matrixcolumns{matrix letter}{terminal subscript}
\newcommand{\matrixcolumns}[2]{\left\lbrack\vect{#1}_{1}|\vect{#1}_{2}|\vect{#1}_{3}|\ldots|\vect{#1}_{#2}\right\rbrack}
%
%%%%%%%%%%%%%%%%%%%%%
%
%     Special Matrices
%
%%%%%%%%%%%%%%%%%%%%%
%
%  Transpose of a matrix
%  Usage:  \transpose{A}
\newcommand{\transpose}[1]{#1^{t}}
%
%  Inverse of a matrix
%  Usage:  \inverse{A}
\newcommand{\inverse}[1]{#1^{-1}}
%
%  Submatrix (for minors, determinants)
%  Usage: \submatrix{matrix-name}{delete-row}{delete-col}
\newcommand{\submatrix}[3]{#1\left(#2|#3\right)}
%
%  Adjoint of a matrix (twice)
%  This macro is a convenience to call \transpose and \conjugate properly
%  It shouldn't need to be modified (or mathematical meanings will change)
%  Usage:  \adj{A}
\newcommand{\adj}[1]{\transpose{\left(\conjugate{#1}\right)}}
%
%  This macro controls the symbol used for the adjoint
%  It can be edited to taste
%  Usage:  \adjoint{A}
\newcommand{\adjoint}[1]{#1^\ast}
%
%%%%%%%%%%%%%%%%%%%%%
%
%     Sets
%
%%%%%%%%%%%%%%%%%%%%%
%
%  A convenience for simple sets
%  Usage:  \set{list of element}
\newcommand{\set}[1]{\left\{#1\right\}}
%
%  Sets with vertical bar, "such that", sized for objects, not condition
%  Usage:  \setparts{objects}{condition}
%
%%\newcommand{\setparts}[2]{\left\{ #1\mid#2\right\}}
%%\newcommand{\setparts}[2]{\left\{\left. #1\right\rvert#2\right\}}
\newcommand{\setparts}[2]{\left\lbrace#1\,\middle|\,#2\right\rbrace}
%
%  Set Cardinality
%  Usage:  \card{a-set-letter}
\newcommand{\card}[1]{\left\lvert#1\right\rvert}
%
%  Set Union
%  Use \cup
%
%  Set Intersection
%  Use \cap
%
%  Set Complement
%  Usage:  \setcomplement{a-set-letter}
\newcommand{\setcomplement}[1]{\overline{#1}}
%
%%%%%%%%%%%%%%%%%%%%%
%
%     Eigenvalues and Eigenspaces
%
%%%%%%%%%%%%%%%%%%%%%
%
%  Characteristic polynomial
%  Usage: \charpoly{matrix-letter}{variable-letter}
\newcommand{\charpoly}[2]{p_{#1}\left(#2\right)}
%
%  Eigenspace
%  Usage: \eigenspace{matrix-letter}{eigenvalue-letter}
\newcommand{\eigenspace}[2]{\mathcal{E}_{#1}\left(#2\right)}
%
%  2013/10/03 Including ampersands is problematic here, 
%  think about fixes later
%  2014/02/22 Limited testing, seems &amp; is fine for HTML and LaTeX
%  2016-07-20 only employed in Archetypes, MBX has gather/align override
%  Eigensystem (presumes wrapped in an mrow within md)
%  Usage: \eigensystem{matrixletter}{eigenvalue}{list of basis vectors}
\newcommand{\eigensystem}[3]{\lambda&amp;=#2&amp;\eigenspace{#1}{#2}&amp;=\spn{\set{#3}}} 
%
%  Generalized Eigenspace
%  Usage: \geneigenspace{lin-trans-letter}{eigenvalue-letter}
\newcommand{\geneigenspace}[2]{\mathcal{G}_{#1}\left(#2\right)}
%
%  Algebraic multiplicty
%  Usage: \algmult{matrix-letter}{eigenvalue-letter}
\newcommand{\algmult}[2]{\alpha_{#1}\left(#2\right)}
%
%  Geometric multiplicty
%  Usage: \geomult{matrix-letter}{eigenvalue-letter}
\newcommand{\geomult}[2]{\gamma_{#1}\left(#2\right)}
%
%  Index (of eigenvalue)
%  Usage: \indx{matrix-letter}{eigenvalue-letter}
\newcommand{\indx}[2]{\iota_{#1}\left(#2\right)}
%
%%%%%%%%%%%%%%%%%%%%%
%
%     Linear Transformations
%
%%%%%%%%%%%%%%%%%%%%%
%
%  MathJax defines \lt to ease XML confusion
%
%  Linear transformation definition
%  Usage: \ltdefn{name-letter}{domain}{range}
\newcommand{\ltdefn}[3]{#1\colon #2\rightarrow#3}
%
%  Linear transformation evaluation
%  Usage: \lteval{name-letter}{input}
%  Replaces old \lt desired by MathJax
\newcommand{\lteval}[2]{#1\left(#2\right)}
%
% Linear transformation inverse
%  Usage: \ltinverse{name-letter}
\newcommand{\ltinverse}[1]{#1^{-1}}
%
%  Linear transformation restriction
%  Usage: \restrict{name-letter}{subspace-letter}
\newcommand{\restrict}[2]{{#1}|_{#2}}
%
%  Linear transformation preimage
%  Usage: \preimage{name-letter}{codomain-element}
\newcommand{\preimage}[2]{#1^{-1}\left(#2\right)}
%
%  Range of a linear transformation
%  TeX uses \range for something else
%  Usage:  \rng{T}
\newcommand{\rng}[1]{\mathcal{R}\!\left(#1\right)}
%
%  Kernel of a linear transformation
%  TeX uses \ker to do something different
%  Usage:  \krn{T}
\newcommand{\krn}[1]{\mathcal{K}\!\left(#1\right)}
%
%  Linear transformation composition
%  Usage: \compose{function-name}{function-name}
\newcommand{\compose}[2]{{#1}\circ{#2}}
%
%  Vector space of linear transformations
%  Usage: \vslt{domains}{codomains}
%  Presumes math mode
\newcommand{\vslt}[2]{\mathcal{LT}\left(#1,\,#2\right)}
%
%%%%%%%%%%%%%%%%%%%%%
%
%     Vector and Matrix Representations
%
%%%%%%%%%%%%%%%%%%%%%
%
%  Isomorphism symbol
%  Usage: \isomorphic
\newcommand{\isomorphic}{\cong}
%
%  Similarity
%  Usage: \similar{inner-matrix}{outer-invertible-matrix}
%  Rearranging this will not "fix" all desired changes throughout
%
\newcommand{\similar}[2]{\inverse{#2}#1#2}
%
%  Vector representation function name
%  Usage: \vectrepname{basis-letter}
\newcommand{\vectrepname}[1]{\rho_{#1}}
%
%  Vector representation output
%  Usage: \vectrep{basis-letter}{input}
\newcommand{\vectrep}[2]{\lteval{\vectrepname{#1}}{#2}}
%
%  Vector representation inverse function name
%  (Added later, not used consistently in FCLA)
%  Usage: \vectrepinvname{basis-letter}
\newcommand{\vectrepinvname}[1]{\ltinverse{\vectrepname{#1}}}
%
%  Vector representation inverse output
%  Usage: \vectrepinv{basis-letter}{input}
\newcommand{\vectrepinv}[2]{\lteval{\ltinverse{\vectrepname{#1}}}{#2}}
%
%  Matrix representation
%  Usage: \matrixrep{transformation-letter}{domain-basis-letter}{codomain-basis-letter}
\newcommand{\matrixrep}[3]{M^{#1}_{#2,#3}}
%
%  Matrix representation column-by-colum
%  2016-07-20 only employed once?
%  Usage: \matrixrepcolumns{transformation-letter}{codomain-basis-letter}{codomain-basis-vector-letter}{final-index}
\newcommand{\matrixrepcolumns}[4]{\left\lbrack \left.\vectrep{#2}{\lteval{#1}{\vect{#3}_{1}}}\right|\left.\vectrep{#2}{\lteval{#1}{\vect{#3}_{2}}}\right|\left.\vectrep{#2}{\lteval{#1}{\vect{#3}_{3}}}\right|\ldots\left|\vectrep{#2}{\lteval{#1}{\vect{#3}_{#4}}}\right.\right\rbrack}
%
%  Change of basis matrix
%  Usage: \cbm{domain-basis-letter}{codomain-basis-letter}
\newcommand{\cbm}[2]{C_{#1,#2}}
%
%%%%%%%%%%%%%%%%%%%%%
%
%     Canonical Forms
%
%%%%%%%%%%%%%%%%%%%%%
%
%  Jordan blocks
%  Usage: \jordan{size}{diagonal-element}
\newcommand{\jordan}[2]{J_{#1}\left(#2\right)}
%
%%%%%%%%%%%%%%%%%%%%%
%
%     Hadamard Matrices
%     Contributed by Elizabeth Million
%
%%%%%%%%%%%%%%%%%%%%%
%
%  Hadamard Product
%  Usage: \hadamard{a-matrix}{a-matrix}
\newcommand{\hadamard}[2]{#1\circ #2}
%
%  Hadamard identity matrix
%  Usage: \hadamardidentity{paired-subscripts-size-of-matrix}
\newcommand{\hadamardidentity}[1]{J_{#1}}
%
%  Hadamard inverse matrix
%  Usage: \hadamardinverse{matrix-expression}
\newcommand{\hadamardinverse}[1]{\widehat{#1}}

\newcommand{\definedTerm}[1]{\textbf{#1}}
\newcommand{\dfn}[1]{\textbf{#1}}

\newcommand{\wt}{\widetilde}
\newcommand{\ov}{\overline}
\newcommand{\inj}{\rightarrowtail}
\newcommand{\surj}{\twoheadrightarrow}
\newcommand{\harpoon}{\overset{\rightharpoonup}}

\newenvironment{amatrix}[1]{%
  \left[\begin{array}{@{}*{#1}{c}|c@{}}
}{%
  \end{array}\right]
}


\title{Consistent Systems}

\begin{document}
\begin{abstract}
  With row reduction we can systematically handle the situation when we
  have infinitely many solutions to a system.
\end{abstract}
\maketitle

We will now be more careful about analyzing the reduced row-echelon form derived from the  augmented matrix of a system of linear equations.    In particular, we will see how to systematically handle the situation when we have infinitely many solutions to a system, and we will prove that every system of linear equations has either zero, one or infinitely many solutions.  With these tools, we will be able to routinely solve any linear system.

\begin{definition}[Consistent System]
\label{definition:CS}
A system of linear equations is \dfn{consistent} if it has at least one solution.  Otherwise, the system is called \dfn{inconsistent}.
\end{definition}

\begin{exercise}
  Is the system
  \begin{align*}
    x_1 + 4x_2 + 3x_3 - x_4 &= 5\\
    x_1 - x_2 + x_3 + 2x_4 &= 6\\
    4x_1 + x_2 + 6x_3 + 5x_4 &= 9
  \end{align*}
  consistent?

  \begin{multipleChoice}
    \choice{Yes, it is consistent.}
    \choice[correct]{No, it is inconsistent.}
  \end{multipleChoice}

  \begin{hint}
    The augmented matrix for the given linear system is
    \[
      \begin{bmatrix}
        1 & 4 & 3 & -1 & 5\\
        1 & -1 & 1 & 2 & 6\\
        4 & 1 & 6 & 5 & 9
      \end{bmatrix}
    \].
  \end{hint}

  \begin{hint}
    The augmented matrix row-reduces to
    \[\begin{bmatrix}
        \leading{1} & 0 & 7/5 & 7/5 & 0\\
        0 & \leading{1} & 2/5 & -3/5 & 0\\
        0 & 0 & 0 & 0 & \leading{1}
      \end{bmatrix}.\]
  \end{hint}

  \begin{hint}
    For this system, we have $n = 4$ and $r = 3$.  However, with a
    pivot column in the last column we see that the original system
    has no solution.
  \end{hint}
\end{exercise}

We will want to first recognize when a system is inconsistent or
consistent, and in the case of consistent systems we will be able to
further refine the types of solutions possible.  We will do this by
analyzing the reduced row-echelon form of a matrix, using the value of
$r$, and the sets of column indices, $D$ and $F$.

Use of the notation for the elements of $D$ and $F$ can be a bit
confusing, since we have subscripted variables that are in turn equal
to integers used to index the matrix.  However, many questions about
matrices and systems of equations can be answered once we know $r$,
$D$ and $F$.  The choice of the letters $D$ and $F$ refer to our
upcoming definition of dependent and free variables.  An example will
help us begin to get comfortable with this aspect of reduced
row-echelon form.

\begin{example}[Reduced row-echelon form notation]
For the $5\times 9$ matrix
\[
B=
\begin{bmatrix}
\leading{1}&5&0&0&2&8&0&5&-1\\
0&0&\leading{1}&0&4&7&0&2&0\\
0&0&0&\leading{1}&3&9&0&3&-6\\
0&0&0&0&0&0&\leading{1}&4&2\\
0&0&0&0&0&0&0&0&0
\end{bmatrix}
\]
in reduced row-echelon form we have
\begin{align*}
r&=4\\
d_1&=\answer{1}
&
d_2&=3 \\
d_3&=4
&
d_4&=7\\
f_1&=\answer{2}
&
f_2&=5 \\
&
f_3&=6
&
f_4&=8\\
&
f_5&=9
\end{align*}

Notice that the sets
\begin{align*}
D&=\set{d_1,\,d_2,\,d_3,\,d_4}\\
  &=\set{\answer{1},\,3,\,4,\,7} \\
F&=\set{f_1,\,f_2,\,f_3,\,f_4,\,f_5}\\
  &=\set{\answer{2},\,5,\,6,\,8,\,9}
\end{align*}
have nothing in common and together account for all of the columns of $B$ (we say it is a \dfn{partition} of the set of column indices).
\end{example}

The number $r$ is \textbf{the single most important piece of
  information we can get from the reduced row-echelon form of a
  matrix.}  It is defined as the number of nonzero rows, but since
each nonzero row has a leading 1, it is also the number of leading 1's
present.  For each leading 1, we have a pivot column, so $r$ is also
the number of pivot columns.  Repeating ourselves, $r$ is the number
of nonzero rows, the number of leading 1's \textit{and} the number of
pivot columns.  Across different situations, each of these
interpretations of the meaning of $r$ will be useful, though it may be
most helpful to think in terms of pivot columns.

Before proving some theorems about the possibilities for solution sets
to systems of equations, let us analyze one particular system with an
infinite solution set very carefully as an example.  We will use this
technique frequently, and shortly we will refine it slightly.


\begin{example}[Describing infinite solution sets]

Consider this system of $m=4$ equations in $n=7$ variables:
\begin{align*}
x_1 +4x_2  - x_4  + 7x_6 - 9x_7 &= 3\\
2x_1 + 8x_2 - x_3 + 3x_4 + 9x_5 - 13x_6 + 7x_7 &= 9\\
 2x_3 -3x_4 -4x_5 +12x_6 -8x_7 &= 1\\
-x_1  - 4x_2 + 2x_3 +4x_4 + 8x_5 - 31x_6 + 37x_7 &= 4
\end{align*}
This system has a $4\times 8$ augmented matrix that is row-equivalent to the following matrix:
\[
\begin{bmatrix}
\leading{1} & 4 & 0 & 0 & 2 & 1 & -3 & 4 \\
0 & 0 & \leading{1}  &  0 & 1 &  -3 & 5 & 2 \\
0 & 0 &  0 & \leading{1} & 2 &  -6 & 6 & 1 \\
0 & 0 &  0 & 0 & 0 & 0 & 0 & 0
\end{bmatrix}
\]
Is this matrix in reduced row-echelon form?
\begin{multipleChoice}
  \choice[correct]{Yes.}
  \choice{No.}
\end{multipleChoice}
We find that $r=\answer{3}$ and
\begin{align*}
  D&=\set{d_1,\,d_2,\,d_3}=\set{1,\,3,\,4}
  &
    F&=\set{f_1,\,f_2,\,f_3,\,f_4,\,f_5}=\set{2,\,5,\,6,\,7,\,8}
\end{align*}
Let $i$ denote any one of the $r$ nonzero rows.  Then the index $d_i$ is a pivot column.  It will be easy in this case to use the equation represented by row $i$ to write an expression for the variable $x_{d_i}$.  It will be a linear function of the variables $x_{f_1},\,x_{f_2},\,x_{f_3},\,x_{f_4}$ (notice that $f_5=8$ does not reference a variable, but does tell us the final column is not a pivot column).  We will now construct these three expressions.  Notice that using subscripts upon subscripts takes some getting used to.
\begin{align*}
(i=1)& & x_{d_1}&=x_1=4-4x_2-2x_5-x_6+3x_7\\
(i=2)& & x_{d_2}&=x_3=2-x_5+3x_6-5x_7\\
(i=3)& & x_{d_3}&=x_4=1-2x_5+6x_6-6x_7
\end{align*}

Each element of the set $F=\set{f_1,\,f_2,\,f_3,\,f_4,\,f_5}=\set{2,\,5,\,6,\,7,\,8}$ is the index of a variable, except for $f_5=8$.  We refer to $x_{f_1}=x_2$, $x_{f_2}=x_5$, $x_{f_3}=x_6$ and $x_{f_4}=x_7$ as ``free'' (or ``independent'') variables since they are allowed to assume any possible combination of values that we can imagine and we can continue on to build a solution to the system by solving individual equations for the values of the other (``dependent'') variables.

Each element of the set $D=\set{d_1,\,d_2,\,d_3}=\set{1,\,3,\,4}$ is the index of a variable.  We refer to the variables $x_{d_1}=x_1$, $x_{d_2}=x_3$ and $x_{d_3}=x_4$ as  
\wordChoice{\choice[correct]{dependent}\choice{independent}} variables since they \textit{depend} on the \textit{independent} variables.  More precisely, for each possible choice of values for the independent variables we get \textit{exactly one} set of values for the dependent variables that combine to form a solution of the system.

To express the solutions as a set, we write
\[
\setparts{
\colvector{
4-4x_2-2x_5-x_6+3x_7\\
x_2\\
2-x_5+3x_6-5x_7\\
1-2x_5+6x_6-6x_7\\
x_5\\
x_6\\
x_7
}
}{
x_2,\,x_5,\,x_6,\,x_7\in\complexes
}
\]

The condition that $x_2,\,x_5,\,x_6,\,x_7\in\complex{\null}$ is how we specify that the variables $x_2,\,x_5,\,x_6,\,x_7$ are ``free'' to assume any possible values.
\end{example}

This systematic approach to solving a system of equations will allow us to create a precise description of the solution set for any consistent system once we have found the reduced row-echelon form of the augmented matrix.  It will work just as well when the set of free variables is empty and we get just a single solution.

Using the reduced row-echelon form of the augmented matrix of a system of equations to determine the nature of the solution set of the system is a very key idea.  So let us look at one more example like the last one.  But first a definition, and then the example.   We mix our metaphors a bit when we call variables free versus dependent.

\begin{definition}[Independent and Dependent Variables]
Suppose $A$ is the augmented matrix of a consistent system of linear equations and $B$ is a row-equivalent matrix in reduced row-echelon form.  Suppose $j$ is the index of a pivot column of $B$.  Then the variable $x_j$ is \dfn{dependent}.  A variable that is not dependent is called \dfn{independent} or \dfn{free}.
\end{definition}

If you studied this definition carefully, you might wonder what to do if the system has $n$ variables and column $n+1$ is a pivot column?  We will see shortly that this never happens for a consistent system.

\begin{example}[Free and dependent variables]

Consider the system of five equations in five variables,
\begin{align*}
 x_1  - x_2  -2 x_3 +  x_4 + 11 x_5 &= 13 \\
x_1 - x_2 +  x_3+  x_4 + 5 x_5 &= 16 \\
 2 x_1  -2 x_2       +  x_4 + 10 x_5 &= 21 \\
 2 x_1  -2 x_2  - x_3 + 3 x_4 + 20 x_5 &= 38 \\
 2 x_1  -2 x_ 2 +  x_3 +  x_4 + 8 x_ 5&= 22
\end{align*}
whose augmented matrix row-reduces to
\[
\begin{bmatrix}
 \leading{1} & -1 & 0 & 0 & 3 & 6 \\
 0 & 0 & \leading{1} & 0 & -2 & 1 \\
 0 & 0 & 0 & \leading{1} & 4 & 9 \\
 0 & 0 & 0 & 0 & 0 & 0 \\
 0 & 0 & 0 & 0 & 0 & 0
\end{bmatrix}
\]

Columns 1, 3 and \answer{4} are pivot columns.

\begin{feedback}[correct]
  Since $D=\set{1,\,3,\,\answer{4}}$, we know that the variables
  $x_1$, $x_3$ and $x_4$ will be dependent variables, and each of the
  $r=\answer{3}$ nonzero rows of the row-reduced matrix will yield an
  expression for one of these three variables.  The set $F$ is all the
  remaining column indices, $F=\set{2,\,5,\,6}$.  The column index $6$
  in $F$ means that the final column is not a pivot column, and thus
  the system is
  \wordChoice{\choice{inconsistent}\choice[correct]{consistent}}.  The
  remaining indices in $F$ indicate free variables, so $x_2$ and $x_5$
  (the remaining variables) are our free variables.  The resulting
  three equations that describe our solution set are then,
  \begin{align*}
    (x_{d_1}=x_1)& & x_1&=6+x_2-3x_5\\
    (x_{d_2}=x_3)& & x_3&=1+2x_5\\
    (x_{d_3}=x_4)& & x_4&=9-4x_5
  \end{align*}
  
  Make sure you understand where these three equations came from, and
  notice how the location of the pivot columns determined the
  variables on the left-hand side of each equation.  We can compactly
  describe the solution set as,
  \[
    S=
    \setparts{
      \colvector{6+x_2-3x_5\\x_2\\1+2x_5\\9-4x_5\\x_5}
    }{x_2,\,x_5\in\complexes}
  \]
  
  Notice how we express the freedom for $x_2$ and $x_5$: $x_2,\,x_5\in\complexes$.
\end{feedback}
\end{example}

Notice that for a consistent system the row-reduced augmented matrix
has $n+1\in F$, so the largest element of $F$ does not refer to a
variable.  Also, for an inconsistent system, $n+1\in D$, and it then
does not make much sense to discuss whether or not variables are free
or dependent since there is no solution.  Take a look back at the
definition and see why we did not need to consider the possibility of
referencing $x_{n+1}$ as a dependent variable.

We can explore the relationships between $r$ and $n$ for a consistent
system.  We can distinguish between the case of a unique solution and
infinitely many solutions, and furthermore, we recognize that these
are the only two possibilities.

\begin{theorem}[Consistent Systems]
\label{theorem:CSRN}

Suppose $A$ is the augmented matrix of a \textit{consistent} system of
linear equations with $n$ variables.  Suppose also that $B$ is a
row-equivalent matrix in reduced row-echelon form with $r$ pivot
columns.  Then $r\leq n$.  If $r=n$, then the system has a unique
solution, and if $r<n$, then the system has infinitely many solutions.

\begin{proof}
  This theorem contains three implications that we must establish.
  Notice first that $B$ has $n+1$ columns, so there can be at most
  $n+1$ pivot columns, i.e., $r\leq n+1$.  If $r=n+1$, then every
  column of $B$ is a pivot column, and in particular, the last column
  is a pivot column.  So the system is inconsistent, contrary to our
  hypothesis. We are left with $r\leq n$.

  When $r=n$, we find $n-r=0$ free variables (i.e., $F=\set{n+1}$) and
  the only solution is given by setting the $n$ variables to the the
  first $n$ entries of column $n+1$ of $B$.

  When $r<n$, we have $n-r>0$ free variables.  Choose one free
  variable and set all the other free variables to zero.  Now, set the
  chosen free variable to any fixed value.  It is possible to then
  determine the values of the dependent variables to create a solution
  to the system.  By setting the chosen free variable to different
  values, in this manner we can create infinitely many solutions.
\end{proof}
\end{theorem}

\end{document}
