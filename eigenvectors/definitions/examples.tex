\documentclass{ximera}

% These macros are automatically generated from the "macros"
% XML element.  Make permanent edits there.
%
% History
%   2004/01/01  Initiated for FCLA, evolved from there
%   2006/09/17  Converted  _, ^  to \sb, \sp for TeX4ht
%   2014/02/01  Updated for MathBook XML projects
%               Obsolete in FCLA: \codeindent, \computerfont, \define
%               Change: MathJax wants \lt, so replaced by \lteval
%   2014/02/22  New: \orderof, \reals, \per
%   2015/08/16  Incorporated into MathBook XML version of FCLA
%
%%%%%%%%%%%%%%%%%%%%%
%
%     Conveniences
%
%%%%%%%%%%%%%%%%%%%%%
%
%  Order of (asymptotically limit of fraction is 1)
%  Usage: \orderof{some function}
%
\newcommand{\orderof}[1]{\sim #1}
%
%  Integers
%  Usage:  \Z
\newcommand{\Z}{\mathbb{Z}}
%
%  Real numbers, as set of scalars
%  Usage:  \reals
\newcommand{\reals}{\mathbb{R}}
%
%  n-space over real field
%  Usage: \complex{integer-dimension}
\newcommand{\real}[1]{\mathbb{R}^{#1}}
%
%  Complex numbers, as set of scalars
%  Usage:  \complexes
\newcommand{\complexes}{\mathbb{C}}
%
%  n-space over complex field
%  Usage: \complex{integer-dimension}
\newcommand{\complex}[1]{\mathbb{C}^{#1}}
\newcommand{\CC}{\mathbb{C}}
%
%  Complex conjugation (scalar, vector, matrix)
%  Usage: \conjugate{object}
\newcommand{\conjugate}[1]{\overline{#1}}
%
%  Complex number modulus
%  Usage: \modulus{a+bi}
%  Presumes math mode
\newcommand{\modulus}[1]{\left\lvert#1\right\rvert}
%
%  Zero vector
%  Usage: \zerovector
\newcommand{\zerovector}{\vect{0}}
%
%  Zero matrix
%  Usage: \zeromatrix, use a subscript when size is important
\newcommand{\zeromatrix}{\mathcal{O}}
%
%  Inner product (brackets, not quadratic form)
%  Usage: \innerproduct{a-vector}{a-vector}
\newcommand{\innerproduct}[2]{\left\langle#1,\,#2\right\rangle}
%
%  Norm of a vector
%  Usage: \norm{a-vector}
\newcommand{\norm}[1]{\left\lVert#1\right\rVert}
%
%  Dimension
%  Usage: \dimension{vector-space-letter}
\newcommand{\dimension}[1]{\dim\left(#1\right)}
%
%  Nullity
%  Usage: \nullity{matrix-or-lintrans-letter}
\newcommand{\nullity}[1]{n\left(#1\right)}
%
%  Rank
%  Usage: \rank{matrix-or-lintrans-letter}
\newcommand{\rank}[1]{r\left(#1\right)}
%
%  Direct sum
%  Usage: \ds between a couple of subspaces
%
\newcommand{\ds}{\oplus}
%
%  Determinant of a matrix (functional)
%  Usage: \detname{A}
\newcommand{\detname}[1]{\det\left(#1\right)}
%
%  Determinant of a matrix (vertical bars)
%  Usage: \detbars{A}
\newcommand{\detbars}[1]{\left\lvert#1\right\rvert}
%
%  Trace of a Matrix
%  Usage: \trace{matrix name}
\newcommand{\trace}[1]{t\left(#1\right)}
%
%  Square Root of a Matrix
%  Usage: \sr{a-matrix}
\newcommand{\sr}[1]{#1^{1/2}}
%
%%%%%%%%%%%%%%%%%%%%%
%
%     Subspace Constructions
%
%%%%%%%%%%%%%%%%%%%%%
%
%  Span of a set of vectors
%  \span and \sp are used by TeX for other things
%  Usage: \spn{set-of-vectors}
\newcommand{\spn}[1]{\left\langle#1\right\rangle}
%
%  Null space of a matrix
%  Usage:  \nsp{A}
\newcommand{\nsp}[1]{\mathcal{N}\!\left(#1\right)}
%
%  Column space of a matrix
%  Usage:  \csp{A}
\newcommand{\csp}[1]{\mathcal{C}\!\left(#1\right)}
%
%  Row space of a matrix
%  Usage:  \rsp{A}
\newcommand{\rsp}[1]{\mathcal{R}\!\left(#1\right)}
%
%  Left null space of a matrix
%  Usage:  \lns{A}
\newcommand{\lns}[1]{\mathcal{L}\!\left(#1\right)}
%
%  Orthogonal complement of a vector space
%  Avoiding TeX's \perp
%  Usage:  \per{A}
\newcommand{\per}[1]{#1^\perp}
%
%%%%%%%%%%%%%%%%%%%%%
%
%     Systems of Equations
%
%%%%%%%%%%%%%%%%%%%%%
%
%  In-line form of an augmented matrix for a system of equations
%  Usage: \augmented{coefficient-matrix}{constant-vector}
\newcommand{\augmented}[2]{\left\lbrack\left.#1\,\right\rvert\,#2\right\rbrack}
%
%  Notation for a linear system before introducing matrix multiplication
%  Usage: \linearsystem{coefficient-matrix}{constant-vector}
\newcommand{\linearsystem}[2]{\mathcal{LS}\!\left(#1,\,#2\right)}
%
%  Notation for a homogenous system before introducing matrix multiplication
%  Usage: \homosystem{coefficient-matrix}
\newcommand{\homosystem}[1]{\linearsystem{#1}{\zerovector}}
%
%%%%%%%%%%%%%%%%%%%%%
%
%     Row Operations, Echelon Form
%
%%%%%%%%%%%%%%%%%%%%%
%
% Row operations on matrices
%
% Three commands to shorten up descriptions of gaussian elimination
%
% Usage: \rowopswap{row-i}{row-j}
% Usage: \rowopmult{scalar}{row-i}
% Usage: \rowopadd{scalar}{row-multiplied}{row-added-to}
\newcommand{\rowopswap}[2]{R_{#1}\leftrightarrow R_{#2}}
\newcommand{\rowopmult}[2]{#1R_{#2}}
\newcommand{\rowopadd}[3]{#1R_{#2}+R_{#3}}
%
% Mark leading 1's in echelon form with fbox
% Usage: \leading{a-1-usually}
\newcommand{\leading}[1]{\fbox{#1}}
%
%  Row-reduce arrow
%  Usage:  \rref inbetween a matrix and its reduced row-echelon form
\newcommand{\rref}{\xrightarrow{\text{RREF}}}
%
%  Elementary Matrices
%  Usage: \elemswap{subscript}{subscript}
%  Usage: \elemmult{scalar}{subscript}
%  Usage: \elemadd{scalar}{subscript-mult}{subscript-target}
%
\newcommand{\elemswap}[2]{E_{#1,#2}}
\newcommand{\elemmult}[2]{E_{#2}\left(#1\right)}
\newcommand{\elemadd}[3]{E_{#2,#3}\left(#1\right)}
%
%%%%%%%%%%%%%%%%%%%%%
%
%     2-D Constructions (Lists, Vectors, Matrices)
%
%%%%%%%%%%%%%%%%%%%%%
%
%  A list of scalars of generic length
%  Usage:  \scalarlist{scalar letter}{terminal subscript}
\newcommand{\scalarlist}[2]{{#1}_{1},\,{#1}_{2},\,{#1}_{3},\,\ldots,\,{#1}_{#2}}
%
%  Vector styling, bold (or use wiggles, arrows, whatever)
%  Subscripts go outside this construction
%  Usage: \vect{a symbol to use as a vector}
%  Have to already be in math mode
%
\newcommand{\vect}[1]{\mathbf{#1}}
%
%  A column vector
%  Usage: \colvector{list-delimited-by-\\}
%
\newcommand{\colvector}[1]{\begin{bmatrix}#1\end{bmatrix}}
%
%  A generic vector with components
%  Usage: \vectorcomponents{component-letter}{final-subscript}
\newcommand{\vectorcomponents}[2]{\colvector{#1_{1}\\#1_{2}\\#1_{3}\\\vdots\\#1_{#2}}}
%
%  A list of vectors of generic length
%  Usage:  \vectorlist{vector letter}{terminal subscript}
\newcommand{\vectorlist}[2]{\vect{#1}_{1},\,\vect{#1}_{2},\,\vect{#1}_{3},\,\ldots,\,\vect{#1}_{#2}}
%
%  Vector entries, entry i of vector v
%  (vector-expession still needs \vect, etc.)
%  Usage:  \vectorentry{vector-expression}{single-subscript}
\newcommand{\vectorentry}[2]{\left\lbrack#1\right\rbrack_{#2}}
%
%  Matrix entries, entry i,j of matrix A
%  Usage:  \matrixentry{matrix-expression}{paired-subscripts}
%
\newcommand{\matrixentry}[2]{\left\lbrack#1\right\rbrack_{#2}}
%
%  A generic linear combination
%  Usage:  \lincombo{scalar letter}{vector letter}{terminal subscript}
\newcommand{\lincombo}[3]{#1_{1}\vect{#2}_{1}+#1_{2}\vect{#2}_{2}+#1_{3}\vect{#2}_{3}+\cdots +#1_{#3}\vect{#2}_{#3}}
%
%  Matrix, column by column, as vectors
%  Usage:  \matrixcolumns{matrix letter}{terminal subscript}
\newcommand{\matrixcolumns}[2]{\left\lbrack\vect{#1}_{1}|\vect{#1}_{2}|\vect{#1}_{3}|\ldots|\vect{#1}_{#2}\right\rbrack}
%
%%%%%%%%%%%%%%%%%%%%%
%
%     Special Matrices
%
%%%%%%%%%%%%%%%%%%%%%
%
%  Transpose of a matrix
%  Usage:  \transpose{A}
\newcommand{\transpose}[1]{#1^{t}}
%
%  Inverse of a matrix
%  Usage:  \inverse{A}
\newcommand{\inverse}[1]{#1^{-1}}
%
%  Submatrix (for minors, determinants)
%  Usage: \submatrix{matrix-name}{delete-row}{delete-col}
\newcommand{\submatrix}[3]{#1\left(#2|#3\right)}
%
%  Adjoint of a matrix (twice)
%  This macro is a convenience to call \transpose and \conjugate properly
%  It shouldn't need to be modified (or mathematical meanings will change)
%  Usage:  \adj{A}
\newcommand{\adj}[1]{\transpose{\left(\conjugate{#1}\right)}}
%
%  This macro controls the symbol used for the adjoint
%  It can be edited to taste
%  Usage:  \adjoint{A}
\newcommand{\adjoint}[1]{#1^\ast}
%
%%%%%%%%%%%%%%%%%%%%%
%
%     Sets
%
%%%%%%%%%%%%%%%%%%%%%
%
%  A convenience for simple sets
%  Usage:  \set{list of element}
\newcommand{\set}[1]{\left\{#1\right\}}
%
%  Sets with vertical bar, "such that", sized for objects, not condition
%  Usage:  \setparts{objects}{condition}
%
%%\newcommand{\setparts}[2]{\left\{ #1\mid#2\right\}}
%%\newcommand{\setparts}[2]{\left\{\left. #1\right\rvert#2\right\}}
\newcommand{\setparts}[2]{\left\lbrace#1\,\middle|\,#2\right\rbrace}
%
%  Set Cardinality
%  Usage:  \card{a-set-letter}
\newcommand{\card}[1]{\left\lvert#1\right\rvert}
%
%  Set Union
%  Use \cup
%
%  Set Intersection
%  Use \cap
%
%  Set Complement
%  Usage:  \setcomplement{a-set-letter}
\newcommand{\setcomplement}[1]{\overline{#1}}
%
%%%%%%%%%%%%%%%%%%%%%
%
%     Eigenvalues and Eigenspaces
%
%%%%%%%%%%%%%%%%%%%%%
%
%  Characteristic polynomial
%  Usage: \charpoly{matrix-letter}{variable-letter}
\newcommand{\charpoly}[2]{p_{#1}\left(#2\right)}
%
%  Eigenspace
%  Usage: \eigenspace{matrix-letter}{eigenvalue-letter}
\newcommand{\eigenspace}[2]{\mathcal{E}_{#1}\left(#2\right)}
%
%  2013/10/03 Including ampersands is problematic here, 
%  think about fixes later
%  2014/02/22 Limited testing, seems &amp; is fine for HTML and LaTeX
%  2016-07-20 only employed in Archetypes, MBX has gather/align override
%  Eigensystem (presumes wrapped in an mrow within md)
%  Usage: \eigensystem{matrixletter}{eigenvalue}{list of basis vectors}
\newcommand{\eigensystem}[3]{\lambda&amp;=#2&amp;\eigenspace{#1}{#2}&amp;=\spn{\set{#3}}} 
%
%  Generalized Eigenspace
%  Usage: \geneigenspace{lin-trans-letter}{eigenvalue-letter}
\newcommand{\geneigenspace}[2]{\mathcal{G}_{#1}\left(#2\right)}
%
%  Algebraic multiplicty
%  Usage: \algmult{matrix-letter}{eigenvalue-letter}
\newcommand{\algmult}[2]{\alpha_{#1}\left(#2\right)}
%
%  Geometric multiplicty
%  Usage: \geomult{matrix-letter}{eigenvalue-letter}
\newcommand{\geomult}[2]{\gamma_{#1}\left(#2\right)}
%
%  Index (of eigenvalue)
%  Usage: \indx{matrix-letter}{eigenvalue-letter}
\newcommand{\indx}[2]{\iota_{#1}\left(#2\right)}
%
%%%%%%%%%%%%%%%%%%%%%
%
%     Linear Transformations
%
%%%%%%%%%%%%%%%%%%%%%
%
%  MathJax defines \lt to ease XML confusion
%
%  Linear transformation definition
%  Usage: \ltdefn{name-letter}{domain}{range}
\newcommand{\ltdefn}[3]{#1\colon #2\rightarrow#3}
%
%  Linear transformation evaluation
%  Usage: \lteval{name-letter}{input}
%  Replaces old \lt desired by MathJax
\newcommand{\lteval}[2]{#1\left(#2\right)}
%
% Linear transformation inverse
%  Usage: \ltinverse{name-letter}
\newcommand{\ltinverse}[1]{#1^{-1}}
%
%  Linear transformation restriction
%  Usage: \restrict{name-letter}{subspace-letter}
\newcommand{\restrict}[2]{{#1}|_{#2}}
%
%  Linear transformation preimage
%  Usage: \preimage{name-letter}{codomain-element}
\newcommand{\preimage}[2]{#1^{-1}\left(#2\right)}
%
%  Range of a linear transformation
%  TeX uses \range for something else
%  Usage:  \rng{T}
\newcommand{\rng}[1]{\mathcal{R}\!\left(#1\right)}
%
%  Kernel of a linear transformation
%  TeX uses \ker to do something different
%  Usage:  \krn{T}
\newcommand{\krn}[1]{\mathcal{K}\!\left(#1\right)}
%
%  Linear transformation composition
%  Usage: \compose{function-name}{function-name}
\newcommand{\compose}[2]{{#1}\circ{#2}}
%
%  Vector space of linear transformations
%  Usage: \vslt{domains}{codomains}
%  Presumes math mode
\newcommand{\vslt}[2]{\mathcal{LT}\left(#1,\,#2\right)}
%
%%%%%%%%%%%%%%%%%%%%%
%
%     Vector and Matrix Representations
%
%%%%%%%%%%%%%%%%%%%%%
%
%  Isomorphism symbol
%  Usage: \isomorphic
\newcommand{\isomorphic}{\cong}
%
%  Similarity
%  Usage: \similar{inner-matrix}{outer-invertible-matrix}
%  Rearranging this will not "fix" all desired changes throughout
%
\newcommand{\similar}[2]{\inverse{#2}#1#2}
%
%  Vector representation function name
%  Usage: \vectrepname{basis-letter}
\newcommand{\vectrepname}[1]{\rho_{#1}}
%
%  Vector representation output
%  Usage: \vectrep{basis-letter}{input}
\newcommand{\vectrep}[2]{\lteval{\vectrepname{#1}}{#2}}
%
%  Vector representation inverse function name
%  (Added later, not used consistently in FCLA)
%  Usage: \vectrepinvname{basis-letter}
\newcommand{\vectrepinvname}[1]{\ltinverse{\vectrepname{#1}}}
%
%  Vector representation inverse output
%  Usage: \vectrepinv{basis-letter}{input}
\newcommand{\vectrepinv}[2]{\lteval{\ltinverse{\vectrepname{#1}}}{#2}}
%
%  Matrix representation
%  Usage: \matrixrep{transformation-letter}{domain-basis-letter}{codomain-basis-letter}
\newcommand{\matrixrep}[3]{M^{#1}_{#2,#3}}
%
%  Matrix representation column-by-colum
%  2016-07-20 only employed once?
%  Usage: \matrixrepcolumns{transformation-letter}{codomain-basis-letter}{codomain-basis-vector-letter}{final-index}
\newcommand{\matrixrepcolumns}[4]{\left\lbrack \left.\vectrep{#2}{\lteval{#1}{\vect{#3}_{1}}}\right|\left.\vectrep{#2}{\lteval{#1}{\vect{#3}_{2}}}\right|\left.\vectrep{#2}{\lteval{#1}{\vect{#3}_{3}}}\right|\ldots\left|\vectrep{#2}{\lteval{#1}{\vect{#3}_{#4}}}\right.\right\rbrack}
%
%  Change of basis matrix
%  Usage: \cbm{domain-basis-letter}{codomain-basis-letter}
\newcommand{\cbm}[2]{C_{#1,#2}}
%
%%%%%%%%%%%%%%%%%%%%%
%
%     Canonical Forms
%
%%%%%%%%%%%%%%%%%%%%%
%
%  Jordan blocks
%  Usage: \jordan{size}{diagonal-element}
\newcommand{\jordan}[2]{J_{#1}\left(#2\right)}
%
%%%%%%%%%%%%%%%%%%%%%
%
%     Hadamard Matrices
%     Contributed by Elizabeth Million
%
%%%%%%%%%%%%%%%%%%%%%
%
%  Hadamard Product
%  Usage: \hadamard{a-matrix}{a-matrix}
\newcommand{\hadamard}[2]{#1\circ #2}
%
%  Hadamard identity matrix
%  Usage: \hadamardidentity{paired-subscripts-size-of-matrix}
\newcommand{\hadamardidentity}[1]{J_{#1}}
%
%  Hadamard inverse matrix
%  Usage: \hadamardinverse{matrix-expression}
\newcommand{\hadamardinverse}[1]{\widehat{#1}}

\newcommand{\definedTerm}[1]{\textbf{#1}}
\newcommand{\dfn}[1]{\textbf{#1}}

\newcommand{\wt}{\widetilde}
\newcommand{\ov}{\overline}
\newcommand{\inj}{\rightarrowtail}
\newcommand{\surj}{\twoheadrightarrow}
\newcommand{\harpoon}{\overset{\rightharpoonup}}

\newenvironment{amatrix}[1]{%
  \left[\begin{array}{@{}*{#1}{c}|c@{}}
}{%
  \end{array}\right]
}


\title{Examples of Computing Eigenvalues and Eigenvectors}

\begin{document}
\begin{abstract}
  A selection of examples illustrates the range of possibilities for the eigenvalues and eigenvectors of a matrix. 
\end{abstract}
\maketitle

These examples can all be done by hand, though the computation of the
characteristic polynomial would be very time-consuming and
error-prone.  It can also be difficult to find roots of an arbitrary
polynomial, though if we were to suggest that most of our eigenvalues
are going to be integers, then it can be easier to hunt for roots.
First, we will sneak in a pair of definitions so we can illustrate
them throughout this sequence of examples.

\begin{definition}[Algebraic Multiplicity of an Eigenvalue]
  Suppose that $A$ is a square matrix and $\lambda$ is an eigenvalue
  of $A$.  Then the \dfn{algebraic multiplicity} of $\lambda$,
  $\algmult{A}{\lambda}$, is the highest power of $(x-\lambda)$ that
  divides the characteristic polynomial, $\charpoly{A}{x}$.
\end{definition}

Since an eigenvalue $\lambda$ is a root of the characteristic
polynomial, there is always a factor of $(x-\lambda)$, and the
algebraic multiplicity is just the power of this factor in a
factorization of $\charpoly{A}{x}$.  So in particular,
$\algmult{A}{\lambda}\geq 1$.  Compare the definition of algebraic
multiplicity with the next definition.

\begin{definition}[Geometric Multiplicity of an Eigenvalue]
  Suppose that $A$ is a square matrix and $\lambda$ is an eigenvalue
  of $A$.  Then the \dfn{geometric multiplicity} of $\lambda$,
  $\geomult{A}{\lambda}$, is the dimension of the eigenspace
  $\eigenspace{A}{\lambda}$.
\end{definition}

Every eigenvalue must have at least one eigenvector, so the associated
eigenspace cannot be trivial, and so $\geomult{A}{\lambda}\geq 1$.

\begin{example}[Eigenvalue multiplicities, matrix of size 4]
  Consider the matrix
  \[
    B=
    \begin{bmatrix}
      -2 & 1 & -2 & -4\\
      12 & 1 & 4 & 9\\
      6 & 5 & -2 & -4\\
      3 & -4 & 5 & 10
    \end{bmatrix}
  \]
  then
  \[
    \charpoly{B}{x}=8-20x+18x^2-7x^3+x^4=(x-1)(x-\answer{2})^3
  \]
  So the eigenvalues are $\lambda=1,\,2$ with algebraic multiplicities
  $\algmult{B}{1}=1$ and $\algmult{B}{2}=\answer{3}$.

  Computing eigenvectors,
  \begin{align*}
    \lambda&=1&B- 1I_4&=
                        \begin{bmatrix}
                          -3 & 1 & -2 & -4\\
                          12 & 0 & 4 & 9\\
                          6 & 5 & -3 & -4\\
                          3 & -4 & 5 & 9
                        \end{bmatrix}
                                       \rref
                                       \begin{bmatrix}
                                         \leading{1} & 0 & \frac{1}{3} & 0\\
                                         0 & \leading{1} & -1 & 0\\
                                         0 & 0 & 0 & \leading{1}\\
                                         0 & 0 & 0 & 0
                                       \end{bmatrix}\\
           &&\eigenspace{B}{1}&=\nsp{B-1I_4}
                                =\spn{\set{\colvector{-\frac{1}{3}\\1\\1\\0}}}
    =\spn{\set{\colvector{-1\\3\\3\\0}}}\\
    \lambda&=2&B-2I_4&=
                       \begin{bmatrix}
                         -4 & 1 & -2 & -4\\
                         12 & -1 & 4 & 9\\
                         6 & 5 & -4 & -4\\
                         3 & -4 & 5 & 8
                       \end{bmatrix}
                                      \rref
                                      \begin{bmatrix}
                                        \leading{1} & 0 & 0 & 1/2\\
                                        0 & \leading{1} & 0 & -1\\
                                        0 & 0 & \leading{1} & 1/2\\
                                        0 & 0 & 0 & 0
                                      \end{bmatrix}\\
           &&\eigenspace{B}{2}&=\nsp{B-2I_4}
                                =\spn{\set{\colvector{-\frac{1}{2}\\1\\-\frac{1}{2}\\1}}}
    =\spn{\set{\colvector{-1\\2\\-1\\2}}}\\
  \end{align*}
  
  So each eigenspace has dimension 1 and so
  $\geomult{B}{1}=\answer{1}$ and $\geomult{B}{2}=\answer{1}$.  This
  example is of interest because of the discrepancy between the two
  multiplicities for $\lambda=\answer{2}$.  In many of our examples
  the algebraic and geometric multiplicities will be equal for all of
  the eigenvalues (as it was for $\lambda=1$ in this example), so keep
  this example in mind.  We will have some explanations for this
  phenomenon later.
\end{example}

\begin{example}[Eigenvalues, symmetric matrix of size 4]
  Consider the matrix
  \[
    C=
    \begin{bmatrix}
      1 &  0 &  1 &  1\\
      0 &  1 &  1 &  1\\
      1 &  1 &  1 &  0\\
      1 &  1 &  0 &  1
    \end{bmatrix}
  \]
  then
  \[
    \charpoly{C}{x}=-3+4x+2x^2-4x^3+x^4=(x-3)(x-1)^2(x+1)
  \]
  So the eigenvalues are $\lambda=3,\,1,\,-1$ with algebraic
  multiplicities $\algmult{C}{3}=1$, $\algmult{C}{1}=\answer{2}$ and
  $\algmult{C}{-1}=1$.

  Computing eigenvectors,
  \begin{align*}
    \lambda&=3&C- 3I_4&=
                        \begin{bmatrix}
                          -2 & 0 & 1 & 1\\
                          0 & -2 & 1 & 1\\
                          1 & 1 & -2 & 0\\
                          1 & 1 & 0 & -2
                        \end{bmatrix}
                                      \rref
                                      \begin{bmatrix}
                                        \leading{1} & 0 & 0 & -1\\
                                        0 & \leading{1} & 0 & -1\\
                                        0 & 0 & \leading{1} & -1\\
                                        0 & 0 & 0 & 0
                                      \end{bmatrix}\\
           &&\eigenspace{C}{3}&=\nsp{C-3I_4}
                                =\spn{\set{\colvector{1\\1\\1\\1}}}\\
    \lambda&=1&C-1I_4&=
                       \begin{bmatrix}
                         0 & 0 & 1 & 1\\
                         0 & 0 & 1 & 1\\
                         1 & 1 & 0 & 0\\
                         1 & 1 & 0 & 0
                       \end{bmatrix}
                                     \rref
                                     \begin{bmatrix}
                                       \leading{1} & 1 & 0 & 0\\
                                       0 & 0 & \leading{1} & 1\\
                                       0 & 0 & 0 & 0\\
                                       0 & 0 & 0 & 0
                                     \end{bmatrix}\\
           &&\eigenspace{C}{1}&=\nsp{C-1I_4}
                                =\spn{\set{\colvector{-1\\1\\0\\0},\,\colvector{0\\0\\-1\\1}}}\\
    \lambda&=-1&C+1I_4&=
                        \begin{bmatrix}
                          2 & 0 & 1 & 1\\
                          0 & 2 & 1 & 1\\
                          1 & 1 & 2 & 0\\
                          1 & 1 & 0 & 2
                        \end{bmatrix}
                                      \rref
                                      \begin{bmatrix}
                                        \leading{1} & 0 & 0 & 1\\
                                        0 & \leading{1} & 0 & 1\\
                                        0 & 0 & \leading{1} & -1\\
                                        0 & 0 & 0 & 0
                                      \end{bmatrix}\\
           &&\eigenspace{C}{-1}&=\nsp{C+1I_4}
                                 =\spn{\set{\colvector{-1\\-1\\1\\1}}}\\
  \end{align*}

  So the eigenspace dimensions yield geometric multiplicities
  $\geomult{C}{3}=1$, $\geomult{C}{1}=\answer{2}$ and
  $\geomult{C}{-1}=1$, the same as for the algebraic multiplicities.
  This example is of interest because $A$ is a symmetric matrix, and
  will be the subject of \ref{theorem:HMRE}.

\end{example}

\begin{example}[High multiplicity eigenvalues, matrix of size 5]
  
  Consider the matrix
  \[
    E=
    \begin{bmatrix}
      29 & 14 & 2 & 6 & -9\\
      -47 & -22 & -1 & -11 & 13\\
      19 & 10 & 5 & 4 & -8\\
      -19 & -10 & -3 & -2 & 8\\
      7 & 4 & 3 & 1 & -3
    \end{bmatrix}
  \]
  then
  \[
    \charpoly{E}{x}=-16+16x+8x^2-16x^3+7x^4-x^5=-(x-2)^4(x+1)
  \]
  So the eigenvalues are $\lambda=2,\,-1$ with algebraic multiplicities $\algmult{E}{2}=\answer{4}$  and $\algmult{E}{-1}=1$.
  
  Computing eigenvectors,
  \begin{align*}
    \lambda&=2\\
    E-2I_5&=
            \begin{bmatrix}
              27 & 14 & 2 & 6 & -9\\
              -47 & -24 & -1 & -11 & 13\\
              19 & 10 & 3 & 4 & -8\\
              -19 & -10 & -3 & -4 & 8\\
              7 & 4 & 3 & 1 & -5
            \end{bmatrix}
                              \rref
                              \begin{bmatrix}
                                \leading{1} & 0 & 0 & 1 & 0\\
                                0 & \leading{1} & 0 & -\frac{3}{2} & -\frac{1}{2}\\
                                0 & 0 & \leading{1} & 0 & -1\\
                                0 & 0 & 0 & 0 & 0\\
                                0 & 0 & 0 & 0 & 0
                              \end{bmatrix}\\
    \eigenspace{E}{2}&=\nsp{E-2I_5}
                       =\spn{\set{\colvector{-1\\\frac{3}{2}\\0\\1\\0},\,\colvector{0\\\frac{1}{2}\\1\\0\\1}}}
    =\spn{\set{\colvector{-2\\3\\0\\2\\0},\,\colvector{0\\1\\2\\0\\2}}}\\
    \lambda&=-1\\
    E+1I_5&=
            \begin{bmatrix}
              30 & 14 & 2 & 6 & -9\\
              -47 & -21 & -1 & -11 & 13\\
              19 & 10 & 6 & 4 & -8\\
              -19 & -10 & -3 & -1 & 8\\
              7 & 4 & 3 & 1 & -2
            \end{bmatrix}
                              \rref
                              \begin{bmatrix}
                                \leading{1} & 0 & 0 & 2 & 0\\
                                0 & \leading{1} & 0 & -4 & 0\\
                                0 & 0 & \leading{1} & 1 & 0\\
                                0 & 0 & 0 & 0 & \leading{1}\\
                                0 & 0 & 0 & 0 & 0
                              \end{bmatrix}\\
    \eigenspace{E}{-1}&=\nsp{E+1I_5}=\spn{\set{\colvector{-2\\4\\-1\\1\\0}}}\\
  \end{align*}
  
  So the eigenspace dimensions yield geometric multiplicities
  $\geomult{E}{2}=\answer{2}$ and $\geomult{E}{-1}=1$.  This example
  is of interest because $\lambda=2$ has such a large algebraic
  multiplicity, which is also not equal to its geometric multiplicity.

\end{example}

\begin{example}[Complex eigenvalues, matrix of size 6]

  Consider the matrix
  \[
    F=
    \begin{bmatrix}
      -59 & -34 & 41 & 12 & 25 & 30\\
      1 & 7 & -46 & -36 & -11 & -29\\
      -233 & -119 & 58 & -35 & 75 & 54\\
      157 & 81 & -43 & 21 & -51 & -39\\
      -91 & -48 & 32 & -5 & 32 & 26\\
      209 & 107 & -55 & 28 & -69 & -50
    \end{bmatrix}
  \]
  then
  \begin{align*}
    \charpoly{F}{x}&=-50+55x+13x^2-50x^3+32x^4-9x^5+x^6\\
                   &=(x-2)(x+1)(x^2-4x+5)^2\\
                   &=(x-2)(x+1)((x-(2+i))(x-(2-i)))^2\\
                   &=(x-2)(x+1)(x-(2+i))^2(x-(2-i))^2\\
  \end{align*}
  So the eigenvalues are $\lambda=2,\,-1,2+i,\,2-i$ with algebraic
  multiplicities $\algmult{F}{2}=1$, $\algmult{F}{-1}=1$,
  $\algmult{F}{2+i}=2$ and $\algmult{F}{2-i}=2$.

  We compute eigenvectors, noting that the last two basis vectors are
  each a scalar multiple of what \ref{theorem:BNS} will provide,
  \begin{align*}
    \lambda&=2\quad\quad F-2I_6=\\
           &
             \begin{bmatrix}
               -61 & -34 & 41 & 12 & 25 & 30\\
               1 & 5 & -46 & -36 & -11 & -29\\
               -233 & -119 & 56 & -35 & 75 & 54\\
               157 & 81 & -43 & 19 & -51 & -39\\
               -91 & -48 & 32 & -5 & 30 & 26\\
               209 & 107 & -55 & 28 & -69 & -52
             \end{bmatrix}
                                            \rref
                                            \begin{bmatrix}
                                              \leading{1} & 0 & 0 & 0 & 0 & \frac{1}{5}\\
                                              0 & \leading{1} & 0 & 0 & 0 & 0\\
                                              0 & 0 & \leading{1} & 0 & 0 & \frac{3}{5}\\
                                              0 & 0 & 0 & \leading{1} & 0 & -\frac{1}{5}\\
                                              0 & 0 & 0 & 0 & \leading{1} & \frac{4}{5}\\
                                              0 & 0 & 0 & 0 & 0 & 0
                                            \end{bmatrix}\\
           &\eigenspace{F}{2}=\nsp{F-2I_6}
             =\spn{\set{\colvector{-\frac{1}{5}\\0\\-\frac{3}{5}\\\frac{1}{5}\\-\frac{4}{5}\\1}}}
    =\spn{\set{\colvector{-1\\0\\-3\\1\\-4\\5}}}\\
  \end{align*}
  \begin{align*}
    \lambda&=-1\quad\quad F+1I_6=\\
           &
             \begin{bmatrix}
               -58 & -34 & 41 & 12 & 25 & 30\\
               1 & 8 & -46 & -36 & -11 & -29\\
               -233 & -119 & 59 & -35 & 75 & 54\\
               157 & 81 & -43 & 22 & -51 & -39\\
               -91 & -48 & 32 & -5 & 33 & 26\\
               209 & 107 & -55 & 28 & -69 & -49
             \end{bmatrix}
                                            \rref
                                            \begin{bmatrix}
                                              \leading{1} & 0 & 0 & 0 & 0 & \frac{1}{2}\\
                                              0 & \leading{1} & 0 & 0 & 0 & -\frac{3}{2}\\
                                              0 & 0 & \leading{1} & 0 & 0 & \frac{1}{2}\\
                                              0 & 0 & 0 & \leading{1} & 0 & 0\\
                                              0 & 0 & 0 & 0 & \leading{1} & -\frac{1}{2}\\
                                              0 & 0 & 0 & 0 & 0 & 0
                                            \end{bmatrix}\\
           &\eigenspace{F}{-1}=\nsp{F+I_6}
             =\spn{\set{\colvector{-\frac{1}{2}\\\frac{3}{2}\\-\frac{1}{2}\\0\\\frac{1}{2}\\1}}}
    =\spn{\set{\colvector{-1\\3\\-1\\0\\1\\2}}}\\
  \end{align*}
  \begin{align*}
    \lambda&=2+i\\
           &F-(2+i)I_6=
             \begin{bmatrix}
               -61-i & -34 & 41 & 12 & 25 & 30\\
               1 & 5-i & -46 & -36 & -11 & -29\\
               -233 & -119 & 56-i & -35 & 75 & 54\\
               157 & 81 & -43 & 19-i & -51 & -39\\
               -91 & -48 & 32 & -5 & 30-i & 26\\
               209 & 107 & -55 & 28 & -69 & -52-i
             \end{bmatrix}\\
           &
             \rref
             \begin{bmatrix}
               \leading{1} & 0 & 0 & 0 & 0 & \frac{1}{5}(7+ i)\\
               0 & \leading{1} & 0 & 0 & 0 & \frac{1}{5}(-9-2i)\\
               0 & 0 & \leading{1} & 0 & 0 & 1\\
               0 & 0 & 0 & \leading{1} & 0 & -1\\
               0 & 0 & 0 & 0 & \leading{1} & 1\\
               0 & 0 & 0 & 0 & 0 & 0
             \end{bmatrix}\\
           &\eigenspace{F}{2+i}=\nsp{F-(2+i)I_6}
             =\spn{\set{\colvector{-7-i\\9+2i\\-5\\5\\-5\\5}}}\\
  \end{align*}
  \begin{align*}
    \lambda&=2-i\\
           &F-(2-i)I_6=
             \begin{bmatrix}
               -61+i & -34 & 41 & 12 & 25 & 30\\
               1 & 5+i & -46 & -36 & -11 & -29\\
               -233 & -119 & 56+i & -35 & 75 & 54\\
               157 & 81 & -43 & 19+i & -51 & -39\\
               -91 & -48 & 32 & -5 & 30+i & 26\\
               209 & 107 & -55 & 28 & -69 & -52+i
             \end{bmatrix}\\
           &\rref
             \begin{bmatrix}
               \leading{1} & 0 & 0 & 0 & 0 & \frac{1}{5}(7-i)\\
               0 & \leading{1} & 0 & 0 & 0 & \frac{1}{5}(-9+2i)\\
               0 & 0 & \leading{1} & 0 & 0 & 1\\
               0 & 0 & 0 & \leading{1} & 0 & -1\\
               0 & 0 & 0 & 0 & \leading{1} & 1\\
               0 & 0 & 0 & 0 & 0 & 0
             \end{bmatrix}\\
           &\eigenspace{F}{2-i}=\nsp{F-(2-i)I_6}
             =\spn{\set{\colvector{-7+i\\9-2i\\-5\\5\\-5\\5}}}\\
  \end{align*}
  
  Eigenspace dimensions yield geometric multiplicities of
  $\geomult{F}{2}=1$, $\geomult{F}{-1}=1$, $\geomult{F}{2+i}=1$ and
  $\geomult{F}{2-i}=1$.  This example demonstrates some of the
  possibilities for the appearance of complex eigenvalues, even when
  all the entries of the matrix are real.  Notice how all the numbers
  in the analysis of $\lambda=2-i$ are conjugates of the corresponding
  number in the analysis of $\lambda=2+i$.  This is the content of the
  upcoming \ref{theorem:ERMCP}.
\end{example}

\begin{example}[Distinct eigenvalues, matrix of size 5]
  Consider the matrix
  \[
    H=
    \begin{bmatrix}
      15 & 18 & -8 & 6 & -5\\
      5 & 3 & 1 & -1 & -3\\
      0 & -4 & 5 & -4 & -2\\
      -43 & -46 & 17 & -14 & 15\\
      26 & 30 & -12 & 8 & -10
    \end{bmatrix}
  \]
  then
  \[
    \charpoly{H}{x}=-6x+x^2+7x^3-x^4-x^5=x(x-2)(x-1)(x+1)(x+3)
  \]
  So the eigenvalues are $\lambda=2,\,1,\,0,\,-1,\,-3$ with algebraic
  multiplicities $\algmult{H}{2}=1$, $\algmult{H}{1}=1$,
  $\algmult{H}{0}=1$, $\algmult{H}{-1}=1$ and $\algmult{H}{-3}=1$.
  
  Computing eigenvectors,
  \begin{align*}
    \lambda&=2\\&H-2I_5=
                  \begin{bmatrix}
                    13 & 18 & -8 & 6 & -5\\
                    5 & 1 & 1 & -1 & -3\\
                    0 & -4 & 3 & -4 & -2\\
                    -43 & -46 & 17 & -16 & 15\\
                    26 & 30 & -12 & 8 & -12
                  \end{bmatrix}
                                        \rref
                                        \begin{bmatrix}
                                          \leading{1} & 0 & 0 & 0 & -1\\
                                          0 & \leading{1} & 0 & 0 & 1\\
                                          0 & 0 & \leading{1} & 0 & 2\\
                                          0 & 0 & 0 & \leading{1} & 1\\
                                          0 & 0 & 0 & 0 & 0
                                        \end{bmatrix}\\
           &\eigenspace{H}{2}=\nsp{H-2I_5}
             =\spn{\set{\colvector{1\\-1\\-2\\-1\\1}}}
  \end{align*}
  \begin{align*}
    \lambda&=1\\&H-1I_5=
                  \begin{bmatrix}
                    14 & 18 & -8 & 6 & -5\\
                    5 & 2 & 1 & -1 & -3\\
                    0 & -4 & 4 & -4 & -2\\
                    -43 & -46 & 17 & -15 & 15\\
                    26 & 30 & -12 & 8 & -11
                  \end{bmatrix}
                                        \rref
                                        \begin{bmatrix}
                                          \leading{1} & 0 & 0 & 0 & -\frac{1}{2}\\
                                          0 & \leading{1} & 0 & 0 & 0\\
                                          0 & 0 & \leading{1} & 0 & \frac{1}{2}\\
                                          0 & 0 & 0 & \leading{1} & 1\\
                                          0 & 0 & 0 & 0 & 0
                                        \end{bmatrix}\\
           &\eigenspace{H}{1}=\nsp{H-1I_5}
             =\spn{\set{\colvector{\frac{1}{2}\\0\\-\frac{1}{2}\\-1\\1}}}
    =\spn{\set{\colvector{1\\0\\-1\\-2\\2}}}
  \end{align*}
  \begin{align*}
    \lambda&=0\\&H-0I_5=
                  \begin{bmatrix}
                    15 & 18 & -8 & 6 & -5\\
                    5 & 3 & 1 & -1 & -3\\
                    0 & -4 & 5 & -4 & -2\\
                    -43 & -46 & 17 & -14 & 15\\
                    26 & 30 & -12 & 8 & -10
                  \end{bmatrix}
                                        \rref
                                        \begin{bmatrix}
                                          \leading{1} & 0 & 0 & 0 & 1\\
                                          0 & \leading{1} & 0 & 0 & -2\\
                                          0 & 0 & \leading{1} & 0 & -2\\
                                          0 & 0 & 0 & \leading{1} & 0\\
                                          0 & 0 & 0 & 0 & 0
                                        \end{bmatrix}\\
           &\eigenspace{H}{0}=\nsp{H-0I_5}
             =\spn{\set{\colvector{-1\\2\\2\\0\\1}}}
  \end{align*}
  \begin{align*}
    \lambda&=-1\\&H+1I_5=
                   \begin{bmatrix}
                     16 & 18 & -8 & 6 & -5\\
                     5 & 4 & 1 & -1 & -3\\
                     0 & -4 & 6 & -4 & -2\\
                     -43 & -46 & 17 & -13 & 15\\
                     26 & 30 & -12 & 8 & -9
                   \end{bmatrix}
                                         \rref
                                         \begin{bmatrix}
                                           \leading{1} & 0 & 0 & 0 & -1/2\\
                                           0 & \leading{1} & 0 & 0 & 0\\
                                           0 & 0 & \leading{1} & 0 & 0\\
                                           0 & 0 & 0 & \leading{1} & 1/2\\
                                           0 & 0 & 0 & 0 & 0
                                         \end{bmatrix}\\
           &\eigenspace{H}{-1}=\nsp{H+1I_5}
             =\spn{\set{\colvector{\frac{1}{2}\\0\\0\\-\frac{1}{2}\\1}}}
    =\spn{\set{\colvector{1\\0\\0\\-1\\2}}}
  \end{align*}
  \begin{align*}
    \lambda&=-3\\&H+3I_5=
                   \begin{bmatrix}
                     18 & 18 & -8 & 6 & -5\\
                     5 & 6 & 1 & -1 & -3\\
                     0 & -4 & 8 & -4 & -2\\
                     -43 & -46 & 17 & -11 & 15\\
                     26 & 30 & -12 & 8 & -7
                   \end{bmatrix}
                                         \rref
                                         \begin{bmatrix}
                                           \leading{1} & 0 & 0 & 0 & -1\\
                                           0 & \leading{1} & 0 & 0 & \frac{1}{2}\\
                                           0 & 0 & \leading{1} & 0 & 1\\
                                           0 & 0 & 0 & \leading{1} & 2\\
                                           0 & 0 & 0 & 0 & 0
                                         \end{bmatrix}\\
           &\eigenspace{H}{-3}=\nsp{H+3I_5}
=\spn{\set{\colvector{1\\-\frac{1}{2}\\-1\\-2\\1}}}
    =\spn{\set{\colvector{-2\\1\\2\\4\\-2}}}
  \end{align*}


  So the eigenspace dimensions yield geometric multiplicities
  $\geomult{H}{2}=\answer{1}$, $\geomult{H}{1}=\answer{1}$,
  $\geomult{H}{0}=\answer{1}$, $\geomult{H}{-1}=\answer{1}$ and
  $\geomult{H}{-3}=\answer{1}$, identical to the algebraic
  multiplicities.  This example is of interest for two reasons.
  First, $\lambda=0$ is an eigenvalue, illustrating the upcoming
  \ref{theorem:SMZE}.  Second, all the eigenvalues are distinct,
  yielding algebraic and geometric multiplicities of 1 for each
  eigenvalue, illustrating \ref{theorem:DED}.
\end{example}

\end{document}
