\documentclass{ximera}

% These macros are automatically generated from the "macros"
% XML element.  Make permanent edits there.
%
% History
%   2004/01/01  Initiated for FCLA, evolved from there
%   2006/09/17  Converted  _, ^  to \sb, \sp for TeX4ht
%   2014/02/01  Updated for MathBook XML projects
%               Obsolete in FCLA: \codeindent, \computerfont, \define
%               Change: MathJax wants \lt, so replaced by \lteval
%   2014/02/22  New: \orderof, \reals, \per
%   2015/08/16  Incorporated into MathBook XML version of FCLA
%
%%%%%%%%%%%%%%%%%%%%%
%
%     Conveniences
%
%%%%%%%%%%%%%%%%%%%%%
%
%  Order of (asymptotically limit of fraction is 1)
%  Usage: \orderof{some function}
%
\newcommand{\orderof}[1]{\sim #1}
%
%  Integers
%  Usage:  \Z
\newcommand{\Z}{\mathbb{Z}}
%
%  Real numbers, as set of scalars
%  Usage:  \reals
\newcommand{\reals}{\mathbb{R}}
%
%  n-space over real field
%  Usage: \complex{integer-dimension}
\newcommand{\real}[1]{\mathbb{R}^{#1}}
%
%  Complex numbers, as set of scalars
%  Usage:  \complexes
\newcommand{\complexes}{\mathbb{C}}
%
%  n-space over complex field
%  Usage: \complex{integer-dimension}
\newcommand{\complex}[1]{\mathbb{C}^{#1}}
\newcommand{\CC}{\mathbb{C}}
%
%  Complex conjugation (scalar, vector, matrix)
%  Usage: \conjugate{object}
\newcommand{\conjugate}[1]{\overline{#1}}
%
%  Complex number modulus
%  Usage: \modulus{a+bi}
%  Presumes math mode
\newcommand{\modulus}[1]{\left\lvert#1\right\rvert}
%
%  Zero vector
%  Usage: \zerovector
\newcommand{\zerovector}{\vect{0}}
%
%  Zero matrix
%  Usage: \zeromatrix, use a subscript when size is important
\newcommand{\zeromatrix}{\mathcal{O}}
%
%  Inner product (brackets, not quadratic form)
%  Usage: \innerproduct{a-vector}{a-vector}
\newcommand{\innerproduct}[2]{\left\langle#1,\,#2\right\rangle}
%
%  Norm of a vector
%  Usage: \norm{a-vector}
\newcommand{\norm}[1]{\left\lVert#1\right\rVert}
%
%  Dimension
%  Usage: \dimension{vector-space-letter}
\newcommand{\dimension}[1]{\dim\left(#1\right)}
%
%  Nullity
%  Usage: \nullity{matrix-or-lintrans-letter}
\newcommand{\nullity}[1]{n\left(#1\right)}
%
%  Rank
%  Usage: \rank{matrix-or-lintrans-letter}
\newcommand{\rank}[1]{r\left(#1\right)}
%
%  Direct sum
%  Usage: \ds between a couple of subspaces
%
\newcommand{\ds}{\oplus}
%
%  Determinant of a matrix (functional)
%  Usage: \detname{A}
\newcommand{\detname}[1]{\det\left(#1\right)}
%
%  Determinant of a matrix (vertical bars)
%  Usage: \detbars{A}
\newcommand{\detbars}[1]{\left\lvert#1\right\rvert}
%
%  Trace of a Matrix
%  Usage: \trace{matrix name}
\newcommand{\trace}[1]{t\left(#1\right)}
%
%  Square Root of a Matrix
%  Usage: \sr{a-matrix}
\newcommand{\sr}[1]{#1^{1/2}}
%
%%%%%%%%%%%%%%%%%%%%%
%
%     Subspace Constructions
%
%%%%%%%%%%%%%%%%%%%%%
%
%  Span of a set of vectors
%  \span and \sp are used by TeX for other things
%  Usage: \spn{set-of-vectors}
\newcommand{\spn}[1]{\left\langle#1\right\rangle}
%
%  Null space of a matrix
%  Usage:  \nsp{A}
\newcommand{\nsp}[1]{\mathcal{N}\!\left(#1\right)}
%
%  Column space of a matrix
%  Usage:  \csp{A}
\newcommand{\csp}[1]{\mathcal{C}\!\left(#1\right)}
%
%  Row space of a matrix
%  Usage:  \rsp{A}
\newcommand{\rsp}[1]{\mathcal{R}\!\left(#1\right)}
%
%  Left null space of a matrix
%  Usage:  \lns{A}
\newcommand{\lns}[1]{\mathcal{L}\!\left(#1\right)}
%
%  Orthogonal complement of a vector space
%  Avoiding TeX's \perp
%  Usage:  \per{A}
\newcommand{\per}[1]{#1^\perp}
%
%%%%%%%%%%%%%%%%%%%%%
%
%     Systems of Equations
%
%%%%%%%%%%%%%%%%%%%%%
%
%  In-line form of an augmented matrix for a system of equations
%  Usage: \augmented{coefficient-matrix}{constant-vector}
\newcommand{\augmented}[2]{\left\lbrack\left.#1\,\right\rvert\,#2\right\rbrack}
%
%  Notation for a linear system before introducing matrix multiplication
%  Usage: \linearsystem{coefficient-matrix}{constant-vector}
\newcommand{\linearsystem}[2]{\mathcal{LS}\!\left(#1,\,#2\right)}
%
%  Notation for a homogenous system before introducing matrix multiplication
%  Usage: \homosystem{coefficient-matrix}
\newcommand{\homosystem}[1]{\linearsystem{#1}{\zerovector}}
%
%%%%%%%%%%%%%%%%%%%%%
%
%     Row Operations, Echelon Form
%
%%%%%%%%%%%%%%%%%%%%%
%
% Row operations on matrices
%
% Three commands to shorten up descriptions of gaussian elimination
%
% Usage: \rowopswap{row-i}{row-j}
% Usage: \rowopmult{scalar}{row-i}
% Usage: \rowopadd{scalar}{row-multiplied}{row-added-to}
\newcommand{\rowopswap}[2]{R_{#1}\leftrightarrow R_{#2}}
\newcommand{\rowopmult}[2]{#1R_{#2}}
\newcommand{\rowopadd}[3]{#1R_{#2}+R_{#3}}
%
% Mark leading 1's in echelon form with fbox
% Usage: \leading{a-1-usually}
\newcommand{\leading}[1]{\fbox{#1}}
%
%  Row-reduce arrow
%  Usage:  \rref inbetween a matrix and its reduced row-echelon form
\newcommand{\rref}{\xrightarrow{\text{RREF}}}
%
%  Elementary Matrices
%  Usage: \elemswap{subscript}{subscript}
%  Usage: \elemmult{scalar}{subscript}
%  Usage: \elemadd{scalar}{subscript-mult}{subscript-target}
%
\newcommand{\elemswap}[2]{E_{#1,#2}}
\newcommand{\elemmult}[2]{E_{#2}\left(#1\right)}
\newcommand{\elemadd}[3]{E_{#2,#3}\left(#1\right)}
%
%%%%%%%%%%%%%%%%%%%%%
%
%     2-D Constructions (Lists, Vectors, Matrices)
%
%%%%%%%%%%%%%%%%%%%%%
%
%  A list of scalars of generic length
%  Usage:  \scalarlist{scalar letter}{terminal subscript}
\newcommand{\scalarlist}[2]{{#1}_{1},\,{#1}_{2},\,{#1}_{3},\,\ldots,\,{#1}_{#2}}
%
%  Vector styling, bold (or use wiggles, arrows, whatever)
%  Subscripts go outside this construction
%  Usage: \vect{a symbol to use as a vector}
%  Have to already be in math mode
%
\newcommand{\vect}[1]{\mathbf{#1}}
%
%  A column vector
%  Usage: \colvector{list-delimited-by-\\}
%
\newcommand{\colvector}[1]{\begin{bmatrix}#1\end{bmatrix}}
%
%  A generic vector with components
%  Usage: \vectorcomponents{component-letter}{final-subscript}
\newcommand{\vectorcomponents}[2]{\colvector{#1_{1}\\#1_{2}\\#1_{3}\\\vdots\\#1_{#2}}}
%
%  A list of vectors of generic length
%  Usage:  \vectorlist{vector letter}{terminal subscript}
\newcommand{\vectorlist}[2]{\vect{#1}_{1},\,\vect{#1}_{2},\,\vect{#1}_{3},\,\ldots,\,\vect{#1}_{#2}}
%
%  Vector entries, entry i of vector v
%  (vector-expession still needs \vect, etc.)
%  Usage:  \vectorentry{vector-expression}{single-subscript}
\newcommand{\vectorentry}[2]{\left\lbrack#1\right\rbrack_{#2}}
%
%  Matrix entries, entry i,j of matrix A
%  Usage:  \matrixentry{matrix-expression}{paired-subscripts}
%
\newcommand{\matrixentry}[2]{\left\lbrack#1\right\rbrack_{#2}}
%
%  A generic linear combination
%  Usage:  \lincombo{scalar letter}{vector letter}{terminal subscript}
\newcommand{\lincombo}[3]{#1_{1}\vect{#2}_{1}+#1_{2}\vect{#2}_{2}+#1_{3}\vect{#2}_{3}+\cdots +#1_{#3}\vect{#2}_{#3}}
%
%  Matrix, column by column, as vectors
%  Usage:  \matrixcolumns{matrix letter}{terminal subscript}
\newcommand{\matrixcolumns}[2]{\left\lbrack\vect{#1}_{1}|\vect{#1}_{2}|\vect{#1}_{3}|\ldots|\vect{#1}_{#2}\right\rbrack}
%
%%%%%%%%%%%%%%%%%%%%%
%
%     Special Matrices
%
%%%%%%%%%%%%%%%%%%%%%
%
%  Transpose of a matrix
%  Usage:  \transpose{A}
\newcommand{\transpose}[1]{#1^{t}}
%
%  Inverse of a matrix
%  Usage:  \inverse{A}
\newcommand{\inverse}[1]{#1^{-1}}
%
%  Submatrix (for minors, determinants)
%  Usage: \submatrix{matrix-name}{delete-row}{delete-col}
\newcommand{\submatrix}[3]{#1\left(#2|#3\right)}
%
%  Adjoint of a matrix (twice)
%  This macro is a convenience to call \transpose and \conjugate properly
%  It shouldn't need to be modified (or mathematical meanings will change)
%  Usage:  \adj{A}
\newcommand{\adj}[1]{\transpose{\left(\conjugate{#1}\right)}}
%
%  This macro controls the symbol used for the adjoint
%  It can be edited to taste
%  Usage:  \adjoint{A}
\newcommand{\adjoint}[1]{#1^\ast}
%
%%%%%%%%%%%%%%%%%%%%%
%
%     Sets
%
%%%%%%%%%%%%%%%%%%%%%
%
%  A convenience for simple sets
%  Usage:  \set{list of element}
\newcommand{\set}[1]{\left\{#1\right\}}
%
%  Sets with vertical bar, "such that", sized for objects, not condition
%  Usage:  \setparts{objects}{condition}
%
%%\newcommand{\setparts}[2]{\left\{ #1\mid#2\right\}}
%%\newcommand{\setparts}[2]{\left\{\left. #1\right\rvert#2\right\}}
\newcommand{\setparts}[2]{\left\lbrace#1\,\middle|\,#2\right\rbrace}
%
%  Set Cardinality
%  Usage:  \card{a-set-letter}
\newcommand{\card}[1]{\left\lvert#1\right\rvert}
%
%  Set Union
%  Use \cup
%
%  Set Intersection
%  Use \cap
%
%  Set Complement
%  Usage:  \setcomplement{a-set-letter}
\newcommand{\setcomplement}[1]{\overline{#1}}
%
%%%%%%%%%%%%%%%%%%%%%
%
%     Eigenvalues and Eigenspaces
%
%%%%%%%%%%%%%%%%%%%%%
%
%  Characteristic polynomial
%  Usage: \charpoly{matrix-letter}{variable-letter}
\newcommand{\charpoly}[2]{p_{#1}\left(#2\right)}
%
%  Eigenspace
%  Usage: \eigenspace{matrix-letter}{eigenvalue-letter}
\newcommand{\eigenspace}[2]{\mathcal{E}_{#1}\left(#2\right)}
%
%  2013/10/03 Including ampersands is problematic here, 
%  think about fixes later
%  2014/02/22 Limited testing, seems &amp; is fine for HTML and LaTeX
%  2016-07-20 only employed in Archetypes, MBX has gather/align override
%  Eigensystem (presumes wrapped in an mrow within md)
%  Usage: \eigensystem{matrixletter}{eigenvalue}{list of basis vectors}
\newcommand{\eigensystem}[3]{\lambda&amp;=#2&amp;\eigenspace{#1}{#2}&amp;=\spn{\set{#3}}} 
%
%  Generalized Eigenspace
%  Usage: \geneigenspace{lin-trans-letter}{eigenvalue-letter}
\newcommand{\geneigenspace}[2]{\mathcal{G}_{#1}\left(#2\right)}
%
%  Algebraic multiplicty
%  Usage: \algmult{matrix-letter}{eigenvalue-letter}
\newcommand{\algmult}[2]{\alpha_{#1}\left(#2\right)}
%
%  Geometric multiplicty
%  Usage: \geomult{matrix-letter}{eigenvalue-letter}
\newcommand{\geomult}[2]{\gamma_{#1}\left(#2\right)}
%
%  Index (of eigenvalue)
%  Usage: \indx{matrix-letter}{eigenvalue-letter}
\newcommand{\indx}[2]{\iota_{#1}\left(#2\right)}
%
%%%%%%%%%%%%%%%%%%%%%
%
%     Linear Transformations
%
%%%%%%%%%%%%%%%%%%%%%
%
%  MathJax defines \lt to ease XML confusion
%
%  Linear transformation definition
%  Usage: \ltdefn{name-letter}{domain}{range}
\newcommand{\ltdefn}[3]{#1\colon #2\rightarrow#3}
%
%  Linear transformation evaluation
%  Usage: \lteval{name-letter}{input}
%  Replaces old \lt desired by MathJax
\newcommand{\lteval}[2]{#1\left(#2\right)}
%
% Linear transformation inverse
%  Usage: \ltinverse{name-letter}
\newcommand{\ltinverse}[1]{#1^{-1}}
%
%  Linear transformation restriction
%  Usage: \restrict{name-letter}{subspace-letter}
\newcommand{\restrict}[2]{{#1}|_{#2}}
%
%  Linear transformation preimage
%  Usage: \preimage{name-letter}{codomain-element}
\newcommand{\preimage}[2]{#1^{-1}\left(#2\right)}
%
%  Range of a linear transformation
%  TeX uses \range for something else
%  Usage:  \rng{T}
\newcommand{\rng}[1]{\mathcal{R}\!\left(#1\right)}
%
%  Kernel of a linear transformation
%  TeX uses \ker to do something different
%  Usage:  \krn{T}
\newcommand{\krn}[1]{\mathcal{K}\!\left(#1\right)}
%
%  Linear transformation composition
%  Usage: \compose{function-name}{function-name}
\newcommand{\compose}[2]{{#1}\circ{#2}}
%
%  Vector space of linear transformations
%  Usage: \vslt{domains}{codomains}
%  Presumes math mode
\newcommand{\vslt}[2]{\mathcal{LT}\left(#1,\,#2\right)}
%
%%%%%%%%%%%%%%%%%%%%%
%
%     Vector and Matrix Representations
%
%%%%%%%%%%%%%%%%%%%%%
%
%  Isomorphism symbol
%  Usage: \isomorphic
\newcommand{\isomorphic}{\cong}
%
%  Similarity
%  Usage: \similar{inner-matrix}{outer-invertible-matrix}
%  Rearranging this will not "fix" all desired changes throughout
%
\newcommand{\similar}[2]{\inverse{#2}#1#2}
%
%  Vector representation function name
%  Usage: \vectrepname{basis-letter}
\newcommand{\vectrepname}[1]{\rho_{#1}}
%
%  Vector representation output
%  Usage: \vectrep{basis-letter}{input}
\newcommand{\vectrep}[2]{\lteval{\vectrepname{#1}}{#2}}
%
%  Vector representation inverse function name
%  (Added later, not used consistently in FCLA)
%  Usage: \vectrepinvname{basis-letter}
\newcommand{\vectrepinvname}[1]{\ltinverse{\vectrepname{#1}}}
%
%  Vector representation inverse output
%  Usage: \vectrepinv{basis-letter}{input}
\newcommand{\vectrepinv}[2]{\lteval{\ltinverse{\vectrepname{#1}}}{#2}}
%
%  Matrix representation
%  Usage: \matrixrep{transformation-letter}{domain-basis-letter}{codomain-basis-letter}
\newcommand{\matrixrep}[3]{M^{#1}_{#2,#3}}
%
%  Matrix representation column-by-colum
%  2016-07-20 only employed once?
%  Usage: \matrixrepcolumns{transformation-letter}{codomain-basis-letter}{codomain-basis-vector-letter}{final-index}
\newcommand{\matrixrepcolumns}[4]{\left\lbrack \left.\vectrep{#2}{\lteval{#1}{\vect{#3}_{1}}}\right|\left.\vectrep{#2}{\lteval{#1}{\vect{#3}_{2}}}\right|\left.\vectrep{#2}{\lteval{#1}{\vect{#3}_{3}}}\right|\ldots\left|\vectrep{#2}{\lteval{#1}{\vect{#3}_{#4}}}\right.\right\rbrack}
%
%  Change of basis matrix
%  Usage: \cbm{domain-basis-letter}{codomain-basis-letter}
\newcommand{\cbm}[2]{C_{#1,#2}}
%
%%%%%%%%%%%%%%%%%%%%%
%
%     Canonical Forms
%
%%%%%%%%%%%%%%%%%%%%%
%
%  Jordan blocks
%  Usage: \jordan{size}{diagonal-element}
\newcommand{\jordan}[2]{J_{#1}\left(#2\right)}
%
%%%%%%%%%%%%%%%%%%%%%
%
%     Hadamard Matrices
%     Contributed by Elizabeth Million
%
%%%%%%%%%%%%%%%%%%%%%
%
%  Hadamard Product
%  Usage: \hadamard{a-matrix}{a-matrix}
\newcommand{\hadamard}[2]{#1\circ #2}
%
%  Hadamard identity matrix
%  Usage: \hadamardidentity{paired-subscripts-size-of-matrix}
\newcommand{\hadamardidentity}[1]{J_{#1}}
%
%  Hadamard inverse matrix
%  Usage: \hadamardinverse{matrix-expression}
\newcommand{\hadamardinverse}[1]{\widehat{#1}}

\newcommand{\definedTerm}[1]{\textbf{#1}}
\newcommand{\dfn}[1]{\textbf{#1}}

\newcommand{\wt}{\widetilde}
\newcommand{\ov}{\overline}
\newcommand{\inj}{\rightarrowtail}
\newcommand{\surj}{\twoheadrightarrow}
\newcommand{\harpoon}{\overset{\rightharpoonup}}

\newenvironment{amatrix}[1]{%
  \left[\begin{array}{@{}*{#1}{c}|c@{}}
}{%
  \end{array}\right]
}


\title{Existence of Eigenvalues and Eigenvectors}

\begin{document}
\begin{abstract}
  Before we embark on computing eigenvalues and eigenvectors, we will prove that every matrix has at least one eigenvalue (and an eigenvector to go with it).
\end{abstract}
\maketitle

Every matrix has at least one eigenvalue.  In \ref{theorem:MNEM}, we
will determine the maximum number of eigenvalues a matrix may have.

\begin{theorem}[Every Matrix Has an Eigenvalue]
\label{theorem:EMHE}

Suppose $A$ is a square matrix.  Then $A$ has at least one eigenvalue.


\begin{proof}
Suppose that $A$ has size $n$, and choose $\vect{x}$ as \textit{any} nonzero vector from $\complex{n}$.  (Notice how much latitude we have in our choice of $\vect{x}$.  Only the zero vector is off-limits.)  Consider the set
\[
S=\set{\vect{x},\,A\vect{x},\,A^2\vect{x},\,A^3\vect{x},\,\ldots,\,A^n\vect{x}}
\]




This is a set of $n+1$ vectors from $\complex{n}$, so by \ref{theorem:MVSLD}, $S$ is linearly dependent.  Let $a_0,\,a_1,\,a_2,\,\ldots,\,a_n$ be a collection of $n+1$ scalars from $\complexes$, not all zero, that provide a relation of linear dependence on $S$.  In other words,
\[
a_0\vect{x}+a_1A\vect{x}+a_2A^2\vect{x}+a_3A^3\vect{x}+\cdots+a_nA^n\vect{x}=\zerovector
\]




Some of the $a_i$ are nonzero.  Suppose that just $a_0\neq 0$, and $a_1=a_2=a_3=\cdots=a_n=0$.  Then $a_0\vect{x}=\zerovector$ and by \ref{theorem:SMEZV}, either $a_0=0$ or $\vect{x}=\zerovector$, which are both contradictions.  So $a_i\neq 0$ for some $i\geq 1$.  Let $m$ be the largest integer such that $a_m\neq 0$.  From this discussion we know that $m\geq 1$.  We can also assume that $a_m=1$, for if not, replace each $a_i$ by $a_i/a_m$ to obtain scalars that serve equally well in providing a relation of linear dependence on $S$.



Define the polynomial
\[
p(x)=a_0+a_1x+a_2x^2+a_3x^3+\cdots+a_mx^m
\]




Because we have consistently used $\complexes$ as our set of scalars (rather than ${\mathbb R}$), we know that we can factor $p(x)$ into linear factors of the form $(x-b_i)$, where $b_i\in\complexes$.  So there are scalars, $\scalarlist{b}{m}$, from $\complexes$ so that,
\[
p(x)=(x-b_m)(x-b_{m-1})\cdots(x-b_3)(x-b_2)(x-b_1)
\]




Put it all together and
\begin{align*}
\zerovector&=a_0\vect{x}+a_1A\vect{x}+a_2A^2\vect{x}+\cdots+a_nA^n\vect{x}\\
&=a_0\vect{x}+a_1A\vect{x}+a_2A^2\vect{x}+\cdots+a_mA^m\vect{x}&&\text{$a_i=0$ for $i>m$}\\
&=\left(a_0I_n+a_1A+a_2A^2+\cdots+a_mA^m\right)\vect{x}&&\ref{theorem:MMDAA}\\
&=p(A)\vect{x}&&\text{Definition of $p(x)$}\\
&=(A-b_mI_n)(A-b_{m-1}I_n)\cdots(A-b_2I_n)(A-b_1I_n)\vect{x}
\end{align*}




Let $k$ be the smallest integer such that
\[
(A-b_kI_n)(A-b_{k-1}I_n)\cdots(A-b_2I_n)(A-b_1I_n)\vect{x}=\zerovector.
\]




From the preceding equation, we know that $k\leq m$.  Define the vector $\vect{z}$ by
\[
\vect{z}=(A-b_{k-1}I_n)\cdots(A-b_2I_n)(A-b_1I_n)\vect{x}
\]




Notice that by the definition of $k$, the vector $\vect{z}$ must be nonzero.  In the case where $k=1$, we understand that $\vect{z}$ is defined by $\vect{z}=\vect{x}$, and $\vect{z}$ is still nonzero.  Now
\[
(A-b_kI_n)\vect{z}=(A-b_kI_n)(A-b_{k-1}I_n)\cdots(A-b_3I_n)(A-b_2I_n)(A-b_1I_n)\vect{x}=\zerovector
\]
which allows us to write
\begin{align*}
A\vect{z}
&=(A+\zeromatrix)\vect{z}&&\ref{property:ZM}\\
&=(A-b_kI_n+b_kI_n)\vect{z}&&\ref{property:AIM}\\
&=(A-b_kI_n)\vect{z}+b_kI_n\vect{z}&&\ref{theorem:MMDAA}\\
&=\zerovector+b_kI_n\vect{z}&&\text{Defining property of $\vect{z}$}\\
&=b_kI_n\vect{z}&&\ref{property:ZM}\\
&=b_k\vect{z}&&\ref{theorem:MMIM}
\end{align*}




Since $\vect{z}\neq\zerovector$, this equation says that $\vect{z}$ is an eigenvector of $A$ for the eigenvalue $\lambda=b_k$ (\ref{definition:EEM}), so we have shown that any square matrix $A$ does have at least one eigenvalue.



\end{proof}
\end{theorem}

The proof of \ref{theorem:EMHE} is constructive (it contains an unambiguous procedure that leads to an eigenvalue), but it is not meant to be practical.  We will illustrate the theorem with an example, the purpose being to provide a companion for studying the proof and not to suggest this is the best procedure for computing an eigenvalue.



\begin{example}[Computing an eigenvalue the hard way]

  This example illustrates the proof of \ref{theorem:EMHE}, and so
  will employ the same notation as the proof --- look there for full
  explanations.  It is \textit{not} meant to be an example of a
  reasonable computational approach to finding eigenvalues and
  eigenvectors.  OK, warnings in place, here we go.

Consider the matrix
\begin{align*}
A&=\begin{bmatrix}
-7 & -1 & 11 & 0 & -4\\
4 & 1 & 0 & 2 & 0\\
-10 & -1 & 14 & 0 & -4\\
8 & 2 & -15 & -1 & 5\\
-10 & -1 & 16 & 0 & -6
\end{bmatrix}
\end{align*}
and choose the vector $\vect{x}$,
\begin{align*}
\vect{x}&=\colvector{3\\0\\3\\-5\\4}
\end{align*}

It is important to notice that the choice of $\vect{x}$ could be
\textit{anything}, so long as it is \textit{not} the zero vector.  We
have not chosen $\vect{x}$ totally at random, but so as to make our
illustration of the theorem as general as possible.  You could
replicate this example with your own choice and the computations are
guaranteed to be reasonable, provided you have a computational tool
that will factor a fifth degree polynomial for you.

The set
\begin{align*}
S&=\set{\vect{x},\,A\vect{x},\,A^2\vect{x},\,A^3\vect{x},\,A^4\vect{x},\,A^5\vect{x}}\\
&=
\set{
\colvector{3\\0\\3\\-5\\4},\,
\colvector{-4\\2\\-4\\4\\-6},\,
\colvector{6\\-6\\6\\-2\\10},\,
\colvector{-10\\14\\-10\\-2\\-18},\,
\colvector{18\\-30\\18\\10\\34},\,
\colvector{-34\\62\\-34\\-26\\-66}
}
\end{align*}
is guaranteed to be linearly dependent, as it has six vectors from $\complex{5}$ (\ref{theorem:MVSLD}).

We will search for a nontrivial relation of linear dependence by
solving a homogeneous system of equations whose coefficient matrix has
the vectors of $S$ as columns through row operations,
\[
\begin{bmatrix}
3 & -4 & 6 & -10 & 18 & -34\\
0 & 2 & -6 & 14 & -30 & 62\\
3 & -4 & 6 & -10 & 18 & -34\\
-5 & 4 & -2 & -2 & 10 & -26\\
4 & -6 & 10 & -18 & 34 & -66
\end{bmatrix}
\rref
\begin{bmatrix}
\leading{1} & 0 & -2 & 6 & -14 & 30\\
0 & \leading{1} & -3 & 7 & -15 & 31\\
0 & 0 & 0 & 0 & 0 & 0\\
0 & 0 & 0 & 0 & 0 & 0\\
0 & 0 & 0 & 0 & 0 & 0
\end{bmatrix}
\]

There are four free variables for describing solutions to this
homogeneous system, so we have our pick of solutions.  The most
expedient choice would be to set $x_3=1$ and $x_4=x_5=x_6=0$.
However, we will again opt to maximize the generality of our
illustration of \ref{theorem:EMHE} and choose $x_3=-8$, $x_4=-3$,
$x_5=1$ and $x_6=0$.  This leads to a solution with $x_1=16$ and
$x_2=12$.

This relation of linear dependence then says that
\begin{align*}
\zerovector&=16\vect{x}+12A\vect{x}-8A^2\vect{x}-3A^3\vect{x}+A^4\vect{x}+0A^5\vect{x}\\
\zerovector&=\left(16+12A-8A^2-3A^3+A^4\right)\vect{x}
\end{align*}

So we define $p(x)=16+12x-8x^2-3x^3+x^4$, and as advertised in the
proof of \ref{theorem:EMHE}, we have a polynomial of degree $m=4>1$
such that $p(A)\vect{x}=\zerovector$.  Now we need to factor $p(x)$
over $\complexes$.  If you made your own choice of $\vect{x}$ at the
start, this is where you might have a fifth degree polynomial, and
where you might need to use a computational tool to find roots and
factors.  We have
\[
p(x)=16+12x-8x^2-3x^3+x^4=(x-4)(x+2)(x-2)(x+1)
\]

So we know that
\[
\zerovector=p(A)\vect{x}=(A-4I_5)(A+\answer{2}I_5)(A-2I_5)(A+1I_5)\vect{x}
\]

We apply one factor at a time, until we get the zero vector, so as to determine the value of $k$ described in the proof of \ref{theorem:EMHE},
\begin{align*}
(A+1I_5)\vect{x}&=
\begin{bmatrix}
-6 & -1 & 11 & 0 & -4\\
4 & 2 & 0 & 2 & 0\\
-10 & -1 & 15 & 0 & -4\\
8 & 2 & -15 & 0 & 5\\
-10 & -1 & 16 & 0 & -5
\end{bmatrix}
\colvector{3\\0\\3\\-5\\4}
=
\colvector{-1\\2\\-1\\-1\\-2}\\
(A-2I_5)(A+1I_5)\vect{x}&=
\begin{bmatrix}
-9 & -1 & 11 & 0 & -4\\
4 & -1 & 0 & 2 & 0\\
-10 & -1 & 12 & 0 & -4\\
8 & 2 & -15 & -3 & 5\\
-10 & -1 & 16 & 0 & -8
\end{bmatrix}
\colvector{-1\\2\\-1\\-1\\-2}
=
\colvector{4\\-8\\4\\4\\8}\\
(A+2I_5)(A-2I_5)(A+1I_5)\vect{x}&=
\begin{bmatrix}
-5 & -1 & 11 & 0 & -4\\
4 & 3 & 0 & 2 & 0\\
-10 & -1 & 16 & 0 & -4\\
8 & 2 & -15 & 1 & 5\\
-10 & -1 & 16 & 0 & -4
\end{bmatrix}
\colvector{4\\-8\\4\\4\\8}
=
\colvector{0\\0\\0\\0\\0}\\
\end{align*}

So $k=3$ and
\[
\vect{z}=(A-2I_5)(A+1I_5)\vect{x}=\colvector{4\\-8\\4\\4\\8}
\]
is an eigenvector of $A$ for the eigenvalue $\lambda=-2$, as you can
check by doing the computation $A\vect{z}$.  If you work through this
example with your own choice of the vector $\vect{x}$ (strongly
recommended) then the eigenvalue you will find may be different, but
will be in the set $\set{3,\,0,\,1,\,-1,\,-2}$.
\end{example}

\end{document}

%%% Local Variables:
%%% mode: latex
%%% TeX-master: t
%%% End:
