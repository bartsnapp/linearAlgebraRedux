\documentclass{ximera}

% These macros are automatically generated from the "macros"
% XML element.  Make permanent edits there.
%
% History
%   2004/01/01  Initiated for FCLA, evolved from there
%   2006/09/17  Converted  _, ^  to \sb, \sp for TeX4ht
%   2014/02/01  Updated for MathBook XML projects
%               Obsolete in FCLA: \codeindent, \computerfont, \define
%               Change: MathJax wants \lt, so replaced by \lteval
%   2014/02/22  New: \orderof, \reals, \per
%   2015/08/16  Incorporated into MathBook XML version of FCLA
%
%%%%%%%%%%%%%%%%%%%%%
%
%     Conveniences
%
%%%%%%%%%%%%%%%%%%%%%
%
%  Order of (asymptotically limit of fraction is 1)
%  Usage: \orderof{some function}
%
\newcommand{\orderof}[1]{\sim #1}
%
%  Integers
%  Usage:  \Z
\newcommand{\Z}{\mathbb{Z}}
%
%  Real numbers, as set of scalars
%  Usage:  \reals
\newcommand{\reals}{\mathbb{R}}
%
%  n-space over real field
%  Usage: \complex{integer-dimension}
\newcommand{\real}[1]{\mathbb{R}^{#1}}
%
%  Complex numbers, as set of scalars
%  Usage:  \complexes
\newcommand{\complexes}{\mathbb{C}}
%
%  n-space over complex field
%  Usage: \complex{integer-dimension}
\newcommand{\complex}[1]{\mathbb{C}^{#1}}
\newcommand{\CC}{\mathbb{C}}
%
%  Complex conjugation (scalar, vector, matrix)
%  Usage: \conjugate{object}
\newcommand{\conjugate}[1]{\overline{#1}}
%
%  Complex number modulus
%  Usage: \modulus{a+bi}
%  Presumes math mode
\newcommand{\modulus}[1]{\left\lvert#1\right\rvert}
%
%  Zero vector
%  Usage: \zerovector
\newcommand{\zerovector}{\vect{0}}
%
%  Zero matrix
%  Usage: \zeromatrix, use a subscript when size is important
\newcommand{\zeromatrix}{\mathcal{O}}
%
%  Inner product (brackets, not quadratic form)
%  Usage: \innerproduct{a-vector}{a-vector}
\newcommand{\innerproduct}[2]{\left\langle#1,\,#2\right\rangle}
%
%  Norm of a vector
%  Usage: \norm{a-vector}
\newcommand{\norm}[1]{\left\lVert#1\right\rVert}
%
%  Dimension
%  Usage: \dimension{vector-space-letter}
\newcommand{\dimension}[1]{\dim\left(#1\right)}
%
%  Nullity
%  Usage: \nullity{matrix-or-lintrans-letter}
\newcommand{\nullity}[1]{n\left(#1\right)}
%
%  Rank
%  Usage: \rank{matrix-or-lintrans-letter}
\newcommand{\rank}[1]{r\left(#1\right)}
%
%  Direct sum
%  Usage: \ds between a couple of subspaces
%
\newcommand{\ds}{\oplus}
%
%  Determinant of a matrix (functional)
%  Usage: \detname{A}
\newcommand{\detname}[1]{\det\left(#1\right)}
%
%  Determinant of a matrix (vertical bars)
%  Usage: \detbars{A}
\newcommand{\detbars}[1]{\left\lvert#1\right\rvert}
%
%  Trace of a Matrix
%  Usage: \trace{matrix name}
\newcommand{\trace}[1]{t\left(#1\right)}
%
%  Square Root of a Matrix
%  Usage: \sr{a-matrix}
\newcommand{\sr}[1]{#1^{1/2}}
%
%%%%%%%%%%%%%%%%%%%%%
%
%     Subspace Constructions
%
%%%%%%%%%%%%%%%%%%%%%
%
%  Span of a set of vectors
%  \span and \sp are used by TeX for other things
%  Usage: \spn{set-of-vectors}
\newcommand{\spn}[1]{\left\langle#1\right\rangle}
%
%  Null space of a matrix
%  Usage:  \nsp{A}
\newcommand{\nsp}[1]{\mathcal{N}\!\left(#1\right)}
%
%  Column space of a matrix
%  Usage:  \csp{A}
\newcommand{\csp}[1]{\mathcal{C}\!\left(#1\right)}
%
%  Row space of a matrix
%  Usage:  \rsp{A}
\newcommand{\rsp}[1]{\mathcal{R}\!\left(#1\right)}
%
%  Left null space of a matrix
%  Usage:  \lns{A}
\newcommand{\lns}[1]{\mathcal{L}\!\left(#1\right)}
%
%  Orthogonal complement of a vector space
%  Avoiding TeX's \perp
%  Usage:  \per{A}
\newcommand{\per}[1]{#1^\perp}
%
%%%%%%%%%%%%%%%%%%%%%
%
%     Systems of Equations
%
%%%%%%%%%%%%%%%%%%%%%
%
%  In-line form of an augmented matrix for a system of equations
%  Usage: \augmented{coefficient-matrix}{constant-vector}
\newcommand{\augmented}[2]{\left\lbrack\left.#1\,\right\rvert\,#2\right\rbrack}
%
%  Notation for a linear system before introducing matrix multiplication
%  Usage: \linearsystem{coefficient-matrix}{constant-vector}
\newcommand{\linearsystem}[2]{\mathcal{LS}\!\left(#1,\,#2\right)}
%
%  Notation for a homogenous system before introducing matrix multiplication
%  Usage: \homosystem{coefficient-matrix}
\newcommand{\homosystem}[1]{\linearsystem{#1}{\zerovector}}
%
%%%%%%%%%%%%%%%%%%%%%
%
%     Row Operations, Echelon Form
%
%%%%%%%%%%%%%%%%%%%%%
%
% Row operations on matrices
%
% Three commands to shorten up descriptions of gaussian elimination
%
% Usage: \rowopswap{row-i}{row-j}
% Usage: \rowopmult{scalar}{row-i}
% Usage: \rowopadd{scalar}{row-multiplied}{row-added-to}
\newcommand{\rowopswap}[2]{R_{#1}\leftrightarrow R_{#2}}
\newcommand{\rowopmult}[2]{#1R_{#2}}
\newcommand{\rowopadd}[3]{#1R_{#2}+R_{#3}}
%
% Mark leading 1's in echelon form with fbox
% Usage: \leading{a-1-usually}
\newcommand{\leading}[1]{\fbox{#1}}
%
%  Row-reduce arrow
%  Usage:  \rref inbetween a matrix and its reduced row-echelon form
\newcommand{\rref}{\xrightarrow{\text{RREF}}}
%
%  Elementary Matrices
%  Usage: \elemswap{subscript}{subscript}
%  Usage: \elemmult{scalar}{subscript}
%  Usage: \elemadd{scalar}{subscript-mult}{subscript-target}
%
\newcommand{\elemswap}[2]{E_{#1,#2}}
\newcommand{\elemmult}[2]{E_{#2}\left(#1\right)}
\newcommand{\elemadd}[3]{E_{#2,#3}\left(#1\right)}
%
%%%%%%%%%%%%%%%%%%%%%
%
%     2-D Constructions (Lists, Vectors, Matrices)
%
%%%%%%%%%%%%%%%%%%%%%
%
%  A list of scalars of generic length
%  Usage:  \scalarlist{scalar letter}{terminal subscript}
\newcommand{\scalarlist}[2]{{#1}_{1},\,{#1}_{2},\,{#1}_{3},\,\ldots,\,{#1}_{#2}}
%
%  Vector styling, bold (or use wiggles, arrows, whatever)
%  Subscripts go outside this construction
%  Usage: \vect{a symbol to use as a vector}
%  Have to already be in math mode
%
\newcommand{\vect}[1]{\mathbf{#1}}
%
%  A column vector
%  Usage: \colvector{list-delimited-by-\\}
%
\newcommand{\colvector}[1]{\begin{bmatrix}#1\end{bmatrix}}
%
%  A generic vector with components
%  Usage: \vectorcomponents{component-letter}{final-subscript}
\newcommand{\vectorcomponents}[2]{\colvector{#1_{1}\\#1_{2}\\#1_{3}\\\vdots\\#1_{#2}}}
%
%  A list of vectors of generic length
%  Usage:  \vectorlist{vector letter}{terminal subscript}
\newcommand{\vectorlist}[2]{\vect{#1}_{1},\,\vect{#1}_{2},\,\vect{#1}_{3},\,\ldots,\,\vect{#1}_{#2}}
%
%  Vector entries, entry i of vector v
%  (vector-expession still needs \vect, etc.)
%  Usage:  \vectorentry{vector-expression}{single-subscript}
\newcommand{\vectorentry}[2]{\left\lbrack#1\right\rbrack_{#2}}
%
%  Matrix entries, entry i,j of matrix A
%  Usage:  \matrixentry{matrix-expression}{paired-subscripts}
%
\newcommand{\matrixentry}[2]{\left\lbrack#1\right\rbrack_{#2}}
%
%  A generic linear combination
%  Usage:  \lincombo{scalar letter}{vector letter}{terminal subscript}
\newcommand{\lincombo}[3]{#1_{1}\vect{#2}_{1}+#1_{2}\vect{#2}_{2}+#1_{3}\vect{#2}_{3}+\cdots +#1_{#3}\vect{#2}_{#3}}
%
%  Matrix, column by column, as vectors
%  Usage:  \matrixcolumns{matrix letter}{terminal subscript}
\newcommand{\matrixcolumns}[2]{\left\lbrack\vect{#1}_{1}|\vect{#1}_{2}|\vect{#1}_{3}|\ldots|\vect{#1}_{#2}\right\rbrack}
%
%%%%%%%%%%%%%%%%%%%%%
%
%     Special Matrices
%
%%%%%%%%%%%%%%%%%%%%%
%
%  Transpose of a matrix
%  Usage:  \transpose{A}
\newcommand{\transpose}[1]{#1^{t}}
%
%  Inverse of a matrix
%  Usage:  \inverse{A}
\newcommand{\inverse}[1]{#1^{-1}}
%
%  Submatrix (for minors, determinants)
%  Usage: \submatrix{matrix-name}{delete-row}{delete-col}
\newcommand{\submatrix}[3]{#1\left(#2|#3\right)}
%
%  Adjoint of a matrix (twice)
%  This macro is a convenience to call \transpose and \conjugate properly
%  It shouldn't need to be modified (or mathematical meanings will change)
%  Usage:  \adj{A}
\newcommand{\adj}[1]{\transpose{\left(\conjugate{#1}\right)}}
%
%  This macro controls the symbol used for the adjoint
%  It can be edited to taste
%  Usage:  \adjoint{A}
\newcommand{\adjoint}[1]{#1^\ast}
%
%%%%%%%%%%%%%%%%%%%%%
%
%     Sets
%
%%%%%%%%%%%%%%%%%%%%%
%
%  A convenience for simple sets
%  Usage:  \set{list of element}
\newcommand{\set}[1]{\left\{#1\right\}}
%
%  Sets with vertical bar, "such that", sized for objects, not condition
%  Usage:  \setparts{objects}{condition}
%
%%\newcommand{\setparts}[2]{\left\{ #1\mid#2\right\}}
%%\newcommand{\setparts}[2]{\left\{\left. #1\right\rvert#2\right\}}
\newcommand{\setparts}[2]{\left\lbrace#1\,\middle|\,#2\right\rbrace}
%
%  Set Cardinality
%  Usage:  \card{a-set-letter}
\newcommand{\card}[1]{\left\lvert#1\right\rvert}
%
%  Set Union
%  Use \cup
%
%  Set Intersection
%  Use \cap
%
%  Set Complement
%  Usage:  \setcomplement{a-set-letter}
\newcommand{\setcomplement}[1]{\overline{#1}}
%
%%%%%%%%%%%%%%%%%%%%%
%
%     Eigenvalues and Eigenspaces
%
%%%%%%%%%%%%%%%%%%%%%
%
%  Characteristic polynomial
%  Usage: \charpoly{matrix-letter}{variable-letter}
\newcommand{\charpoly}[2]{p_{#1}\left(#2\right)}
%
%  Eigenspace
%  Usage: \eigenspace{matrix-letter}{eigenvalue-letter}
\newcommand{\eigenspace}[2]{\mathcal{E}_{#1}\left(#2\right)}
%
%  2013/10/03 Including ampersands is problematic here, 
%  think about fixes later
%  2014/02/22 Limited testing, seems &amp; is fine for HTML and LaTeX
%  2016-07-20 only employed in Archetypes, MBX has gather/align override
%  Eigensystem (presumes wrapped in an mrow within md)
%  Usage: \eigensystem{matrixletter}{eigenvalue}{list of basis vectors}
\newcommand{\eigensystem}[3]{\lambda&amp;=#2&amp;\eigenspace{#1}{#2}&amp;=\spn{\set{#3}}} 
%
%  Generalized Eigenspace
%  Usage: \geneigenspace{lin-trans-letter}{eigenvalue-letter}
\newcommand{\geneigenspace}[2]{\mathcal{G}_{#1}\left(#2\right)}
%
%  Algebraic multiplicty
%  Usage: \algmult{matrix-letter}{eigenvalue-letter}
\newcommand{\algmult}[2]{\alpha_{#1}\left(#2\right)}
%
%  Geometric multiplicty
%  Usage: \geomult{matrix-letter}{eigenvalue-letter}
\newcommand{\geomult}[2]{\gamma_{#1}\left(#2\right)}
%
%  Index (of eigenvalue)
%  Usage: \indx{matrix-letter}{eigenvalue-letter}
\newcommand{\indx}[2]{\iota_{#1}\left(#2\right)}
%
%%%%%%%%%%%%%%%%%%%%%
%
%     Linear Transformations
%
%%%%%%%%%%%%%%%%%%%%%
%
%  MathJax defines \lt to ease XML confusion
%
%  Linear transformation definition
%  Usage: \ltdefn{name-letter}{domain}{range}
\newcommand{\ltdefn}[3]{#1\colon #2\rightarrow#3}
%
%  Linear transformation evaluation
%  Usage: \lteval{name-letter}{input}
%  Replaces old \lt desired by MathJax
\newcommand{\lteval}[2]{#1\left(#2\right)}
%
% Linear transformation inverse
%  Usage: \ltinverse{name-letter}
\newcommand{\ltinverse}[1]{#1^{-1}}
%
%  Linear transformation restriction
%  Usage: \restrict{name-letter}{subspace-letter}
\newcommand{\restrict}[2]{{#1}|_{#2}}
%
%  Linear transformation preimage
%  Usage: \preimage{name-letter}{codomain-element}
\newcommand{\preimage}[2]{#1^{-1}\left(#2\right)}
%
%  Range of a linear transformation
%  TeX uses \range for something else
%  Usage:  \rng{T}
\newcommand{\rng}[1]{\mathcal{R}\!\left(#1\right)}
%
%  Kernel of a linear transformation
%  TeX uses \ker to do something different
%  Usage:  \krn{T}
\newcommand{\krn}[1]{\mathcal{K}\!\left(#1\right)}
%
%  Linear transformation composition
%  Usage: \compose{function-name}{function-name}
\newcommand{\compose}[2]{{#1}\circ{#2}}
%
%  Vector space of linear transformations
%  Usage: \vslt{domains}{codomains}
%  Presumes math mode
\newcommand{\vslt}[2]{\mathcal{LT}\left(#1,\,#2\right)}
%
%%%%%%%%%%%%%%%%%%%%%
%
%     Vector and Matrix Representations
%
%%%%%%%%%%%%%%%%%%%%%
%
%  Isomorphism symbol
%  Usage: \isomorphic
\newcommand{\isomorphic}{\cong}
%
%  Similarity
%  Usage: \similar{inner-matrix}{outer-invertible-matrix}
%  Rearranging this will not "fix" all desired changes throughout
%
\newcommand{\similar}[2]{\inverse{#2}#1#2}
%
%  Vector representation function name
%  Usage: \vectrepname{basis-letter}
\newcommand{\vectrepname}[1]{\rho_{#1}}
%
%  Vector representation output
%  Usage: \vectrep{basis-letter}{input}
\newcommand{\vectrep}[2]{\lteval{\vectrepname{#1}}{#2}}
%
%  Vector representation inverse function name
%  (Added later, not used consistently in FCLA)
%  Usage: \vectrepinvname{basis-letter}
\newcommand{\vectrepinvname}[1]{\ltinverse{\vectrepname{#1}}}
%
%  Vector representation inverse output
%  Usage: \vectrepinv{basis-letter}{input}
\newcommand{\vectrepinv}[2]{\lteval{\ltinverse{\vectrepname{#1}}}{#2}}
%
%  Matrix representation
%  Usage: \matrixrep{transformation-letter}{domain-basis-letter}{codomain-basis-letter}
\newcommand{\matrixrep}[3]{M^{#1}_{#2,#3}}
%
%  Matrix representation column-by-colum
%  2016-07-20 only employed once?
%  Usage: \matrixrepcolumns{transformation-letter}{codomain-basis-letter}{codomain-basis-vector-letter}{final-index}
\newcommand{\matrixrepcolumns}[4]{\left\lbrack \left.\vectrep{#2}{\lteval{#1}{\vect{#3}_{1}}}\right|\left.\vectrep{#2}{\lteval{#1}{\vect{#3}_{2}}}\right|\left.\vectrep{#2}{\lteval{#1}{\vect{#3}_{3}}}\right|\ldots\left|\vectrep{#2}{\lteval{#1}{\vect{#3}_{#4}}}\right.\right\rbrack}
%
%  Change of basis matrix
%  Usage: \cbm{domain-basis-letter}{codomain-basis-letter}
\newcommand{\cbm}[2]{C_{#1,#2}}
%
%%%%%%%%%%%%%%%%%%%%%
%
%     Canonical Forms
%
%%%%%%%%%%%%%%%%%%%%%
%
%  Jordan blocks
%  Usage: \jordan{size}{diagonal-element}
\newcommand{\jordan}[2]{J_{#1}\left(#2\right)}
%
%%%%%%%%%%%%%%%%%%%%%
%
%     Hadamard Matrices
%     Contributed by Elizabeth Million
%
%%%%%%%%%%%%%%%%%%%%%
%
%  Hadamard Product
%  Usage: \hadamard{a-matrix}{a-matrix}
\newcommand{\hadamard}[2]{#1\circ #2}
%
%  Hadamard identity matrix
%  Usage: \hadamardidentity{paired-subscripts-size-of-matrix}
\newcommand{\hadamardidentity}[1]{J_{#1}}
%
%  Hadamard inverse matrix
%  Usage: \hadamardinverse{matrix-expression}
\newcommand{\hadamardinverse}[1]{\widehat{#1}}

\newcommand{\definedTerm}[1]{\textbf{#1}}
\newcommand{\dfn}[1]{\textbf{#1}}

\newcommand{\wt}{\widetilde}
\newcommand{\ov}{\overline}
\newcommand{\inj}{\rightarrowtail}
\newcommand{\surj}{\twoheadrightarrow}
\newcommand{\harpoon}{\overset{\rightharpoonup}}

\newenvironment{amatrix}[1]{%
  \left[\begin{array}{@{}*{#1}{c}|c@{}}
}{%
  \end{array}\right]
}


\title{Multiplicities of Eigenvalues}

\begin{document}
\begin{abstract}
  Our understanding of the roots of a polynomial informs our
  understanding of the algebraic multiplicity of eigenvalues.
\end{abstract}
\maketitle

A polynomial of degree $n$ will have exactly $n$ roots.  From this fact about polynomial equations we can say more about the algebraic multiplicities of eigenvalues.

\begin{theorem}[Degree of the Characteristic Polynomial]
\label{theorem:DCP}

Suppose that $A$ is a square matrix of size $n$.  Then the
characteristic polynomial of $A$, $\charpoly{A}{x}$, has degree $n$.

\begin{proof}
  We will prove a more general result by induction Then the theorem
  will be true as a special case.  We will carefully state this result
  as a proposition indexed by $m$, $m\geq 1$.

  $P(m)$: Suppose that $A$ is an $m\times m$ matrix whose entries are
  complex numbers or linear polynomials in the variable $x$ of the
  form $c-x$, where $c$ is a complex number.  Suppose further that
  there are exactly $k$ entries that contain $x$ and that no row or
  column contains more than one such entry.  Then, when $k=m$,
  $\detname{A}$ is a polynomial in $x$ of degree $m$, with leading
  coefficient $\pm 1$, and when $k<m$, $\detname{A}$ is a polynomial
  in $x$ of degree $k$ or less.

  Base Case: Suppose $A$ is a $1\times 1$ matrix.  Then its
  determinant is equal to the lone entry (\ref{definition:DM}).  When
  $k=m=1$, the entry is of the form $c-x$, a polynomial in $x$ of
  degree $m=1$ with leading coefficient $-1$.  When $k<m$, then $k=0$
  and the entry is simply a complex number, a polynomial of degree
  $0\leq k$.  So $P(1)$ is true.

  Induction Step: Assume $P(m)$ is true, and that $A$ is an
  $(m+1)\times(m+1)$ matrix with $k$ entries of the form $c-x$.  There
  are two cases to consider.

  Suppose $k=m+1$.  Then every row and every column will contain an
  entry of the form $c-x$.  Suppose that for the first row, this entry
  is in column $t$.  Compute the determinant of $A$ by an expansion
  about this first row (\ref{definition:DM}).  The term associated
  with entry $t$ of this row will be of the form
  \[
    (c-x)(-1)^{1+t}\detname{\submatrix{A}{1}{t}}
  \]
  The submatrix $\submatrix{A}{1}{t}$ is an $m\times m$ matrix with
  $k=m$ terms of the form $c-x$, no more than one per row or column.
  By the induction hypothesis, $\detname{\submatrix{A}{1}{t}}$ will be
  a polynomial in $x$ of degree $m$ with coefficient $\pm 1$.  So this
  entire term is then a polynomial of degree $m+1$ with leading
  coefficient $\pm 1$.

  The remaining terms (which constitute the sum that is the
  determinant of $A$) are products of complex numbers from the first
  row with cofactors built from submatrices that lack the first row of
  $A$ and lack some column of $A$, other than column $t$.  As such,
  these submatrices are $m\times m$ matrices with $k=m-1<m$ entries of
  the form $c-x$, no more than one per row or column.  Applying the
  induction hypothesis, we see that these terms are polynomials in $x$
  of degree $m-1$ or less.  Adding the single term from the entry in
  column $t$ with all these others, we see that $\detname{A}$ is a
  polynomial in $x$ of degree $m+1$ and leading coefficient $\pm 1$.


  The second case occurs when $k<m+1$.  Now there is a row of $A$ that
  does not contain an entry of the form $c-x$.  We consider the
  determinant of $A$ by expanding about this row (\ref{theorem:DER}),
  whose entries are all complex numbers.  The cofactors employed are
  built from submatrices that are $m\times m$ matrices with either $k$
  or $k-1$ entries of the form $c-x$, no more than one per row or
  column.  In either case, $k\leq m$, and we can apply the induction
  hypothesis to see that the determinants computed for the cofactors
  are all polynomials of degree $k$ or less.  Summing these
  contributions to the determinant of $A$ yields a polynomial in $x$
  of degree $k$ or less, as desired.

  The characteristic polynomial of an $n\times n$ matrix is the
  determinant of a matrix having exactly $n$ entries of the form
  $c-x$, no more than one per row or column.  As such we can apply
  $P(n)$ to see that the characteristic polynomial has degree $n$.
\end{proof}
\end{theorem}

\begin{theorem}[Number of Eigenvalues of a Matrix]
\label{theorem:NEM}

Suppose that $\scalarlist{\lambda}{k}$ are the distinct eigenvalues of
a square matrix $A$ of size $n$.  Then
\[
  \sum_{i=1}^{k}\algmult{A}{\lambda_i}=\answer{n}.
\]

\begin{proof}
  By the definition of the algebraic multiplicity (\ref{definition:AME}), we can factor the characteristic polynomial as
\[
  \charpoly{A}{x}=c(x-\lambda_1)^{\algmult{A}{\lambda_1}}(x-\lambda_2)^{\algmult{A}{\lambda_2}}(x-\lambda_3)^{\algmult{A}{\lambda_3}}\cdots(x-\lambda_k)^{\algmult{A}{\lambda_k}}
\]
where $c$ is a nonzero constant.  (We could prove that $c=(-1)^{n}$,
but we do not need that specificity right now.  The left-hand side is
a polynomial of degree $n$ by \ref{theorem:DCP} and the right-hand
side is a polynomial of degree $\sum_{i=1}^{k}\algmult{A}{\lambda_i}$.
So the equality of the polynomials' degrees gives the equality
$\sum_{i=1}^{k}\algmult{A}{\lambda_i}=n$.
\end{proof}
\end{theorem}

\begin{theorem}[Multiplicities of an Eigenvalue]
\label{theorem:ME}

Suppose that $A$ is a square matrix of size $n$ and $\lambda$ is an eigenvalue.  Then
\[
1\leq\geomult{A}{\lambda}\leq\algmult{A}{\lambda}\leq \answer{n}
\]

\begin{proof}
Since $\lambda$ is an eigenvalue of $A$, there is an eigenvector of $A$ for $\lambda$, $\vect{x}$.  Then $\vect{x}\in\eigenspace{A}{\lambda}$, so $\geomult{A}{\lambda}\geq 1$, since we can extend $\set{\vect{x}}$ into a basis of $\eigenspace{A}{\lambda}$ (\ref{theorem:ELIS}).

To show $\geomult{A}{\lambda}\leq\algmult{A}{\lambda}$ is the most involved portion of this proof.  To this end, let $g=\geomult{A}{\lambda}$ and let $\vectorlist{x}{g}$ be a basis for the eigenspace of $\lambda$, $\eigenspace{A}{\lambda}$.  Construct another $n-g$ vectors, $\vectorlist{y}{n-g}$, so that
\[
\set{\vectorlist{x}{g},\,\vectorlist{y}{n-g}}
\]
is a basis of $\complex{n}$.  This can be done by repeated applications of \ref{theorem:ELIS}.



Finally, define a matrix $S$ by
\[
S=[\vect{x}_1|\vect{x}_2|\vect{x}_3|\ldots|\vect{x}_g|\vect{y}_1|\vect{y}_2|\vect{y}_3|\ldots|\vect{y}_{n-g}]
=[\vect{x}_1|\vect{x}_2|\vect{x}_3|\ldots|\vect{x}_g|R]
\]
where $R$ is an $n\times(n-g)$ matrix whose columns are $\vectorlist{y}{n-g}$.  The columns of $S$ are linearly independent by design, so $S$ is nonsingular (\ref{theorem:NMLIC}) and therefore invertible (\ref{theorem:NI}).



Then,
\begin{align*}
\matrixcolumns{e}{n}&=I_n\\
&=\inverse{S}S\\
&=\inverse{S}[\vect{x}_1|\vect{x}_2|\vect{x}_3|\ldots|\vect{x}_g|R]\\
&=[\inverse{S}\vect{x}_1|\inverse{S}\vect{x}_2|\inverse{S}\vect{x}_3|\ldots|\inverse{S}\vect{x}_g|\inverse{S}R]
\end{align*}




So
\[
\inverse{S}\vect{x}_i=\vect{e}_i\quad 1\leq i\leq g\tag{$*$}
\]




Preparations in place, we compute the characteristic polynomial of $A$,
\begin{align*}
\charpoly{A}{x}&=\detname{A-xI_n}&&\ref{definition:CP}\\
&=1\detname{A-xI_n}&&\ref{property:OCN}\\
&=\detname{I_n}\detname{A-xI_n}&&\ref{definition:DM}\\
&=\detname{\inverse{S}S}\detname{A-xI_n}&&\ref{definition:MI}\\
&=\detname{\inverse{S}}\detname{S}\detname{A-xI_n}&&\ref{theorem:DRMM}\\
&=\detname{\inverse{S}}\detname{A-xI_n}\detname{S}&&\ref{property:CMCN}\\
&=\detname{\inverse{S}\left(A-xI_n\right)S}&&\ref{theorem:DRMM}\\
&=\detname{\inverse{S}AS-\inverse{S}xI_nS}&&\ref{theorem:MMDAA}\\
&=\detname{\inverse{S}AS-x\inverse{S}I_nS}&&\ref{theorem:MMSMM}\\
&=\detname{\inverse{S}AS-x\inverse{S}S}&&\ref{theorem:MMIM}\\
&=\detname{\inverse{S}AS-xI_n}&&\ref{definition:MI}\\
&=\charpoly{\inverse{S}AS}{x}&&\ref{definition:CP}
\end{align*}




What can we learn then about the matrix $\inverse{S}AS$?
\begin{align*}
\inverse{S}AS&=\inverse{S}A[\vect{x}_1|\vect{x}_2|\vect{x}_3|\ldots|\vect{x}_g|R]\\
&=\inverse{S}[A\vect{x}_1|A\vect{x}_2|A\vect{x}_3|\ldots|A\vect{x}_g|AR]&&\ref{definition:MM}\\
&=\inverse{S}[\lambda\vect{x}_1|\lambda\vect{x}_2|\lambda\vect{x}_3|\ldots|\lambda\vect{x}_g|AR]&&\ref{definition:EEM}\\
&=[\inverse{S}\lambda\vect{x}_1|\inverse{S}\lambda\vect{x}_2|\inverse{S}\lambda\vect{x}_3|\ldots|\inverse{S}\lambda\vect{x}_g|\inverse{S}AR]&&\ref{definition:MM}\\
&=[\lambda\inverse{S}\vect{x}_1|\lambda\inverse{S}\vect{x}_2|\lambda\inverse{S}\vect{x}_3|\ldots|\lambda\inverse{S}\vect{x}_g|\inverse{S}AR]&&\ref{theorem:MMSMM}\\
&=[\lambda\vect{e}_1|\lambda\vect{e}_2|\lambda\vect{e}_3|\ldots|\lambda\vect{e}_g|\inverse{S}AR]&&\text{$\inverse{S}S=I_n$, (($*$) above)}
\end{align*}




Now imagine computing the characteristic polynomial of $A$ by computing the characteristic polynomial of $\inverse{S}AS$ using the form just obtained.  The first $g$ columns of $\inverse{S}AS$ are all zero, save for a $\lambda$ on the diagonal.  So if we compute the determinant by expanding about the first column, successively, we will get successive factors of $(\lambda-x)$.  More precisely, let $T$ be the square matrix of size $n-g$ that is formed from the last $n-g$ rows and last $n-g$ columns of $\inverse{S}AR$.  Then
\[
\charpoly{A}{x}=\charpoly{\inverse{S}AS}{x}=(\lambda-x)^g\charpoly{T}{x}.
\]




This says that $(x-\lambda)$ is a factor of the characteristic polynomial <em>at least</em> $g$ times, so the algebraic multiplicity of $\lambda$ as an eigenvalue of $A$ is greater than or equal to $g$ (\ref{definition:AME}).  In other words,
\[
\geomult{A}{\lambda}=g\leq\algmult{A}{\lambda}
\]
as desired.



\ref{theorem:NEM} says that the sum of the algebraic multiplicities for \textit{all} the eigenvalues of $A$ is equal to $n$.  Since the algebraic multiplicity is a positive quantity, no single algebraic multiplicity can exceed $n$ without the sum of all of the algebraic multiplicities doing the same.



\end{proof}
\end{theorem}

\begin{theorem}[Maximum Number of Eigenvalues of a Matrix]
\label{theorem:MNEM}

Suppose that $A$ is a square matrix of size $n$.  Then $A$ cannot have more than $\answer{n}$ distinct eigenvalues.

\begin{proof}
Suppose that $A$ has $k$ distinct eigenvalues, $\scalarlist{\lambda}{k}$.  Then
\begin{align*}
k&=\sum_{i=1}^{k}1\\
&\leq\sum_{i=1}^{k}\algmult{A}{\lambda_i}&&\ref{theorem:ME}\\
&=n&&\ref{theorem:NEM}
\end{align*}




\end{proof}
\end{theorem}

\end{document}

%%% Local Variables:
%%% mode: latex
%%% TeX-master: t
%%% End:
