\documentclass{ximera}

% These macros are automatically generated from the "macros"
% XML element.  Make permanent edits there.
%
% History
%   2004/01/01  Initiated for FCLA, evolved from there
%   2006/09/17  Converted  _, ^  to \sb, \sp for TeX4ht
%   2014/02/01  Updated for MathBook XML projects
%               Obsolete in FCLA: \codeindent, \computerfont, \define
%               Change: MathJax wants \lt, so replaced by \lteval
%   2014/02/22  New: \orderof, \reals, \per
%   2015/08/16  Incorporated into MathBook XML version of FCLA
%
%%%%%%%%%%%%%%%%%%%%%
%
%     Conveniences
%
%%%%%%%%%%%%%%%%%%%%%
%
%  Order of (asymptotically limit of fraction is 1)
%  Usage: \orderof{some function}
%
\newcommand{\orderof}[1]{\sim #1}
%
%  Integers
%  Usage:  \Z
\newcommand{\Z}{\mathbb{Z}}
%
%  Real numbers, as set of scalars
%  Usage:  \reals
\newcommand{\reals}{\mathbb{R}}
%
%  n-space over real field
%  Usage: \complex{integer-dimension}
\newcommand{\real}[1]{\mathbb{R}^{#1}}
%
%  Complex numbers, as set of scalars
%  Usage:  \complexes
\newcommand{\complexes}{\mathbb{C}}
%
%  n-space over complex field
%  Usage: \complex{integer-dimension}
\newcommand{\complex}[1]{\mathbb{C}^{#1}}
\newcommand{\CC}{\mathbb{C}}
%
%  Complex conjugation (scalar, vector, matrix)
%  Usage: \conjugate{object}
\newcommand{\conjugate}[1]{\overline{#1}}
%
%  Complex number modulus
%  Usage: \modulus{a+bi}
%  Presumes math mode
\newcommand{\modulus}[1]{\left\lvert#1\right\rvert}
%
%  Zero vector
%  Usage: \zerovector
\newcommand{\zerovector}{\vect{0}}
%
%  Zero matrix
%  Usage: \zeromatrix, use a subscript when size is important
\newcommand{\zeromatrix}{\mathcal{O}}
%
%  Inner product (brackets, not quadratic form)
%  Usage: \innerproduct{a-vector}{a-vector}
\newcommand{\innerproduct}[2]{\left\langle#1,\,#2\right\rangle}
%
%  Norm of a vector
%  Usage: \norm{a-vector}
\newcommand{\norm}[1]{\left\lVert#1\right\rVert}
%
%  Dimension
%  Usage: \dimension{vector-space-letter}
\newcommand{\dimension}[1]{\dim\left(#1\right)}
%
%  Nullity
%  Usage: \nullity{matrix-or-lintrans-letter}
\newcommand{\nullity}[1]{n\left(#1\right)}
%
%  Rank
%  Usage: \rank{matrix-or-lintrans-letter}
\newcommand{\rank}[1]{r\left(#1\right)}
%
%  Direct sum
%  Usage: \ds between a couple of subspaces
%
\newcommand{\ds}{\oplus}
%
%  Determinant of a matrix (functional)
%  Usage: \detname{A}
\newcommand{\detname}[1]{\det\left(#1\right)}
%
%  Determinant of a matrix (vertical bars)
%  Usage: \detbars{A}
\newcommand{\detbars}[1]{\left\lvert#1\right\rvert}
%
%  Trace of a Matrix
%  Usage: \trace{matrix name}
\newcommand{\trace}[1]{t\left(#1\right)}
%
%  Square Root of a Matrix
%  Usage: \sr{a-matrix}
\newcommand{\sr}[1]{#1^{1/2}}
%
%%%%%%%%%%%%%%%%%%%%%
%
%     Subspace Constructions
%
%%%%%%%%%%%%%%%%%%%%%
%
%  Span of a set of vectors
%  \span and \sp are used by TeX for other things
%  Usage: \spn{set-of-vectors}
\newcommand{\spn}[1]{\left\langle#1\right\rangle}
%
%  Null space of a matrix
%  Usage:  \nsp{A}
\newcommand{\nsp}[1]{\mathcal{N}\!\left(#1\right)}
%
%  Column space of a matrix
%  Usage:  \csp{A}
\newcommand{\csp}[1]{\mathcal{C}\!\left(#1\right)}
%
%  Row space of a matrix
%  Usage:  \rsp{A}
\newcommand{\rsp}[1]{\mathcal{R}\!\left(#1\right)}
%
%  Left null space of a matrix
%  Usage:  \lns{A}
\newcommand{\lns}[1]{\mathcal{L}\!\left(#1\right)}
%
%  Orthogonal complement of a vector space
%  Avoiding TeX's \perp
%  Usage:  \per{A}
\newcommand{\per}[1]{#1^\perp}
%
%%%%%%%%%%%%%%%%%%%%%
%
%     Systems of Equations
%
%%%%%%%%%%%%%%%%%%%%%
%
%  In-line form of an augmented matrix for a system of equations
%  Usage: \augmented{coefficient-matrix}{constant-vector}
\newcommand{\augmented}[2]{\left\lbrack\left.#1\,\right\rvert\,#2\right\rbrack}
%
%  Notation for a linear system before introducing matrix multiplication
%  Usage: \linearsystem{coefficient-matrix}{constant-vector}
\newcommand{\linearsystem}[2]{\mathcal{LS}\!\left(#1,\,#2\right)}
%
%  Notation for a homogenous system before introducing matrix multiplication
%  Usage: \homosystem{coefficient-matrix}
\newcommand{\homosystem}[1]{\linearsystem{#1}{\zerovector}}
%
%%%%%%%%%%%%%%%%%%%%%
%
%     Row Operations, Echelon Form
%
%%%%%%%%%%%%%%%%%%%%%
%
% Row operations on matrices
%
% Three commands to shorten up descriptions of gaussian elimination
%
% Usage: \rowopswap{row-i}{row-j}
% Usage: \rowopmult{scalar}{row-i}
% Usage: \rowopadd{scalar}{row-multiplied}{row-added-to}
\newcommand{\rowopswap}[2]{R_{#1}\leftrightarrow R_{#2}}
\newcommand{\rowopmult}[2]{#1R_{#2}}
\newcommand{\rowopadd}[3]{#1R_{#2}+R_{#3}}
%
% Mark leading 1's in echelon form with fbox
% Usage: \leading{a-1-usually}
\newcommand{\leading}[1]{\fbox{#1}}
%
%  Row-reduce arrow
%  Usage:  \rref inbetween a matrix and its reduced row-echelon form
\newcommand{\rref}{\xrightarrow{\text{RREF}}}
%
%  Elementary Matrices
%  Usage: \elemswap{subscript}{subscript}
%  Usage: \elemmult{scalar}{subscript}
%  Usage: \elemadd{scalar}{subscript-mult}{subscript-target}
%
\newcommand{\elemswap}[2]{E_{#1,#2}}
\newcommand{\elemmult}[2]{E_{#2}\left(#1\right)}
\newcommand{\elemadd}[3]{E_{#2,#3}\left(#1\right)}
%
%%%%%%%%%%%%%%%%%%%%%
%
%     2-D Constructions (Lists, Vectors, Matrices)
%
%%%%%%%%%%%%%%%%%%%%%
%
%  A list of scalars of generic length
%  Usage:  \scalarlist{scalar letter}{terminal subscript}
\newcommand{\scalarlist}[2]{{#1}_{1},\,{#1}_{2},\,{#1}_{3},\,\ldots,\,{#1}_{#2}}
%
%  Vector styling, bold (or use wiggles, arrows, whatever)
%  Subscripts go outside this construction
%  Usage: \vect{a symbol to use as a vector}
%  Have to already be in math mode
%
\newcommand{\vect}[1]{\mathbf{#1}}
%
%  A column vector
%  Usage: \colvector{list-delimited-by-\\}
%
\newcommand{\colvector}[1]{\begin{bmatrix}#1\end{bmatrix}}
%
%  A generic vector with components
%  Usage: \vectorcomponents{component-letter}{final-subscript}
\newcommand{\vectorcomponents}[2]{\colvector{#1_{1}\\#1_{2}\\#1_{3}\\\vdots\\#1_{#2}}}
%
%  A list of vectors of generic length
%  Usage:  \vectorlist{vector letter}{terminal subscript}
\newcommand{\vectorlist}[2]{\vect{#1}_{1},\,\vect{#1}_{2},\,\vect{#1}_{3},\,\ldots,\,\vect{#1}_{#2}}
%
%  Vector entries, entry i of vector v
%  (vector-expession still needs \vect, etc.)
%  Usage:  \vectorentry{vector-expression}{single-subscript}
\newcommand{\vectorentry}[2]{\left\lbrack#1\right\rbrack_{#2}}
%
%  Matrix entries, entry i,j of matrix A
%  Usage:  \matrixentry{matrix-expression}{paired-subscripts}
%
\newcommand{\matrixentry}[2]{\left\lbrack#1\right\rbrack_{#2}}
%
%  A generic linear combination
%  Usage:  \lincombo{scalar letter}{vector letter}{terminal subscript}
\newcommand{\lincombo}[3]{#1_{1}\vect{#2}_{1}+#1_{2}\vect{#2}_{2}+#1_{3}\vect{#2}_{3}+\cdots +#1_{#3}\vect{#2}_{#3}}
%
%  Matrix, column by column, as vectors
%  Usage:  \matrixcolumns{matrix letter}{terminal subscript}
\newcommand{\matrixcolumns}[2]{\left\lbrack\vect{#1}_{1}|\vect{#1}_{2}|\vect{#1}_{3}|\ldots|\vect{#1}_{#2}\right\rbrack}
%
%%%%%%%%%%%%%%%%%%%%%
%
%     Special Matrices
%
%%%%%%%%%%%%%%%%%%%%%
%
%  Transpose of a matrix
%  Usage:  \transpose{A}
\newcommand{\transpose}[1]{#1^{t}}
%
%  Inverse of a matrix
%  Usage:  \inverse{A}
\newcommand{\inverse}[1]{#1^{-1}}
%
%  Submatrix (for minors, determinants)
%  Usage: \submatrix{matrix-name}{delete-row}{delete-col}
\newcommand{\submatrix}[3]{#1\left(#2|#3\right)}
%
%  Adjoint of a matrix (twice)
%  This macro is a convenience to call \transpose and \conjugate properly
%  It shouldn't need to be modified (or mathematical meanings will change)
%  Usage:  \adj{A}
\newcommand{\adj}[1]{\transpose{\left(\conjugate{#1}\right)}}
%
%  This macro controls the symbol used for the adjoint
%  It can be edited to taste
%  Usage:  \adjoint{A}
\newcommand{\adjoint}[1]{#1^\ast}
%
%%%%%%%%%%%%%%%%%%%%%
%
%     Sets
%
%%%%%%%%%%%%%%%%%%%%%
%
%  A convenience for simple sets
%  Usage:  \set{list of element}
\newcommand{\set}[1]{\left\{#1\right\}}
%
%  Sets with vertical bar, "such that", sized for objects, not condition
%  Usage:  \setparts{objects}{condition}
%
%%\newcommand{\setparts}[2]{\left\{ #1\mid#2\right\}}
%%\newcommand{\setparts}[2]{\left\{\left. #1\right\rvert#2\right\}}
\newcommand{\setparts}[2]{\left\lbrace#1\,\middle|\,#2\right\rbrace}
%
%  Set Cardinality
%  Usage:  \card{a-set-letter}
\newcommand{\card}[1]{\left\lvert#1\right\rvert}
%
%  Set Union
%  Use \cup
%
%  Set Intersection
%  Use \cap
%
%  Set Complement
%  Usage:  \setcomplement{a-set-letter}
\newcommand{\setcomplement}[1]{\overline{#1}}
%
%%%%%%%%%%%%%%%%%%%%%
%
%     Eigenvalues and Eigenspaces
%
%%%%%%%%%%%%%%%%%%%%%
%
%  Characteristic polynomial
%  Usage: \charpoly{matrix-letter}{variable-letter}
\newcommand{\charpoly}[2]{p_{#1}\left(#2\right)}
%
%  Eigenspace
%  Usage: \eigenspace{matrix-letter}{eigenvalue-letter}
\newcommand{\eigenspace}[2]{\mathcal{E}_{#1}\left(#2\right)}
%
%  2013/10/03 Including ampersands is problematic here, 
%  think about fixes later
%  2014/02/22 Limited testing, seems &amp; is fine for HTML and LaTeX
%  2016-07-20 only employed in Archetypes, MBX has gather/align override
%  Eigensystem (presumes wrapped in an mrow within md)
%  Usage: \eigensystem{matrixletter}{eigenvalue}{list of basis vectors}
\newcommand{\eigensystem}[3]{\lambda&amp;=#2&amp;\eigenspace{#1}{#2}&amp;=\spn{\set{#3}}} 
%
%  Generalized Eigenspace
%  Usage: \geneigenspace{lin-trans-letter}{eigenvalue-letter}
\newcommand{\geneigenspace}[2]{\mathcal{G}_{#1}\left(#2\right)}
%
%  Algebraic multiplicty
%  Usage: \algmult{matrix-letter}{eigenvalue-letter}
\newcommand{\algmult}[2]{\alpha_{#1}\left(#2\right)}
%
%  Geometric multiplicty
%  Usage: \geomult{matrix-letter}{eigenvalue-letter}
\newcommand{\geomult}[2]{\gamma_{#1}\left(#2\right)}
%
%  Index (of eigenvalue)
%  Usage: \indx{matrix-letter}{eigenvalue-letter}
\newcommand{\indx}[2]{\iota_{#1}\left(#2\right)}
%
%%%%%%%%%%%%%%%%%%%%%
%
%     Linear Transformations
%
%%%%%%%%%%%%%%%%%%%%%
%
%  MathJax defines \lt to ease XML confusion
%
%  Linear transformation definition
%  Usage: \ltdefn{name-letter}{domain}{range}
\newcommand{\ltdefn}[3]{#1\colon #2\rightarrow#3}
%
%  Linear transformation evaluation
%  Usage: \lteval{name-letter}{input}
%  Replaces old \lt desired by MathJax
\newcommand{\lteval}[2]{#1\left(#2\right)}
%
% Linear transformation inverse
%  Usage: \ltinverse{name-letter}
\newcommand{\ltinverse}[1]{#1^{-1}}
%
%  Linear transformation restriction
%  Usage: \restrict{name-letter}{subspace-letter}
\newcommand{\restrict}[2]{{#1}|_{#2}}
%
%  Linear transformation preimage
%  Usage: \preimage{name-letter}{codomain-element}
\newcommand{\preimage}[2]{#1^{-1}\left(#2\right)}
%
%  Range of a linear transformation
%  TeX uses \range for something else
%  Usage:  \rng{T}
\newcommand{\rng}[1]{\mathcal{R}\!\left(#1\right)}
%
%  Kernel of a linear transformation
%  TeX uses \ker to do something different
%  Usage:  \krn{T}
\newcommand{\krn}[1]{\mathcal{K}\!\left(#1\right)}
%
%  Linear transformation composition
%  Usage: \compose{function-name}{function-name}
\newcommand{\compose}[2]{{#1}\circ{#2}}
%
%  Vector space of linear transformations
%  Usage: \vslt{domains}{codomains}
%  Presumes math mode
\newcommand{\vslt}[2]{\mathcal{LT}\left(#1,\,#2\right)}
%
%%%%%%%%%%%%%%%%%%%%%
%
%     Vector and Matrix Representations
%
%%%%%%%%%%%%%%%%%%%%%
%
%  Isomorphism symbol
%  Usage: \isomorphic
\newcommand{\isomorphic}{\cong}
%
%  Similarity
%  Usage: \similar{inner-matrix}{outer-invertible-matrix}
%  Rearranging this will not "fix" all desired changes throughout
%
\newcommand{\similar}[2]{\inverse{#2}#1#2}
%
%  Vector representation function name
%  Usage: \vectrepname{basis-letter}
\newcommand{\vectrepname}[1]{\rho_{#1}}
%
%  Vector representation output
%  Usage: \vectrep{basis-letter}{input}
\newcommand{\vectrep}[2]{\lteval{\vectrepname{#1}}{#2}}
%
%  Vector representation inverse function name
%  (Added later, not used consistently in FCLA)
%  Usage: \vectrepinvname{basis-letter}
\newcommand{\vectrepinvname}[1]{\ltinverse{\vectrepname{#1}}}
%
%  Vector representation inverse output
%  Usage: \vectrepinv{basis-letter}{input}
\newcommand{\vectrepinv}[2]{\lteval{\ltinverse{\vectrepname{#1}}}{#2}}
%
%  Matrix representation
%  Usage: \matrixrep{transformation-letter}{domain-basis-letter}{codomain-basis-letter}
\newcommand{\matrixrep}[3]{M^{#1}_{#2,#3}}
%
%  Matrix representation column-by-colum
%  2016-07-20 only employed once?
%  Usage: \matrixrepcolumns{transformation-letter}{codomain-basis-letter}{codomain-basis-vector-letter}{final-index}
\newcommand{\matrixrepcolumns}[4]{\left\lbrack \left.\vectrep{#2}{\lteval{#1}{\vect{#3}_{1}}}\right|\left.\vectrep{#2}{\lteval{#1}{\vect{#3}_{2}}}\right|\left.\vectrep{#2}{\lteval{#1}{\vect{#3}_{3}}}\right|\ldots\left|\vectrep{#2}{\lteval{#1}{\vect{#3}_{#4}}}\right.\right\rbrack}
%
%  Change of basis matrix
%  Usage: \cbm{domain-basis-letter}{codomain-basis-letter}
\newcommand{\cbm}[2]{C_{#1,#2}}
%
%%%%%%%%%%%%%%%%%%%%%
%
%     Canonical Forms
%
%%%%%%%%%%%%%%%%%%%%%
%
%  Jordan blocks
%  Usage: \jordan{size}{diagonal-element}
\newcommand{\jordan}[2]{J_{#1}\left(#2\right)}
%
%%%%%%%%%%%%%%%%%%%%%
%
%     Hadamard Matrices
%     Contributed by Elizabeth Million
%
%%%%%%%%%%%%%%%%%%%%%
%
%  Hadamard Product
%  Usage: \hadamard{a-matrix}{a-matrix}
\newcommand{\hadamard}[2]{#1\circ #2}
%
%  Hadamard identity matrix
%  Usage: \hadamardidentity{paired-subscripts-size-of-matrix}
\newcommand{\hadamardidentity}[1]{J_{#1}}
%
%  Hadamard inverse matrix
%  Usage: \hadamardinverse{matrix-expression}
\newcommand{\hadamardinverse}[1]{\widehat{#1}}

\newcommand{\definedTerm}[1]{\textbf{#1}}
\newcommand{\dfn}[1]{\textbf{#1}}

\newcommand{\wt}{\widetilde}
\newcommand{\ov}{\overline}
\newcommand{\inj}{\rightarrowtail}
\newcommand{\surj}{\twoheadrightarrow}
\newcommand{\harpoon}{\overset{\rightharpoonup}}

\newenvironment{amatrix}[1]{%
  \left[\begin{array}{@{}*{#1}{c}|c@{}}
}{%
  \end{array}\right]
}


\title{Basic Properties of Eigenvalues}

\begin{document}
\begin{abstract}
  Eigenvectors and eigenvalues enjoy many properties.
\end{abstract}
\maketitle

This activity will be more about theorems, and the various properties
eigenvalues and eigenvectors enjoy.  Like a good
$4\times 100\text{ meter}$ relay, we will lead-off with one of our
better theorems and we'll save the very best for the anchor leg.

\begin{theorem}[Eigenvectors with Distinct Eigenvalues are Linearly Independent]
\label{theorem:EDELI}

Suppose that $A$ is an $n\times n$ square matrix and
$S=\set{\vectorlist{x}{p}}$ is a set of eigenvectors with eigenvalues
$\scalarlist{\lambda}{p}$ such that $\lambda_i\neq\lambda_j$ whenever
$i\neq j$.  Then $S$ is a linearly independent set.

\begin{proof}
  If $p=1$, then the set $S=\set{\vect{x}_1}$ is linearly independent
  since eigenvectors are nonzero (\ref{definition:EEM}), so assume for
  the remainder that $p\geq 2$.

  We will prove this result by contradiction (\ref{technique:CD}).
  Suppose to the contrary that $S$ is a linearly \wordChoice{\choice[correct]{dependent}\choice{independent}} set.
  Define $S_i=\set{\vectorlist{x}{i}}$ and let $k$ be an integer such
  that $S_{k-1}=\set{\vectorlist{x}{k-1}}$ is linearly independent and
  $S_k=\set{\vectorlist{x}{k}}$ is linearly dependent.  We have to ask
  if there is even such an integer $k$?  First, since eigenvectors are
  nonzero, the set $\set{\vect{x}_1}$ is linearly independent.  Since
  we are assuming that $S=S_p$ is linearly dependent, there must be an
  integer $k$, $2\leq k\leq p$, where the sets $S_i$ transition from
  linear independence to linear dependence (and stay that way). In
  other words, $\vect{x}_k$ is the vector with the smallest index that
  is a linear combination of just vectors with smaller indices.

  Since $\set{\vectorlist{x}{k}}$ is a linearly dependent set there
  must be scalars, $\scalarlist{a}{k}$, not all zero
  (\ref{definition:LI}), so that
  \begin{align*}
    \zerovector=\lincombo{a}{x}{k}
  \end{align*}

  Then,
  \begin{align*}
    \zerovector
    &=\left(A-\lambda_kI_n\right)\zerovector
    &&\ref{theorem:ZVSM}\\
    &=\left(A-\lambda_kI_n\right)\left(\lincombo{a}{x}{k}\right)
    &&\ref{definition:RLD}\\
    &=\left(A-\lambda_kI_n\right)a_1\vect{x}_1+
      \cdots+
      \left(A-\lambda_kI_n\right)a_k\vect{x}_k
    &&\ref{theorem:MMDAA}\\
    &=a_1\left(A-\lambda_kI_n\right)\vect{x}_1+
      \cdots+
      a_k\left(A-\lambda_kI_n\right)\vect{x}_k
    &&\ref{theorem:MMSMM}\\
    &=a_1\left(A\vect{x}_1-\lambda_kI_n\vect{x}_1\right)+
      \cdots+
      a_k\left(A\vect{x}_k-\lambda_kI_n\vect{x}_k\right)
    &&\ref{theorem:MMDAA}\\
    &=a_1\left(A\vect{x}_1-\lambda_k\vect{x}_1\right)+
      \cdots+
      a_k\left(A\vect{x}_k-\lambda_k\vect{x}_k\right)
    &&\ref{theorem:MMIM}\\
    &=a_1\left(\lambda_1\vect{x}_1-\lambda_k\vect{x}_1\right)+
      \cdots+
      a_k\left(\lambda_k\vect{x}_k-\lambda_k\vect{x}_k\right)
    &&\ref{definition:EEM}\\
    &=a_1\left(\lambda_1-\lambda_k\right)\vect{x}_1+
      \cdots+
      a_k\left(\lambda_k-\lambda_k\right)\vect{x}_k
    &&\ref{theorem:MMDAA}\\
    &=a_1\left(\lambda_1-\lambda_k\right)\vect{x}_1+
      \cdots+
      a_{k-1}\left(\lambda_{k-1}-\lambda_k\right)\vect{x}_{k-1}+
      a_k\left(0\right)\vect{x}_k
    &&\ref{property:AICN}\\
    &=a_1\left(\lambda_1-\lambda_k\right)\vect{x}_1+
      \cdots+
      a_{k-1}\left(\lambda_{k-1}-\lambda_k\right)\vect{x}_{k-1}+
      \zerovector
    &&\ref{theorem:ZSSM}\\
    &=a_1\left(\lambda_1-\lambda_k\right)\vect{x}_1+
      \cdots+
      a_{k-1}\left(\lambda_{k-1}-\lambda_k\right)\vect{x}_{k-1}
    &&\ref{property:Z}
  \end{align*}
  This equation is a relation of linear dependence on the linearly
  independent set $\set{\vectorlist{x}{k-1}}$, so the scalars must all
  be zero.  That is, $a_i\left(\lambda_i-\lambda_k\right)=0$ for
  $1\leq i\leq k-1$.  However, we have the hypothesis that the
  eigenvalues are distinct, so $\lambda_i\neq\lambda_k$ for
  $1\leq i\leq k-1$.  Thus $a_i=0$ for $1\leq i\leq k-1$.

  This reduces the original relation of linear dependence on
  $\set{\vectorlist{x}{k}}$ to the simpler equation
  $a_k\vect{x}_k=\zerovector$.  By \ref{theorem:SMEZV} we conclude
  that $a_k=0$ or $\vect{x}_k=\zerovector$.  Eigenvectors are never
  the zero vector (\ref{definition:EEM}), so $a_k=0$.  So all of the
  scalars $a_i$, $1\leq i\leq k$ are zero, contradicting their
  introduction as the scalars creating a nontrivial relation of linear
  dependence on the set $\set{\vectorlist{x}{k}}$.  With a
  contradiction in hand, we conclude that $S$ must be linearly
  independent.
\end{proof}
\end{theorem}

There is a simple connection between the eigenvalues of a matrix and whether or not the matrix is nonsingular.

\begin{theorem}[Singular Matrices have Zero Eigenvalues]
\label{theorem:SMZE}

Suppose $A$ is a square matrix.  Then $A$ is singular if and only if
$\lambda=\answer{0}$ is an eigenvalue of $A$.

\begin{proof}
  We have the following equivalences:
  \begin{align*}
    \text{$A$ is singular}&\iff\text{there exists $\vect{x}\neq\zerovector$, $A\vect{x}=\zerovector$}&&\ref{definition:NM}\\
                          &\iff\text{there exists $\vect{x}\neq\zerovector$, $A\vect{x}=0\vect{x}$}&&\ref{theorem:ZSSM}\\
                          &\iff\text{$\lambda=0$ is an eigenvalue of $A$}&&\ref{definition:EEM}
\end{align*}

\end{proof}
\end{theorem}

With an equivalence about singular matrices we can update our list of equivalences about nonsingular matrices.

\begin{theorem}[Nonsingular Matrix Equivalences, Round 8]
Suppose that $A$ is a square matrix of size $n$.  The following are equivalent.
\begin{enumerate}\item $A$ is nonsingular.
\item $A$ row-reduces to the identity matrix.
\item The null space of $A$ contains only the zero vector, $\nsp{A}=\set{\zerovector}$.
\item The linear system $\linearsystem{A}{\vect{b}}$ has a unique solution for every possible choice of $\vect{b}$.
\item The columns of $A$ are a linearly independent set.
\item $A$ is invertible.
\item The column space of $A$ is $\complex{n}$, $\csp{A}=\complex{n}$.
\item The columns of $A$ are a basis for $\complex{n}$.
\item The rank of $A$ is $n$, $\rank{A}=n$.
\item The nullity of $A$ is zero, $\nullity{A}=0$.
\item The determinant of $A$ is nonzero, $\detname{A}\neq 0$.
\item $\lambda=0$ is not an eigenvalue of $A$.
\end{enumerate}

\begin{proof}
  The equivalence of the first and last statements is
  \ref{theorem:SMZE}, reformulated by negating each statement in the
  equivalence.  So we are able to improve our previous result with
  this addition.
\end{proof}
\end{theorem}

Certain changes to a matrix change its eigenvalues in a predictable way.

\begin{theorem}[Eigenvalues of a Scalar Multiple of a Matrix]
\label{theorem:ESMM}

Suppose $A$ is a square matrix and $\lambda$ is an eigenvalue of $A$.
Then $\alpha\lambda$ is an eigenvalue of $\alpha A$.

\begin{proof}
Let $\vect{x}\neq\zerovector$ be one eigenvector of $A$ for $\lambda$.  Then
\begin{align*}
\left(\alpha A\right)\vect{x}&=\alpha\left(A\vect{x}\right)&&\ref{theorem:MMSMM}\\
&=\alpha\left(\lambda\vect{x}\right)&&\ref{definition:EEM}\\
&=\left(\alpha\lambda\right)\vect{x}&&\ref{property:SMAC}
\end{align*}

So $\vect{x}\neq\zerovector$ is an eigenvector of $\alpha A$ for the eigenvalue $\alpha\lambda$.

\end{proof}
\end{theorem}

Unfortunately, there are not parallel theorems about the sum or product of arbitrary matrices.  But we can prove a similar result for powers of a matrix.

\begin{theorem}[Eigenvalues Of Matrix Powers]
\label{theorem:EOMP}

Suppose $A$ is a square matrix, $\lambda$ is an eigenvalue of $A$, and $s\geq 0$ is an integer.  Then $\lambda^s$ is an eigenvalue of $A^s$.

\begin{proof}
Let $\vect{x}\neq\zerovector$ be one eigenvector of $A$ for $\lambda$.  Suppose $A$ has size $n$.  Then we proceed by induction on $s$.  First, for $s=0$,
\begin{align*}
A^s\vect{x}&=A^0\vect{x}\\
&=I_n\vect{x}\\
&=\vect{x}&&\ref{theorem:MMIM}\\
&=1\vect{x}&&\ref{property:OC}\\
&=\lambda^0\vect{x}\\
&=\lambda^s\vect{x}\\
\end{align*}
so $\lambda^s$ is an eigenvalue of $A^s$ in this special case.  If we assume the theorem is true for $s$, then we find
\begin{align*}
A^{s+1}\vect{x}&=A^sA\vect{x}\\
&=A^s\left(\lambda\vect{x}\right)&&\ref{definition:EEM}\\
&=\lambda\left(A^s\vect{x}\right)&&\ref{theorem:MMSMM}\\
&=\lambda\left(\lambda^s\vect{x}\right)&&\text{Induction hypothesis}\\
&=\left(\lambda\lambda^s\right)\vect{x}&&\ref{property:SMAC}\\
&=\lambda^{s+1}\vect{x}
\end{align*}

So $\vect{x}\neq\zerovector$ is an eigenvector of $A^{s+1}$ for $\lambda^{s+1}$, and induction tells us the theorem is true for all $s\geq 0$.

\end{proof}
\end{theorem}

While we cannot prove that the sum of two arbitrary matrices behaves
in any reasonable way with regard to eigenvalues, we can work with the
sum of dissimilar powers of the \textit{same} matrix.  We have already
seen two connections between eigenvalues and polynomials, in the proof
of \ref{theorem:EMHE} and the characteristic polynomial
(\ref{definition:CP}).  Our next theorem strengthens this connection.

\begin{theorem}[Eigenvalues of the Polynomial of a Matrix]
\label{theorem:EPM}

Suppose $A$ is a square matrix and $\lambda$ is an eigenvalue of $A$.
Let $q(x)$ be a polynomial in the variable $x$.  Then $q(\lambda)$ is
an eigenvalue of the matrix $q(A)$.





\begin{proof}
Let $\vect{x}\neq\zerovector$ be one eigenvector of $A$ for $\lambda$, and write $q(x)=a_0+a_1x+a_2x^2+\cdots+a_mx^m$.  Then
\begin{align*}
q(A)\vect{x}&=\left(a_0A^0+a_1A^1+a_2A^2+\cdots+a_mA^m\right)\vect{x}\\
&=(a_0A^0)\vect{x}+(a_1A^1)\vect{x}+(a_2A^2)\vect{x}+\cdots+(a_mA^m)\vect{x}&&\ref{theorem:MMDAA}\\
&=a_0(A^0\vect{x})+a_1(A^1\vect{x})+a_2(A^2\vect{x})+\cdots+a_m(A^m\vect{x})&&\ref{theorem:MMSMM}\\
&=a_0(\lambda^0\vect{x})+a_1(\lambda^1\vect{x})+a_2(\lambda^2\vect{x})+\cdots+a_m(\lambda^m\vect{x})&&\ref{theorem:EOMP}\\
&=(a_0\lambda^0)\vect{x}+(a_1\lambda^1)\vect{x}+(a_2\lambda^2)\vect{x}+\cdots+(a_m\lambda^m)\vect{x}&&\ref{property:SMAC}\\
&=\left(a_0\lambda^0+a_1\lambda^1+a_2\lambda^2+\cdots+a_m\lambda^m\right)\vect{x}&&\ref{property:DSAC}\\
&=q(\lambda)\vect{x}
\end{align*}




So $\vect{x}\neq 0$ is an eigenvector of $q(A)$ for the eigenvalue $q(\lambda)$.



\end{proof}
\end{theorem}

\begin{example}[Building desired eigenvalues]

The $4\times 4$ symmetric matrix
\[
C=
\begin{bmatrix}
1 &  0 &  1 &  1\\
0 &  1 &  1 &  1\\
1 &  1 &  1 &  0\\
1 &  1 &  0 &  1
\end{bmatrix}
\]
has three eigenvalues $\lambda=3,\,1,\,-1$.  Suppose we wanted a $4\times 4$ matrix that has the three eigenvalues $\lambda=4,\,0,\,-2$.  We can employ \ref{theorem:EPM} by finding a polynomial that converts $3$ to $4$, $1$ to $0$, and $-1$ to $-2$.  Such a polynomial is called an \dfn{interpolating polynomial}, and in this example we can use
\[
r(x)=\frac{1}{4}x^2+x-\frac{5}{4}
\]
We will not discuss how to concoct this polynomial, but a text on
numerical analysis should provide the details.  For now, simply verify
that $r(3)=\answer{4}$, $r(1)=\answer{0}$ and $r(-1)=-2$.

Now compute
\begin{align*}
r(C)&=\frac{1}{4}C^2+C-\frac{5}{4}I_4\\
&=
\frac{1}{4}
\begin{bmatrix}
3 &  2 &  2 &  2\\
2 &  3 &  2 &  2\\
2 &  2 &  3 &  2\\
2 &  2 &  2 &  3
\end{bmatrix}
+
\begin{bmatrix}
1 &  0 &  1 &  1\\
0 &  1 &  1 &  1\\
1 &  1 &  1 &  0\\
1 &  1 &  0 &  1
\end{bmatrix}
-\frac{5}{4}
\begin{bmatrix}
1 &  0 &  0 &  0\\
0 &  1 &  0 &  0\\
0 &  0 &  1 &  0\\
0 &  0 &  0 &  1
\end{bmatrix}
=
\frac{1}{2}
\begin{bmatrix}
1 &  1 &  3 &  3\\
1 &  1 &  3 &  3\\
3 &  3 &  1 &  1\\
3 &  3 &  1 &  1
\end{bmatrix}
\end{align*}


\ref{theorem:EPM} tells us that if $r(x)$ transforms the eigenvalues in the desired manner, then $r(C)$ will have the desired eigenvalues.  You can check this by computing the eigenvalues of $r(C)$ directly.  Furthermore, notice that the multiplicities are the same, and the eigenspaces of $C$ and $r(C)$ are identical.

\end{example}

Inverses and transposes also behave predictably with regard to their eigenvalues.



\begin{theorem}[Eigenvalues of the Inverse of a Matrix]
\label{theorem:EIM}

Suppose $A$ is a square nonsingular matrix and $\lambda$ is an
eigenvalue of $A$.  Then $\lambda^{-1}$ is an eigenvalue of the matrix
$\inverse{A}$.


\begin{proof}
Notice that since $A$ is assumed nonsingular, $\inverse{A}$ exists by \ref{theorem:NI}, but more importantly, $\lambda^{-1}=1/\lambda$  does not involve division by zero since \ref{theorem:SMZE} prohibits this possibility.

Let $\vect{x}\neq\zerovector$ be one eigenvector of $A$ for $\lambda$. Suppose $A$ has size $n$.  Then
\begin{align*}
\inverse{A}\vect{x}&=\inverse{A}(1\vect{x})&&\ref{property:OC}\\
&=\inverse{A}(\frac{1}{\lambda}\lambda\vect{x})&&\ref{property:MICN}\\
&=\frac{1}{\lambda}\inverse{A}(\lambda\vect{x})&&\ref{theorem:MMSMM}\\
&=\frac{1}{\lambda}\inverse{A}(A\vect{x})&&\ref{definition:EEM}\\
&=\frac{1}{\lambda}(\inverse{A}A)\vect{x}&&\ref{theorem:MMA}\\
&=\frac{1}{\lambda}I_n\vect{x}&&\ref{definition:MI}\\
&=\frac{1}{\lambda}\vect{x}&&\ref{theorem:MMIM}
\end{align*}

So $\vect{x}\neq\zerovector$ is an eigenvector of $\inverse{A}$ for the eigenvalue $\frac{1}{\lambda}$.

\end{proof}
\end{theorem}

The proofs of the theorems above have a similar style to them.  They all begin by grabbing an eigenvalue-eigenvector pair and adjusting it in some way to reach the desired conclusion.  You should add this to your toolkit as a general approach to proving theorems about eigenvalues.



So far we have been able to reserve the characteristic polynomial for strictly computational purposes.  However, sometimes a theorem about eigenvalues can be proved easily by employing the characteristic polynomial (rather than using an eigenvalue-eigenvector pair).  The next theorem is an example of this.




\begin{theorem}[Eigenvalues of the Transpose of a Matrix]
\label{theorem:ETM}


Suppose $A$ is a square matrix and $\lambda$ is an eigenvalue of $A$.  Then $\lambda$ is an eigenvalue of the matrix $\transpose{A}$.


\begin{proof}
Suppose $A$ has size $n$.  Then
\begin{align*}
\charpoly{A}{x}
&=\detname{A-xI_n}&&\ref{definition:CP}\\
&=\detname{\transpose{\left(A-xI_n\right)}}&&\ref{theorem:DT}\\
&=\detname{\transpose{A}-\transpose{\left(xI_n\right)}}&&\ref{theorem:TMA}\\
&=\detname{\transpose{A}-x\transpose{I_n}}&&\ref{theorem:TMSM}\\
&=\detname{\transpose{A}-xI_n}&&\ref{definition:IM}\\
&=\charpoly{\transpose{A}}{x}&&\ref{definition:CP}\\
\end{align*}


So $A$ and $\transpose{A}$ have 
\begin{multipleChoice}
\choice[correct]{the same characteristic polynomial,}
\choice{different characteristic polynomials,}
\end{multipleChoice}
and by \ref{theorem:EMRCP}, their eigenvalues are identical and have
equal algebraic multiplicities.  Notice that what we have proved here
is a bit stronger than the stated conclusion in the theorem.

\end{proof}
\end{theorem}

If a matrix has only real entries, then the computation of the characteristic polynomial (\ref{definition:CP}) will result in a polynomial with coefficients that are real numbers.  Complex numbers could result as roots of this polynomial, but they are roots of quadratic factors with real coefficients, and as such, come in conjugate pairs.  The next theorem proves this, and a bit more, without mentioning the characteristic polynomial.



\begin{theorem}[Eigenvalues of Real Matrices come in Conjugate Pairs]
\label{theorem:ERMCP}

Suppose $A$ is a square matrix with real entries and $\vect{x}$ is an eigenvector of $A$ for the eigenvalue $\lambda$.  Then $\conjugate{\vect{x}}$ is an eigenvector of $A$ for the eigenvalue $\conjugate{\lambda}$.


\begin{proof}

\begin{align*}
A\conjugate{\vect{x}}&=\conjugate{A}\conjugate{\vect{x}}&&\text{$A$ has real entries}\\
&=\conjugate{A\vect{x}}&&\ref{theorem:MMCC}\\
&=\conjugate{\lambda\vect{x}}&&\ref{definition:EEM}\\
&=\conjugate{\lambda}\conjugate{\vect{x}}&&\ref{theorem:CRSM}
\end{align*}

So $\conjugate{\vect{x}}$ is an eigenvector of $A$ for the eigenvalue $\conjugate{\lambda}$.

\end{proof}
\end{theorem}

This result can be a time-saver for computing eigenvalues and eigenvectors of real matrices with complex eigenvalues, since the conjugate eigenvalue and eigenspace can be inferred from the theorem rather than computed.

\end{document}