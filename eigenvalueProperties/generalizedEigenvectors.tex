\documentclass{ximera}
% These macros are automatically generated from the "macros"
% XML element.  Make permanent edits there.
%
% History
%   2004/01/01  Initiated for FCLA, evolved from there
%   2006/09/17  Converted  _, ^  to \sb, \sp for TeX4ht
%   2014/02/01  Updated for MathBook XML projects
%               Obsolete in FCLA: \codeindent, \computerfont, \define
%               Change: MathJax wants \lt, so replaced by \lteval
%   2014/02/22  New: \orderof, \reals, \per
%   2015/08/16  Incorporated into MathBook XML version of FCLA
%
%%%%%%%%%%%%%%%%%%%%%
%
%     Conveniences
%
%%%%%%%%%%%%%%%%%%%%%
%
%  Order of (asymptotically limit of fraction is 1)
%  Usage: \orderof{some function}
%
\newcommand{\orderof}[1]{\sim #1}
%
%  Integers
%  Usage:  \Z
\newcommand{\Z}{\mathbb{Z}}
%
%  Real numbers, as set of scalars
%  Usage:  \reals
\newcommand{\reals}{\mathbb{R}}
%
%  n-space over real field
%  Usage: \complex{integer-dimension}
\newcommand{\real}[1]{\mathbb{R}^{#1}}
%
%  Complex numbers, as set of scalars
%  Usage:  \complexes
\newcommand{\complexes}{\mathbb{C}}
%
%  n-space over complex field
%  Usage: \complex{integer-dimension}
\newcommand{\complex}[1]{\mathbb{C}^{#1}}
\newcommand{\CC}{\mathbb{C}}
%
%  Complex conjugation (scalar, vector, matrix)
%  Usage: \conjugate{object}
\newcommand{\conjugate}[1]{\overline{#1}}
%
%  Complex number modulus
%  Usage: \modulus{a+bi}
%  Presumes math mode
\newcommand{\modulus}[1]{\left\lvert#1\right\rvert}
%
%  Zero vector
%  Usage: \zerovector
\newcommand{\zerovector}{\vect{0}}
%
%  Zero matrix
%  Usage: \zeromatrix, use a subscript when size is important
\newcommand{\zeromatrix}{\mathcal{O}}
%
%  Inner product (brackets, not quadratic form)
%  Usage: \innerproduct{a-vector}{a-vector}
\newcommand{\innerproduct}[2]{\left\langle#1,\,#2\right\rangle}
%
%  Norm of a vector
%  Usage: \norm{a-vector}
\newcommand{\norm}[1]{\left\lVert#1\right\rVert}
%
%  Dimension
%  Usage: \dimension{vector-space-letter}
\newcommand{\dimension}[1]{\dim\left(#1\right)}
%
%  Nullity
%  Usage: \nullity{matrix-or-lintrans-letter}
\newcommand{\nullity}[1]{n\left(#1\right)}
%
%  Rank
%  Usage: \rank{matrix-or-lintrans-letter}
\newcommand{\rank}[1]{r\left(#1\right)}
%
%  Direct sum
%  Usage: \ds between a couple of subspaces
%
\newcommand{\ds}{\oplus}
%
%  Determinant of a matrix (functional)
%  Usage: \detname{A}
\newcommand{\detname}[1]{\det\left(#1\right)}
%
%  Determinant of a matrix (vertical bars)
%  Usage: \detbars{A}
\newcommand{\detbars}[1]{\left\lvert#1\right\rvert}
%
%  Trace of a Matrix
%  Usage: \trace{matrix name}
\newcommand{\trace}[1]{t\left(#1\right)}
%
%  Square Root of a Matrix
%  Usage: \sr{a-matrix}
\newcommand{\sr}[1]{#1^{1/2}}
%
%%%%%%%%%%%%%%%%%%%%%
%
%     Subspace Constructions
%
%%%%%%%%%%%%%%%%%%%%%
%
%  Span of a set of vectors
%  \span and \sp are used by TeX for other things
%  Usage: \spn{set-of-vectors}
\newcommand{\spn}[1]{\left\langle#1\right\rangle}
%
%  Null space of a matrix
%  Usage:  \nsp{A}
\newcommand{\nsp}[1]{\mathcal{N}\!\left(#1\right)}
%
%  Column space of a matrix
%  Usage:  \csp{A}
\newcommand{\csp}[1]{\mathcal{C}\!\left(#1\right)}
%
%  Row space of a matrix
%  Usage:  \rsp{A}
\newcommand{\rsp}[1]{\mathcal{R}\!\left(#1\right)}
%
%  Left null space of a matrix
%  Usage:  \lns{A}
\newcommand{\lns}[1]{\mathcal{L}\!\left(#1\right)}
%
%  Orthogonal complement of a vector space
%  Avoiding TeX's \perp
%  Usage:  \per{A}
\newcommand{\per}[1]{#1^\perp}
%
%%%%%%%%%%%%%%%%%%%%%
%
%     Systems of Equations
%
%%%%%%%%%%%%%%%%%%%%%
%
%  In-line form of an augmented matrix for a system of equations
%  Usage: \augmented{coefficient-matrix}{constant-vector}
\newcommand{\augmented}[2]{\left\lbrack\left.#1\,\right\rvert\,#2\right\rbrack}
%
%  Notation for a linear system before introducing matrix multiplication
%  Usage: \linearsystem{coefficient-matrix}{constant-vector}
\newcommand{\linearsystem}[2]{\mathcal{LS}\!\left(#1,\,#2\right)}
%
%  Notation for a homogenous system before introducing matrix multiplication
%  Usage: \homosystem{coefficient-matrix}
\newcommand{\homosystem}[1]{\linearsystem{#1}{\zerovector}}
%
%%%%%%%%%%%%%%%%%%%%%
%
%     Row Operations, Echelon Form
%
%%%%%%%%%%%%%%%%%%%%%
%
% Row operations on matrices
%
% Three commands to shorten up descriptions of gaussian elimination
%
% Usage: \rowopswap{row-i}{row-j}
% Usage: \rowopmult{scalar}{row-i}
% Usage: \rowopadd{scalar}{row-multiplied}{row-added-to}
\newcommand{\rowopswap}[2]{R_{#1}\leftrightarrow R_{#2}}
\newcommand{\rowopmult}[2]{#1R_{#2}}
\newcommand{\rowopadd}[3]{#1R_{#2}+R_{#3}}
%
% Mark leading 1's in echelon form with fbox
% Usage: \leading{a-1-usually}
\newcommand{\leading}[1]{\fbox{#1}}
%
%  Row-reduce arrow
%  Usage:  \rref inbetween a matrix and its reduced row-echelon form
\newcommand{\rref}{\xrightarrow{\text{RREF}}}
%
%  Elementary Matrices
%  Usage: \elemswap{subscript}{subscript}
%  Usage: \elemmult{scalar}{subscript}
%  Usage: \elemadd{scalar}{subscript-mult}{subscript-target}
%
\newcommand{\elemswap}[2]{E_{#1,#2}}
\newcommand{\elemmult}[2]{E_{#2}\left(#1\right)}
\newcommand{\elemadd}[3]{E_{#2,#3}\left(#1\right)}
%
%%%%%%%%%%%%%%%%%%%%%
%
%     2-D Constructions (Lists, Vectors, Matrices)
%
%%%%%%%%%%%%%%%%%%%%%
%
%  A list of scalars of generic length
%  Usage:  \scalarlist{scalar letter}{terminal subscript}
\newcommand{\scalarlist}[2]{{#1}_{1},\,{#1}_{2},\,{#1}_{3},\,\ldots,\,{#1}_{#2}}
%
%  Vector styling, bold (or use wiggles, arrows, whatever)
%  Subscripts go outside this construction
%  Usage: \vect{a symbol to use as a vector}
%  Have to already be in math mode
%
\newcommand{\vect}[1]{\mathbf{#1}}
%
%  A column vector
%  Usage: \colvector{list-delimited-by-\\}
%
\newcommand{\colvector}[1]{\begin{bmatrix}#1\end{bmatrix}}
%
%  A generic vector with components
%  Usage: \vectorcomponents{component-letter}{final-subscript}
\newcommand{\vectorcomponents}[2]{\colvector{#1_{1}\\#1_{2}\\#1_{3}\\\vdots\\#1_{#2}}}
%
%  A list of vectors of generic length
%  Usage:  \vectorlist{vector letter}{terminal subscript}
\newcommand{\vectorlist}[2]{\vect{#1}_{1},\,\vect{#1}_{2},\,\vect{#1}_{3},\,\ldots,\,\vect{#1}_{#2}}
%
%  Vector entries, entry i of vector v
%  (vector-expession still needs \vect, etc.)
%  Usage:  \vectorentry{vector-expression}{single-subscript}
\newcommand{\vectorentry}[2]{\left\lbrack#1\right\rbrack_{#2}}
%
%  Matrix entries, entry i,j of matrix A
%  Usage:  \matrixentry{matrix-expression}{paired-subscripts}
%
\newcommand{\matrixentry}[2]{\left\lbrack#1\right\rbrack_{#2}}
%
%  A generic linear combination
%  Usage:  \lincombo{scalar letter}{vector letter}{terminal subscript}
\newcommand{\lincombo}[3]{#1_{1}\vect{#2}_{1}+#1_{2}\vect{#2}_{2}+#1_{3}\vect{#2}_{3}+\cdots +#1_{#3}\vect{#2}_{#3}}
%
%  Matrix, column by column, as vectors
%  Usage:  \matrixcolumns{matrix letter}{terminal subscript}
\newcommand{\matrixcolumns}[2]{\left\lbrack\vect{#1}_{1}|\vect{#1}_{2}|\vect{#1}_{3}|\ldots|\vect{#1}_{#2}\right\rbrack}
%
%%%%%%%%%%%%%%%%%%%%%
%
%     Special Matrices
%
%%%%%%%%%%%%%%%%%%%%%
%
%  Transpose of a matrix
%  Usage:  \transpose{A}
\newcommand{\transpose}[1]{#1^{t}}
%
%  Inverse of a matrix
%  Usage:  \inverse{A}
\newcommand{\inverse}[1]{#1^{-1}}
%
%  Submatrix (for minors, determinants)
%  Usage: \submatrix{matrix-name}{delete-row}{delete-col}
\newcommand{\submatrix}[3]{#1\left(#2|#3\right)}
%
%  Adjoint of a matrix (twice)
%  This macro is a convenience to call \transpose and \conjugate properly
%  It shouldn't need to be modified (or mathematical meanings will change)
%  Usage:  \adj{A}
\newcommand{\adj}[1]{\transpose{\left(\conjugate{#1}\right)}}
%
%  This macro controls the symbol used for the adjoint
%  It can be edited to taste
%  Usage:  \adjoint{A}
\newcommand{\adjoint}[1]{#1^\ast}
%
%%%%%%%%%%%%%%%%%%%%%
%
%     Sets
%
%%%%%%%%%%%%%%%%%%%%%
%
%  A convenience for simple sets
%  Usage:  \set{list of element}
\newcommand{\set}[1]{\left\{#1\right\}}
%
%  Sets with vertical bar, "such that", sized for objects, not condition
%  Usage:  \setparts{objects}{condition}
%
%%\newcommand{\setparts}[2]{\left\{ #1\mid#2\right\}}
%%\newcommand{\setparts}[2]{\left\{\left. #1\right\rvert#2\right\}}
\newcommand{\setparts}[2]{\left\lbrace#1\,\middle|\,#2\right\rbrace}
%
%  Set Cardinality
%  Usage:  \card{a-set-letter}
\newcommand{\card}[1]{\left\lvert#1\right\rvert}
%
%  Set Union
%  Use \cup
%
%  Set Intersection
%  Use \cap
%
%  Set Complement
%  Usage:  \setcomplement{a-set-letter}
\newcommand{\setcomplement}[1]{\overline{#1}}
%
%%%%%%%%%%%%%%%%%%%%%
%
%     Eigenvalues and Eigenspaces
%
%%%%%%%%%%%%%%%%%%%%%
%
%  Characteristic polynomial
%  Usage: \charpoly{matrix-letter}{variable-letter}
\newcommand{\charpoly}[2]{p_{#1}\left(#2\right)}
%
%  Eigenspace
%  Usage: \eigenspace{matrix-letter}{eigenvalue-letter}
\newcommand{\eigenspace}[2]{\mathcal{E}_{#1}\left(#2\right)}
%
%  2013/10/03 Including ampersands is problematic here, 
%  think about fixes later
%  2014/02/22 Limited testing, seems &amp; is fine for HTML and LaTeX
%  2016-07-20 only employed in Archetypes, MBX has gather/align override
%  Eigensystem (presumes wrapped in an mrow within md)
%  Usage: \eigensystem{matrixletter}{eigenvalue}{list of basis vectors}
\newcommand{\eigensystem}[3]{\lambda&amp;=#2&amp;\eigenspace{#1}{#2}&amp;=\spn{\set{#3}}} 
%
%  Generalized Eigenspace
%  Usage: \geneigenspace{lin-trans-letter}{eigenvalue-letter}
\newcommand{\geneigenspace}[2]{\mathcal{G}_{#1}\left(#2\right)}
%
%  Algebraic multiplicty
%  Usage: \algmult{matrix-letter}{eigenvalue-letter}
\newcommand{\algmult}[2]{\alpha_{#1}\left(#2\right)}
%
%  Geometric multiplicty
%  Usage: \geomult{matrix-letter}{eigenvalue-letter}
\newcommand{\geomult}[2]{\gamma_{#1}\left(#2\right)}
%
%  Index (of eigenvalue)
%  Usage: \indx{matrix-letter}{eigenvalue-letter}
\newcommand{\indx}[2]{\iota_{#1}\left(#2\right)}
%
%%%%%%%%%%%%%%%%%%%%%
%
%     Linear Transformations
%
%%%%%%%%%%%%%%%%%%%%%
%
%  MathJax defines \lt to ease XML confusion
%
%  Linear transformation definition
%  Usage: \ltdefn{name-letter}{domain}{range}
\newcommand{\ltdefn}[3]{#1\colon #2\rightarrow#3}
%
%  Linear transformation evaluation
%  Usage: \lteval{name-letter}{input}
%  Replaces old \lt desired by MathJax
\newcommand{\lteval}[2]{#1\left(#2\right)}
%
% Linear transformation inverse
%  Usage: \ltinverse{name-letter}
\newcommand{\ltinverse}[1]{#1^{-1}}
%
%  Linear transformation restriction
%  Usage: \restrict{name-letter}{subspace-letter}
\newcommand{\restrict}[2]{{#1}|_{#2}}
%
%  Linear transformation preimage
%  Usage: \preimage{name-letter}{codomain-element}
\newcommand{\preimage}[2]{#1^{-1}\left(#2\right)}
%
%  Range of a linear transformation
%  TeX uses \range for something else
%  Usage:  \rng{T}
\newcommand{\rng}[1]{\mathcal{R}\!\left(#1\right)}
%
%  Kernel of a linear transformation
%  TeX uses \ker to do something different
%  Usage:  \krn{T}
\newcommand{\krn}[1]{\mathcal{K}\!\left(#1\right)}
%
%  Linear transformation composition
%  Usage: \compose{function-name}{function-name}
\newcommand{\compose}[2]{{#1}\circ{#2}}
%
%  Vector space of linear transformations
%  Usage: \vslt{domains}{codomains}
%  Presumes math mode
\newcommand{\vslt}[2]{\mathcal{LT}\left(#1,\,#2\right)}
%
%%%%%%%%%%%%%%%%%%%%%
%
%     Vector and Matrix Representations
%
%%%%%%%%%%%%%%%%%%%%%
%
%  Isomorphism symbol
%  Usage: \isomorphic
\newcommand{\isomorphic}{\cong}
%
%  Similarity
%  Usage: \similar{inner-matrix}{outer-invertible-matrix}
%  Rearranging this will not "fix" all desired changes throughout
%
\newcommand{\similar}[2]{\inverse{#2}#1#2}
%
%  Vector representation function name
%  Usage: \vectrepname{basis-letter}
\newcommand{\vectrepname}[1]{\rho_{#1}}
%
%  Vector representation output
%  Usage: \vectrep{basis-letter}{input}
\newcommand{\vectrep}[2]{\lteval{\vectrepname{#1}}{#2}}
%
%  Vector representation inverse function name
%  (Added later, not used consistently in FCLA)
%  Usage: \vectrepinvname{basis-letter}
\newcommand{\vectrepinvname}[1]{\ltinverse{\vectrepname{#1}}}
%
%  Vector representation inverse output
%  Usage: \vectrepinv{basis-letter}{input}
\newcommand{\vectrepinv}[2]{\lteval{\ltinverse{\vectrepname{#1}}}{#2}}
%
%  Matrix representation
%  Usage: \matrixrep{transformation-letter}{domain-basis-letter}{codomain-basis-letter}
\newcommand{\matrixrep}[3]{M^{#1}_{#2,#3}}
%
%  Matrix representation column-by-colum
%  2016-07-20 only employed once?
%  Usage: \matrixrepcolumns{transformation-letter}{codomain-basis-letter}{codomain-basis-vector-letter}{final-index}
\newcommand{\matrixrepcolumns}[4]{\left\lbrack \left.\vectrep{#2}{\lteval{#1}{\vect{#3}_{1}}}\right|\left.\vectrep{#2}{\lteval{#1}{\vect{#3}_{2}}}\right|\left.\vectrep{#2}{\lteval{#1}{\vect{#3}_{3}}}\right|\ldots\left|\vectrep{#2}{\lteval{#1}{\vect{#3}_{#4}}}\right.\right\rbrack}
%
%  Change of basis matrix
%  Usage: \cbm{domain-basis-letter}{codomain-basis-letter}
\newcommand{\cbm}[2]{C_{#1,#2}}
%
%%%%%%%%%%%%%%%%%%%%%
%
%     Canonical Forms
%
%%%%%%%%%%%%%%%%%%%%%
%
%  Jordan blocks
%  Usage: \jordan{size}{diagonal-element}
\newcommand{\jordan}[2]{J_{#1}\left(#2\right)}
%
%%%%%%%%%%%%%%%%%%%%%
%
%     Hadamard Matrices
%     Contributed by Elizabeth Million
%
%%%%%%%%%%%%%%%%%%%%%
%
%  Hadamard Product
%  Usage: \hadamard{a-matrix}{a-matrix}
\newcommand{\hadamard}[2]{#1\circ #2}
%
%  Hadamard identity matrix
%  Usage: \hadamardidentity{paired-subscripts-size-of-matrix}
\newcommand{\hadamardidentity}[1]{J_{#1}}
%
%  Hadamard inverse matrix
%  Usage: \hadamardinverse{matrix-expression}
\newcommand{\hadamardinverse}[1]{\widehat{#1}}

\newcommand{\definedTerm}[1]{\textbf{#1}}
\newcommand{\dfn}[1]{\textbf{#1}}

\newcommand{\wt}{\widetilde}
\newcommand{\ov}{\overline}
\newcommand{\inj}{\rightarrowtail}
\newcommand{\surj}{\twoheadrightarrow}
\newcommand{\harpoon}{\overset{\rightharpoonup}}

\newenvironment{amatrix}[1]{%
  \left[\begin{array}{@{}*{#1}{c}|c@{}}
}{%
  \end{array}\right]
}

\title{Generalized eigenvectors}
\author{Crichton Ogle}
\pgfplotsset{compat=1.15}
\begin{document}
\begin{abstract}
For an $n\times n$ complex matrix $A$, $\mathbb C^n$ does not necessarily have a basis consisting of eigenvectors of $A$. But it will always have a basis consisting of generalized eigenvectors of $A$.
\end{abstract}
\maketitle

In this section we assume all vector spaces and matrices are complex. As we have already seen, the matrix $A = \begin{bmatrix} 1 & 1\\0 & 1\end{bmatrix}$ is not diagonalizable. In fact, $p_A(t) = (1-t)^2$, indicating that the single eigenvalue $\lambda=1$ has algebraic multiplicity $m_a(1) = 2$, while
\[
m_g(1) = Dim(E_1(A)) = Null(A - 1\cdot Id) = Null\left(\begin{bmatrix} 0 & 1\\0 & 0\end{bmatrix}\right) = 1
\]
implying that $A$ is defective. 
\vskip.2in

However this is not the end of the story. Letting $B = (A - 1\cdot Id)$, we see that
\[
B^2 = B*B = 0^{2\times 2}
\]
is the zero matrix. Moreover, ${\bf e}_1 = B*{\bf e}_2$, where $E_1(A) = Span\{{\bf e}_1\}$. We then see that ${\bf e}_2$ is not an eigenvector of $A$, but $B*{\bf e}_2$ is. There is an inclusion
\[
\mathbb C\cong E_1(A) = N(B)\subset N(B^2) = \mathbb C^2
\]
In this example, the vector ${\bf e}_2$ is referred to as a \textbf{\textit{generalized eigenvector}} of the matrix $A$; it satisfies the property that the vector itself is not necessarily an eigenvector of $A$, but $B^k*{\bf e}_2$ is for some $k\ge 1$.
One other observation worth noting: in this example, the smallest exponent $m$ of $B$ satisfying the property
\[
N(B^m) = N(B^{m+1})
\]
is $m = m_a(1) = 2$, the algebraic multiplicity of the eigenvalue $\lambda = 1$. This is not an accident.

\begin{theorem} If $A$ is an $n\times n$ numerical matrix and $\lambda$ is an eigenvalue of $A$, then
\[
Null\left((A - \lambda I)^{m_a(\lambda)}\right) = m_a(\lambda)
\]
\end{theorem}

\begin{definition} The \textbf{\textit{generalized eigenspace of $\lambda$}} (for the matrix $A$) is the space $E^g_{\lambda}(A) := 
N\left((A - \lambda I)^{m_a(\lambda)}\right)$. A non-zero element of $E^g_{\lambda}(A)$ is referred to as a \textbf{\textit{generalized eigenvector}} of $A$.
\end{definition}

Letting $E_\lambda^k(A) := N\left((A - \lambda I)^{k}\right)$, we have a sequence of inclusions
\[
E_\lambda(A) = E^1_\lambda(A)\subset E_\lambda^2(A)\subset\dots\subset E_\lambda^{m_a(\lambda)} = E^g_\lambda(A)
\]

\begin{theorem} If $\lambda_1\ne \lambda_2\ne\dots\ne \lambda_k$ are the distinct eigenvalues of an $n\times n$ matrix $A$ then
\[
E^g_{\lambda_1}(A) + E^g_{\lambda_2}(A) + \dots + E^g_{\lambda_k}(A) = 
E^g_{\lambda_1}(A)\oplus E^g_{\lambda_2}(A)\oplus \dots \oplus E^g_{\lambda_k}(A) = \mathbb C^n
\]
\end{theorem}



The generalized eigenvalue problem is to find a basis $S^g_\lambda$ for each generalized eigenspace compatible with this filtration. This means that for each $k$, the vectors of $S^g_\lambda$ lying in $E_\lambda^k(A)$ is a basis for that subspace.

This turns out to be more involved than the earlier problem of finding a basis for $E_\lambda(A) = E_\lambda^1(A)$, and an algorithm for finding such a basis will be deferred until Module IV.

One thing that can often be done, however, is to find a \textbf{\textit{Jordan chain}}. We will first need to define some terminology.

\begin{definition} If ${\bf v}\in E^g_\lambda(A)$ is a generalized eigenvector of $A$, the \textbf{\textit{rank}} of ${\bf v}$ is the unique integer $m\ge 1$ for which $(A - \lambda I)^m*{\bf v} = {\bf 0}, (A - \lambda)^{m-1}*{\bf v}\ne {\bf 0}$.
\end{definition}

By the above Theorem, such an $m$ always exists. A generalized eigenvector of $A$, then, is an eigenvector of $A$ iff its rank equals $1$. For an eigenvalue $\lambda$ of $A$, we will abbreviate $(A - \lambda I)$ as $A_\lambda$.

\begin{definition} Given a generalized eigenvector ${\bf v}_m$ of $A$ of rank $m$, the \textbf{\textit{Jordan chain}} associated to ${\bf v}_m$ is the sequence of vectors
\[
J({\bf v}_m) := \{{\bf v}_m,{\bf v}_{m-1},{\bf v}_{m-2},\dots,{\bf v}_1\}
\]
where ${\bf v}_{m-i} := A_\lambda^i*{\bf v}_m$
\end{definition}

By definition of rank, it is easy to see that every vector in a Jordan chain must be non-zero. In fact, more is true

\begin{theorem} If ${\bf v}_m$ is a generalized eigenvector of $A$ of rank $m$ (corresponding to the eigenvalue $\lambda$), then the Jordan chain $J({\bf v}_m)$ corresponding to ${\bf v}_m$ consists of $m$ linearly independent eigenvectors.
\end{theorem}

In this way, a rank $m$ generalized eigenvector of $A$ ${\bf v}_m$ (corresponding to the eigenvalue $\lambda$) will generate an $m$-dimensional subspace $Span(J({\bf v}_m))$ of the generalized eigenspace $E^g_\lambda(A)$ with basis given by the Jordan chain $J({\bf v}_m)$ associated with ${\bf v}_m$.

\begin{definition} A \textbf{\textit{Jordan canonical basis}} for $E^g_\lambda(A)$ is one that consists entirely of Jordan chains.
\end{definition}

\begin{example} Suppose $\lambda$ is a non-defective eigenvalue of $A$; that is, one with $m_g(\lambda) = m_a(\lambda)$. Then $E^g_\lambda(A) = E_\lambda(A)$ - every generalized eigenvector of $A$ has rank $1$. Therefore every Jordan chain has length $1$, and the number of Jordan chains we would need to find a basis for $E^g_\lambda(A)$ is $m_g(A) = m_a(A)$.
\end{example}

\begin{example} let $A = \begin{bmatrix} 1 & 1\\0 & 1\end{bmatrix}$, as above. Then $\lambda = 1$ is defective, as $1 = m_g(1) < m_a(1) = 2$. We also saw that $A_1*{\bf e}_2 = {\bf e}_1$, where ${\bf e}_1$ is an eigenvector of $A$ corresponding to the eigenvalue $\lambda = 1$.

Thus ${\bf e}_2$ is a generalized eigenvector of $A$ of rank $2$, and the Jordan chain $\{{\bf e}_2, {\bf e}_1\}$ is a basis for $E^g_1(A) = \mathbb C^2$ (in fact, it is the standard basis).
\end{example}

\begin{example} let $A = \begin{bmatrix} 2 & 1 & 0\\0 & 2 & 1\\0 & 0 & 2\end{bmatrix}$. In this case, it is easy to see that the characteristic polynomial is $p_A(t) = (2-t)^3$, so that the only eigenvalue is $\lambda = 2$. Also $m_1(2) = 3$, and one can verify that $m_g(2) = 1$. 

Thus there is a gap of two between the dimension of the generalized eigenspace $E^g_2(A) = \mathbb C^3$, and that of the regular eigenspace $E_1(A)$. Note also that ${\bf e}_1$ is an eigenvector for $A$.

Can we find a Jordan chain which provides a basis for the generalized eigenspace $E^g_2(A)$, which we see is all of $\mathbb C^3$? If a single Jordan chain is going to do the job, it must have length $3$, and therefore be the Jordan chain associated to a generalized eigenvector of rank $3$.

Now $A_2 = A - 2Id = \begin{bmatrix} 0 & 1 & 0\\  0 & 0 & 1\\ 0 & 0 & 0\end{bmatrix}$, $A_2^2 = \begin{bmatrix} 0 & 0 & 1\\ 0 & 0 & 0 \\ 0 & 0 & 0\end{bmatrix}$, with $A_2^3 = {\bf 0}^{3\times 3}$. We are then looking for a vector ${\bf v}\in\mathbb C^3$ with $A_2^3*{\bf v} = \bf 0$ (which is automatically the case), but with $A_2^2*{\bf v}\ne 0$. We see that this last condition is satisfied iff the third coordinate of ${\bf v}$ is non-zero.

There is clearly a choice involved. The simplest choice here is to take ${\bf v} = {\bf v}_3 = {\bf e}_3$.  Then ${\bf v}_2 = A_2*{\bf v}_3 = {\bf e}_2$, and ${\bf v}_1 = A_2*{\bf v}_2 = {\bf e}_1$. So in this case we see
\[
J({\bf e}_3) = \{{\bf e}_3, {\bf e}_2, {\bf e}_1\}
\]
\end{example}

The previous examples were designed to be able to easily find a Jordan chain. The following is a bit more involved.

\begin{example} Let $A = \begin{bmatrix} 1 & -1 & 0\\ -1 & 4 & -1\\ -4 & 13 & -3\end{bmatrix}$. Then $p_A(t) = (-t)(1-t)^2$. The eigenvalue $\lambda = 0$ has algebraic multiplicity $1$, and therefore cannot be defective. The eigenvalue $\lambda = 1$ has $m_a(1) = 2$, while $m_g(1) = 1$. Thus for $\lambda = 0$, $E^g_0(A) = E_0(A) = N(A)$, with basis given by any non-zero vector of the nullspace of $A$.

For $\lambda = 1$, we cannot have two linearly independent Jordan chains of length $1$, because that would give $m_g(1) = 2$. So we must have a single Jordan chain of length 2. Now $A_1^2 = \begin{bmatrix} 1 & -3 & 1\\ 1 & -3 & 1\\ 3 & -9 & 3\end{bmatrix}$. Inspection shows the vector ${\bf v}_2 = \begin{bmatrix} 1\\ 1\\ 2\end{bmatrix}$ is in $N(A_1^2)$ but not in $N(A_1)$. Letting ${\bf v}_1 = A_1*{\bf v}_2 = \begin{bmatrix} -1\\ 0\\ 1\end{bmatrix}$ yields a Jordan chain of length 2:
\[
J({\bf v}_2) = \{{\bf v}_2, {\bf v}_1\}
\]
which is then a basis for $E_1^g(A)$.
\end{example}

\end{document}
