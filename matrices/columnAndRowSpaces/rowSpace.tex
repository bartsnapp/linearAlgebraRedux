\documentclass{ximera}

% These macros are automatically generated from the "macros"
% XML element.  Make permanent edits there.
%
% History
%   2004/01/01  Initiated for FCLA, evolved from there
%   2006/09/17  Converted  _, ^  to \sb, \sp for TeX4ht
%   2014/02/01  Updated for MathBook XML projects
%               Obsolete in FCLA: \codeindent, \computerfont, \define
%               Change: MathJax wants \lt, so replaced by \lteval
%   2014/02/22  New: \orderof, \reals, \per
%   2015/08/16  Incorporated into MathBook XML version of FCLA
%
%%%%%%%%%%%%%%%%%%%%%
%
%     Conveniences
%
%%%%%%%%%%%%%%%%%%%%%
%
%  Order of (asymptotically limit of fraction is 1)
%  Usage: \orderof{some function}
%
\newcommand{\orderof}[1]{\sim #1}
%
%  Integers
%  Usage:  \Z
\newcommand{\Z}{\mathbb{Z}}
%
%  Real numbers, as set of scalars
%  Usage:  \reals
\newcommand{\reals}{\mathbb{R}}
%
%  n-space over real field
%  Usage: \complex{integer-dimension}
\newcommand{\real}[1]{\mathbb{R}^{#1}}
%
%  Complex numbers, as set of scalars
%  Usage:  \complexes
\newcommand{\complexes}{\mathbb{C}}
%
%  n-space over complex field
%  Usage: \complex{integer-dimension}
\newcommand{\complex}[1]{\mathbb{C}^{#1}}
\newcommand{\CC}{\mathbb{C}}
%
%  Complex conjugation (scalar, vector, matrix)
%  Usage: \conjugate{object}
\newcommand{\conjugate}[1]{\overline{#1}}
%
%  Complex number modulus
%  Usage: \modulus{a+bi}
%  Presumes math mode
\newcommand{\modulus}[1]{\left\lvert#1\right\rvert}
%
%  Zero vector
%  Usage: \zerovector
\newcommand{\zerovector}{\vect{0}}
%
%  Zero matrix
%  Usage: \zeromatrix, use a subscript when size is important
\newcommand{\zeromatrix}{\mathcal{O}}
%
%  Inner product (brackets, not quadratic form)
%  Usage: \innerproduct{a-vector}{a-vector}
\newcommand{\innerproduct}[2]{\left\langle#1,\,#2\right\rangle}
%
%  Norm of a vector
%  Usage: \norm{a-vector}
\newcommand{\norm}[1]{\left\lVert#1\right\rVert}
%
%  Dimension
%  Usage: \dimension{vector-space-letter}
\newcommand{\dimension}[1]{\dim\left(#1\right)}
%
%  Nullity
%  Usage: \nullity{matrix-or-lintrans-letter}
\newcommand{\nullity}[1]{n\left(#1\right)}
%
%  Rank
%  Usage: \rank{matrix-or-lintrans-letter}
\newcommand{\rank}[1]{r\left(#1\right)}
%
%  Direct sum
%  Usage: \ds between a couple of subspaces
%
\newcommand{\ds}{\oplus}
%
%  Determinant of a matrix (functional)
%  Usage: \detname{A}
\newcommand{\detname}[1]{\det\left(#1\right)}
%
%  Determinant of a matrix (vertical bars)
%  Usage: \detbars{A}
\newcommand{\detbars}[1]{\left\lvert#1\right\rvert}
%
%  Trace of a Matrix
%  Usage: \trace{matrix name}
\newcommand{\trace}[1]{t\left(#1\right)}
%
%  Square Root of a Matrix
%  Usage: \sr{a-matrix}
\newcommand{\sr}[1]{#1^{1/2}}
%
%%%%%%%%%%%%%%%%%%%%%
%
%     Subspace Constructions
%
%%%%%%%%%%%%%%%%%%%%%
%
%  Span of a set of vectors
%  \span and \sp are used by TeX for other things
%  Usage: \spn{set-of-vectors}
\newcommand{\spn}[1]{\left\langle#1\right\rangle}
%
%  Null space of a matrix
%  Usage:  \nsp{A}
\newcommand{\nsp}[1]{\mathcal{N}\!\left(#1\right)}
%
%  Column space of a matrix
%  Usage:  \csp{A}
\newcommand{\csp}[1]{\mathcal{C}\!\left(#1\right)}
%
%  Row space of a matrix
%  Usage:  \rsp{A}
\newcommand{\rsp}[1]{\mathcal{R}\!\left(#1\right)}
%
%  Left null space of a matrix
%  Usage:  \lns{A}
\newcommand{\lns}[1]{\mathcal{L}\!\left(#1\right)}
%
%  Orthogonal complement of a vector space
%  Avoiding TeX's \perp
%  Usage:  \per{A}
\newcommand{\per}[1]{#1^\perp}
%
%%%%%%%%%%%%%%%%%%%%%
%
%     Systems of Equations
%
%%%%%%%%%%%%%%%%%%%%%
%
%  In-line form of an augmented matrix for a system of equations
%  Usage: \augmented{coefficient-matrix}{constant-vector}
\newcommand{\augmented}[2]{\left\lbrack\left.#1\,\right\rvert\,#2\right\rbrack}
%
%  Notation for a linear system before introducing matrix multiplication
%  Usage: \linearsystem{coefficient-matrix}{constant-vector}
\newcommand{\linearsystem}[2]{\mathcal{LS}\!\left(#1,\,#2\right)}
%
%  Notation for a homogenous system before introducing matrix multiplication
%  Usage: \homosystem{coefficient-matrix}
\newcommand{\homosystem}[1]{\linearsystem{#1}{\zerovector}}
%
%%%%%%%%%%%%%%%%%%%%%
%
%     Row Operations, Echelon Form
%
%%%%%%%%%%%%%%%%%%%%%
%
% Row operations on matrices
%
% Three commands to shorten up descriptions of gaussian elimination
%
% Usage: \rowopswap{row-i}{row-j}
% Usage: \rowopmult{scalar}{row-i}
% Usage: \rowopadd{scalar}{row-multiplied}{row-added-to}
\newcommand{\rowopswap}[2]{R_{#1}\leftrightarrow R_{#2}}
\newcommand{\rowopmult}[2]{#1R_{#2}}
\newcommand{\rowopadd}[3]{#1R_{#2}+R_{#3}}
%
% Mark leading 1's in echelon form with fbox
% Usage: \leading{a-1-usually}
\newcommand{\leading}[1]{\fbox{#1}}
%
%  Row-reduce arrow
%  Usage:  \rref inbetween a matrix and its reduced row-echelon form
\newcommand{\rref}{\xrightarrow{\text{RREF}}}
%
%  Elementary Matrices
%  Usage: \elemswap{subscript}{subscript}
%  Usage: \elemmult{scalar}{subscript}
%  Usage: \elemadd{scalar}{subscript-mult}{subscript-target}
%
\newcommand{\elemswap}[2]{E_{#1,#2}}
\newcommand{\elemmult}[2]{E_{#2}\left(#1\right)}
\newcommand{\elemadd}[3]{E_{#2,#3}\left(#1\right)}
%
%%%%%%%%%%%%%%%%%%%%%
%
%     2-D Constructions (Lists, Vectors, Matrices)
%
%%%%%%%%%%%%%%%%%%%%%
%
%  A list of scalars of generic length
%  Usage:  \scalarlist{scalar letter}{terminal subscript}
\newcommand{\scalarlist}[2]{{#1}_{1},\,{#1}_{2},\,{#1}_{3},\,\ldots,\,{#1}_{#2}}
%
%  Vector styling, bold (or use wiggles, arrows, whatever)
%  Subscripts go outside this construction
%  Usage: \vect{a symbol to use as a vector}
%  Have to already be in math mode
%
\newcommand{\vect}[1]{\mathbf{#1}}
%
%  A column vector
%  Usage: \colvector{list-delimited-by-\\}
%
\newcommand{\colvector}[1]{\begin{bmatrix}#1\end{bmatrix}}
%
%  A generic vector with components
%  Usage: \vectorcomponents{component-letter}{final-subscript}
\newcommand{\vectorcomponents}[2]{\colvector{#1_{1}\\#1_{2}\\#1_{3}\\\vdots\\#1_{#2}}}
%
%  A list of vectors of generic length
%  Usage:  \vectorlist{vector letter}{terminal subscript}
\newcommand{\vectorlist}[2]{\vect{#1}_{1},\,\vect{#1}_{2},\,\vect{#1}_{3},\,\ldots,\,\vect{#1}_{#2}}
%
%  Vector entries, entry i of vector v
%  (vector-expession still needs \vect, etc.)
%  Usage:  \vectorentry{vector-expression}{single-subscript}
\newcommand{\vectorentry}[2]{\left\lbrack#1\right\rbrack_{#2}}
%
%  Matrix entries, entry i,j of matrix A
%  Usage:  \matrixentry{matrix-expression}{paired-subscripts}
%
\newcommand{\matrixentry}[2]{\left\lbrack#1\right\rbrack_{#2}}
%
%  A generic linear combination
%  Usage:  \lincombo{scalar letter}{vector letter}{terminal subscript}
\newcommand{\lincombo}[3]{#1_{1}\vect{#2}_{1}+#1_{2}\vect{#2}_{2}+#1_{3}\vect{#2}_{3}+\cdots +#1_{#3}\vect{#2}_{#3}}
%
%  Matrix, column by column, as vectors
%  Usage:  \matrixcolumns{matrix letter}{terminal subscript}
\newcommand{\matrixcolumns}[2]{\left\lbrack\vect{#1}_{1}|\vect{#1}_{2}|\vect{#1}_{3}|\ldots|\vect{#1}_{#2}\right\rbrack}
%
%%%%%%%%%%%%%%%%%%%%%
%
%     Special Matrices
%
%%%%%%%%%%%%%%%%%%%%%
%
%  Transpose of a matrix
%  Usage:  \transpose{A}
\newcommand{\transpose}[1]{#1^{t}}
%
%  Inverse of a matrix
%  Usage:  \inverse{A}
\newcommand{\inverse}[1]{#1^{-1}}
%
%  Submatrix (for minors, determinants)
%  Usage: \submatrix{matrix-name}{delete-row}{delete-col}
\newcommand{\submatrix}[3]{#1\left(#2|#3\right)}
%
%  Adjoint of a matrix (twice)
%  This macro is a convenience to call \transpose and \conjugate properly
%  It shouldn't need to be modified (or mathematical meanings will change)
%  Usage:  \adj{A}
\newcommand{\adj}[1]{\transpose{\left(\conjugate{#1}\right)}}
%
%  This macro controls the symbol used for the adjoint
%  It can be edited to taste
%  Usage:  \adjoint{A}
\newcommand{\adjoint}[1]{#1^\ast}
%
%%%%%%%%%%%%%%%%%%%%%
%
%     Sets
%
%%%%%%%%%%%%%%%%%%%%%
%
%  A convenience for simple sets
%  Usage:  \set{list of element}
\newcommand{\set}[1]{\left\{#1\right\}}
%
%  Sets with vertical bar, "such that", sized for objects, not condition
%  Usage:  \setparts{objects}{condition}
%
%%\newcommand{\setparts}[2]{\left\{ #1\mid#2\right\}}
%%\newcommand{\setparts}[2]{\left\{\left. #1\right\rvert#2\right\}}
\newcommand{\setparts}[2]{\left\lbrace#1\,\middle|\,#2\right\rbrace}
%
%  Set Cardinality
%  Usage:  \card{a-set-letter}
\newcommand{\card}[1]{\left\lvert#1\right\rvert}
%
%  Set Union
%  Use \cup
%
%  Set Intersection
%  Use \cap
%
%  Set Complement
%  Usage:  \setcomplement{a-set-letter}
\newcommand{\setcomplement}[1]{\overline{#1}}
%
%%%%%%%%%%%%%%%%%%%%%
%
%     Eigenvalues and Eigenspaces
%
%%%%%%%%%%%%%%%%%%%%%
%
%  Characteristic polynomial
%  Usage: \charpoly{matrix-letter}{variable-letter}
\newcommand{\charpoly}[2]{p_{#1}\left(#2\right)}
%
%  Eigenspace
%  Usage: \eigenspace{matrix-letter}{eigenvalue-letter}
\newcommand{\eigenspace}[2]{\mathcal{E}_{#1}\left(#2\right)}
%
%  2013/10/03 Including ampersands is problematic here, 
%  think about fixes later
%  2014/02/22 Limited testing, seems &amp; is fine for HTML and LaTeX
%  2016-07-20 only employed in Archetypes, MBX has gather/align override
%  Eigensystem (presumes wrapped in an mrow within md)
%  Usage: \eigensystem{matrixletter}{eigenvalue}{list of basis vectors}
\newcommand{\eigensystem}[3]{\lambda&amp;=#2&amp;\eigenspace{#1}{#2}&amp;=\spn{\set{#3}}} 
%
%  Generalized Eigenspace
%  Usage: \geneigenspace{lin-trans-letter}{eigenvalue-letter}
\newcommand{\geneigenspace}[2]{\mathcal{G}_{#1}\left(#2\right)}
%
%  Algebraic multiplicty
%  Usage: \algmult{matrix-letter}{eigenvalue-letter}
\newcommand{\algmult}[2]{\alpha_{#1}\left(#2\right)}
%
%  Geometric multiplicty
%  Usage: \geomult{matrix-letter}{eigenvalue-letter}
\newcommand{\geomult}[2]{\gamma_{#1}\left(#2\right)}
%
%  Index (of eigenvalue)
%  Usage: \indx{matrix-letter}{eigenvalue-letter}
\newcommand{\indx}[2]{\iota_{#1}\left(#2\right)}
%
%%%%%%%%%%%%%%%%%%%%%
%
%     Linear Transformations
%
%%%%%%%%%%%%%%%%%%%%%
%
%  MathJax defines \lt to ease XML confusion
%
%  Linear transformation definition
%  Usage: \ltdefn{name-letter}{domain}{range}
\newcommand{\ltdefn}[3]{#1\colon #2\rightarrow#3}
%
%  Linear transformation evaluation
%  Usage: \lteval{name-letter}{input}
%  Replaces old \lt desired by MathJax
\newcommand{\lteval}[2]{#1\left(#2\right)}
%
% Linear transformation inverse
%  Usage: \ltinverse{name-letter}
\newcommand{\ltinverse}[1]{#1^{-1}}
%
%  Linear transformation restriction
%  Usage: \restrict{name-letter}{subspace-letter}
\newcommand{\restrict}[2]{{#1}|_{#2}}
%
%  Linear transformation preimage
%  Usage: \preimage{name-letter}{codomain-element}
\newcommand{\preimage}[2]{#1^{-1}\left(#2\right)}
%
%  Range of a linear transformation
%  TeX uses \range for something else
%  Usage:  \rng{T}
\newcommand{\rng}[1]{\mathcal{R}\!\left(#1\right)}
%
%  Kernel of a linear transformation
%  TeX uses \ker to do something different
%  Usage:  \krn{T}
\newcommand{\krn}[1]{\mathcal{K}\!\left(#1\right)}
%
%  Linear transformation composition
%  Usage: \compose{function-name}{function-name}
\newcommand{\compose}[2]{{#1}\circ{#2}}
%
%  Vector space of linear transformations
%  Usage: \vslt{domains}{codomains}
%  Presumes math mode
\newcommand{\vslt}[2]{\mathcal{LT}\left(#1,\,#2\right)}
%
%%%%%%%%%%%%%%%%%%%%%
%
%     Vector and Matrix Representations
%
%%%%%%%%%%%%%%%%%%%%%
%
%  Isomorphism symbol
%  Usage: \isomorphic
\newcommand{\isomorphic}{\cong}
%
%  Similarity
%  Usage: \similar{inner-matrix}{outer-invertible-matrix}
%  Rearranging this will not "fix" all desired changes throughout
%
\newcommand{\similar}[2]{\inverse{#2}#1#2}
%
%  Vector representation function name
%  Usage: \vectrepname{basis-letter}
\newcommand{\vectrepname}[1]{\rho_{#1}}
%
%  Vector representation output
%  Usage: \vectrep{basis-letter}{input}
\newcommand{\vectrep}[2]{\lteval{\vectrepname{#1}}{#2}}
%
%  Vector representation inverse function name
%  (Added later, not used consistently in FCLA)
%  Usage: \vectrepinvname{basis-letter}
\newcommand{\vectrepinvname}[1]{\ltinverse{\vectrepname{#1}}}
%
%  Vector representation inverse output
%  Usage: \vectrepinv{basis-letter}{input}
\newcommand{\vectrepinv}[2]{\lteval{\ltinverse{\vectrepname{#1}}}{#2}}
%
%  Matrix representation
%  Usage: \matrixrep{transformation-letter}{domain-basis-letter}{codomain-basis-letter}
\newcommand{\matrixrep}[3]{M^{#1}_{#2,#3}}
%
%  Matrix representation column-by-colum
%  2016-07-20 only employed once?
%  Usage: \matrixrepcolumns{transformation-letter}{codomain-basis-letter}{codomain-basis-vector-letter}{final-index}
\newcommand{\matrixrepcolumns}[4]{\left\lbrack \left.\vectrep{#2}{\lteval{#1}{\vect{#3}_{1}}}\right|\left.\vectrep{#2}{\lteval{#1}{\vect{#3}_{2}}}\right|\left.\vectrep{#2}{\lteval{#1}{\vect{#3}_{3}}}\right|\ldots\left|\vectrep{#2}{\lteval{#1}{\vect{#3}_{#4}}}\right.\right\rbrack}
%
%  Change of basis matrix
%  Usage: \cbm{domain-basis-letter}{codomain-basis-letter}
\newcommand{\cbm}[2]{C_{#1,#2}}
%
%%%%%%%%%%%%%%%%%%%%%
%
%     Canonical Forms
%
%%%%%%%%%%%%%%%%%%%%%
%
%  Jordan blocks
%  Usage: \jordan{size}{diagonal-element}
\newcommand{\jordan}[2]{J_{#1}\left(#2\right)}
%
%%%%%%%%%%%%%%%%%%%%%
%
%     Hadamard Matrices
%     Contributed by Elizabeth Million
%
%%%%%%%%%%%%%%%%%%%%%
%
%  Hadamard Product
%  Usage: \hadamard{a-matrix}{a-matrix}
\newcommand{\hadamard}[2]{#1\circ #2}
%
%  Hadamard identity matrix
%  Usage: \hadamardidentity{paired-subscripts-size-of-matrix}
\newcommand{\hadamardidentity}[1]{J_{#1}}
%
%  Hadamard inverse matrix
%  Usage: \hadamardinverse{matrix-expression}
\newcommand{\hadamardinverse}[1]{\widehat{#1}}

\newcommand{\definedTerm}[1]{\textbf{#1}}
\newcommand{\dfn}[1]{\textbf{#1}}

\newcommand{\wt}{\widetilde}
\newcommand{\ov}{\overline}
\newcommand{\inj}{\rightarrowtail}
\newcommand{\surj}{\twoheadrightarrow}
\newcommand{\harpoon}{\overset{\rightharpoonup}}

\newenvironment{amatrix}[1]{%
  \left[\begin{array}{@{}*{#1}{c}|c@{}}
}{%
  \end{array}\right]
}


\title{Row Space of a Matrix}

\begin{document}
\begin{abstract}
  The row space is the column space of the transpose.
\end{abstract}
\maketitle

The rows of a matrix can be viewed as vectors, since they are just
lists of numbers, arranged horizontally.  So we will transpose a
matrix, turning rows into columns, so we can then manipulate rows as
column vectors.  As a result we will be able to make some new
connections between row operations and solutions to systems of
equations.

\begin{definition}[Row Space of a Matrix]
  Suppose $A$ is an $m\times n$ matrix.  Then the \dfn{row space} of
  $A$, $\rsp{A}$, is the column space of $\transpose{A}$, i.e.,
  $\rsp{A}=\csp{\transpose{A}}$.
\end{definition}

Informally, the row space is the set of all linear combinations of the
rows of $A$.  However, we write the rows as column vectors, thus the
necessity of using the transpose to make the rows into columns.
Additionally, with the row space defined in terms of the column space,
all of the previous results of this section can be applied to row
spaces.

Notice that if $A$ is a rectangular $m\times n$ matrix, then
$\csp{A}\subseteq\complex{m}$, while $\rsp{A}\subseteq\complex{n}$ and
the two sets are not comparable since they do not even hold objects of
the same type.

\begin{question}
  When $A$ is square of size $n$, both $\csp{A}$ and $\rsp{A}$ are
  subsets of $\complex{n}$.  Then it is usually the case that
  \begin{multipleChoice}
    \choice[correct]{$\rsp(A) \neq \csp(A)$}
    \choice{$\rsp(A) = \csp(A)$}
  \end{multipleChoice}
\end{question}

\begin{example}
  The system of linear equations
  \begin{align*}
    x_1 +4x_2  - x_4  + 7x_6 - 9x_7 &= 3\\
    2x_1 + 8x_2 - x_3 + 3x_4 + 9x_5 - 13x_6 + 7x_7 &= 9\\
    2x_3 -3x_4 -4x_5 +12x_6 -8x_7 &= 1\\
    -x_1  - 4x_2 + 2x_3 +4x_4 + 8x_5 - 31x_6 + 37x_7 &= 4
  \end{align*}
  has coefficient matrix
  \[
    M = \begin{bmatrix}
      1 & 4 & 0 & -1 & 0 & 7 &  -9 \\
      2 & 8 &  -1 & 3 & 9 &  -13 & 7\\
      0 & 0 &  2 & -3 & -4 & 12 &  -8\\
      -1 &  -4 & 2 & 4 & 8 &  -31 & 37
    \end{bmatrix}
  \]
  To build the row space, we transpose the matrix,
  \[
    \transpose{M}=
    \begin{bmatrix}
      1 & \answer{2} & 0 & -1\\
      \answer{4} & 8 & 0 & -4\\
      0 & -1 & 2 & 2\\
      -1 & 3 & -3 & 4\\
      0 & 9 & -4 & 8\\
      7 & -13 & 12 & -31\\
      -9 & 7 & -8 & 37
    \end{bmatrix}
  \]

  Then the columns of this matrix are used in a span to build the row space,
  \[
    \rsp{M}=\csp{\transpose{M}}=
    \spn{\set{
        \colvector{1\\4\\0\\-1\\0\\7\\-9},\,
        \colvector{2\\8\\-1\\3\\9\\-13\\7},\,
        \colvector{0\\0\\2\\-3\\-4\\12\\-8},\,
        \colvector{-1\\-4\\2\\4\\8\\-31\\37}
      }}.
  \]
  
  However, we can use \ref{theorem:BCS} to get a slightly better description.  First, row-reduce $\transpose{M}$,
  \[
    \begin{bmatrix}
      \leading{1} & 0 & 0 & -\frac{31}{7}\\
      0 & \leading{1} & 0 & \frac{12}{7}\\
      0 & 0 & \leading{1} & \frac{13}{7}\\
      0 & 0 & 0 & 0\\
      0 & 0 & 0 & 0\\
      0 & 0 & 0 & 0\\
      0 & 0 & 0 & 0
    \end{bmatrix}.
  \]

  Since the pivot columns have indices $D=\set{1,\,2,\,3}$, the column
  space of $\transpose{M}$ can be spanned by just the first three
  columns of $\transpose{M}$,
  \[
    \rsp{M}=\csp{\transpose{M}}=
    \spn{\set{
        \colvector{1\\4\\0\\-1\\0\\7\\-9},\,
        \colvector{2\\8\\-1\\3\\9\\-13\\7},\,
        \colvector{0\\0\\2\\-3\\-4\\12\\-8}
      }}.
  \]
\end{example}

The row space would not be too interesting if it was simply the column
space of the transpose.  However, when we do row operations on a
matrix we have no effect on the many linear combinations that can be
formed with the rows of the matrix.  This is stated more carefully in
the following theorem.

\begin{theorem}[Row-Equivalent Matrices have equal Row Spaces]
  \label{theorem:REMRS}Suppose $A$ and $B$ are row-equivalent
  matrices.  Then $\rsp{A}=\rsp{B}$.

  \begin{proof}
    Two matrices are row-equivalent (\ref{definition:REM}) if one can
    be obtained from another by a sequence of (possibly many) row
    operations.  We will prove the theorem for two matrices that
    differ by a single row operation, and then this result can be
    applied repeatedly to get the full statement of the theorem.  The
    row spaces of $A$ and $B$ are spans of the columns of their
    transposes.  For each row operation we perform on a matrix, we can
    define an analogous operation on the columns.  Perhaps we should
    call these \dfn{column operations}.  Instead, we will still call
    them row operations, but we will apply them to the columns of the
    transposes.

    Refer to the columns of $\transpose{A}$ and $\transpose{B}$ as
    $\vect{A}_i$ and $\vect{B}_i$, $1\leq i\leq m$.  The row operation
    that switches rows will switch columns of the transposed matrices.
    This will have \wordChoice{\choice[correct]{no}\choice{an}} effect
    on the possible linear combinations formed by the columns.

    Suppose that $\transpose{B}$ is formed from $\transpose{A}$ by
    multiplying column $\vect{A}_t$ by $\alpha\neq 0$.  In other
    words, $\vect{B}_t=\alpha\vect{A}_t$, and $\vect{B}_i=\vect{A}_i$
    for all $i\neq t$.  We need to establish that two sets are equal,
    $\csp{\transpose{A}}=\csp{\transpose{B}}$.  We will take a generic
    element of one and show that it is contained in the other.
    \begin{align*}
      \beta_1\vect{B}_1+
      &
        \beta_2\vect{B}_2+
        \beta_3\vect{B}_3+
        \cdots+
        \beta_t\vect{B}_t+
        \cdots+
        \beta_m\vect{B}_m\\
      &=
        \beta_1\vect{A}_1+
        \beta_2\vect{A}_2+
        \beta_3\vect{A}_3+
        \cdots+
        \beta_t\left(\alpha\vect{A}_t\right)+
        \cdots+
        \beta_m\vect{A}_m\\
      &=
        \beta_1\vect{A}_1+
        \beta_2\vect{A}_2+
        \beta_3\vect{A}_3+
        \cdots+
        \left(\alpha\beta_t\right)\vect{A}_t+
        \cdots+
        \beta_m\vect{A}_m
    \end{align*}
    says that $\csp{\transpose{B}}\subseteq\csp{\transpose{A}}$.  Similarly,
    \begin{align*}
      \gamma_1\vect{A}_1+
      &
        \gamma_2\vect{A}_2+
        \gamma_3\vect{A}_3+
        \cdots+
        \gamma_t\vect{A}_t+
        \cdots+
        \gamma_m\vect{A}_m\\
      &=
        \gamma_1\vect{A}_1+
        \gamma_2\vect{A}_2+
        \gamma_3\vect{A}_3+
        \cdots+
        \left(\frac{\gamma_t}{\alpha}\alpha\right)\vect{A}_t+
        \cdots+
        \gamma_m\vect{A}_m\\
      &=
        \gamma_1\vect{A}_1+
        \gamma_2\vect{A}_2+
        \gamma_3\vect{A}_3+
        \cdots+
        \frac{\gamma_t}{\alpha}\left(\alpha\vect{A}_t\right)+
        \cdots+
        \gamma_m\vect{A}_m\\
      &=
        \gamma_1\vect{B}_1+
        \gamma_2\vect{B}_2+
        \gamma_3\vect{B}_3+
        \cdots+
        \frac{\gamma_t}{\alpha}\vect{B}_t+
        \cdots+
        \gamma_m\vect{B}_m
    \end{align*}
    says that $\csp{\transpose{A}}\subseteq\csp{\transpose{B}}$.  So
    $\rsp{A}=\csp{\transpose{A}}=\csp{\transpose{B}}=\rsp{B}$ when a
    single row operation of the second type is performed.

    Suppose now that $\transpose{B}$ is formed from $\transpose{A}$ by
    replacing $\vect{A}_t$ with $\alpha\vect{A}_s+\vect{A}_t$ for some
    $\alpha\in\complexes$ and $s\neq t$.  In other words,
    $\vect{B}_t=\alpha\vect{A}_s+\vect{A}_t$, and
    $\vect{B}_i=\vect{A}_i$ for $i\neq t$.
    \begin{align*}
      \beta_1\vect{B}_1+
      &
        \beta_2\vect{B}_2+
        \cdots+
        \beta_s\vect{B}_s+
        \cdots+
        \beta_t\vect{B}_t+
        \cdots+
        \beta_m\vect{B}_m\\
      &=
        \beta_1\vect{A}_1+
        \beta_2\vect{A}_2+
        \cdots+
        \beta_s\vect{A}_s+
        \cdots+
        \beta_t\left(\alpha\vect{A}_s+\vect{A}_t\right)+
        \cdots+
        \beta_m\vect{A}_m\\
      &=
        \beta_1\vect{A}_1+
        \beta_2\vect{A}_2+
        \cdots+
        \beta_s\vect{A}_s+
        \cdots+
        \left(\beta_t\alpha\right)\vect{A}_s+\beta_t\vect{A}_t+
        \cdots+
        \beta_m\vect{A}_m\\
      &=
        \beta_1\vect{A}_1+
        \beta_2\vect{A}_2+
        \cdots+
        \beta_s\vect{A}_s+\left(\beta_t\alpha\right)\vect{A}_s+
        \cdots+
        \beta_t\vect{A}_t+
        \cdots+
        \beta_m\vect{A}_m\\
      &=
        \beta_1\vect{A}_1+
        \beta_2\vect{A}_2+
        \cdots+
        \left(\beta_s+\beta_t\alpha\right)\vect{A}_s+
        \cdots+
        \beta_t\vect{A}_t+
        \cdots+
        \beta_m\vect{A}_m
    \end{align*}
    says that $\csp{\transpose{B}}\subseteq\csp{\transpose{A}}$.  Similarly,
    \begin{align*}
      \gamma_1&\vect{A}_1+
                \gamma_2\vect{A}_2+
                \cdots+
                \gamma_s\vect{A}_s+
                \cdots+
                \gamma_t\vect{A}_t+
                \cdots+
                \gamma_m\vect{A}_m\\
              &=
                \gamma_1\vect{A}_1+
                \gamma_2\vect{A}_2+
                \cdots+
                \gamma_s\vect{A}_s+
                \cdots+
                \left(-\alpha\gamma_t\vect{A}_s + \alpha\gamma_t\vect{A}_s\right)+\gamma_t\vect{A}_t+
                \cdots+
                \gamma_m\vect{A}_m\\
              &=
                \gamma_1\vect{A}_1+
                \gamma_2\vect{A}_2+
                \cdots+
                \left(-\alpha\gamma_t+\gamma_s\right)\vect{A}_s+
                \cdots+
                \gamma_t\left(\alpha\vect{A}_s+\vect{A}_t\right)+
                \cdots+
                \gamma_m\vect{A}_m\\
              &=
                \gamma_1\vect{B}_1+
                \gamma_2\vect{B}_2+
                \cdots+
                \left(-\alpha\gamma_t+\gamma_s\right)\vect{B}_s+
                \cdots+
                \gamma_t\vect{B}_t+
                \cdots+
                \gamma_m\vect{B}_m
    \end{align*}
    says that $\csp{\transpose{A}}\subseteq\csp{\transpose{B}}$.  So
    $\rsp{A}=\csp{\transpose{A}}=\csp{\transpose{B}}=\rsp{B}$ when a
    single row operation of the third type is performed.

    So the row space of a matrix is preserved by each row operation,
    and hence row spaces of row-equivalent matrices are equal.
\end{proof}
\end{theorem}

\begin{example}[Row spaces of two row-equivalent matrices]
  The matrices
  \begin{align*}
    A&=\begin{bmatrix}
      2&-1&3&4\\
      5&2&-2&3\\
      1&1&0&6
    \end{bmatrix}
  \end{align*}
  and
  \begin{align*}
    B&=\begin{bmatrix}
      1&1&0&6\\
      3&0&-2&-9\\
      2&-1&3&4
    \end{bmatrix}
  \end{align*}
  are row-equivalent by demonstrating a sequence of two row operations
  that converted $A$ into $B$.  Applying \ref{theorem:REMRS} we can
  conclude
  \begin{multipleChoice}
    \choice[correct]{$\rsp{A} = \rsp{B}$}
    \choice{$\rsp{A} \neq \rsp{B}$}
  \end{multipleChoice}
  
  \begin{feedback}
    \begin{align*}
      \rsp{A} &=
                \spn{\set{\colvector{2\\-1\\3\\4},\,\colvector{5\\2\\-2\\3},\,\colvector{1\\1\\0\\6}}} \\
              &=\spn{\set{\colvector{1\\1\\0\\6},\,\colvector{3\\0\\-2\\-9},\,\colvector{2\\-1\\3\\4}}} \\
              &=\rsp{B}.
    \end{align*}    
  \end{feedback}

\end{example}

\ref{theorem:REMRS} is at its best when one of the row-equivalent
matrices is in reduced row-echelon form.  The vectors that are zero
rows can be ignored.  The echelon pattern insures that the nonzero
rows yield vectors that are linearly independent.  Here is the
theorem.

\begin{theorem}[Basis for the Row Space]
  \label{theorem:BRS}

  Suppose that $A$ is a matrix and $B$ is a row-equivalent matrix in
  reduced row-echelon form.  Let $S$ be the set of nonzero columns of
  $\transpose{B}$.  Then
  \begin{enumerate}
  \item $\rsp{A}=\spn{S}$.
  \item $S$ is a linearly independent set.
  \end{enumerate}

  \begin{proof}
    From \ref{theorem:REMRS} we know that $\rsp{A}=\rsp{B}$.  If $B$
    has any zero rows, these are columns of $\transpose{B}$ that are
    the zero vector.  We can safely toss out the zero vector in the
    span construction, since it can be recreated from the nonzero
    vectors by a linear combination where all the scalars are zero.
    So $\rsp{A}=\spn{S}$.

    Suppose $B$ has $r$ nonzero rows and let
    $D=\set{d_1,\,d_2,\,d_3,\,\ldots,\,d_r}$ denote the indices of the
    pivot columns of $B$.  Denote the $r$ column vectors of
    $\transpose{B}$, the vectors in $S$, as $\vectorlist{B}{r}$.  To
    show that $S$ is linearly independent, start with a relation of
    linear dependence
    \[
      \lincombo{\alpha}{B}{r}=\zerovector
    \]

    Now consider this vector equality in location $d_i$.  Since $B$ is
    in reduced row-echelon form, the entries of column $d_i$ of $B$
    are all zero, except for a leading 1 in row $i$.  Thus, in
    $\transpose{B}$, row $d_i$ is all zeros, excepting a 1 in column
    $i$.  So, for $1\leq i\leq r$,
    \begin{align*}
      0
      &=\vectorentry{\zerovector}{d_i}&&\ref{definition:ZCV}\\
      &=\vectorentry{\lincombo{\alpha}{B}{r}}{d_i}
                                      &&\ref{definition:RLDCV}\\
      &=
        \vectorentry{\alpha_1\vect{B}_1}{d_i}+
        \vectorentry{\alpha_2\vect{B}_2}{d_i}+
        \vectorentry{\alpha_3\vect{B}_3}{d_i}+
        \cdots+
        \vectorentry{\alpha_r\vect{B}_r}{d_i}
                                      &&\ref{definition:MA}\\
      &=
        \alpha_1\vectorentry{\vect{B}_1}{d_i}+
        \alpha_2\vectorentry{\vect{B}_2}{d_i}+
        \alpha_3\vectorentry{\vect{B}_3}{d_i}+
        \cdots+
        \alpha_r\vectorentry{\vect{B}_r}{d_i}
                                      &&\ref{definition:MSM}\\
      &=
        \alpha_1(0)+
        \alpha_2(0)+
        \alpha_3(0)+
        \cdots+
        \alpha_i(1)+
        \cdots+
        \alpha_r(0)
                                      &&\ref{definition:RREF}\\
      &=\alpha_i
    \end{align*}

    So we conclude that $\alpha_i=0$ for all $1\leq i\leq r$,
    establishing the linear independence of $S$
    (\ref{definition:LICV}).
  \end{proof}
\end{theorem}

\begin{example}[Improving a span]

  Suppose in the course of analyzing a matrix (its column space, its
  null space, its ldots ) we encounter the following set of vectors,
  described by a span
  \[
    X=\spn{\set{
        \colvector{1\\2\\1\\6\\6},\,
        \colvector{3\\-1\\2\\-1\\6},\,
        \colvector{1\\-1\\0\\-1\\-2},\,
        \colvector{-3\\2\\-3\\6\\-10}
      }}
  \]

  Let $A$ be the matrix whose rows are the vectors in $X$, so by
  design $X=\rsp{A}$,
  \[
    A=
    \begin{bmatrix}
      1 & 2 & 1 & 6 & 6\\
      3 & -1 & 2 & -1 & 6\\
      1 & -1 & 0 & -1 & -2\\
      -3 & 2 & -3 & 6 & -10
    \end{bmatrix}
  \]

  Row-reduce $A$ to form a row-equivalent matrix in reduced
  row-echelon form,
  \[
    B=
    \begin{bmatrix}
      \leading{1} & 0 & 0 & 2 & -1\\
      0 & \leading{1} & 0 & 3 & 1\\
      0 & 0 & \leading{1} & -2 & 5\\
      0 & 0 & 0 & 0 & 0
    \end{bmatrix}
  \]
  
  Then \ref{theorem:BRS} says we can grab the nonzero columns of
  $\transpose{B}$ and write
  \[
    X=\rsp{A}=\rsp{B}=
    \spn{\set{
        \colvector{1\\0\\0\\2\\-1},\,
        \colvector{0\\1\\0\\3\\1},\,
        \colvector{0\\0\\1\\-2\\5}
      }}
  \]

  These three vectors provide a much-improved description of $X$.
  There are fewer vectors, and the pattern of zeros and ones in the
  first three entries makes it easier to determine membership in $X$.
\end{example}

Notice that in \ref{example:IAS} all we had to do was row-reduce the
right matrix and toss out a zero row.  Next to row operations
themselves, \ref{theorem:BRS} is probably \textbf{the most powerful
  computational technique at your disposal} as it quickly provides a
much improved description of a span in the context of computing the
row space, the column space, \ldots

\ref{theorem:BRS} and the techniques of \ref{example:IAS} will provide
yet another description of the column space of a matrix.  First we
state a triviality as a theorem, so we can reference it later.

\begin{theorem}[Column Space, Row Space, Transpose]
  \label{theorem:CSRST}

  Suppose $A$ is a matrix.  Then $\csp{A}=\rsp{\transpose{A}}$.

  \begin{proof}
    \begin{align*}
      \csp{A}
      &=\csp{\transpose{\left(\transpose{A}\right)}}
      &&\ref{theorem:TT}\\
      &=\rsp{\transpose{A}}
      &&\ref{definition:RSM}
    \end{align*}
  \end{proof}
\end{theorem}

So to find another expression for the column space of a matrix, build
its transpose, row-reduce it, toss out the zero rows, and convert the
nonzero rows to column vectors to yield an improved set for the span
construction.

\begin{example}[Column space from row operations]
  Consider again the matrix
  \[
    M = \begin{bmatrix}
      1 & 4 & 0 & -1 & 0 & 7 &  -9 \\
      2 & 8 &  -1 & 3 & 9 &  -13 & 7\\
      0 & 0 &  2 & -3 & -4 & 12 &  -8\\
      -1 &  -4 & 2 & 4 & 8 &  -31 & 37
    \end{bmatrix}
  \]
  The transpose is
  \[
    \begin{bmatrix}
      1 & 2 & 0 & -1\\
      4 & 8 & 0 & -4\\
      0 & -1 & 2 & 2\\
      -1 & 3 & -3 & 4\\
      0 & 9 & -4 & 8\\
      7 & -13 & 12 & -31\\
      -9 & 7 & -8 & 37
    \end{bmatrix}.
  \]

  Row-reduced this becomes,
  \[
    \begin{bmatrix}
      \leading{1} & 0 & 0 & -\frac{31}{7}\\
      0 & \leading{1} & 0 & \frac{12}{7}\\
      0 & 0 & \leading{1} & \frac{13}{7}\\
      0 & 0 & 0 & 0\\
      0 & 0 & 0 & 0\\
      0 & 0 & 0 & 0\\
      0 & 0 & 0 & 0
    \end{bmatrix}.
  \]

  Now, using \ref{theorem:CSRST} and \ref{theorem:BRS}
  \[
    \csp{M}=\rsp{\transpose{M}}=
    \spn{\set{
        \colvector{1\\0\\0\\-\frac{31}{7}},\,
        \colvector{0\\1\\0\\\frac{12}{7}},\,
        \colvector{0\\0\\1\\\frac{13}{7}}
      }}.
  \]

  This is a very nice description of the column space.  Fewer vectors
  than the 7 involved in the definition, and the pattern of the zeros
  and ones in the first 3 slots can be used to advantage.  For
  example, if we set
  \[
    \vect{b}=\colvector{3\\9\\1\\4}.
  \]
  then $\linearsystem{M}{\vect{b}}$ is consistent.  And since
  $\linearsystem{M}{\vect{b}}$ is consistent, \ref{theorem:CSCS} tells
  us that $\vect{b}\in\csp{M}$.  But we could see this quickly with
  the following computation, which really only involves any work in
  the 4th entry of the vectors as the scalars in the linear
  combination are \textit{dictated} by the first three entries of
  $\vect{b}$.
  \[
    \vect{b}=\colvector{3\\9\\1\\4}=
    3\colvector{1\\0\\0\\-\frac{31}{7}}+
    9\colvector{0\\1\\0\\\frac{12}{7}}+
    1\colvector{0\\0\\1\\\frac{13}{7}}
  \]

  Can you now rapidly construct several vectors, $\vect{b}$, so that
  $\linearsystem{M}{\vect{b}}$ is consistent, and several more so that
  the system is inconsistent?
\end{example}

\end{document}
