\documentclass{ximera}

% These macros are automatically generated from the "macros"
% XML element.  Make permanent edits there.
%
% History
%   2004/01/01  Initiated for FCLA, evolved from there
%   2006/09/17  Converted  _, ^  to \sb, \sp for TeX4ht
%   2014/02/01  Updated for MathBook XML projects
%               Obsolete in FCLA: \codeindent, \computerfont, \define
%               Change: MathJax wants \lt, so replaced by \lteval
%   2014/02/22  New: \orderof, \reals, \per
%   2015/08/16  Incorporated into MathBook XML version of FCLA
%
%%%%%%%%%%%%%%%%%%%%%
%
%     Conveniences
%
%%%%%%%%%%%%%%%%%%%%%
%
%  Order of (asymptotically limit of fraction is 1)
%  Usage: \orderof{some function}
%
\newcommand{\orderof}[1]{\sim #1}
%
%  Integers
%  Usage:  \Z
\newcommand{\Z}{\mathbb{Z}}
%
%  Real numbers, as set of scalars
%  Usage:  \reals
\newcommand{\reals}{\mathbb{R}}
%
%  n-space over real field
%  Usage: \complex{integer-dimension}
\newcommand{\real}[1]{\mathbb{R}^{#1}}
%
%  Complex numbers, as set of scalars
%  Usage:  \complexes
\newcommand{\complexes}{\mathbb{C}}
%
%  n-space over complex field
%  Usage: \complex{integer-dimension}
\newcommand{\complex}[1]{\mathbb{C}^{#1}}
\newcommand{\CC}{\mathbb{C}}
%
%  Complex conjugation (scalar, vector, matrix)
%  Usage: \conjugate{object}
\newcommand{\conjugate}[1]{\overline{#1}}
%
%  Complex number modulus
%  Usage: \modulus{a+bi}
%  Presumes math mode
\newcommand{\modulus}[1]{\left\lvert#1\right\rvert}
%
%  Zero vector
%  Usage: \zerovector
\newcommand{\zerovector}{\vect{0}}
%
%  Zero matrix
%  Usage: \zeromatrix, use a subscript when size is important
\newcommand{\zeromatrix}{\mathcal{O}}
%
%  Inner product (brackets, not quadratic form)
%  Usage: \innerproduct{a-vector}{a-vector}
\newcommand{\innerproduct}[2]{\left\langle#1,\,#2\right\rangle}
%
%  Norm of a vector
%  Usage: \norm{a-vector}
\newcommand{\norm}[1]{\left\lVert#1\right\rVert}
%
%  Dimension
%  Usage: \dimension{vector-space-letter}
\newcommand{\dimension}[1]{\dim\left(#1\right)}
%
%  Nullity
%  Usage: \nullity{matrix-or-lintrans-letter}
\newcommand{\nullity}[1]{n\left(#1\right)}
%
%  Rank
%  Usage: \rank{matrix-or-lintrans-letter}
\newcommand{\rank}[1]{r\left(#1\right)}
%
%  Direct sum
%  Usage: \ds between a couple of subspaces
%
\newcommand{\ds}{\oplus}
%
%  Determinant of a matrix (functional)
%  Usage: \detname{A}
\newcommand{\detname}[1]{\det\left(#1\right)}
%
%  Determinant of a matrix (vertical bars)
%  Usage: \detbars{A}
\newcommand{\detbars}[1]{\left\lvert#1\right\rvert}
%
%  Trace of a Matrix
%  Usage: \trace{matrix name}
\newcommand{\trace}[1]{t\left(#1\right)}
%
%  Square Root of a Matrix
%  Usage: \sr{a-matrix}
\newcommand{\sr}[1]{#1^{1/2}}
%
%%%%%%%%%%%%%%%%%%%%%
%
%     Subspace Constructions
%
%%%%%%%%%%%%%%%%%%%%%
%
%  Span of a set of vectors
%  \span and \sp are used by TeX for other things
%  Usage: \spn{set-of-vectors}
\newcommand{\spn}[1]{\left\langle#1\right\rangle}
%
%  Null space of a matrix
%  Usage:  \nsp{A}
\newcommand{\nsp}[1]{\mathcal{N}\!\left(#1\right)}
%
%  Column space of a matrix
%  Usage:  \csp{A}
\newcommand{\csp}[1]{\mathcal{C}\!\left(#1\right)}
%
%  Row space of a matrix
%  Usage:  \rsp{A}
\newcommand{\rsp}[1]{\mathcal{R}\!\left(#1\right)}
%
%  Left null space of a matrix
%  Usage:  \lns{A}
\newcommand{\lns}[1]{\mathcal{L}\!\left(#1\right)}
%
%  Orthogonal complement of a vector space
%  Avoiding TeX's \perp
%  Usage:  \per{A}
\newcommand{\per}[1]{#1^\perp}
%
%%%%%%%%%%%%%%%%%%%%%
%
%     Systems of Equations
%
%%%%%%%%%%%%%%%%%%%%%
%
%  In-line form of an augmented matrix for a system of equations
%  Usage: \augmented{coefficient-matrix}{constant-vector}
\newcommand{\augmented}[2]{\left\lbrack\left.#1\,\right\rvert\,#2\right\rbrack}
%
%  Notation for a linear system before introducing matrix multiplication
%  Usage: \linearsystem{coefficient-matrix}{constant-vector}
\newcommand{\linearsystem}[2]{\mathcal{LS}\!\left(#1,\,#2\right)}
%
%  Notation for a homogenous system before introducing matrix multiplication
%  Usage: \homosystem{coefficient-matrix}
\newcommand{\homosystem}[1]{\linearsystem{#1}{\zerovector}}
%
%%%%%%%%%%%%%%%%%%%%%
%
%     Row Operations, Echelon Form
%
%%%%%%%%%%%%%%%%%%%%%
%
% Row operations on matrices
%
% Three commands to shorten up descriptions of gaussian elimination
%
% Usage: \rowopswap{row-i}{row-j}
% Usage: \rowopmult{scalar}{row-i}
% Usage: \rowopadd{scalar}{row-multiplied}{row-added-to}
\newcommand{\rowopswap}[2]{R_{#1}\leftrightarrow R_{#2}}
\newcommand{\rowopmult}[2]{#1R_{#2}}
\newcommand{\rowopadd}[3]{#1R_{#2}+R_{#3}}
%
% Mark leading 1's in echelon form with fbox
% Usage: \leading{a-1-usually}
\newcommand{\leading}[1]{\fbox{#1}}
%
%  Row-reduce arrow
%  Usage:  \rref inbetween a matrix and its reduced row-echelon form
\newcommand{\rref}{\xrightarrow{\text{RREF}}}
%
%  Elementary Matrices
%  Usage: \elemswap{subscript}{subscript}
%  Usage: \elemmult{scalar}{subscript}
%  Usage: \elemadd{scalar}{subscript-mult}{subscript-target}
%
\newcommand{\elemswap}[2]{E_{#1,#2}}
\newcommand{\elemmult}[2]{E_{#2}\left(#1\right)}
\newcommand{\elemadd}[3]{E_{#2,#3}\left(#1\right)}
%
%%%%%%%%%%%%%%%%%%%%%
%
%     2-D Constructions (Lists, Vectors, Matrices)
%
%%%%%%%%%%%%%%%%%%%%%
%
%  A list of scalars of generic length
%  Usage:  \scalarlist{scalar letter}{terminal subscript}
\newcommand{\scalarlist}[2]{{#1}_{1},\,{#1}_{2},\,{#1}_{3},\,\ldots,\,{#1}_{#2}}
%
%  Vector styling, bold (or use wiggles, arrows, whatever)
%  Subscripts go outside this construction
%  Usage: \vect{a symbol to use as a vector}
%  Have to already be in math mode
%
\newcommand{\vect}[1]{\mathbf{#1}}
%
%  A column vector
%  Usage: \colvector{list-delimited-by-\\}
%
\newcommand{\colvector}[1]{\begin{bmatrix}#1\end{bmatrix}}
%
%  A generic vector with components
%  Usage: \vectorcomponents{component-letter}{final-subscript}
\newcommand{\vectorcomponents}[2]{\colvector{#1_{1}\\#1_{2}\\#1_{3}\\\vdots\\#1_{#2}}}
%
%  A list of vectors of generic length
%  Usage:  \vectorlist{vector letter}{terminal subscript}
\newcommand{\vectorlist}[2]{\vect{#1}_{1},\,\vect{#1}_{2},\,\vect{#1}_{3},\,\ldots,\,\vect{#1}_{#2}}
%
%  Vector entries, entry i of vector v
%  (vector-expession still needs \vect, etc.)
%  Usage:  \vectorentry{vector-expression}{single-subscript}
\newcommand{\vectorentry}[2]{\left\lbrack#1\right\rbrack_{#2}}
%
%  Matrix entries, entry i,j of matrix A
%  Usage:  \matrixentry{matrix-expression}{paired-subscripts}
%
\newcommand{\matrixentry}[2]{\left\lbrack#1\right\rbrack_{#2}}
%
%  A generic linear combination
%  Usage:  \lincombo{scalar letter}{vector letter}{terminal subscript}
\newcommand{\lincombo}[3]{#1_{1}\vect{#2}_{1}+#1_{2}\vect{#2}_{2}+#1_{3}\vect{#2}_{3}+\cdots +#1_{#3}\vect{#2}_{#3}}
%
%  Matrix, column by column, as vectors
%  Usage:  \matrixcolumns{matrix letter}{terminal subscript}
\newcommand{\matrixcolumns}[2]{\left\lbrack\vect{#1}_{1}|\vect{#1}_{2}|\vect{#1}_{3}|\ldots|\vect{#1}_{#2}\right\rbrack}
%
%%%%%%%%%%%%%%%%%%%%%
%
%     Special Matrices
%
%%%%%%%%%%%%%%%%%%%%%
%
%  Transpose of a matrix
%  Usage:  \transpose{A}
\newcommand{\transpose}[1]{#1^{t}}
%
%  Inverse of a matrix
%  Usage:  \inverse{A}
\newcommand{\inverse}[1]{#1^{-1}}
%
%  Submatrix (for minors, determinants)
%  Usage: \submatrix{matrix-name}{delete-row}{delete-col}
\newcommand{\submatrix}[3]{#1\left(#2|#3\right)}
%
%  Adjoint of a matrix (twice)
%  This macro is a convenience to call \transpose and \conjugate properly
%  It shouldn't need to be modified (or mathematical meanings will change)
%  Usage:  \adj{A}
\newcommand{\adj}[1]{\transpose{\left(\conjugate{#1}\right)}}
%
%  This macro controls the symbol used for the adjoint
%  It can be edited to taste
%  Usage:  \adjoint{A}
\newcommand{\adjoint}[1]{#1^\ast}
%
%%%%%%%%%%%%%%%%%%%%%
%
%     Sets
%
%%%%%%%%%%%%%%%%%%%%%
%
%  A convenience for simple sets
%  Usage:  \set{list of element}
\newcommand{\set}[1]{\left\{#1\right\}}
%
%  Sets with vertical bar, "such that", sized for objects, not condition
%  Usage:  \setparts{objects}{condition}
%
%%\newcommand{\setparts}[2]{\left\{ #1\mid#2\right\}}
%%\newcommand{\setparts}[2]{\left\{\left. #1\right\rvert#2\right\}}
\newcommand{\setparts}[2]{\left\lbrace#1\,\middle|\,#2\right\rbrace}
%
%  Set Cardinality
%  Usage:  \card{a-set-letter}
\newcommand{\card}[1]{\left\lvert#1\right\rvert}
%
%  Set Union
%  Use \cup
%
%  Set Intersection
%  Use \cap
%
%  Set Complement
%  Usage:  \setcomplement{a-set-letter}
\newcommand{\setcomplement}[1]{\overline{#1}}
%
%%%%%%%%%%%%%%%%%%%%%
%
%     Eigenvalues and Eigenspaces
%
%%%%%%%%%%%%%%%%%%%%%
%
%  Characteristic polynomial
%  Usage: \charpoly{matrix-letter}{variable-letter}
\newcommand{\charpoly}[2]{p_{#1}\left(#2\right)}
%
%  Eigenspace
%  Usage: \eigenspace{matrix-letter}{eigenvalue-letter}
\newcommand{\eigenspace}[2]{\mathcal{E}_{#1}\left(#2\right)}
%
%  2013/10/03 Including ampersands is problematic here, 
%  think about fixes later
%  2014/02/22 Limited testing, seems &amp; is fine for HTML and LaTeX
%  2016-07-20 only employed in Archetypes, MBX has gather/align override
%  Eigensystem (presumes wrapped in an mrow within md)
%  Usage: \eigensystem{matrixletter}{eigenvalue}{list of basis vectors}
\newcommand{\eigensystem}[3]{\lambda&amp;=#2&amp;\eigenspace{#1}{#2}&amp;=\spn{\set{#3}}} 
%
%  Generalized Eigenspace
%  Usage: \geneigenspace{lin-trans-letter}{eigenvalue-letter}
\newcommand{\geneigenspace}[2]{\mathcal{G}_{#1}\left(#2\right)}
%
%  Algebraic multiplicty
%  Usage: \algmult{matrix-letter}{eigenvalue-letter}
\newcommand{\algmult}[2]{\alpha_{#1}\left(#2\right)}
%
%  Geometric multiplicty
%  Usage: \geomult{matrix-letter}{eigenvalue-letter}
\newcommand{\geomult}[2]{\gamma_{#1}\left(#2\right)}
%
%  Index (of eigenvalue)
%  Usage: \indx{matrix-letter}{eigenvalue-letter}
\newcommand{\indx}[2]{\iota_{#1}\left(#2\right)}
%
%%%%%%%%%%%%%%%%%%%%%
%
%     Linear Transformations
%
%%%%%%%%%%%%%%%%%%%%%
%
%  MathJax defines \lt to ease XML confusion
%
%  Linear transformation definition
%  Usage: \ltdefn{name-letter}{domain}{range}
\newcommand{\ltdefn}[3]{#1\colon #2\rightarrow#3}
%
%  Linear transformation evaluation
%  Usage: \lteval{name-letter}{input}
%  Replaces old \lt desired by MathJax
\newcommand{\lteval}[2]{#1\left(#2\right)}
%
% Linear transformation inverse
%  Usage: \ltinverse{name-letter}
\newcommand{\ltinverse}[1]{#1^{-1}}
%
%  Linear transformation restriction
%  Usage: \restrict{name-letter}{subspace-letter}
\newcommand{\restrict}[2]{{#1}|_{#2}}
%
%  Linear transformation preimage
%  Usage: \preimage{name-letter}{codomain-element}
\newcommand{\preimage}[2]{#1^{-1}\left(#2\right)}
%
%  Range of a linear transformation
%  TeX uses \range for something else
%  Usage:  \rng{T}
\newcommand{\rng}[1]{\mathcal{R}\!\left(#1\right)}
%
%  Kernel of a linear transformation
%  TeX uses \ker to do something different
%  Usage:  \krn{T}
\newcommand{\krn}[1]{\mathcal{K}\!\left(#1\right)}
%
%  Linear transformation composition
%  Usage: \compose{function-name}{function-name}
\newcommand{\compose}[2]{{#1}\circ{#2}}
%
%  Vector space of linear transformations
%  Usage: \vslt{domains}{codomains}
%  Presumes math mode
\newcommand{\vslt}[2]{\mathcal{LT}\left(#1,\,#2\right)}
%
%%%%%%%%%%%%%%%%%%%%%
%
%     Vector and Matrix Representations
%
%%%%%%%%%%%%%%%%%%%%%
%
%  Isomorphism symbol
%  Usage: \isomorphic
\newcommand{\isomorphic}{\cong}
%
%  Similarity
%  Usage: \similar{inner-matrix}{outer-invertible-matrix}
%  Rearranging this will not "fix" all desired changes throughout
%
\newcommand{\similar}[2]{\inverse{#2}#1#2}
%
%  Vector representation function name
%  Usage: \vectrepname{basis-letter}
\newcommand{\vectrepname}[1]{\rho_{#1}}
%
%  Vector representation output
%  Usage: \vectrep{basis-letter}{input}
\newcommand{\vectrep}[2]{\lteval{\vectrepname{#1}}{#2}}
%
%  Vector representation inverse function name
%  (Added later, not used consistently in FCLA)
%  Usage: \vectrepinvname{basis-letter}
\newcommand{\vectrepinvname}[1]{\ltinverse{\vectrepname{#1}}}
%
%  Vector representation inverse output
%  Usage: \vectrepinv{basis-letter}{input}
\newcommand{\vectrepinv}[2]{\lteval{\ltinverse{\vectrepname{#1}}}{#2}}
%
%  Matrix representation
%  Usage: \matrixrep{transformation-letter}{domain-basis-letter}{codomain-basis-letter}
\newcommand{\matrixrep}[3]{M^{#1}_{#2,#3}}
%
%  Matrix representation column-by-colum
%  2016-07-20 only employed once?
%  Usage: \matrixrepcolumns{transformation-letter}{codomain-basis-letter}{codomain-basis-vector-letter}{final-index}
\newcommand{\matrixrepcolumns}[4]{\left\lbrack \left.\vectrep{#2}{\lteval{#1}{\vect{#3}_{1}}}\right|\left.\vectrep{#2}{\lteval{#1}{\vect{#3}_{2}}}\right|\left.\vectrep{#2}{\lteval{#1}{\vect{#3}_{3}}}\right|\ldots\left|\vectrep{#2}{\lteval{#1}{\vect{#3}_{#4}}}\right.\right\rbrack}
%
%  Change of basis matrix
%  Usage: \cbm{domain-basis-letter}{codomain-basis-letter}
\newcommand{\cbm}[2]{C_{#1,#2}}
%
%%%%%%%%%%%%%%%%%%%%%
%
%     Canonical Forms
%
%%%%%%%%%%%%%%%%%%%%%
%
%  Jordan blocks
%  Usage: \jordan{size}{diagonal-element}
\newcommand{\jordan}[2]{J_{#1}\left(#2\right)}
%
%%%%%%%%%%%%%%%%%%%%%
%
%     Hadamard Matrices
%     Contributed by Elizabeth Million
%
%%%%%%%%%%%%%%%%%%%%%
%
%  Hadamard Product
%  Usage: \hadamard{a-matrix}{a-matrix}
\newcommand{\hadamard}[2]{#1\circ #2}
%
%  Hadamard identity matrix
%  Usage: \hadamardidentity{paired-subscripts-size-of-matrix}
\newcommand{\hadamardidentity}[1]{J_{#1}}
%
%  Hadamard inverse matrix
%  Usage: \hadamardinverse{matrix-expression}
\newcommand{\hadamardinverse}[1]{\widehat{#1}}

\newcommand{\definedTerm}[1]{\textbf{#1}}
\newcommand{\dfn}[1]{\textbf{#1}}

\newcommand{\wt}{\widetilde}
\newcommand{\ov}{\overline}
\newcommand{\inj}{\rightarrowtail}
\newcommand{\surj}{\twoheadrightarrow}
\newcommand{\harpoon}{\overset{\rightharpoonup}}

\newenvironment{amatrix}[1]{%
  \left[\begin{array}{@{}*{#1}{c}|c@{}}
}{%
  \end{array}\right]
}


\title{Extended Echelon Form}

\begin{document}
\begin{abstract}
  Computing the column space of a matrix as the null space of an
  associated matrix feels similar to the computation of the inverse of
  a nonsingular matrix.
\end{abstract}
\maketitle

\begin{definition}[Extended Echelon Form]
  Suppose $A$ is an $m\times n$ matrix.  Extend $A$ on its right side
  with the addition of an $m\times m$ identity matrix to form an
  $m\times (n + m)$ matrix $M$.  Use row operations to bring $M$ to
  reduced row-echelon form and call the result $N$.  $N$ is the
  \dfn{extended reduced row-echelon form} of $A$, and we will
  standardize on names for five submatrices ($B$, $C$, $J$, $K$, $L$)
  of $N$.

  Let $B$ denote the $m\times n$ matrix formed from the first $n$
  columns of $N$ and let $J$ denote the $m\times m$ matrix formed from
  the last $m$ columns of $N$.  Suppose that $B$ has $r$ nonzero rows.
  Further partition $N$ by letting $C$ denote the $r\times n$ matrix
  formed from all of the nonzero rows of $B$.  Let $K$ be the
  $r\times m$ matrix formed from the first $r$ rows of $J$, while $L$
  will be the $(m-r)\times m$ matrix formed from the bottom $m-r$ rows
  of $J$.  Pictorially,
  \[
    M=[A\vert I_m]
    \rref
    N=[B\vert J]
    =
    \left[\begin{array}{c|c}C&K\\\hline0&L\end{array}\right]
  \]
\end{definition}

\begin{example}[Submatrices of extended echelon form]

  We illustrate extended echelon form with the matrix $A$,
  \[
    A=
    \begin{bmatrix}
      1 & -1 & -2 & 7 & 1 & 6 \\
      -6 & 2 & -4 & -18 & -3 & -26 \\
      4 & -1 & 4 & 10 & 2 & 17 \\
      3 & -1 & 2 & 9 & 1 & 12
    \end{bmatrix}
  \]

  Augmenting with the $4\times 4$ identity matrix,
  \[
    M=
    \begin{bmatrix}
      1 & -1 & -2 & 7 & 1 & 6 & 1 & 0 & 0 & 0 \\
      -6 & 2 & -4 & -18 & -3 & -26 & 0 & 1 & 0 & 0 \\
      4 & -1 & 4 & 10 & 2 & 17 & 0 & 0 & 1 & 0 \\
      3 & -1 & 2 & 9 & 1 & 12 & 0 & 0 & 0 & 1
    \end{bmatrix}
  \]
  and row-reducing, we obtain
  \[
    N=
    \begin{bmatrix}
      \leading{1} & 0 & 2 & 1 & 0 & 3 & 0 & 1 & 1 & 1\\
      0 & \leading{1} & 4 & -6 & 0 & -1 & 0 & 2 & 3 & 0 \\
      0 & 0 & 0 & 0 & \leading{1} & 2 & 0 & -1 & 0 & -2 \\
      0 & 0 & 0 & 0 & 0 & 0 & \leading{1} & 2 & 2 & 1
    \end{bmatrix}
  \]
  
  So we then obtain
  \begin{align*}
    B&=
       \begin{bmatrix}
         \leading{1} & 0 & 2 & 1 & 0 & 3 \\
         0 & \leading{1} & 4 & -6 & 0 & -1 \\
         0 & 0 & 0 & 0 & \leading{1} & 2 \\
         0 & 0 & 0 & 0 & 0 & 0
       \end{bmatrix}\\
    C&=
       \begin{bmatrix}
         \leading{1} & 0 & 2 & 1 & 0 & 3 \\
         0 & \leading{1} & 4 & -6 & 0 & -1 \\
         0 & 0 & 0 & 0 & \leading{1} & 2
       \end{bmatrix}\\
    J&=
       \begin{bmatrix}
         0 & 1 & 1 & 1\\
         0 & 2 & 3 & 0 \\
         0 & -1 & 0 & -2 \\
         \leading{1} & 2 & 2 & 1
       \end{bmatrix}\\
    K&=
       \begin{bmatrix}
         0 & 1 & 1 & 1\\
         0 & 2 & 3 & 0 \\
         0 & -1 & 0 & -2
       \end{bmatrix}\\
    L&=
       \begin{bmatrix}
         \leading{1} & \answer{2} & 2 & 1
       \end{bmatrix}
  \end{align*}
\end{example}

You can observe (or verify) the properties of the following theorem
with the preceeding example.

\begin{theorem}[Properties of Extended Echelon Form]
  \label{theorem:PEEF}

  Suppose that $A$ is an $m\times n$ matrix and that $N$ is its extended echelon form.  Then
  \begin{enumerate}
  \item $J$ is nonsingular.
  \item $B=JA$.
  \item If $\vect{x}\in\complex{n}$ and $\vect{y}\in\complex{m}$, then $A\vect{x}=\vect{y}$ if and only if $B\vect{x}=J\vect{y}$.
  \item $C$ is in reduced row-echelon form, has no zero rows and has $r$ pivot columns.
  \item $L$ is in reduced row-echelon form, has no zero rows and has $m-r$ pivot columns.
  \end{enumerate}

  \begin{proof}
    $J$ is the result of applying a sequence of row operations to
    $I_m$, and therefore $J$ and $I_m$ are row-equivalent.
    $\homosystem{I_m}$ has only the zero solution, since $I_m$ is
    nonsingular (\ref{theorem:NMRRI}).  Thus, $\homosystem{J}$ also
    has only the zero solution (\ref{theorem:REMES},
    \ref{definition:ESYS}) and $J$ is therefore nonsingular
    (\ref{definition:NSM}).

    To prove the second part of this conclusion, first convince
    yourself that row operations and the matrix-vector product are
    associative operations.  By this we mean the following.  Suppose
    that $F$ is an $m\times n$ matrix that is row-equivalent to the
    matrix $G$.  Apply to the column vector $F\vect{w}$ the same
    sequence of row operations that converts $F$ to $G$.  Then the
    result is 
    \begin{multipleChoice}
      \choice[correct]{$G\vect{w}$.}
      \choice{$F\vect{w}$.}
    \end{multipleChoice}

    So we can do row operations on the matrix, then do a matrix-vector
    product, \textit{or} do a matrix-vector product and then do row
    operations on a column vector, and the result will be the same
    either way.  Since matrix multiplication is defined by a
    collection of matrix-vector products (\ref{definition:MM}), the
    matrix product $FH$ will become $GH$ if we apply the same sequence
    of row operations to $FH$ that convert $F$ to $G$.  Now apply
    these observations to $A$.

    Write $AI_n=I_mA$ and apply the row operations that convert $M$ to
    $N$.  $A$ is converted to $B$, while $I_m$ is converted to $J$, so
    we have $BI_n=JA$.  Simplifying the left side gives the desired
    conclusion.

    For the third conclusion, we now establish the two equivalences
    \begin{align*}
      A\vect{x}&=\vect{y} &
      &\iff &
              JA\vect{x}&=J\vect{y} &
      &\iff &
              B\vect{x}&=J\vect{y}
    \end{align*}

    The forward direction of the first equivalence is accomplished by
    multiplying both sides of the matrix equality by $J$, while the
    backward direction is accomplished by multiplying by the inverse
    of $J$ (which we know exists by \ref{theorem:NI} since $J$ is
    nonsingular).  The second equivalence is obtained simply by the
    substitutions given by $JA=B$.

    The first $r$ rows of $N$ are in reduced row-echelon form, since
    any contiguous collection of rows taken from a matrix in reduced
    row-echelon form will form a matrix that is again in reduced
    row-echelon form.  Since the matrix $C$ is formed by removing the
    last $n$ entries of each these rows, the remainder is still in
    reduced row-echelon form.  By its construction, $C$ has no zero
    rows. $C$ has $r$ rows and each contains a leading 1, so there are
    $r$ pivot columns in $C$.

    The final $m-r$ rows of $N$ are in reduced row-echelon form, since
    any contiguous collection of rows taken from a matrix in reduced
    row-echelon form will form a matrix that is again in reduced
    row-echelon form.  Since the matrix $L$ is formed by removing the
    first $n$ entries of each these rows, and these entries are all
    zero (they form the zero rows of $B$), the remainder is still in
    reduced row-echelon form.  $L$ is the final $m-r$ rows of the
    nonsingular matrix $J$, so none of these rows can be totally zero,
    or $J$ would not row-reduce to the identity matrix.  $L$ has $m-r$
    rows and each contains a leading 1, so there are $\answer{m-r}$
    pivot columns in $L$.
\end{proof}
\end{theorem}

Notice that in the case where $A$ is a nonsingular matrix we know that
the reduced row-echelon form of $A$ is the identity matrix
(\ref{theorem:NMRRI}), so $B=I_n$.  Then the second conclusion above
says $JA=B=I_n$, so $J$ is the inverse of $A$.  Thus this theorem
generalizes \ref{theorem:CINM}, though the result is a
``left-inverse'' of $A$ rather than a ``right-inverse.''

The third conclusion of \ref{theorem:PEEF} is the most telling.  It
says that $\vect{x}$ is a solution to the linear system
$\linearsystem{A}{\vect{y}}$ if and only if $\vect{x}$ is a solution
to the linear system $\linearsystem{B}{J\vect{y}}$.  Or said
differently, if we row-reduce the augmented matrix
$\augmented{A}{\vect{y}}$ we will get the augmented matrix
$\augmented{B}{J\vect{y}}$.  The matrix $J$ tracks the cumulative
effect of the row operations that converts $A$ to reduced row-echelon
form, here effectively applying them to the vector of constants in a
system of equations having $A$ as a coefficient matrix.  When $A$
row-reduces to a matrix with zero rows, then $J\vect{y}$ should also
have zero entries in the same rows if the system is to be consistent.

\end{document}
