\documentclass{ximera}
% These macros are automatically generated from the "macros"
% XML element.  Make permanent edits there.
%
% History
%   2004/01/01  Initiated for FCLA, evolved from there
%   2006/09/17  Converted  _, ^  to \sb, \sp for TeX4ht
%   2014/02/01  Updated for MathBook XML projects
%               Obsolete in FCLA: \codeindent, \computerfont, \define
%               Change: MathJax wants \lt, so replaced by \lteval
%   2014/02/22  New: \orderof, \reals, \per
%   2015/08/16  Incorporated into MathBook XML version of FCLA
%
%%%%%%%%%%%%%%%%%%%%%
%
%     Conveniences
%
%%%%%%%%%%%%%%%%%%%%%
%
%  Order of (asymptotically limit of fraction is 1)
%  Usage: \orderof{some function}
%
\newcommand{\orderof}[1]{\sim #1}
%
%  Integers
%  Usage:  \Z
\newcommand{\Z}{\mathbb{Z}}
%
%  Real numbers, as set of scalars
%  Usage:  \reals
\newcommand{\reals}{\mathbb{R}}
%
%  n-space over real field
%  Usage: \complex{integer-dimension}
\newcommand{\real}[1]{\mathbb{R}^{#1}}
%
%  Complex numbers, as set of scalars
%  Usage:  \complexes
\newcommand{\complexes}{\mathbb{C}}
%
%  n-space over complex field
%  Usage: \complex{integer-dimension}
\newcommand{\complex}[1]{\mathbb{C}^{#1}}
\newcommand{\CC}{\mathbb{C}}
%
%  Complex conjugation (scalar, vector, matrix)
%  Usage: \conjugate{object}
\newcommand{\conjugate}[1]{\overline{#1}}
%
%  Complex number modulus
%  Usage: \modulus{a+bi}
%  Presumes math mode
\newcommand{\modulus}[1]{\left\lvert#1\right\rvert}
%
%  Zero vector
%  Usage: \zerovector
\newcommand{\zerovector}{\vect{0}}
%
%  Zero matrix
%  Usage: \zeromatrix, use a subscript when size is important
\newcommand{\zeromatrix}{\mathcal{O}}
%
%  Inner product (brackets, not quadratic form)
%  Usage: \innerproduct{a-vector}{a-vector}
\newcommand{\innerproduct}[2]{\left\langle#1,\,#2\right\rangle}
%
%  Norm of a vector
%  Usage: \norm{a-vector}
\newcommand{\norm}[1]{\left\lVert#1\right\rVert}
%
%  Dimension
%  Usage: \dimension{vector-space-letter}
\newcommand{\dimension}[1]{\dim\left(#1\right)}
%
%  Nullity
%  Usage: \nullity{matrix-or-lintrans-letter}
\newcommand{\nullity}[1]{n\left(#1\right)}
%
%  Rank
%  Usage: \rank{matrix-or-lintrans-letter}
\newcommand{\rank}[1]{r\left(#1\right)}
%
%  Direct sum
%  Usage: \ds between a couple of subspaces
%
\newcommand{\ds}{\oplus}
%
%  Determinant of a matrix (functional)
%  Usage: \detname{A}
\newcommand{\detname}[1]{\det\left(#1\right)}
%
%  Determinant of a matrix (vertical bars)
%  Usage: \detbars{A}
\newcommand{\detbars}[1]{\left\lvert#1\right\rvert}
%
%  Trace of a Matrix
%  Usage: \trace{matrix name}
\newcommand{\trace}[1]{t\left(#1\right)}
%
%  Square Root of a Matrix
%  Usage: \sr{a-matrix}
\newcommand{\sr}[1]{#1^{1/2}}
%
%%%%%%%%%%%%%%%%%%%%%
%
%     Subspace Constructions
%
%%%%%%%%%%%%%%%%%%%%%
%
%  Span of a set of vectors
%  \span and \sp are used by TeX for other things
%  Usage: \spn{set-of-vectors}
\newcommand{\spn}[1]{\left\langle#1\right\rangle}
%
%  Null space of a matrix
%  Usage:  \nsp{A}
\newcommand{\nsp}[1]{\mathcal{N}\!\left(#1\right)}
%
%  Column space of a matrix
%  Usage:  \csp{A}
\newcommand{\csp}[1]{\mathcal{C}\!\left(#1\right)}
%
%  Row space of a matrix
%  Usage:  \rsp{A}
\newcommand{\rsp}[1]{\mathcal{R}\!\left(#1\right)}
%
%  Left null space of a matrix
%  Usage:  \lns{A}
\newcommand{\lns}[1]{\mathcal{L}\!\left(#1\right)}
%
%  Orthogonal complement of a vector space
%  Avoiding TeX's \perp
%  Usage:  \per{A}
\newcommand{\per}[1]{#1^\perp}
%
%%%%%%%%%%%%%%%%%%%%%
%
%     Systems of Equations
%
%%%%%%%%%%%%%%%%%%%%%
%
%  In-line form of an augmented matrix for a system of equations
%  Usage: \augmented{coefficient-matrix}{constant-vector}
\newcommand{\augmented}[2]{\left\lbrack\left.#1\,\right\rvert\,#2\right\rbrack}
%
%  Notation for a linear system before introducing matrix multiplication
%  Usage: \linearsystem{coefficient-matrix}{constant-vector}
\newcommand{\linearsystem}[2]{\mathcal{LS}\!\left(#1,\,#2\right)}
%
%  Notation for a homogenous system before introducing matrix multiplication
%  Usage: \homosystem{coefficient-matrix}
\newcommand{\homosystem}[1]{\linearsystem{#1}{\zerovector}}
%
%%%%%%%%%%%%%%%%%%%%%
%
%     Row Operations, Echelon Form
%
%%%%%%%%%%%%%%%%%%%%%
%
% Row operations on matrices
%
% Three commands to shorten up descriptions of gaussian elimination
%
% Usage: \rowopswap{row-i}{row-j}
% Usage: \rowopmult{scalar}{row-i}
% Usage: \rowopadd{scalar}{row-multiplied}{row-added-to}
\newcommand{\rowopswap}[2]{R_{#1}\leftrightarrow R_{#2}}
\newcommand{\rowopmult}[2]{#1R_{#2}}
\newcommand{\rowopadd}[3]{#1R_{#2}+R_{#3}}
%
% Mark leading 1's in echelon form with fbox
% Usage: \leading{a-1-usually}
\newcommand{\leading}[1]{\fbox{#1}}
%
%  Row-reduce arrow
%  Usage:  \rref inbetween a matrix and its reduced row-echelon form
\newcommand{\rref}{\xrightarrow{\text{RREF}}}
%
%  Elementary Matrices
%  Usage: \elemswap{subscript}{subscript}
%  Usage: \elemmult{scalar}{subscript}
%  Usage: \elemadd{scalar}{subscript-mult}{subscript-target}
%
\newcommand{\elemswap}[2]{E_{#1,#2}}
\newcommand{\elemmult}[2]{E_{#2}\left(#1\right)}
\newcommand{\elemadd}[3]{E_{#2,#3}\left(#1\right)}
%
%%%%%%%%%%%%%%%%%%%%%
%
%     2-D Constructions (Lists, Vectors, Matrices)
%
%%%%%%%%%%%%%%%%%%%%%
%
%  A list of scalars of generic length
%  Usage:  \scalarlist{scalar letter}{terminal subscript}
\newcommand{\scalarlist}[2]{{#1}_{1},\,{#1}_{2},\,{#1}_{3},\,\ldots,\,{#1}_{#2}}
%
%  Vector styling, bold (or use wiggles, arrows, whatever)
%  Subscripts go outside this construction
%  Usage: \vect{a symbol to use as a vector}
%  Have to already be in math mode
%
\newcommand{\vect}[1]{\mathbf{#1}}
%
%  A column vector
%  Usage: \colvector{list-delimited-by-\\}
%
\newcommand{\colvector}[1]{\begin{bmatrix}#1\end{bmatrix}}
%
%  A generic vector with components
%  Usage: \vectorcomponents{component-letter}{final-subscript}
\newcommand{\vectorcomponents}[2]{\colvector{#1_{1}\\#1_{2}\\#1_{3}\\\vdots\\#1_{#2}}}
%
%  A list of vectors of generic length
%  Usage:  \vectorlist{vector letter}{terminal subscript}
\newcommand{\vectorlist}[2]{\vect{#1}_{1},\,\vect{#1}_{2},\,\vect{#1}_{3},\,\ldots,\,\vect{#1}_{#2}}
%
%  Vector entries, entry i of vector v
%  (vector-expession still needs \vect, etc.)
%  Usage:  \vectorentry{vector-expression}{single-subscript}
\newcommand{\vectorentry}[2]{\left\lbrack#1\right\rbrack_{#2}}
%
%  Matrix entries, entry i,j of matrix A
%  Usage:  \matrixentry{matrix-expression}{paired-subscripts}
%
\newcommand{\matrixentry}[2]{\left\lbrack#1\right\rbrack_{#2}}
%
%  A generic linear combination
%  Usage:  \lincombo{scalar letter}{vector letter}{terminal subscript}
\newcommand{\lincombo}[3]{#1_{1}\vect{#2}_{1}+#1_{2}\vect{#2}_{2}+#1_{3}\vect{#2}_{3}+\cdots +#1_{#3}\vect{#2}_{#3}}
%
%  Matrix, column by column, as vectors
%  Usage:  \matrixcolumns{matrix letter}{terminal subscript}
\newcommand{\matrixcolumns}[2]{\left\lbrack\vect{#1}_{1}|\vect{#1}_{2}|\vect{#1}_{3}|\ldots|\vect{#1}_{#2}\right\rbrack}
%
%%%%%%%%%%%%%%%%%%%%%
%
%     Special Matrices
%
%%%%%%%%%%%%%%%%%%%%%
%
%  Transpose of a matrix
%  Usage:  \transpose{A}
\newcommand{\transpose}[1]{#1^{t}}
%
%  Inverse of a matrix
%  Usage:  \inverse{A}
\newcommand{\inverse}[1]{#1^{-1}}
%
%  Submatrix (for minors, determinants)
%  Usage: \submatrix{matrix-name}{delete-row}{delete-col}
\newcommand{\submatrix}[3]{#1\left(#2|#3\right)}
%
%  Adjoint of a matrix (twice)
%  This macro is a convenience to call \transpose and \conjugate properly
%  It shouldn't need to be modified (or mathematical meanings will change)
%  Usage:  \adj{A}
\newcommand{\adj}[1]{\transpose{\left(\conjugate{#1}\right)}}
%
%  This macro controls the symbol used for the adjoint
%  It can be edited to taste
%  Usage:  \adjoint{A}
\newcommand{\adjoint}[1]{#1^\ast}
%
%%%%%%%%%%%%%%%%%%%%%
%
%     Sets
%
%%%%%%%%%%%%%%%%%%%%%
%
%  A convenience for simple sets
%  Usage:  \set{list of element}
\newcommand{\set}[1]{\left\{#1\right\}}
%
%  Sets with vertical bar, "such that", sized for objects, not condition
%  Usage:  \setparts{objects}{condition}
%
%%\newcommand{\setparts}[2]{\left\{ #1\mid#2\right\}}
%%\newcommand{\setparts}[2]{\left\{\left. #1\right\rvert#2\right\}}
\newcommand{\setparts}[2]{\left\lbrace#1\,\middle|\,#2\right\rbrace}
%
%  Set Cardinality
%  Usage:  \card{a-set-letter}
\newcommand{\card}[1]{\left\lvert#1\right\rvert}
%
%  Set Union
%  Use \cup
%
%  Set Intersection
%  Use \cap
%
%  Set Complement
%  Usage:  \setcomplement{a-set-letter}
\newcommand{\setcomplement}[1]{\overline{#1}}
%
%%%%%%%%%%%%%%%%%%%%%
%
%     Eigenvalues and Eigenspaces
%
%%%%%%%%%%%%%%%%%%%%%
%
%  Characteristic polynomial
%  Usage: \charpoly{matrix-letter}{variable-letter}
\newcommand{\charpoly}[2]{p_{#1}\left(#2\right)}
%
%  Eigenspace
%  Usage: \eigenspace{matrix-letter}{eigenvalue-letter}
\newcommand{\eigenspace}[2]{\mathcal{E}_{#1}\left(#2\right)}
%
%  2013/10/03 Including ampersands is problematic here, 
%  think about fixes later
%  2014/02/22 Limited testing, seems &amp; is fine for HTML and LaTeX
%  2016-07-20 only employed in Archetypes, MBX has gather/align override
%  Eigensystem (presumes wrapped in an mrow within md)
%  Usage: \eigensystem{matrixletter}{eigenvalue}{list of basis vectors}
\newcommand{\eigensystem}[3]{\lambda&amp;=#2&amp;\eigenspace{#1}{#2}&amp;=\spn{\set{#3}}} 
%
%  Generalized Eigenspace
%  Usage: \geneigenspace{lin-trans-letter}{eigenvalue-letter}
\newcommand{\geneigenspace}[2]{\mathcal{G}_{#1}\left(#2\right)}
%
%  Algebraic multiplicty
%  Usage: \algmult{matrix-letter}{eigenvalue-letter}
\newcommand{\algmult}[2]{\alpha_{#1}\left(#2\right)}
%
%  Geometric multiplicty
%  Usage: \geomult{matrix-letter}{eigenvalue-letter}
\newcommand{\geomult}[2]{\gamma_{#1}\left(#2\right)}
%
%  Index (of eigenvalue)
%  Usage: \indx{matrix-letter}{eigenvalue-letter}
\newcommand{\indx}[2]{\iota_{#1}\left(#2\right)}
%
%%%%%%%%%%%%%%%%%%%%%
%
%     Linear Transformations
%
%%%%%%%%%%%%%%%%%%%%%
%
%  MathJax defines \lt to ease XML confusion
%
%  Linear transformation definition
%  Usage: \ltdefn{name-letter}{domain}{range}
\newcommand{\ltdefn}[3]{#1\colon #2\rightarrow#3}
%
%  Linear transformation evaluation
%  Usage: \lteval{name-letter}{input}
%  Replaces old \lt desired by MathJax
\newcommand{\lteval}[2]{#1\left(#2\right)}
%
% Linear transformation inverse
%  Usage: \ltinverse{name-letter}
\newcommand{\ltinverse}[1]{#1^{-1}}
%
%  Linear transformation restriction
%  Usage: \restrict{name-letter}{subspace-letter}
\newcommand{\restrict}[2]{{#1}|_{#2}}
%
%  Linear transformation preimage
%  Usage: \preimage{name-letter}{codomain-element}
\newcommand{\preimage}[2]{#1^{-1}\left(#2\right)}
%
%  Range of a linear transformation
%  TeX uses \range for something else
%  Usage:  \rng{T}
\newcommand{\rng}[1]{\mathcal{R}\!\left(#1\right)}
%
%  Kernel of a linear transformation
%  TeX uses \ker to do something different
%  Usage:  \krn{T}
\newcommand{\krn}[1]{\mathcal{K}\!\left(#1\right)}
%
%  Linear transformation composition
%  Usage: \compose{function-name}{function-name}
\newcommand{\compose}[2]{{#1}\circ{#2}}
%
%  Vector space of linear transformations
%  Usage: \vslt{domains}{codomains}
%  Presumes math mode
\newcommand{\vslt}[2]{\mathcal{LT}\left(#1,\,#2\right)}
%
%%%%%%%%%%%%%%%%%%%%%
%
%     Vector and Matrix Representations
%
%%%%%%%%%%%%%%%%%%%%%
%
%  Isomorphism symbol
%  Usage: \isomorphic
\newcommand{\isomorphic}{\cong}
%
%  Similarity
%  Usage: \similar{inner-matrix}{outer-invertible-matrix}
%  Rearranging this will not "fix" all desired changes throughout
%
\newcommand{\similar}[2]{\inverse{#2}#1#2}
%
%  Vector representation function name
%  Usage: \vectrepname{basis-letter}
\newcommand{\vectrepname}[1]{\rho_{#1}}
%
%  Vector representation output
%  Usage: \vectrep{basis-letter}{input}
\newcommand{\vectrep}[2]{\lteval{\vectrepname{#1}}{#2}}
%
%  Vector representation inverse function name
%  (Added later, not used consistently in FCLA)
%  Usage: \vectrepinvname{basis-letter}
\newcommand{\vectrepinvname}[1]{\ltinverse{\vectrepname{#1}}}
%
%  Vector representation inverse output
%  Usage: \vectrepinv{basis-letter}{input}
\newcommand{\vectrepinv}[2]{\lteval{\ltinverse{\vectrepname{#1}}}{#2}}
%
%  Matrix representation
%  Usage: \matrixrep{transformation-letter}{domain-basis-letter}{codomain-basis-letter}
\newcommand{\matrixrep}[3]{M^{#1}_{#2,#3}}
%
%  Matrix representation column-by-colum
%  2016-07-20 only employed once?
%  Usage: \matrixrepcolumns{transformation-letter}{codomain-basis-letter}{codomain-basis-vector-letter}{final-index}
\newcommand{\matrixrepcolumns}[4]{\left\lbrack \left.\vectrep{#2}{\lteval{#1}{\vect{#3}_{1}}}\right|\left.\vectrep{#2}{\lteval{#1}{\vect{#3}_{2}}}\right|\left.\vectrep{#2}{\lteval{#1}{\vect{#3}_{3}}}\right|\ldots\left|\vectrep{#2}{\lteval{#1}{\vect{#3}_{#4}}}\right.\right\rbrack}
%
%  Change of basis matrix
%  Usage: \cbm{domain-basis-letter}{codomain-basis-letter}
\newcommand{\cbm}[2]{C_{#1,#2}}
%
%%%%%%%%%%%%%%%%%%%%%
%
%     Canonical Forms
%
%%%%%%%%%%%%%%%%%%%%%
%
%  Jordan blocks
%  Usage: \jordan{size}{diagonal-element}
\newcommand{\jordan}[2]{J_{#1}\left(#2\right)}
%
%%%%%%%%%%%%%%%%%%%%%
%
%     Hadamard Matrices
%     Contributed by Elizabeth Million
%
%%%%%%%%%%%%%%%%%%%%%
%
%  Hadamard Product
%  Usage: \hadamard{a-matrix}{a-matrix}
\newcommand{\hadamard}[2]{#1\circ #2}
%
%  Hadamard identity matrix
%  Usage: \hadamardidentity{paired-subscripts-size-of-matrix}
\newcommand{\hadamardidentity}[1]{J_{#1}}
%
%  Hadamard inverse matrix
%  Usage: \hadamardinverse{matrix-expression}
\newcommand{\hadamardinverse}[1]{\widehat{#1}}

\newcommand{\definedTerm}[1]{\textbf{#1}}
\newcommand{\dfn}[1]{\textbf{#1}}

\newcommand{\wt}{\widetilde}
\newcommand{\ov}{\overline}
\newcommand{\inj}{\rightarrowtail}
\newcommand{\surj}{\twoheadrightarrow}
\newcommand{\harpoon}{\overset{\rightharpoonup}}

\newenvironment{amatrix}[1]{%
  \left[\begin{array}{@{}*{#1}{c}|c@{}}
}{%
  \end{array}\right]
}

\title{Polynomial data fitting}
\author{Crichton Ogle}

\begin{document}
\begin{abstract}
\end{abstract}
\maketitle

We suppose given $n$ points $\{(x_1,y_1), (x_2,y_2),\dots,(x_n,y_n)\}$ in the plane $\mathbb R^2$, with distinct $x$-coordinates (in practice, such sets of points can arise as data based on the measurement of some quantity - recorded as the $y$-coordinate - as a function of some parameter recorded as the $x$-coordinate). Then we would like to find the equation of the line that {\it best fits} these points (by exactly what measurement the line represents a best possible fit is explained below). if we write the equation of the line as $y = l(x) = c_0 + c_1x$ for indeterminants $c_0, c_1$, then what we are looking for is a {\it least-squares solution} to the $n\times 2$ system of equations
\begin{align*}
c_0 + c_1 x_1 &= y_1\\
c_0 + c_1 x_2 &= y_2\\
&\vdots\\
c_0 + c_1 x_n &= y_n
\end{align*}

Note that, in this system, the $x_i$ and $y_j$ are constants, and we are trying to solve for $c_0$ and $c_1$. For $n\le 2$ there will be a solution, but in the overdetermined case there almost always fails to be one. Hence the need to work in the least-squares setting.
\vskip.2in

\begin{example} We wish to find the least-squares fit by a linear equation to the set of points $(2,3), (4,6), (7,10), (9,14)$. This problem can be represented by the matrix equation
\[
A1*{\bf c} = {\bf y}
\]
Where $A1 = \begin{bmatrix} 1 & 2\\1 & 4\\1 & 7\\1 & 9\end{bmatrix}$, ${\bf c} = \begin{bmatrix}c_0\\c_1\end{bmatrix}$, and ${\bf y} = \begin{bmatrix} 3\\6\\10\\14\end{bmatrix}$. We note that this matrix is full rank. Therefore least-squares solution is unique and given by
\[
{\bf c} = (A1^T*A1)^{-1}*A1^T*{\bf y} = \begin{bmatrix}-.18966\\ 1.53448\end{bmatrix}
\]
Thus the desired equation is given by
\[
l(x) = -.18966 + 1.53448 x
\]
We can also measure the degree to which this comes close to being an actual solution (which would only exist if the points were colinear). Given $\bf c$, the vector
\[
{\bf y}_1 := A1*{\bf c} = \begin{bmatrix} 2.8793\\ 5.9483\\ 10.5517\\ 13.6207\end{bmatrix}
\]
is (by the above) the least-squares approximation to $\bf y$ by a vector in the column space of $A1$ (accurate to 4 decimal places). The accuracy can then be estimated by the distance of this approximation to the original vector $\bf y$:
\[
e_1 := \|{\bf y} - {\bf y}_1\| = 0.68229
\]
\end{example}

The last computation in this example indicates what is being minimized when one fits data points in this way.

\begin{remark} Using least-squares linear approximation techniques to find the best linear fit to a set of $n$ data points $\{(x_1,y_1), (x_2,y_2),\dots,(x_n,y_n)\}$ results in the equation of a line $l(x) = c_0 + c_1(x)$ which minimizes the sum of the squares of the {\it vertical} distances from the given points to the line:
\[
\sum_{i=1}^n (y_i - l(x_i))^2
\]
Note that, unless the line is horizontal, the vertical distance will be slightly larger than the {\it actual} distance, which is measured in the direction orthogonal to the line, and minimizing the sum of squares of those distances would correspond geometrically to what one might normally think of as constituting a least-squares fit. However, the computation needed to find the best fit with respect to this sum is quite a bit more involved,. This linear algebraic approach provides a simple and efficient method for finding a good approximation by a line which will be exact whenever the points are colinear.
\end{remark}

The setup above provides a method for finding not just linear approximations, but higher order ones as well. The linear algebra is essentially the same. To illustrate,

\begin{example} Suppose instead we were asked to find the least-squares fit by a {\it quadratic} equation to the same set set of points $(2,3), (4,6), (7,10), (9,14)$. As before, this problem can be represented by the matrix equation
\[
A2*{\bf c} = {\bf y}
\]
Where $A2 = \begin{bmatrix} 1 & 2 & 4\\1 & 4 & 16\\1 & 7 & 49\\1 & 9 & 81\end{bmatrix}$, ${\bf c} = \begin{bmatrix}c_0\\c_1\\c_2\end{bmatrix}$, and ${\bf y} = \begin{bmatrix} 3\\6\\10\\14\end{bmatrix}$. We note that the matrix $A_2$ is again full rank (it has rank 3). Therefore least-squares solution is unique and given by
\[
{\bf c} = (A2^T*A2)^{-1}*A2^T*{\bf y} = \begin{bmatrix} 0.960345\\ 0.984483\\0.050000\end{bmatrix}
\]
Thus the desired equation is given by
\[
q(x) = 0.960345 + 0.984483 x + 0.050000 x^2
\]
Measuring the degree to which this comes close to being an actual solution (which would only exist if the points all lay on the same quadratic graph), we compute
\[
{\bf y}_2 := A2*{\bf c} = \begin{bmatrix} 3.1293\\ 5.6983\\ 10.3017\\ 13.8707\end{bmatrix}
\]
is (by the above) the least-squares approximation to $\bf y$ by a vector in the column space of $A2$ (accurate to 4 decimal places). The accuracy can then be estimated by the distance of this approximation to the original vector $\bf y$:
\[
e_2 := \|{\bf y} - {\bf y}_2\| = 0.46424
\]
As with the linear fit, the quantity being minimized is the sum of squares of vertical distances of the original points to the graph of this quadratic function. Notice the modest improvement; from $0.68229$ to $0.46424$. Because the column space of $A2$ contains the columns space of $A1$, the least-squares approximation ${\bf y}_2$ has to be at least as good as the linear one ${\bf y}_1$, and almost always will be closer to the original vector $\bf y$.
\end{example}

We will illustrate our final point by looking at what happens if we go one degree higher.

\begin{example} We will find the least-squares fit by a {\it cubic} equation to the same set set of points $(2,3), (4,6), (7,10), (9,14)$. As before, this problem can be represented by the matrix equation
\[
A3*{\bf c} = {\bf y}
\]
Where $A3 = \begin{bmatrix} 1 & 2 & 4 & 8\\1 & 4 & 16 & 64\\1 & 7 & 49 & 343\\1 & 9 & 81 & 729\end{bmatrix}$, ${\bf c} = \begin{bmatrix}c_0\\c_1\\c_2\\c_3\end{bmatrix}$, and ${\bf y} = \begin{bmatrix} 3\\6\\10\\14\end{bmatrix}$. The matrix $A_3$ is still full rank (it has rank 4). Therefore least-squares solution is unique and given by
\[
{\bf c} = (A3^T*A3)^{-1}*A3^T*{\bf y} = \begin{bmatrix} -1.6\\ 2.890476\\-0.342857\\0.023810\end{bmatrix}
\]
Thus the desired equation is given by
\[
f(x) = -1.6 + 2.890476 x + -0.342857 x^2 + 0.023810 x^3
\]
However, now when we compute the least-squares approximation we get
\[
{\bf y}_3 := A3*{\bf c} = \begin{bmatrix} 3\\ 6\\ 10\\ 14\end{bmatrix}
\]
which is not just an approximation but rather the vector $\bf y$ on the nose; $e_3 = 0$.. In other words, given these four points, {\it there is a unique cubic equation which fits the points exactly}. Inspecting the computation more carefully, we see why: the matrix $A3$ is both full rank and {\it square}. In other words, non-singular. In this case the system of equations is no longer over determined but rather balanced. And with a non-singular coefficient matrix, we get a unique solution. Symbolically, this can be seen by noting that the non-singularity of $A3$ results in a simplified expression for $\bf c$, confirming it is indeed an exact solution:
\[
{\bf c} = (A3^T*A3)^{-1}*A3^T*{\bf y} = A3^{-1}*(A3^T)^{-1}*A3^T*{\bf y} = A3^{-1}*y
\]
\end{example}

This set of examples, in which we compute successively higher order approximations to a set of $n$ data points until we finally arrive at an exact fit, is part of a more general phenomenon, which we record without proof by the following theorem.

\begin{theorem} Given $n$ points in $\mathbb R^2$ with distinct $x$-coordinates $\{(x_1,y_1), (x_2,y_2),\dots,(x_n,y_n)\}$, the least-squares fit by a polynomial of degree $k$ is computed by finding the least-squares solution to the matrix equation
\[
A_k*{\bf c} = {\bf y}
\]
where ${\bf y} = [y_1\ y_2\ \dots y_n]$ and  $A_k$ is the $n\times (k+1)$ matrix with $A_k(i,j) = x_i^{j-1}$. The matrix $A_k$ will have full column rank for all $k\le (n-1)$, and so the least-squares solution $\bf c$ is unique and given by
\[
[c_0\ c_1\ \dots c_k]^T = {\bf c} = (A_k^T*A_k)^{-1}*A_k^T*{\bf y}
\]
with degree $k$ polynomial least-squares fit given by
\[
p_k(x) = \sum_{i=0}^k c_i x^i
\]
Because $A_{n-1}$  is non-singular, there will be a polynomial of degree at most $(n-1)$ which fits the points exactly. Moreover, the polynomial of degree at most $(n-1)$ which accomplishes this will be unique.
\end{theorem}
\vskip.3in

%%%%%%%%%%%%%%%%%%%%%%%%%%%%%%%%%%%%%%


\end{document}
