\documentclass{ximera}

% These macros are automatically generated from the "macros"
% XML element.  Make permanent edits there.
%
% History
%   2004/01/01  Initiated for FCLA, evolved from there
%   2006/09/17  Converted  _, ^  to \sb, \sp for TeX4ht
%   2014/02/01  Updated for MathBook XML projects
%               Obsolete in FCLA: \codeindent, \computerfont, \define
%               Change: MathJax wants \lt, so replaced by \lteval
%   2014/02/22  New: \orderof, \reals, \per
%   2015/08/16  Incorporated into MathBook XML version of FCLA
%
%%%%%%%%%%%%%%%%%%%%%
%
%     Conveniences
%
%%%%%%%%%%%%%%%%%%%%%
%
%  Order of (asymptotically limit of fraction is 1)
%  Usage: \orderof{some function}
%
\newcommand{\orderof}[1]{\sim #1}
%
%  Integers
%  Usage:  \Z
\newcommand{\Z}{\mathbb{Z}}
%
%  Real numbers, as set of scalars
%  Usage:  \reals
\newcommand{\reals}{\mathbb{R}}
%
%  n-space over real field
%  Usage: \complex{integer-dimension}
\newcommand{\real}[1]{\mathbb{R}^{#1}}
%
%  Complex numbers, as set of scalars
%  Usage:  \complexes
\newcommand{\complexes}{\mathbb{C}}
%
%  n-space over complex field
%  Usage: \complex{integer-dimension}
\newcommand{\complex}[1]{\mathbb{C}^{#1}}
\newcommand{\CC}{\mathbb{C}}
%
%  Complex conjugation (scalar, vector, matrix)
%  Usage: \conjugate{object}
\newcommand{\conjugate}[1]{\overline{#1}}
%
%  Complex number modulus
%  Usage: \modulus{a+bi}
%  Presumes math mode
\newcommand{\modulus}[1]{\left\lvert#1\right\rvert}
%
%  Zero vector
%  Usage: \zerovector
\newcommand{\zerovector}{\vect{0}}
%
%  Zero matrix
%  Usage: \zeromatrix, use a subscript when size is important
\newcommand{\zeromatrix}{\mathcal{O}}
%
%  Inner product (brackets, not quadratic form)
%  Usage: \innerproduct{a-vector}{a-vector}
\newcommand{\innerproduct}[2]{\left\langle#1,\,#2\right\rangle}
%
%  Norm of a vector
%  Usage: \norm{a-vector}
\newcommand{\norm}[1]{\left\lVert#1\right\rVert}
%
%  Dimension
%  Usage: \dimension{vector-space-letter}
\newcommand{\dimension}[1]{\dim\left(#1\right)}
%
%  Nullity
%  Usage: \nullity{matrix-or-lintrans-letter}
\newcommand{\nullity}[1]{n\left(#1\right)}
%
%  Rank
%  Usage: \rank{matrix-or-lintrans-letter}
\newcommand{\rank}[1]{r\left(#1\right)}
%
%  Direct sum
%  Usage: \ds between a couple of subspaces
%
\newcommand{\ds}{\oplus}
%
%  Determinant of a matrix (functional)
%  Usage: \detname{A}
\newcommand{\detname}[1]{\det\left(#1\right)}
%
%  Determinant of a matrix (vertical bars)
%  Usage: \detbars{A}
\newcommand{\detbars}[1]{\left\lvert#1\right\rvert}
%
%  Trace of a Matrix
%  Usage: \trace{matrix name}
\newcommand{\trace}[1]{t\left(#1\right)}
%
%  Square Root of a Matrix
%  Usage: \sr{a-matrix}
\newcommand{\sr}[1]{#1^{1/2}}
%
%%%%%%%%%%%%%%%%%%%%%
%
%     Subspace Constructions
%
%%%%%%%%%%%%%%%%%%%%%
%
%  Span of a set of vectors
%  \span and \sp are used by TeX for other things
%  Usage: \spn{set-of-vectors}
\newcommand{\spn}[1]{\left\langle#1\right\rangle}
%
%  Null space of a matrix
%  Usage:  \nsp{A}
\newcommand{\nsp}[1]{\mathcal{N}\!\left(#1\right)}
%
%  Column space of a matrix
%  Usage:  \csp{A}
\newcommand{\csp}[1]{\mathcal{C}\!\left(#1\right)}
%
%  Row space of a matrix
%  Usage:  \rsp{A}
\newcommand{\rsp}[1]{\mathcal{R}\!\left(#1\right)}
%
%  Left null space of a matrix
%  Usage:  \lns{A}
\newcommand{\lns}[1]{\mathcal{L}\!\left(#1\right)}
%
%  Orthogonal complement of a vector space
%  Avoiding TeX's \perp
%  Usage:  \per{A}
\newcommand{\per}[1]{#1^\perp}
%
%%%%%%%%%%%%%%%%%%%%%
%
%     Systems of Equations
%
%%%%%%%%%%%%%%%%%%%%%
%
%  In-line form of an augmented matrix for a system of equations
%  Usage: \augmented{coefficient-matrix}{constant-vector}
\newcommand{\augmented}[2]{\left\lbrack\left.#1\,\right\rvert\,#2\right\rbrack}
%
%  Notation for a linear system before introducing matrix multiplication
%  Usage: \linearsystem{coefficient-matrix}{constant-vector}
\newcommand{\linearsystem}[2]{\mathcal{LS}\!\left(#1,\,#2\right)}
%
%  Notation for a homogenous system before introducing matrix multiplication
%  Usage: \homosystem{coefficient-matrix}
\newcommand{\homosystem}[1]{\linearsystem{#1}{\zerovector}}
%
%%%%%%%%%%%%%%%%%%%%%
%
%     Row Operations, Echelon Form
%
%%%%%%%%%%%%%%%%%%%%%
%
% Row operations on matrices
%
% Three commands to shorten up descriptions of gaussian elimination
%
% Usage: \rowopswap{row-i}{row-j}
% Usage: \rowopmult{scalar}{row-i}
% Usage: \rowopadd{scalar}{row-multiplied}{row-added-to}
\newcommand{\rowopswap}[2]{R_{#1}\leftrightarrow R_{#2}}
\newcommand{\rowopmult}[2]{#1R_{#2}}
\newcommand{\rowopadd}[3]{#1R_{#2}+R_{#3}}
%
% Mark leading 1's in echelon form with fbox
% Usage: \leading{a-1-usually}
\newcommand{\leading}[1]{\fbox{#1}}
%
%  Row-reduce arrow
%  Usage:  \rref inbetween a matrix and its reduced row-echelon form
\newcommand{\rref}{\xrightarrow{\text{RREF}}}
%
%  Elementary Matrices
%  Usage: \elemswap{subscript}{subscript}
%  Usage: \elemmult{scalar}{subscript}
%  Usage: \elemadd{scalar}{subscript-mult}{subscript-target}
%
\newcommand{\elemswap}[2]{E_{#1,#2}}
\newcommand{\elemmult}[2]{E_{#2}\left(#1\right)}
\newcommand{\elemadd}[3]{E_{#2,#3}\left(#1\right)}
%
%%%%%%%%%%%%%%%%%%%%%
%
%     2-D Constructions (Lists, Vectors, Matrices)
%
%%%%%%%%%%%%%%%%%%%%%
%
%  A list of scalars of generic length
%  Usage:  \scalarlist{scalar letter}{terminal subscript}
\newcommand{\scalarlist}[2]{{#1}_{1},\,{#1}_{2},\,{#1}_{3},\,\ldots,\,{#1}_{#2}}
%
%  Vector styling, bold (or use wiggles, arrows, whatever)
%  Subscripts go outside this construction
%  Usage: \vect{a symbol to use as a vector}
%  Have to already be in math mode
%
\newcommand{\vect}[1]{\mathbf{#1}}
%
%  A column vector
%  Usage: \colvector{list-delimited-by-\\}
%
\newcommand{\colvector}[1]{\begin{bmatrix}#1\end{bmatrix}}
%
%  A generic vector with components
%  Usage: \vectorcomponents{component-letter}{final-subscript}
\newcommand{\vectorcomponents}[2]{\colvector{#1_{1}\\#1_{2}\\#1_{3}\\\vdots\\#1_{#2}}}
%
%  A list of vectors of generic length
%  Usage:  \vectorlist{vector letter}{terminal subscript}
\newcommand{\vectorlist}[2]{\vect{#1}_{1},\,\vect{#1}_{2},\,\vect{#1}_{3},\,\ldots,\,\vect{#1}_{#2}}
%
%  Vector entries, entry i of vector v
%  (vector-expession still needs \vect, etc.)
%  Usage:  \vectorentry{vector-expression}{single-subscript}
\newcommand{\vectorentry}[2]{\left\lbrack#1\right\rbrack_{#2}}
%
%  Matrix entries, entry i,j of matrix A
%  Usage:  \matrixentry{matrix-expression}{paired-subscripts}
%
\newcommand{\matrixentry}[2]{\left\lbrack#1\right\rbrack_{#2}}
%
%  A generic linear combination
%  Usage:  \lincombo{scalar letter}{vector letter}{terminal subscript}
\newcommand{\lincombo}[3]{#1_{1}\vect{#2}_{1}+#1_{2}\vect{#2}_{2}+#1_{3}\vect{#2}_{3}+\cdots +#1_{#3}\vect{#2}_{#3}}
%
%  Matrix, column by column, as vectors
%  Usage:  \matrixcolumns{matrix letter}{terminal subscript}
\newcommand{\matrixcolumns}[2]{\left\lbrack\vect{#1}_{1}|\vect{#1}_{2}|\vect{#1}_{3}|\ldots|\vect{#1}_{#2}\right\rbrack}
%
%%%%%%%%%%%%%%%%%%%%%
%
%     Special Matrices
%
%%%%%%%%%%%%%%%%%%%%%
%
%  Transpose of a matrix
%  Usage:  \transpose{A}
\newcommand{\transpose}[1]{#1^{t}}
%
%  Inverse of a matrix
%  Usage:  \inverse{A}
\newcommand{\inverse}[1]{#1^{-1}}
%
%  Submatrix (for minors, determinants)
%  Usage: \submatrix{matrix-name}{delete-row}{delete-col}
\newcommand{\submatrix}[3]{#1\left(#2|#3\right)}
%
%  Adjoint of a matrix (twice)
%  This macro is a convenience to call \transpose and \conjugate properly
%  It shouldn't need to be modified (or mathematical meanings will change)
%  Usage:  \adj{A}
\newcommand{\adj}[1]{\transpose{\left(\conjugate{#1}\right)}}
%
%  This macro controls the symbol used for the adjoint
%  It can be edited to taste
%  Usage:  \adjoint{A}
\newcommand{\adjoint}[1]{#1^\ast}
%
%%%%%%%%%%%%%%%%%%%%%
%
%     Sets
%
%%%%%%%%%%%%%%%%%%%%%
%
%  A convenience for simple sets
%  Usage:  \set{list of element}
\newcommand{\set}[1]{\left\{#1\right\}}
%
%  Sets with vertical bar, "such that", sized for objects, not condition
%  Usage:  \setparts{objects}{condition}
%
%%\newcommand{\setparts}[2]{\left\{ #1\mid#2\right\}}
%%\newcommand{\setparts}[2]{\left\{\left. #1\right\rvert#2\right\}}
\newcommand{\setparts}[2]{\left\lbrace#1\,\middle|\,#2\right\rbrace}
%
%  Set Cardinality
%  Usage:  \card{a-set-letter}
\newcommand{\card}[1]{\left\lvert#1\right\rvert}
%
%  Set Union
%  Use \cup
%
%  Set Intersection
%  Use \cap
%
%  Set Complement
%  Usage:  \setcomplement{a-set-letter}
\newcommand{\setcomplement}[1]{\overline{#1}}
%
%%%%%%%%%%%%%%%%%%%%%
%
%     Eigenvalues and Eigenspaces
%
%%%%%%%%%%%%%%%%%%%%%
%
%  Characteristic polynomial
%  Usage: \charpoly{matrix-letter}{variable-letter}
\newcommand{\charpoly}[2]{p_{#1}\left(#2\right)}
%
%  Eigenspace
%  Usage: \eigenspace{matrix-letter}{eigenvalue-letter}
\newcommand{\eigenspace}[2]{\mathcal{E}_{#1}\left(#2\right)}
%
%  2013/10/03 Including ampersands is problematic here, 
%  think about fixes later
%  2014/02/22 Limited testing, seems &amp; is fine for HTML and LaTeX
%  2016-07-20 only employed in Archetypes, MBX has gather/align override
%  Eigensystem (presumes wrapped in an mrow within md)
%  Usage: \eigensystem{matrixletter}{eigenvalue}{list of basis vectors}
\newcommand{\eigensystem}[3]{\lambda&amp;=#2&amp;\eigenspace{#1}{#2}&amp;=\spn{\set{#3}}} 
%
%  Generalized Eigenspace
%  Usage: \geneigenspace{lin-trans-letter}{eigenvalue-letter}
\newcommand{\geneigenspace}[2]{\mathcal{G}_{#1}\left(#2\right)}
%
%  Algebraic multiplicty
%  Usage: \algmult{matrix-letter}{eigenvalue-letter}
\newcommand{\algmult}[2]{\alpha_{#1}\left(#2\right)}
%
%  Geometric multiplicty
%  Usage: \geomult{matrix-letter}{eigenvalue-letter}
\newcommand{\geomult}[2]{\gamma_{#1}\left(#2\right)}
%
%  Index (of eigenvalue)
%  Usage: \indx{matrix-letter}{eigenvalue-letter}
\newcommand{\indx}[2]{\iota_{#1}\left(#2\right)}
%
%%%%%%%%%%%%%%%%%%%%%
%
%     Linear Transformations
%
%%%%%%%%%%%%%%%%%%%%%
%
%  MathJax defines \lt to ease XML confusion
%
%  Linear transformation definition
%  Usage: \ltdefn{name-letter}{domain}{range}
\newcommand{\ltdefn}[3]{#1\colon #2\rightarrow#3}
%
%  Linear transformation evaluation
%  Usage: \lteval{name-letter}{input}
%  Replaces old \lt desired by MathJax
\newcommand{\lteval}[2]{#1\left(#2\right)}
%
% Linear transformation inverse
%  Usage: \ltinverse{name-letter}
\newcommand{\ltinverse}[1]{#1^{-1}}
%
%  Linear transformation restriction
%  Usage: \restrict{name-letter}{subspace-letter}
\newcommand{\restrict}[2]{{#1}|_{#2}}
%
%  Linear transformation preimage
%  Usage: \preimage{name-letter}{codomain-element}
\newcommand{\preimage}[2]{#1^{-1}\left(#2\right)}
%
%  Range of a linear transformation
%  TeX uses \range for something else
%  Usage:  \rng{T}
\newcommand{\rng}[1]{\mathcal{R}\!\left(#1\right)}
%
%  Kernel of a linear transformation
%  TeX uses \ker to do something different
%  Usage:  \krn{T}
\newcommand{\krn}[1]{\mathcal{K}\!\left(#1\right)}
%
%  Linear transformation composition
%  Usage: \compose{function-name}{function-name}
\newcommand{\compose}[2]{{#1}\circ{#2}}
%
%  Vector space of linear transformations
%  Usage: \vslt{domains}{codomains}
%  Presumes math mode
\newcommand{\vslt}[2]{\mathcal{LT}\left(#1,\,#2\right)}
%
%%%%%%%%%%%%%%%%%%%%%
%
%     Vector and Matrix Representations
%
%%%%%%%%%%%%%%%%%%%%%
%
%  Isomorphism symbol
%  Usage: \isomorphic
\newcommand{\isomorphic}{\cong}
%
%  Similarity
%  Usage: \similar{inner-matrix}{outer-invertible-matrix}
%  Rearranging this will not "fix" all desired changes throughout
%
\newcommand{\similar}[2]{\inverse{#2}#1#2}
%
%  Vector representation function name
%  Usage: \vectrepname{basis-letter}
\newcommand{\vectrepname}[1]{\rho_{#1}}
%
%  Vector representation output
%  Usage: \vectrep{basis-letter}{input}
\newcommand{\vectrep}[2]{\lteval{\vectrepname{#1}}{#2}}
%
%  Vector representation inverse function name
%  (Added later, not used consistently in FCLA)
%  Usage: \vectrepinvname{basis-letter}
\newcommand{\vectrepinvname}[1]{\ltinverse{\vectrepname{#1}}}
%
%  Vector representation inverse output
%  Usage: \vectrepinv{basis-letter}{input}
\newcommand{\vectrepinv}[2]{\lteval{\ltinverse{\vectrepname{#1}}}{#2}}
%
%  Matrix representation
%  Usage: \matrixrep{transformation-letter}{domain-basis-letter}{codomain-basis-letter}
\newcommand{\matrixrep}[3]{M^{#1}_{#2,#3}}
%
%  Matrix representation column-by-colum
%  2016-07-20 only employed once?
%  Usage: \matrixrepcolumns{transformation-letter}{codomain-basis-letter}{codomain-basis-vector-letter}{final-index}
\newcommand{\matrixrepcolumns}[4]{\left\lbrack \left.\vectrep{#2}{\lteval{#1}{\vect{#3}_{1}}}\right|\left.\vectrep{#2}{\lteval{#1}{\vect{#3}_{2}}}\right|\left.\vectrep{#2}{\lteval{#1}{\vect{#3}_{3}}}\right|\ldots\left|\vectrep{#2}{\lteval{#1}{\vect{#3}_{#4}}}\right.\right\rbrack}
%
%  Change of basis matrix
%  Usage: \cbm{domain-basis-letter}{codomain-basis-letter}
\newcommand{\cbm}[2]{C_{#1,#2}}
%
%%%%%%%%%%%%%%%%%%%%%
%
%     Canonical Forms
%
%%%%%%%%%%%%%%%%%%%%%
%
%  Jordan blocks
%  Usage: \jordan{size}{diagonal-element}
\newcommand{\jordan}[2]{J_{#1}\left(#2\right)}
%
%%%%%%%%%%%%%%%%%%%%%
%
%     Hadamard Matrices
%     Contributed by Elizabeth Million
%
%%%%%%%%%%%%%%%%%%%%%
%
%  Hadamard Product
%  Usage: \hadamard{a-matrix}{a-matrix}
\newcommand{\hadamard}[2]{#1\circ #2}
%
%  Hadamard identity matrix
%  Usage: \hadamardidentity{paired-subscripts-size-of-matrix}
\newcommand{\hadamardidentity}[1]{J_{#1}}
%
%  Hadamard inverse matrix
%  Usage: \hadamardinverse{matrix-expression}
\newcommand{\hadamardinverse}[1]{\widehat{#1}}

\newcommand{\definedTerm}[1]{\textbf{#1}}
\newcommand{\dfn}[1]{\textbf{#1}}

\newcommand{\wt}{\widetilde}
\newcommand{\ov}{\overline}
\newcommand{\inj}{\rightarrowtail}
\newcommand{\surj}{\twoheadrightarrow}
\newcommand{\harpoon}{\overset{\rightharpoonup}}

\newenvironment{amatrix}[1]{%
  \left[\begin{array}{@{}*{#1}{c}|c@{}}
}{%
  \end{array}\right]
}


\title{Gram-Schmidt}

\begin{document}
\begin{abstract}
  A linearly independent set of vectors can be transformed into an
  orthogonal set of vectors while preserving the span.
\end{abstract}
\maketitle

The Gram-Schmidt Procedure is really a theorem.  It says that if we
begin with a linearly independent set of $p$ vectors, $S$, then we can
do a number of calculations with these vectors and produce an
orthogonal set of $p$ vectors, $T$, so that $\spn{S}=\spn{T}$.  Given
the large number of computations involved, it is indeed a procedure to
do all the necessary computations, and it is best employed on a
computer.  However, it also has value in proofs where we may on
occasion wish to replace a linearly independent set by an orthogonal
set.

This is our first occasion to use the technique of ``mathematical
induction'' for a proof, a technique we will see again several times.

\begin{theorem}[Gram-Schmidt Procedure]
  \label{theorem:GSP}

  Suppose that $S=\set{\vectorlist{v}{p}}$ is a linearly independent set of vectors in $\complex{m}$.  Define the vectors $\vect{u}_i$, $1\leq i\leq p$ by
  \[
    \vect{u}_i=\vect{v}_i
    -\frac{\innerproduct{\vect{u}_1}{\vect{v}_i}}{\innerproduct{\vect{u}_1}{\vect{u}_1}}\vect{u}_1
    -\frac{\innerproduct{\vect{u}_2}{\vect{v}_i}}{\innerproduct{\vect{u}_2}{\vect{u}_2}}\vect{u}_2
    -\frac{\innerproduct{\vect{u}_3}{\vect{v}_i}}{\innerproduct{\vect{u}_3}{\vect{u}_3}}\vect{u}_3
    -\cdots
    -\frac{\innerproduct{\vect{u}_{i-1}}{\vect{v}_i}}{\innerproduct{\vect{u}_{i-1}}{\vect{u}_{i-1}}}\vect{u}_{i-1}
  \]
  
  
  Let $T=\set{\vectorlist{u}{p}}$.  Then $T$ is an orthogonal set of nonzero vectors, and $\spn{T}=\spn{S}$.
  
  \begin{proof}
    We will prove the result by using induction on $p$.
    
    To begin, we prove that $T$ has the desired properties when
    $p=\answer{1}$.  In this case $\vect{u}_1=\vect{v}_1$ and
    $T=\set{\vect{u}_1}=\set{\vect{v}_1}=S$.  Because $S$ and $T$ are
    equal, $\spn{S}=\spn{T}$.  Equally trivial, $T$ is an orthogonal
    set.  If $\vect{u}_1=\zerovector$, then $S$ would be a linearly
    dependent set, a contradiction.

    Suppose that the theorem is true for any set of $p-1$ linearly
    independent vectors.  Let $S=\set{\vectorlist{v}{p}}$ be a
    linearly independent set of $p$ vectors.  Then
    $S^\prime=\set{\vectorlist{v}{p-1}}$ is also linearly independent.
    So we can apply the theorem to $S^\prime$ and construct the
    vectors $T^\prime=\set{\vectorlist{u}{p-1}}$.  $T^\prime$ is
    therefore an orthogonal set of nonzero vectors and
    $\spn{S^\prime}=\spn{T^\prime}$.  Define
    \[
      \vect{u}_p=\vect{v}_p
      -\frac{\innerproduct{\vect{u}_1}{\vect{v}_p}}{\innerproduct{\vect{u}_1}{\vect{u}_1}}\vect{u}_1
      -\frac{\innerproduct{\vect{u}_2}{\vect{v}_p}}{\innerproduct{\vect{u}_2}{\vect{u}_2}}\vect{u}_2
      -\frac{\innerproduct{\vect{u}_3}{\vect{v}_p}}{\innerproduct{\vect{u}_3}{\vect{u}_3}}\vect{u}_3
      -\cdots
      -\frac{\innerproduct{\vect{u}_{p-1}}{\vect{v}_p}}{\innerproduct{\vect{u}_{p-1}}{\vect{u}_{p-1}}}\vect{u}_{p-1}
    \]
    and let $T=T^\prime\cup\set{\vect{u}_p}$.  We need to now show
    that $T$ has several properties by building on what we know about
    $T^\prime$.  But first notice that the above equation has no
    problems with the denominators
    ($\innerproduct{\vect{u}_i}{\vect{u}_i}$) being zero, since the
    $\vect{u}_i$ are from $T^\prime$, which is composed of nonzero
    vectors.

    We show that $\spn{T}=\spn{S}$, by first establishing that
    $\spn{T}\subseteq\spn{S}$.  Suppose $\vect{x}\in\spn{T}$, so
    \[
      \vect{x}=\lincombo{a}{u}{p}
    \]
    The term $a_p\vect{u}_p$ is a linear combination of vectors from
    $T^\prime$ and the vector $\vect{v}_p$, while the remaining terms
    are a linear combination of vectors from $T^\prime$.  Since
    $\spn{T^\prime}=\spn{S^\prime}$, any term that is a multiple of a
    vector from $T^\prime$ can be rewritten as a linear combination of
    vectors from $S^\prime$.  The remaining term $a_p\vect{v}_p$ is a
    multiple of a vector in $S$.  So we see that $\vect{x}$ can be
    rewritten as a linear combination of vectors from $S$, i.e.,
    $\vect{x}\in\spn{S}$.

    To show that $\spn{S}\subseteq\spn{T}$, begin with
    $\vect{y}\in\spn{S}$, so
    \[
      \vect{y}=\lincombo{a}{v}{p}
    \]

    Rearrange our defining equation for $\vect{u}_p$ by solving for
    $\vect{v}_p$.  Then the term $a_p\vect{v}_p$ is a multiple of a
    linear combination of elements of $T$.  The remaining terms are a
    linear combination of $\vectorlist{v}{p-1}$, hence an element of
    $\spn{S^\prime}=\spn{T^\prime}$.  Thus these remaining terms can
    be written as a linear combination of the vectors in $T^\prime$.
    So $\vect{y}$ is a linear combination of vectors from $T$, i.e.,
    $\vect{y}\in\spn{T}$.


    The elements of $T^\prime$ are nonzero, but what about
    $\vect{u}_p$?  Suppose to the contrary that
    $\vect{u}_p=\zerovector$,
    \begin{align*}
      \zerovector&=\vect{u}_p=\vect{v}_p
                   -\frac{\innerproduct{\vect{u}_1}{\vect{v}_p}}{\innerproduct{\vect{u}_1}{\vect{u}_1}}\vect{u}_1
                   -\frac{\innerproduct{\vect{u}_2}{\vect{v}_p}}{\innerproduct{\vect{u}_2}{\vect{u}_2}}\vect{u}_2
                   -\frac{\innerproduct{\vect{u}_3}{\vect{v}_p}}{\innerproduct{\vect{u}_3}{\vect{u}_3}}\vect{u}_3
                   -\cdots
                   -\frac{\innerproduct{\vect{u}_{p-1}}{\vect{v}_p}}{\innerproduct{\vect{u}_{p-1}}{\vect{u}_{p-1}}}\vect{u}_{p-1}\\
                 &\vect{v}_p=
                   \frac{\innerproduct{\vect{u}_1}{\vect{v}_p}}{\innerproduct{\vect{u}_1}{\vect{u}_1}}\vect{u}_1
                   +\frac{\innerproduct{\vect{u}_2}{\vect{v}_p}}{\innerproduct{\vect{u}_2}{\vect{u}_2}}\vect{u}_2
                   +\frac{\innerproduct{\vect{u}_3}{\vect{v}_p}}{\innerproduct{\vect{u}_3}{\vect{u}_3}}\vect{u}_3
                   +\cdots
                   +\frac{\innerproduct{\vect{u}_{p-1}}{\vect{v}_p}}{\innerproduct{\vect{u}_{p-1}}{\vect{u}_{p-1}}}\vect{u}_{p-1}
    \end{align*}

    Since $\spn{S^\prime}=\spn{T^\prime}$ we can write the vectors
    $\vectorlist{u}{p-1}$ on the right side of this equation in terms
    of the vectors $\vectorlist{v}{p-1}$ and we then have the vector
    $\vect{v}_p$ expressed as a linear combination of the other $p-1$
    vectors in $S$, implying that $S$ is a linearly dependent set
    (\ref{theorem:DLDS}), contrary to our lone hypothesis about $S$.

    Finally, it is a simple matter to establish that $T$ is an
    orthogonal set, though it will not appear so simple looking.
    Think about your objects as you work through the following---what
    is a vector and what is a scalar.  Since $T^\prime$ is an
    orthogonal set by induction, most pairs of elements in $T$ are
    already known to be orthogonal.  We just need to test ``new''
    inner products, between $\vect{u}_p$ and $\vect{u}_i$, for
    $1\leq i\leq p-1$.  Here we go, using summation notation,
    \begin{align*}
      \innerproduct{\vect{u}_i}{\vect{u}_p}&=
                                             \innerproduct{\vect{u}_i}{
                                             \vect{v}_p-\sum_{k=1}^{p-1}\frac{\innerproduct{\vect{u}_k}{\vect{v}_p}}{\innerproduct{\vect{u}_k}{\vect{u}_k}}\vect{u}_k
                                             }
      \\
                                           &=
                                             \innerproduct{\vect{u}_i}{\vect{v}_p}
                                             -
                                             \innerproduct{\vect{u}_i}{
                                             \sum_{k=1}^{p-1}\frac{\innerproduct{\vect{u}_k}{\vect{v}_p}}{\innerproduct{\vect{u}_k}{\vect{u}_k}}\vect{u}_k
                                             }
      \\ %&&\ref{theorem:IPVA}\\
                                           &=
                                             \innerproduct{\vect{u}_i}{\vect{v}_p}
                                             -
                                             \sum_{k=1}^{p-1}\innerproduct{\vect{u}_i}{
                                             \frac{\innerproduct{\vect{u}_k}{\vect{v}_p}}{\innerproduct{\vect{u}_k}{\vect{u}_k}}\vect{u}_k
                                             }
      \\ %&&\ref{theorem:IPVA}\\
                                           &=
                                             \innerproduct{\vect{u}_i}{\vect{v}_p}
                                             -
                                             \sum_{k=1}^{p-1}\frac{\innerproduct{\vect{u}_k}{\vect{v}_p}}{\innerproduct{\vect{u}_k}{\vect{u}_k}}\innerproduct{\vect{u}_i}{\vect{u}_k}
      \\ %&&\ref{theorem:IPSM}\\
                                           &=
                                             \innerproduct{\vect{u}_i}{\vect{v}_p}
                                             -
                                             \frac{\innerproduct{\vect{u}_i}{\vect{v}_p}}{\innerproduct{\vect{u}_i}{\vect{u}_i}}\innerproduct{\vect{u}_i}{\vect{u}_i}
                                             -
                                             \sum_{k\neq i}\frac{\innerproduct{\vect{u}_k}{\vect{v}_p}}{\innerproduct{\vect{u}_k}{\vect{u}_k}}(0)
\\ %&&\text{Induction Hypothesis}\\
                                           &=
                                             \innerproduct{\vect{u}_i}{\vect{v}_p}
                                             -
                                             \innerproduct{\vect{u}_i}{\vect{v}_p}
                                             -
                                             \sum_{k\neq i}0\\
                                           &=0
    \end{align*}
    
  \end{proof}
\end{theorem}

\begin{example}[Gram-Schmidt of three vectors]

  We will illustrate the Gram-Schmidt process with three vectors.  Begin with the linearly independent set
  \[
    S=\set{\vect{v}_1,\,\vect{v}_2,\,\vect{v}_3}=\set{
      \colvector{1\\1+i\\1},\,
      \colvector{-i\\1\\1+i},\,
      \colvector{0\\i\\i}
    }
  \]
  
  Then
  \begin{align*}
    \vect{u}_1&=\vect{v_1}=\colvector{1\\1+i\\1}\\
    \vect{u}_2&=\vect{v}_2
                -\frac{\innerproduct{\vect{u}_1}{\vect{v}_2}}{\innerproduct{\vect{u}_1}{\vect{u}_1}}\vect{u}_1
                =\frac{1}{4}\colvector{-2-3i\\1-i\\2+5i}\\
    \vect{u}_3&=\vect{v}_3
                -\frac{\innerproduct{\vect{u}_1}{\vect{v}_3}}{\innerproduct{\vect{u}_1}{\vect{u}_1}}\vect{u}_1
                -\frac{\innerproduct{\vect{u}_2}{\vect{v}_3}}{\innerproduct{\vect{u}_2}{\vect{u}_2}}\vect{u}_2
                =\frac{1}{11}\colvector{-3-i\\1+3i\\-1-i}
  \end{align*}
  and
  \[
    T=\set{\vect{u}_1,\,\vect{u}_2,\,\vect{u}_3}
    =\set{
      \colvector{1\\1+i\\1},\,
      \frac{1}{4}\colvector{-2-3i\\1-i\\2+5i},\,
      \frac{1}{11}\colvector{-3-i\\1+3i\\-1-i}
    }
  \]
  is an orthogonal set (which you can check) of nonzero vectors and
  $\spn{T}=\spn{S}$ (all by \ref{theorem:GSP}).  Of course, as a
  by-product of orthogonality, the set $T$ is also linearly
  independent (\ref{theorem:OSLI}).

\end{example}

One final definition related to orthogonal vectors.

\begin{definition}[OrthoNormal Set]

  Suppose $S=\set{\vectorlist{u}{n}}$ is an orthogonal set of vectors
  such that $\norm{\vect{u}_i}=1$ for all $1\leq i\leq n$.  Then $S$
  is an \dfn{orthonormal} set of vectors.

\end{definition}

Once you have an orthogonal set, it is easy to convert it to an
orthonormal set---multiply each vector by the reciprocal of its norm,
and the resulting vector will have norm 1.  This scaling of each
vector will not affect the orthogonality properties (apply
\ref{theorem:IPSM}).

\begin{example}[Orthonormal set, three vectors]

The set
\[
T=\set{\vect{u}_1,\,\vect{u}_2,\,\vect{u}_3}
=\set{
\colvector{1\\1+i\\1},\,
\frac{1}{4}\colvector{-2-3i\\1-i\\2+5i},\,
\frac{1}{11}\colvector{-3-i\\1+3i\\-1-i}
}
\]
from \ref{example:GSTV} is an orthogonal set.

We compute the norm of each vector,
\begin{align*}
\norm{\vect{u}_1}=\answer{2}&&
\norm{\vect{u}_2}=\frac{1}{2}\sqrt{11}&&
\norm{\vect{u}_3}=\frac{\sqrt{2}}{\sqrt{11}}
\end{align*}




Converting each vector to a norm of 1, yields an orthonormal set,
\begin{align*}
\vect{w}_1&=\frac{1}{\answer{2}}\colvector{1\\1+i\\1}\\
\vect{w}_2&=\frac{1}{\frac{1}{2}\sqrt{11}}\frac{1}{4}\colvector{-2-3i\\1-i\\2+5i}=\frac{1}{2\sqrt{11}}\colvector{-2-3i\\1-i\\2+5i}\\
\vect{w}_3&=\frac{1}{\frac{\sqrt{2}}{\sqrt{11}}}\frac{1}{11}\colvector{-3-i\\1+3i\\-1-i}=\frac{1}{\sqrt{22}}\colvector{-3-i\\1+3i\\-1-i}
\end{align*}

\end{example}

\begin{example}[Orthonormal set, four vectors]
  
  As an exercise convert the linearly independent set
  \[
    S=\set{
      \colvector{1+i\\1\\1-i\\i},\,
      \colvector{i\\1+i\\-1\\-i},\,
      \colvector{i\\-i\\ -1+i\\1},\,
      \colvector{-1-i\\i\\1\\-1}
    }
  \]
  to an orthogonal set via the Gram-Schmidt Process (\ref{theorem:GSP}) and then scale the vectors to norm 1 to create an orthonormal set.  You should get the same set you would if you scaled the orthogonal set of \ref{example:AOS} to become an orthonormal set.

\end{example}

We will see orthonormal sets again because they are intimately related
to unitary matrices (\ref{definition:UM}) through \ref{theorem:CUMOS}.
Some of the utility of orthonormal sets is captured by
\ref{theorem:COB}.  Orthonormal sets appear once again in
\ref{section:OD} where they are key in orthonormal diagonalization.

\end{document}
